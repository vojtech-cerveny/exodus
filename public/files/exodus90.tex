% ===============================================
% ===== SETTINGS FOR DOCUMENT
% ===============================================
\documentclass[11pt]{article}
\usepackage{geometry}
\geometry{
  top=3cm,
  bottom=3cm
}
\author{Exodus 90}
\usepackage[czech]{babel}
\font\myfont=cmr12 at 40pt
\setlength{\parskip}{1em plus 0.1em minus 0.2em} % Adjusts space between paragraphs
\usepackage{graphicx}
\usepackage{amssymb}
\usepackage{enumitem}
\setlist[enumerate]{itemsep=-0.4em} % Globálně nastavuje odsazení pro všechny seznamy
% Upravení prostoru okolo nečíslovaných podsekcí
\usepackage{titlesec}

% Upravení prostoru okolo nečíslovaných podsekcí
\titleformat*{\subsection}{\normalsize\bfseries}
\titlespacing*{\subsection}{0pt}{2.5ex minus 2ex}{1.3ex minus 1ex}

% ===============================================
% ===== CUSTOM COMMANDS
% ===============================================
\newcommand{\zacatekPrvniTyden}{
  \textbf{Jste v Egyptě} \newline
  Jste ochotni přijmout, že Egypt vás zotročil? „Ano“ bude vyžadovat úplnou změnu ve vašem způsobu života.

\subsection*{Týdenní úkony (ukazatelé cesty)}
\begin{enumerate}
  \item Vzdejte se kontroly
  \item Zavázejte se svému bratrstvu
  \item Najít si čas pro každodenní modlitbu
  \item Buďte radostní
  \item Každou noc zkoumejte svůj den
\end{enumerate}
Modlete se, aby Pán osvobodil vás a vaše bratrství. \newline
Modleme se za svobodu všech mužů v exodu, stejně tak, jako se oni modlí za vás.\newline
Ve jménu Otce i Syna i Ducha svatého …  Otče náš… Amen
}

\newcommand{\zacatekDruhyTyden}{
  \textbf{Jste v Egyptě} \newline
  Disciplíny duchovního cvičení zvětštily naši dříve nevědomou náklonnost k lidskému komfortu.
  Nebylo by snazší toto duchovní cvičení opustit a zůstat v pohodlí otroctví navěky?

  \subsection*{Týdenní úkony (ukazatelé cesty)}
\begin{enumerate}
  \item Dobře se vyzpovídejte
  \item Držte se denních reflexí
  \item Navštěvujte jednu mši v týdnu navíc
  \item Zvažte přečtení Průvodce terénem
  \item Uvědomte si (zjistěte), kde je vaše kotva
\end{enumerate}
Modlete se, aby Pán osvobodil vás a vaše bratrství. \newline
Modleme se za svobodu všech mužů v exodu, stejně tak, jako se oni modlí za vás.\newline
Ve jménu Otce i Syna i Ducha svatého …  Otče náš… Amen
}

\newcommand{\zacatekTretiTyden}{
  \textbf{Jste v Egyptě} \newline
  Život se stal náročnějším; Zdá se, že Mojžíši a Áronovi se nedaří získat větší svobodu. Navzdory tomu všemu,
komu se tento týden rozhodnete sloužit: Bohu, nebo faraonovi?

\subsection*{Týdenní úkony (ukazatelé cesty)}
\begin{enumerate}
  \item Pokračujte ve zkoumání svého dne
  \item Přistupujte upřímně k vaší denní svaté hodině
  \item Neodbývejte úkony/úkoly (don’t cut corners-neřežte rohy)
  \item Pamatujte si své proč
  \item Zůstaňte radostní
\end{enumerate}
Modlete se, aby Pán osvobodil vás a vaše bratrství. \newline
Modleme se za svobodu všech mužů v exodu, stejně tak, jako se oni modlí za vás.\newline
Ve jménu Otce i Syna i Ducha svatého …  Otče náš… Amen
}

\newcommand{\zacatekCtvrtyTyden}{
  \textbf{Jste v Egyptě} \newline
  Izraelité konečně zří pravou cenu opuštění Egypta. Uvědomte si, jaké falešné bohy ničí Bůh ve vašem životě.
Dovolte, aby vám tato realita přinesla změnu srdce, kterou potřebujete, abyste sloužili pouze Bohu samotnému.

\subsection*{Týdenní úkony (ukazatelé cesty)}
\begin{enumerate}
  \item Zavázejte se (odevzdejte se) svému bratrstvu
  \item Vytvořte si cvičící plán
  \item Uvědomte si Boží moc
  \item Jděte/vyjděte ven
\end{enumerate}
Modlete se, aby Pán osvobodil vás a vaše bratrství. \newline
Modleme se za svobodu všech mužů v exodu, stejně tak, jako se oni modlí za vás.\newline
Ve jménu Otce i Syna i Ducha svatého …  Otče náš… Amen
}

\newcommand{\zacatekPatyTyden}{
  \textbf{Východ z/od Egypta (východně od Egypta), útěk do pouště} \newline
  Izraelci zabili egyptského boha (beránka) a veřejně rozmazali jehněčí krev na jejich veřeje. Nyní vstupujete
  do pouště. Vyplatí se vám zůstat velmi blízko Bohu a vašemu bratrstvu.

\subsection*{Týdenní úkony (ukazatelé cesty)}
\begin{enumerate}
  \item Vzdejte se kontroly
  \item Zkontaktujte svou kotvu
  \item Znovu si připomeňte své proč
  \item Zvažte přečtení Průvodce terénem
\end{enumerate}
Modlete se, aby Pán osvobodil vás a vaše bratrství. \newline
Modleme se za svobodu všech mužů v exodu, stejně tak, jako se oni modlí za vás.\newline
Ve jménu Otce i Syna i Ducha svatého …  Otče náš… Amen
}

\newcommand{\zacatekSestyTyden}{
  \textbf{Severozápadní břeh Rudého moře} \newline 
  Faraon a jeho armáda opustili Egypt v neúprosném pronásledování Izraelitů. Jen co si Izraelité pomysleli, že jsou zcela na svobodě, ocitli se uvězněni mezi zuřící armádou a zdánlivě neprůchodným vodním útvarem. Ztratit zde naději by znamenalo vzdát se víry v Boha, a vyústění by vedlo k ještě tvrdšímu zotročení než kdy dříve. Nacházíme se na podobném místě. Naše dřívější zvyky na nás útočí. Pokud ztratíme víru, skončíme a otočíme se nazpět, budeme znovu zotročeni. Pokud však zůstaneme oddaní naší víře, Pán nás povede skrz vody. Očistí nás a oddělí nás od našich nepřátel jako nikdy předtím. Co si tento týden zvolíte?

\subsection*{Týdenní úkony (ukazatelé cesty)}
\begin{enumerate}
  \item Udělejte si čas na dobrou zpověď
  \item Držte se reflexí
  \item Vstupte do Slova
\end{enumerate}
Modlete se, aby Pán osvobodil vás a vaše bratrství. \newline
Modleme se za svobodu všech mužů v exodu, stejně tak, jako se oni modlí za vás.\newline
Ve jménu Otce i Syna i Ducha svatého …  Otče náš… Amen
}

\newcommand{\zacatekSedmyTyden}{
  \textbf{Jste v drsné hornaté poušti na úpatí hory Sinaj} \newline 
  Ve svých rukou držíte plán svobody: modlitba, askeze a bratrství. Stále je před námi mnoho práce. Nové zvyky vyžadují dostatek času (značný čas), aby se vytvořily.

  \subsection*{Týdenní úkony (ukazatelé cesty)}
\begin{enumerate}
  \item Ctěte svou svatou hodinu
  \item Běžte ke zdroji
  \item Přistupujte k vašemu nočnímu examen zodpovědně
  \item Zůstaňte radostní
\end{enumerate}
Modlete se, aby Pán osvobodil vás a vaše bratrství. \newline
Modleme se za svobodu všech mužů v exodu, stejně tak, jako se oni modlí za vás.\newline
Ve jménu Otce i Syna i Ducha svatého …  Otče náš… Amen
}

\newcommand{\zacatekOsmyTyden}{
  \textbf{Drsná hornatá poušť, úpatí hory Sinaj} \newline 
  Izraelité se utábořili na úpatí hory Sinaj, kde čekají na Boží slovo. Zůstanete zde ještě s Izraelity příští čtyři týdny. Pro vás je to čas obnovy. Čas být v horské poušti čekající na Boží slovo. Budete-li poslouchat pozorně, mnoho dostanete. Budete-li žít prostopášně, poputujete zpět do otroctví, ze kterého jste byli zrovna propuštěni. Přestaňte se dívat na sebe a na dny minulosti. Upřete zrak na drsné hory před vámi. Nechte se mocností skal a výškou vrcholů inspirovat k lepší změně.
  \subsection*{Týdenní úkony (ukazatelé cesty)}
\begin{enumerate}
  \item Stále prahněte po svobodě (mějte stále touhu po vaší svobodě)
  \item Držte se disciplín
  \item Víte, kde je vaše kotva
\end{enumerate}
Modlete se, aby Pán osvobodil vás a vaše bratrství. \newline
Modleme se za svobodu všech mužů v exodu, stejně tak, jako se oni modlí za vás.\newline
Ve jménu Otce i Syna i Ducha svatého …  Otče náš… Amen
}

\newcommand{\zacatekDevatyTyden}{
\textbf{Drsná hornatá poušť, úpatí hory Sinaj} \newline 
Izraelité se učí, jak postavit a vyzdobit Svatostánek, posvátné místo, kde by Bůh přebýval mezi svým lidem. „Vaše tělo je chrámem Ducha svatého, který ve vás přebývá a jejž máte od Boha,“ (1 Kor 6,19). Jak vyzdobujete svůj chrám? Jakým věcem dovolujete vstup na toto posvátné místo? Co děláte pro to, abyste zajistili, že toto místo bude posvátné jako Svatostánek, aby ve vás mohl přebývat Bůh?

\subsection*{Týdenní úkony (ukazatelé cesty)}
\begin{enumerate}
  \item Držte se pravidelného cvičení
  \item Dodržujte svůj závazek vůči svému bratrstvu
  \item Udělejte si čas na vaše bratrstvo i mimo pravidelné setkání
  \item Dodržujte pravidelné noční examen (s pečlivostí)
  \item Zůstaňte radostní
\end{enumerate}
Modlete se, aby Pán osvobodil vás a vaše bratrství. \newline
Modleme se za svobodu všech mužů v exodu, stejně tak, jako se oni modlí za vás.\newline
Ve jménu Otce i Syna i Ducha svatého …  Otče náš… Amen
}

\newcommand{\zacatekDesatyTyden}{
\textbf{Drsná hornatá poušť, úpatí hory Sinaj} \newline 
Přemýšlejte s Izraelity v poušti, jakým způsobem svou nově nabytou svobodu využijete. Použijete ji ke službě Pánu, nebo ke službě sobě samému /aby jsi sloužil sám sobě? Jenom jedna cesta vás zanechá svobodnými.

\subsection*{Týdenní úkony (ukazatelé cesty)}
\begin{enumerate}
  \item Zkontaktujte svou kotvu
  \item Držte se disciplín
  \item Stále se spoléhejte na Pána
\end{enumerate}
Modlete se, aby Pán osvobodil vás a vaše bratrství. \newline
Modleme se za svobodu všech mužů v exodu, stejně tak, jako se oni modlí za vás.\newline
Ve jménu Otce i Syna i Ducha svatého …  Otče náš… Amen
}

\newcommand{\zacatekJedenactyTyden}{
\textbf{Drsná hornatá poušť, úpatí hory Sinaj} \newline 
Levité dokazují, že ve zhýralém světě je možné nebojácně/odvážně důvěřovat/být věrný Hospodinu. Pán však dokazuje něco ještě většího. Prokazuje/ukazuje, že navždy zůstane věrný své smlouvě (k nám), i když Jej zklameme.

\subsection*{Týdenní úkony (ukazatelé cesty)}
\begin{enumerate}
  \item Vzdejte se kontroly
  \item Vzpomeňte si na své proč
  \item Dobře se potřetí vyzpovídejte
  \item Rozdávejte/sdílejte radost
\end{enumerate}
Modlete se, aby Pán osvobodil vás a vaše bratrství. \newline
Modleme se za svobodu všech mužů v exodu, stejně tak, jako se oni modlí za vás.\newline
Ve jménu Otce i Syna i Ducha svatého …  Otče náš… Amen
}

\newcommand{\zacatekDvanactyTyden}{
\textbf{Drsná hornatá poušť, úpatí hory Sinaj} \newline 
Před vámi jsou rozhodující/zásadní zbývající dny exodu. Stále jste očišťováni. Zavázali jste se k devadesáti dnům exodu, nikoli sedmdesát, osmdesát nebo dokonce osmdesát devět. Prodlužte svůj krok a zvyšte svoji dynamiku.

\subsection*{Týdenní úkony (ukazatelé cesty)}
\begin{enumerate}
  \item Znovu se zavázejte k dennímu rozjímání
  \item Pokračujte ve svém závazku ke Slovu
  \item Upevněte svůj plán pro Den 91
\end{enumerate}
Modlete se, aby Pán osvobodil vás a vaše bratrství. \newline
Modleme se za svobodu všech mužů v exodu, stejně tak, jako se oni modlí za vás.\newline
Ve jménu Otce i Syna i Ducha svatého …  Otče náš… Amen
}

\newcommand{\zacatekTrinactyTyden}{
\textbf{Nacházíte se na východním okraji země zaslíbené} \newline 
Dozvěděli/učili jste se, co to znamená každý den brát na sebe svůj kříž a následovat Krista. Učili jste se cestě, která vede ke svobodě. Žijte ji, sdílejte ji a zůstaňte svobodní v 91. dni.

\subsection*{Týdenní úkony (ukazatelé cesty)}
\begin{enumerate}
  \item Ohlédněte (reflektujte) se na vaše uplynulé tři měsíce modlitby.
  \item Uvědomte si svou potřebu Boha.
  \item Uvědomte si, jak jste se v těchto posledních měsících věnovali Písmu.
\end{enumerate}
Modlete se, aby Pán osvobodil vás a vaše bratrství. \newline
Modleme se za svobodu všech mužů v exodu, stejně tak, jako se oni modlí za vás.\newline
Ve jménu Otce i Syna i Ducha svatého …  Otče náš… Amen
}

% ===============================================
% ===== DOCUMENT BEGINS
% ===============================================

\title{\myfont Exodus 90 - Česká verze}
\date{}							% Activate to display a given date or no date

\begin{document}

\maketitle
\vspace*{\fill}
Pokud preferuješ tištěnou verzi Exodusu, pak jsi zde správě. Tato verze je připravená na tisk tak, aby se dobře četla. Užij si ji a i s ní i celý Exodus 90!

% ===============================================
% ===== PRVNI TYDEN
% ===============================================
%ukony
\newpage
\section*{Úkony (ukazatel cesty) pro 1. týden}

\textbf{Místo:} Egypt (jste v Egyptě)

Tento týden se ocitnete s Izraelity v Egyptě. Egyptské prostředí se za celá staletí stalo nepřátelským a utlačujícím, a přesto jste zůstali k otroctví slepí. Naštěstí byl Mojžíš pověřen, aby vám zvěstoval pravdu. Jste ochotni uznat, že vás Egypt zotročil? Řeknete-li „ano“, bude od vás požadována kompletní změna života.

\subsection*{1. Vzdejte se kontroly}
Disciplíny Exodus 90 vám poskytují příležitost vzdát se kontroly nad svým životem a předat ji Bohu. Učte se nově svěřovat kontrolu do Božích rukou. Jako by rytíř položil svůj meč na oltář a na oplátku by obdržel Boží moc, musíte udělat totéž.

\subsection*{2. Zavázat se k vašemu bratrstvu}
Tahle cesta je těžká. Budete vyzkoušeni a testováni. Budete potřebovat své bratry a oni budou potřebovat vás. Týdenní setkání jsou nutností. Modlete se jako Izraelité, kteří byli zachráněni jako kmen (a ne jednotlivě), aby Bůh vysvobodil vaše společenství a také všechny lidi, kteří čekají na vysvobození ze závislosti, sobectví, apatie a ovládání.

\subsection*{3. Najít si čas pro každodenní modlitbu}
Strávit hodinu času v modlitbě každý den. Pokud je to nemožné, trávit co nejvíce času, jak je to možné, s minimálně dvaceti minutami tiché modlitby denně. Je dobré si naplánovat konkrétní čas během dne, nebo toho pravděpodobně zanecháte.

\subsection*{4. Buďte radostní}
Přijali jste Kristův plán svobody. Ano, bude to těžké, ale to by vás nemělo zarmoutit. Spíše se těšte z naděje na svobodu, která vás čeká. Boží pozvání do tohoto duchovního cvičení by vám mělo přinést bohatou radost.

\subsection*{5. Každou noc zkoumejte svůj den}
Exodus 90 obsahuje mnoho disciplín, na které můžete každým dnem odpovídat „ano“. Na konci každého dne před usnutím si nalezněte čas na to, abyste si prošli a zkoumali svůj den. To vám pomůže nejen úspěšně přistupovat na jednotlivé disciplíny, ale také vám pomůže vidět váš pokrok, který jste udělali na cestě ke svobodě. (Jak na to se dozvíte v příručce Exodu v sekci „Jak se modlit noční examen“).Možná verze Examenu přímo zde.

\subsection*{Modlitba}
Modlete se, aby Pán osvobodil vás a vaše bratrství.

Modleme se za svobodu všech mužů v exodu, stejně tak, jako se oni modlí za vás.

Když se učedníci zeptali Ježíše, jak se mají modlit, naučil je „Otče náš“ (viz Lukáš 11: 1–4, Matouš 6: 9–13). Připojte se ke svým bratrům Exodu po celém světě a každý den se modlete tuto mocnou modlitbu za výše uvedené úmysly.

Ve jménu Otce i Syna i Ducha svatého …  Otče náš… Ve jménu Otce i Syna i Ducha svatého … Amen.

%newday
\newpage
\section{Den 1 - JEDINĚ BŮH PŘINÁŠÍ SVOBODU}
\zacatekPrvniTyden
\subsection*{Čtení na den}
\textbf{Exodus 1, 1-7}
\newline
\textit{
\textsuperscript{1}Toto jsou jména synů Izraelových, kteří přišli do Egypta s Jákobem; každý přišel se svou rodinou:
\textsuperscript{2}Rúben, Šimeón, Lévi a Juda,
\textsuperscript{3}Isachar, Zabulón a Benjamín,
\textsuperscript{4}Dan a Neftalí, Gád a Ašer.
\textsuperscript{5}Všech, kdo vzešli z Jákobových beder, bylo sedmdesát. Josef už byl v Egyptě.
\textsuperscript{6}Potom zemřel Josef a všichni jeho bratři i celé to pokolení.
\textsuperscript{7}Ale Izraelci se rozplodili, až se to jimi hemžilo, převelice se rozmnožili a byli velice zdatní; byla jich plná země.
}
\subsection*{Reflexe}
Začínáme náročnou cestu od otroctví ke svobodě rozjímáním nad úvodním odstavcem starověké knihy Exodus - příběhu cesty Božího lidu z Egyptského otroctví ke svobodě.

Na první pohled se můžeme podivovat nad tím, jak by nám právě tento úryvek mohl být užitečný. Ale nepodceňujme Boží slovo!

Vzhledem k tomu, že kniha Exodus je příběhem cesty Izraelitů z otroctví, mělo by nám připadat zvláštní, že Izraelité byli „mimořádně silní; takže země byla jimi naplněna. “
Jak se dozvíme při další četbě, egyptský faraon se dokonce Izraelitů bál. Jak je tedy možné, že byli Izraelité zotročeni, když byli „mimořádně silní“? Proč nepovstali a neosvobodili se? Jak to, že navzdory své velké síle zůstali zotročeni tyranem?

Kniha Exodus, jak uvidíme, je také naším příběhem. Tento starodávný text není jen historií Izraelitů. Je to také určitá metafora moderního muže. Jsme-li zotročení chtíčem, technologiemi, jídlem, pitím nebo něčím jiným, nacházíme se ve stejném otroctví. Ve skutečnosti to, že jsme zotročeni, neznamená, že jsme slabí.
Ve většině případů jsou naše mozky a těla ve skutečnosti docela silné.
Jenže je více než pravděpodobné, že právě síla je tím pravým důvodem, proč jsme se nechali zotročit.

Muži jsou silní, ale když se snaží vypořádat se s životem a jeho mnoha obtížemi, vrhnou se dychtivě po čemkoliv, co jim přinese útěchu a bezpečí. V průběhu svého života (a zejména v mladém věku) objeví a začnou využívat věci nebo činnosti, o nichž si myslí, že je učiní šťastnými. Užívají těchto věcí, protože je vnímají jako prospěšné (pro svůj život).
Jenže postupem času si začínají uvědomovat, že byli podvedeni. Zjistí, že jim tyto věci nepomáhají, že jim nepřinášejí štěstí, po kterém touží. Ale i když pak velmi touží po svobodě, jejich mozek i nadále požaduje to, k čemu byl veden, že je to prospěšné. Pravdu sice objeví, ale chybí vůle.
Ale to není ostuda. Závislý muž není slabý. Může být naopak velmi silný. A stejně jako Izraelité je v této situaci schopen pochopit jednu z největších pravd Písma svatého: jedině Bůh nás může vysvobodit.
Izraelité byli mimořádně silní, ale nedokázali se osvobodit. Moderní muž to zjišťuje také.

Kolikrát jste se pokoušeli „osvobodit se“, jen abyste zjistili, že to nejde? Tisíckrát?
Když začínáme tuto cestu, nikdy (zdůrazňuji nikdy) nezapomeňte na tuto úžasnou pravdu: vy to můžete zvládnout ... ale bude to Bůh, kdo vás osvobodí.

%newday
\newpage
\section{Den 2 - KAŽDODENNÍ ÚKOLY MOHOU MUŽE ZATĚŽOVAT A OSLABOVAT}
\zacatekPrvniTyden
\subsection*{Čtení na den}
\textbf{Exodus 1, 8-14}
\newline
\textit{
\textsuperscript{8}V Egyptě však nastoupil nový král, který o Josefovi nevěděl.
\textsuperscript{9}Ten řekl svému lidu: „Hle, izraelský lid je početnější a zdatnější než my.
\textsuperscript{10}Musíme s ním nakládat moudře, aby se nerozmnožil. Kdyby došlo k válce, jistě by se připojil k těm, kdo nás nenávidí, bojoval by proti nám a odtáhl by ze země.“
\textsuperscript{11}Ustanovili tedy nad ním dráby, aby jej ujařmovali robotou. Musel stavět faraónovi města pro sklady, Pitom a Raamses.
\textsuperscript{12}Avšak jakkoli jej ujařmovali, množil se a rozmáhal dále, takže měli z Izraelců hrůzu.
\textsuperscript{13}Proto začali Egypťané Izraelce surově zotročovat.
\textsuperscript{14}Ztrpčovali jim život tvrdou otročinou při výrobě cihel a všelijakou prací na poli. Všechnu otročinu, kterou na ně uvalili, jim ještě ztěžovali surovostí.  
}\subsection*{Reflexe}
Kniha Exodus je fascinující tím, že je to skutečně příběh každého člověka, což můžeme jasně vidět v dnešním úryvku Písma.

Egypťané byli plni obav, že by se hebrejský lid mohl stát "příliš mocným", než aby ho mohli ovládat, a že by mohl "bojovat proti nim" (Egypťanům). Egypťané na to šli chytře. „Ustanovili nad nimi úkoláře (dráby), aby je trápili těžkými břemeny.“ Jinými slovy, zaměstnávali muže úkoly, mnoha úkoly.
Jak byli Izraelité stále více zatěžováni každodenní prací, přestali se zajímat o svou svobodu a moc.

Žádný člověk nemůže být hrdinou, když je tak zatížen, že ani nemá čas vzhlédnout a zvažovat svou situaci. A tak Izraelité pracovali pro svého pána ještě usilovněji, ale tím nečekaně rostla jejich moc, i když zůstávali stále zotročeni.

Zamyslete se nad tím, jaké „cihly a malty“ používá Dráb (ten Zlý) ve vašich životech, aby vás ovládal, aby vás udržel daleko od vašeho pravého synovství, abyste se nestali příliš silnými. Stejně jako faraon, závistivý k moci, je lstivý v metodách, které používá, aby nás držel.
Přemýšlejte nad „maltou a cihlami“, které vás obklopují: nekonečná práce, zběsilá činnost, neustálý tlak na to, abyste se dostali dopředu. A zamyslete se nad všemi ostatními věcmi, které ďábel používá k tomu, aby vás zotročil: alkohol, pornografie, chtíč, pýcha, strach z neúspěchu, konkurence s ostatními, peníze, sport a postavení… Všechny tyto věci způsobují, že se váš život stává hořkým a smutným.

Ale nemusíte být jimi utlačováni. Rozhodnutí o jejich odstranění vás naučí, že můžete žít bez nich a můžete uniknout jejich poutům, když vás odvádějí od důležitějších věcí.
Je jasné, že muži mají mnoho povinností. Ale většině moderních mužů by velmi prospělo zjednodušení jejich života.

Zkus dnes strávit čas přemýšlením nad oblastmi, kde by mohl být tvůj život zjednodušen. Pozvi Pána do tohoto rozhovoru a zapiš si své závěry. Jedná se o počáteční krok ke svobodě.

\textbf{Poznámka pro ženaté muže:} Jakákoli zjednodušení nebo změny ve vašem způsobu života by měly být konzultovány a prodiskutovány s vaší manželkou.

%newday
\newpage
\section{Den 3 - POSUNOUT SE VPŘED}
\zacatekPrvniTyden
\subsection*{Čtení na den}
\textbf{Exodus 1, 15-22}
\newline
\textit{
\textsuperscript{15}Egyptský král poručil hebrejským porodním bábám, z nichž jedna se jmenovala Šifra a druhá Púa:
\textsuperscript{16}„Když budete pomáhat Hebrejkám při porodu a při slehnutí zjistíte, že to je syn, usmrťte jej; bude-li to dcera, aťsi je naživu.“
\textsuperscript{17}Avšak porodní báby se bály Boha a rozkazem egyptského krále se neřídily. Nechávaly hochy naživu.
\textsuperscript{18}Egyptský král si porodní báby předvolal a řekl jim: „Co to děláte, že necháváte hochy naživu?“
\textsuperscript{19}Porodní báby faraónovi odvětily: „Hebrejky nejsou jako ženy egyptské; jsou plné života. Porodí dříve, než k nim porodní bába přijde.“
\textsuperscript{20}Bůh pak těm porodním bábám prokazoval dobrodiní a lid se množil a byl velmi zdatný.
\textsuperscript{21}Protože se porodní báby bály Boha, požehnal jejich domům.
\textsuperscript{22}Ale farao všemu svému lidu rozkázal: „Každého syna, který se jim narodí, hoďte do Nilu; každou dceru nechte naživu.“
}

\subsection*{Reflexe}

Faraon měl tak velké obavy z toho, aby se Izraelité neosvobodili ze svého zotročení, že nařídil porodním bábám, aby udělaly něco naprosto zvláštního - aby zabily každého chlapce, kterého porodí, a tím aby udusily budoucnost Izraele. Zatímco porodní báby se tomuto požadavku hrdinně vyhýbají, faraon je neúprosný a vyzývá k utopení dětí mužského pohlaví v řece Nil.

Svatý Metoděj považuje faraóna za „předobraz ďábla“. Stejně jako faraón nařídil zabití izraelských chlapců, ďábel se pokouší zabít lidskou ctnost. Vysvětluje, že vody Nilu jsou obrazem našich vášní a ten zlý chce, aby se naše duše vrhly do těchto vod, aby se utopily. Každý člověk zná bolest vnitřního udušení: osamělost pornografie, prázdnota alkoholu, nuda neprozkoumaného života a nekonečná snaha o zábavu.

Dnes je třetí den vašeho odhodlání osvobodit se od těchto věcí. To, co vás kdysi zotročovalo, je nyní prostředkem, díky němuž se stáváte svobodnými, pokud se toho snažíte zbavit.

Svatý Augustin si všímá jisté ironie v příběhu Exodus: Izraelci kráčeli vodami Rudého moře na svobodu. Ti, kteří byli zotročeni a odsouzeni k utonutí, nyní procházejí mořem na cestě ke svobodě.

Naše kultura nás obklopuje neustálým vybízením k bezduchým a ničivým požitkům. I když se od nich vzdalujeme, připadá nám, jako bychom jimi procházeli. V tom se podobáme Izraelitů, kteří procházeli Rudým mořem a měli dvě obří vodní stěny - po levici a pravici.

Ale když Boží síla otevírá cestu, naším jediným úkolem je pokročit vpřed.
Děkujte Pánu, že vám dnes otevřel cestu, a získejte odvahu jít kupředu.


%newday
\newpage
\section{Den 4 - DAR NOVÉHO ŽIVOTA}
\zacatekPrvniTyden
\subsection*{Čtení na den}
\textbf{Exodus 2, 1-10}
\newline
\textit{
\textsuperscript{1}Muž z Léviova domu šel a vzal si lévijskou dceru.
\textsuperscript{2}Žena otěhotněla a porodila syna. Když viděla, jak je půvabný, ukrývala ho po tři měsíce.
\textsuperscript{3}Ale déle už ho ukrývat nemohla. Proto pro něho připravila ze třtiny ošatku, vymazala ji asfaltem a smolou, položila do ní dítě a vložila do rákosí při břehu Nilu.
\textsuperscript{4}Jeho sestra se postavila opodál, aby zvěděla, co se s ním stane.
\textsuperscript{5}Tu sestoupila faraónova dcera, aby se omývala v Nilu, a její dívky se procházely podél Nilu. Vtom uviděla v rákosí ošatku a poslala svou otrokyni, aby ji přinesla.
\textsuperscript{6}Otevřela ji a spatřila dítě, plačícího chlapce. Bylo jí ho líto a řekla: „Je z hebrejských dětí.“
\textsuperscript{7}Jeho sestra se faraónovy dcery otázala: „Mám jít a zavolat kojnou z hebrejských žen, aby ti dítě odkojila?“
\textsuperscript{8}Faraónova dcera jí řekla: „Jdi!“ Děvče tedy šlo a zavolalo matku dítěte.
\textsuperscript{9}Faraónova dcera jí poručila: „Odnes to dítě, odkoj mi je a já ti zaplatím.“ Žena vzala dítě a odkojila je.
\textsuperscript{10}Když dítě odrostlo, přivedla je k faraónově dceři a ona je přijala za syna. Pojmenovala ho Mojžíš (to je Vytahující). Řekla: „Vždyť jsem ho vytáhla z vody.“}

\subsection*{Reflexe}

Když se v našem životě nebo v životě Božího lidu stane něco nepředvídaného a radikálního, můžeme si být téměř jisti, že to jedná Bůh. Vidíme to i v dnešním úryvku z Písma.

Izraelité jsou zotročeni v Egyptě a nemají téměř žádnou naději, že se někdy dočkají svobody, když tu najednou Bůh vzbudí osvoboditele. Jak uvidíme v následujících dnech, narození Mojžíše a jeho povolání jako osvoboditele je předzvěstí velkého osvoboditele, který přijde: Ježíše Krista.

Již nyní můžeme začít vidět podobnosti mezi těmito dvěma postavami:
Mojžíš je zplozen nejmenovaným mužem, je to "hodné" dítě a je na tři dny umístěn v koši na řece. Připomeňme si, že u Marie bylo zjištěno, že "čeká dítě" bez pomoci svého snoubence.

Oba jsou hodné děti: o Ježíšovi čteme, že poté, co byl nalezen v chrámu, „poslouchal je (Josefa a Marii) a prospíval na duchu i na těle a byl milý Bohu i lidem.“ (Lk 2,51-52).

Všimněte si také, že obě děti byly zachráněny před bezohlednými a paranoidními vůdci, kteří se je snažili raději zabít, než aby ztratili svou moc (Mt 2,16).
Číslo tři by mělo vyvolat vzpomínku na mnoho událostí v Kristově životě: tři dny, kdy se dítě ztratí a najde v chrámu, tři dny, kdy je Kristus v hrobě, jeho veřejné působení začalo v jeho třicátém roce.

Ježíš se zjeví jako nový osvoboditel a bude hrát velkou roli v našem vlastním hledání svobody.
A konečně, a to je nejvýznamnější, Mojžíš dostává své jméno, protože faraonova dcera prohlásila: "Vytáhla jsem ho z vody." Mojžíš je tedy v tomto případě zjevením, které je v souladu se skutečností. I vy jste byli vytaženi "z vody", když jste byli pokřtěni, nejspíše jako nemluvně. Byli jste zachráněni z tyranie Zlého a bylo vám dáno vše, co je třeba k tomu, abyste byli synem Nejvyššího.

Mojžíš byl zachráněn skrze vodu, Izraelité byli zachráněni skrze vodu (Ex 14) a i vy jste byli zachráněni skrze vodu křtu.
Křest je dnes často opomíjeným přechodovým rituálem. Na milost a moc svátosti křtu se většinou zapomíná. Svatý Pavel přesto trval na svém významu: „Nevíte snad, že všichni, kteří jsme pokřtěni v Krista Ježíše, byli jsme pokřtěni v jeho smrt? Byli jsme tedy křtem spolu s ním pohřbeni ve smrt, abychom – jako Kristus byl vzkříšen z mrtvých slavnou mocí svého Otce – i my vstoupili na cestu nového života.“ (Řím 6,3-4)

Svoboda, kterou prostřednictvím tohoto exodu hledáte, pramení z nového života, který jste obdrželi při křtu.
Ježíš to řekl jasně: "Kdo uvěří a dá se pokřtít, bude spasen...". Udělali bychom tedy dobře, kdybychom si připomněli milost vlastního křtu a "oživili Boží dar, který je v tobě" (2 Tim 1,6), abyste měli vše, co je nezbytné k získání skutečné a trvalé svobody!

Připomeňte si dnes milosti svého křtu a podívejte se na toho, kdo vám tyto milosti dal. Touží ve vás ještě jednou vzbudit dar nového života. Mluvte s ním dnes otevřeně. Poděkujte mu za pozvání na toto duchovní cvičení a zeptejte se ho, proč vám tak velkoryse znovu nabízí dar nového života. Odpověď bude velká láska.

%newday
\newpage
\section{Den 5 - MUŽ PRO OSTATNÍ}
\zacatekPrvniTyden
\subsection*{Čtení na den}
\textbf{Exodus 2,11-25}
\newline
\textit{
\textsuperscript{11}V oněch dnech, když Mojžíš dospěl, vyšel ke svým bratřím a viděl jejich robotu. Spatřil nějakého Egypťana, jak ubíjí Hebreje, jednoho z jeho bratří.
\textsuperscript{12}Rozhlédl se na všechny strany, a když viděl, že tam nikdo není, ubil Egypťana a zahrabal do písku.
\textsuperscript{13}Když vyšel druhého dne, spatřil dva Hebreje, jak se rvali. Řekl tomu, který nebyl v právu: „Proč chceš ubít svého druha?“
\textsuperscript{14}Ohradil se: „Kdo tě ustanovil nad námi za velitele a soudce? Máš v úmyslu mě zavraždit, jako jsi zavraždil toho Egypťana?“ Mojžíš se ulekl a řekl si: „Jistě se o věci už ví!“
\textsuperscript{15}Farao o tom vskutku uslyšel a chtěl dát Mojžíše zavraždit. Ale Mojžíš před faraónem uprchl a usadil se v midjánské zemi; posadil se u studny.
\textsuperscript{16}Midjánský kněz měl sedm dcer. Ty přišly, vážily vodu a plnily žlaby, aby napojily stádo svého otce.
\textsuperscript{17}Tu přišli pastýři a odháněli je. Ale Mojžíš vstal, ochránil je a napojil jejich stádo.
\textsuperscript{18}Když přišly ke svému otci Reúelovi, zeptal se: „Jak to, že jste dnes přišly tak brzo?“
\textsuperscript{19}Odpověděly: „Nějaký Egypťan nás vysvobodil z rukou pastýřů. Také nám ochotně navážil vodu a napojil stádo.“
\textsuperscript{20}Reúel se zeptal svých dcer: „Kde je? Proč jste tam toho muže nechaly? Zavolejte ho, ať pojí chléb!“
\textsuperscript{21}Mojžíš se rozhodl, že u toho muže zůstane, a on mu dal svou dceru Siporu za manželku. 
\textsuperscript{22}Ta porodila syna a Mojžíš mu dal jméno Geršóm (to je Hostem-tam). Řekl: „Byl jsem hostem v cizí zemi.“
\textsuperscript{23}Po mnoha letech egyptský král zemřel, ale Izraelci vzdychali a úpěli v otročině dál. Jejich volání o pomoc vystupovalo z té otročiny k Bohu.
\textsuperscript{24}Bůh vyslyšel jejich sténání, Bůh se rozpomněl na svou smlouvu s Abrahamem, Izákem a Jákobem,
\textsuperscript{25}Bůh na syny Izraele pohleděl, Bůh se k nim přiznal.
}
\subsection*{Reflexe}

Muži jsou v tom nejlepším, když jsou skutečně „muži pro druhé“. Zralý a sebeovládající se muž, který je zformovaný Boží rukou, má velkou moc, která vychází z nového života v něm.
Ale to, co dělá muže skutečně velkým, je jeho ochota sloužit druhým, používat svou moc/sílu pro druhé - ať už je to jeho žena, děti, bratři, sousedé, církev nebo země. V moderní době jsme upadli do zlozvyku dávat na první místo své vlastní potřeby a touhy, a až na druhé, pokud vůbec, brát v úvahu potřeby ostatních.

V dnešním úryvku z Písma vidíme Mojžíše, jak je solidární s chudými a utlačovanými a jak využívá svou mladickou moc k tomu, aby zasáhl a pomohl druhým. Vidí, že dcery Reuela (Jethra) se nedokážou samy o sebe postarat a postavit se proti darebáckým pastýřům.
Když se zaměříme na péči, podporu a obranu slabších a potřebných, Bůh si nás a naši sílu může použít a použije pro dobro druhých. Brzy se o tom přesvědčíme, až Bůh udělá z Mojžíše velkého osvoboditele a soudce svého lidu.

Dnes a denně se musíme odpoutávat od současných kulturních zvyklostí a rozhodnout se (v případě potřeby každý den) překonat své vlastní potřeby ve prospěch lidí kolem nás.
Během dnešní modlitby se ptejte sami sebe: „Kdo je na vás závislý? Kdo od vás hledá ochranu nebo pomoc tváří v tvář životním nespravedlnostem a nebezpečím? Kdo se vám svěřil v naději, že mu budete pevnou oporou, mužem, na kterého je spolehnutí? Kdo vám věří, že nebudete jen přihlížet, když bude potřeba se angažovat, a to i když to pro vás bude nepohodlné?

Život má skutečně smysl a význam, když se velkoryse věnujeme druhým, nějaké věci nebo církvi.
Jste ochotni takový život žít? Jste ochotni vykročit vpřed a být mužem pro ostatní?

%newday
\newpage
\section{Den 6 - BŮH SI NÁS VOLÍ PŘEDTÍM, NEŽ SI MY ZVOLÍME JEJ}
\zacatekPrvniTyden
\subsection*{Čtení na den}
\textbf{Exodus 3, 1-6}
\newline
\textit{\textsuperscript{1}Mojžíš pásl ovce svého tchána Jitra, midjánského kněze. Jednou vedl ovce až za step a přišel k Boží hoře, k Chorébu.
\textsuperscript{2}Tu se mu ukázal Hospodinův posel v plápolajícím ohni uprostřed trnitého keře. Mojžíš viděl, jak keř v ohni hoří, ale není jím stráven.
\textsuperscript{3}Řekl si : „Zajdu se podívat na ten veliký úkaz, proč keř neshoří.“
\textsuperscript{4}Hospodin viděl, že odbočuje, aby se podíval. I zavolal na něho Bůh zprostředku keře: „Mojžíši, Mojžíši!“ Odpověděl: „Tu jsem.“
\textsuperscript{5}Řekl: „Nepřibližuj se sem! Zuj si opánky, neboť místo, na kterém stojíš, je půda svatá.“
\textsuperscript{6}A pokračoval: „Já jsem Bůh tvého otce, Bůh Abrahamův, Bůh Izákův a Bůh Jákobův.“ Mojžíš si zakryl tvář, neboť se bál na Boha pohledět.
}

\subsection*{Reflexe}

Všimněte si způsobu, jakým Bůh a Mojžíš začali své hluboké přátelství. Ne Mojžíš šel hledat a najít Boha. To se stává velmi zřídka, pokud vůbec. Ale byl to Bůh, kdo přišel s Mojžíšem jako první.  Mojžíš řeší své každodenní starosti, když se mu Bůh zjeví a dovolí mu odpovědět.  

Svatá Terezie z Avily často používala k popisu tohoto jevu příměr se slunečnicí. Když ráno vyjde slunce, jeho paprsky zalijí krajinu a slunečnice k němu otočí hlavu. Může to však udělat pouze tehdy, když na ni svítí slunce. Podobně když se duše obrací k Bohu, je to proto, že Bůh udělal první krok. To je základ duchovního života.

Může existovat tisíc důvodů, proč jste se rozhodli naplnit Exodus 90. Ale nebyli jste to vy, kdo se rozhodl to udělat - byl to Bůh, kdo vás k tomu povolal. Touto výzvou, tímto pozváním k vykonání těchto duchovních cvičení vám Bůh otevřel cestu k hlubšímu vztahu s ním. Během těchto 90 dní je Božím záměrem zjevit se vám více. Tento krok je Jeho, ale je na vás, jak na něj zareagujete. Využijte tento požehnaný čas k tomu, abyste "obrátili hlavu k Bohu" a objevili Ho tak, jak se vám dovolil zjevit.

Volejte dnes k Pánu v modlitbě. Požádejte Ho, aby se vám zjevil více než kdykoli předtím.

%newday
\newpage
\section{Den 7 - BŮH DÁVÁ ČLOVĚKU SÍLU}
\zacatekPrvniTyden
\subsection*{Čtení na den}
\textbf{}
\newline
\textit{
\textsuperscript{7}Hospodin dále řekl: „Dobře jsem viděl ujařmení svého lidu, který je v Egyptě. Slyšel jsem jeho úpění pro bezohlednost jeho poháněčů. Znám jeho bolesti.
\textsuperscript{8}Sestoupil jsem, abych jej vysvobodil z moci Egypta a vyvedl jej z oné země do země dobré a prostorné, do země oplývající mlékem a medem, na místo Kenaanců, Chetejců, Emorejců, Perizejců, Chivejců a Jebúsejců.
\textsuperscript{9}Věru, úpění Izraelců dolehlo nyní ke mně. Viděl jsem také útlak, jak je Egypťané utlačují.
\textsuperscript{10}Nuže pojď, pošlu tě k faraónovi a vyvedeš můj lid, Izraelce, z Egypta.“
\textsuperscript{11}Ale Mojžíš Bohu namítal: „Kdo jsem já, abych šel k faraónovi a vyvedl Izraelce z Egypta?“
\textsuperscript{12}Odpověděl: „Já budu s tebou! A toto ti bude znamením, že jsem tě poslal: Až vyvedeš lid z Egypta, budete sloužit Bohu na této hoře.“
\textsuperscript{13}Avšak Mojžíš Bohu namítl: „Hle, já přijdu k Izraelcům a řeknu jim: Posílá mě k vám Bůh vašich otců. Až se mě však zeptají, jaké je jeho jméno, co jim odpovím?“
\textsuperscript{14}Bůh řekl Mojžíšovi: „JSEM, KTERÝ JSEM.“ A pokračoval: „Řekni Izraelcům toto: JSEM posílá mě k vám.“
\textsuperscript{15}Bůh dále Mojžíšovi poručil: „Řekni Izraelcům toto: ‚Posílá mě k vám Hospodin, Bůh vašich otců, Bůh Abrahamův, Bůh Izákův a Bůh Jákobův.‘ To je navěky mé jméno, jím si mě budou připomínat od pokolení do pokolení.
\textsuperscript{16}Jdi, shromažď izraelské starší a pověz jim: ,Ukázal se mi Hospodin, Bůh vašich otců, Bůh Abrahamův, Izákův a Jákobův, a řekl: Rozhodl jsem se vás navštívit, vím, jak s vámi v Egyptě nakládají,
\textsuperscript{17}a prohlásil jsem: Vyvedu vás z egyptského ujařmení do země Kenaanců, Chetejců, Emorejců, Perizejců, Chivejců a Jebúsejců, do země oplývající mlékem a medem.‘
\textsuperscript{18}Až tě vyslechnou, půjdeš ty a izraelští starší k egyptskému králi a řeknete mu: ‚Potkal se s námi Hospodin, Bůh Hebrejů. Dovol nám nyní odejít do pouště na vzdálenost tří dnů cesty a přinést oběť Hospodinu, našemu Bohu.‘
\textsuperscript{19}Vím, že vám egyptský král nedovolí jít, leda z donucení.
\textsuperscript{20}Proto vztáhnu ruku a budu bít Egypt všemožnými svými divy, které učiním uprostřed něho. Potom vás propustí.
\textsuperscript{21}Zjednám tomuto lidu u Egypťanů přízeň. Až budete odcházet, nepůjdete s prázdnou.
\textsuperscript{22}Každá žena si vyžádá od sousedky a spolubydlící stříbrné a zlaté ozdoby a pláště. Vložíte je na své syny a dcery. Tak vypleníte Egypt.“
}

\subsection*{Reflexe}

Bůh dává vysvobození. Vidíme to na příběhu Izraelitů. V dnešním čtení říká Bůh Mojžíšovi, že nejprve propustí
Izraelity z ruky Egypťanů. Poté je dovede do „zaslíbené země“, která je ovšem obývána Kennaanci, Chetejci,
Emorejsi, Perizejsi, Chivejci a Jebúsejci – všemi nepřáteli Izraele. Dokážete si představit, co si Mojžíš mohl myslet?
„Chceš nás osvobodit od našich otrokářů (což nemůže dopadnout dobře) jenom proto, abys nás mohl dovést
doprostřed našich napřátel?“ Copak je divu, že chtěl být Mojžíš sám? Ale Bůh, aby dal Mojžíšovi odvahu, dělá něco
naprosto nemyslitelného. Zjevuje mu svoje svaté jméno: „Jsem, který jsem.“

V moderním světě jsme zapomněli na důležitost teologie jmen. Ve starověku vědět něčí jméno znamenalo mít nad
ním nějakou moc. Proto Adam pojmenoval všechna zvířata v zahradě Edenu. Prohlašoval tím svou nadřazenost nad
zvířecí říší (Gen 2,20). Když dává Bůh poznat své jméno Mojžíšovi, také mu tím propůjčuje svou božskou moc. Jak
by mohl teď Mojžíš pochybovat o příslibu jemu a jeho lidu? Pouze potřebuje jít za Bohem v důvěře.

Také si povšimněte významnosti úkolu danému Mojžíšovi a zprávy, která mu je dána k předání Izraelitům: „Jsem
mě poslal k vám.“ Znovu Mojžíš předznamenává Ježíše Krista. Tak jako byl Mojžíš poslán Bohem k Izraelitům,
Kristus byl poslán Otcem k osvobození nás všech. „Neboj se“, říká Ježíš, „jen věř,“ (Lk 8,50). Když procházíte
disciplínami Exodu 90, zapíráte se a celou dobu bojujete, mějte oči upřeny na Ježíše. Byl poslán, aby vás vykoupil
od vás samotných, vašich hříchů, zotročení. Když bojujete za svobodu v dennodenním boji, Bůh tam bojuje s vámi
a pro vás.

Zavolejte na Boha. Čeká, aby vám dal svou moc a sílu.


% ===============================================
% ===== DRUHY TYDEN
% ===============================================
%ukony
\newpage
\section*{Úkony (ukazatel cesty) pro 2. týden}

\textbf{Místo:} Egypt (jste v Egyptě)

Izraelité jsou daleko od svobody. Mojžíšova a Áronova poslušnost Bohu pouze zhoršila jejich situaci a zvýraznila jejich otroctví více než kdy dříve. Na tomto místě našeho exodu se také naše otroctví stalo viditelnějším. Disciplíny duchovního cvičení zvětštily naši dříve nevědomou náklonnost k lidskému komfortu. Tato rutina nás formuje, ale nejsme ještě vůbec blízko svobody. Proč si stěžujeme život poslušností Bohu? Nebylo by snazší toto duchovní cvičení opustit a zůstat v pohodlí otroctví navěky? Této úvaze čelí Izraelité i my tento týden.

\subsection*{1. Dobře se vyzpovídejte}
Pokoušet se začít Exodus 90, aniž bychom šli nejdříve ke zpovědi, je jako pokoušet se vylézt na vrchol nebezpečné americké sopky Mount Rainier s pětadevadesáti kily kamení v krosně. Proveditelné, ale pošetilé. Kříž, který na sebe musíte brát každý den je dost těžký tak, jak je. Nechte Boha vyndat ono kamení z vaší krosny. Běžte ke zpovědi, abyste mohli zdolat tuto horu a být opravdu svobodní.
\subsection*{2. Držte se denních reflexí}
Pokud denní čtení a reflexe neprovádíte, neděláte vůbec Exodus 90. Denní čtení Písma a rozjímání nad ním dovoluje Ježíši Kristu, Slovu, aby vás vedl na cestě vaším exodem. Neskončíte se studenou sprchou jen proto, že vám nevyhovuje, tak neskončujte ani se čtením Písma jen proto, že se vám nechce. Držte se tohoto rozjímání. Udrží vás a vaše bratrstvo jednotné po dobu exodu.
\subsection*{3. Navštěvujte jednu mši v týdnu navíc}
Na otázku, co by vyzvalo lidi k tomu, aby více rostli ve víře, odpověděl kněz Augustinského Institutu takto: „Ať chodí na další mši svatou během dní v týdnu.“ Nyní nastal ten čas. Zvolte si den v týdnu a konkrétní čas mše, na kterou budete chodit každý týden, vedle povinných mší svatých v neděli a o svátcích. (Pro více informací o účincích navštěvování více mší svatých v týdnu a způsobu, jak toho využívala různá bratrstva v minulosti, nahlédněte do sekce \textit{Posílit své bratrství} pod pilířem Bratrství v příručce Exodu.)
\subsection*{4. Zvažte přečtení \textit{Průvodce terénem}}
Pokud jste si nenašli čas na přečtení \textit{Průvodce terénem} Exodu 90 předtím, než jste Exodus začali, zvažte jeho přečtení dnes, nebo tuto neděli. Tento průvodce rámcuje celou zkušenost Exodu 90 a pomůže vám pochopit důvod každé části vašeho exodu. Porozumění těmto důvodům často pomáhá k tomu, abyste se zavázali k disciplínám s větší radostí a lehčím srdcem. (Nejdůležitější části \textit{Průvodce terénem} jsou: \textit{Začněte zde: Co je vaše proč}, \textit{Pilíře Exodu 90}, a pro ženaté muže \textit{Muž Exodu a jeho manželka.})
\subsection*{5. Uvědomte si (zjistěte), kde je vaše kotva}
Zavázali jste se k denní komunikaci s vaší kotvou. On na vás spoléhá v tom, že budete naplňovat váš závazek. Jestli jste tak činili, skvěle. Jestli ne, teď je čas začít. Brzy budete potřebovat vaši kotvu stejně tak, jako on potřebuje vás. (Myslíte si, že každodenní komunikace s vaší kotvou není důležitá? Přečtěte si sekci \textit{Kotva (Nepřeskakujte: Smrt je pravděpodobná)} pod pilířem Bratrství v \textit{Pilířích Exodu 90.})

\subsection*{Modlitba}
Modlete se, aby Pán osvobodil vás a vaše bratrství \newline
Modleme se za svobodu všech mužů v exodu, stejně tak, jako se oni modlí za vás.\newline
Ve jménu Otce i Syna i Ducha svatého … Otče náš… Ve jménu Otce i Syna i Ducha svatého … Amen.

%newday
\newpage
\section{Den 8 - HOSPODIN JE JEDINÝ BŮH}
\zacatekDruhyTyden
\subsection*{Čtení na den}
\textbf{Exodus 4,1-9}
\newline
\textit{
\textsuperscript{1}Mojžíš však znovu namítal: „Nikoli, neuvěří mi a neuposlechnou mě, ale řeknou: Hospodin se ti neukázal.“
\textsuperscript{2}Hospodin mu řekl: „Co to máš v ruce?“ Odpověděl: „Hůl.“
\textsuperscript{3}Hospodin řekl: „Hoď ji na zem.“ Hodil ji na zem a stal se z ní had. Mojžíš se dal před ním na útěk.
\textsuperscript{4}Ale Hospodin Mojžíšovi poručil: „Vztáhni ruku a chyť ho za ocas.“ Vztáhl tedy ruku, uchopil ho a v dlani se mu z něho stala hůl.
\textsuperscript{5}„Aby uvěřili, že se ti ukázal Hospodin, Bůh jejich otců, Bůh Abrahamův, Bůh Izákův a Bůh Jákobův.“
\textsuperscript{6}Dále mu Hospodin řekl: „Vlož si ruku za ňadra.“ Vložil tedy ruku za ňadra. Když ruku vytáhl, byla malomocná, bílá jako sníh.
\textsuperscript{7}Tu poručil: „Dej ruku zpět za ňadra.“ Dal ruku zpět za ňadra. Když ji ze záňadří vytáhl, byla opět jako ostatní tělo.
\textsuperscript{8}„A tak jestliže ti neuvěří a nedají na první znamení, uvěří druhému znamení.
\textsuperscript{9}Jestliže však neuvěří ani těmto dvěma znamením a neuposlechnou tě, nabereš vodu z Nilu a vyleješ ji na suchou zemi. Z vody, kterou nabereš z Nilu, se stane na suché zemi krev.“
}

\subsection*{Reflexe}
Skrze Starý Zákon pracuje Bůh nepřetržitě, aby svému lidu ukázal, že je jediný Bůh. Ale jeho lid bojuje
s tím, aby uvěřil. Znamení po znamení ukazuje Bůh svou moc svému lidu a ostatním národům jako Egyptu.
Ti, kteří vidí tato znamení a uvěří, zakusí Boží lásku. Ti, kteří vidí a neuvěří, jdou směrem ke své vlastní
zkáze.

Věříte, že Pán je jediný Bůh? Žijete tak, jako byste tomu věřili? Dobrým způsobem, jak to vyzkoušet, je
podívat se na vaši neděli a první a poslední věc, co uděláte každý den. Co je nejdůležitější částí vaší neděle?
Sportovní utkání? Práce na zahradě? Co je první věcí, co uděláte každé ráno, když se probudíte, a každý
večer předtím, než usnete? Zkontrolovat svůj telefon? Zapnout zprávy? Bůh vás hledá. Chce, abyste věděli,
že On je jediný Bůh. Kontrolování telefonu vám nezaručí vysvobození, ale klečení na kolenou každé ráno
a každou noc vedle své postele před tím, kdo má moc vás vysvobodit, ano. Skrze tyto činy vám může dát
Bůh vysvobození.

Když pokračujete se čtením úžasných věcí, které Bůh učinil pro jeho lid Izrael, obraťte pozornost také na
svůj vlastní život. Vidíte ty úžasné věci, které pro vás, ve vašem životě a na oltáři, udělal? Pohleďte na tyto
věci a posilte dnes svou víru v to, že Pán je Bůh. Ano, Hospodin je jediný Bůh.


%newday
\newpage
\section{Den 9 - DŮVĚŘUJ BOHU}
\zacatekDruhyTyden
\subsection*{Čtení na den}
\textbf{Exodus 4,10-17}
\newline
\textit{
\textsuperscript{10}Ale Mojžíš Hospodinu namítal: „Prosím, Panovníku, nejsem člověk výmluvný; nebyl jsem dříve, nejsem ani nyní, když ke svému služebníku mluvíš. Mám neobratná ústa a neobratný jazyk.“
\textsuperscript{11}Hospodin mu však řekl: „Kdo dal člověku ústa? Kdo působí, že je člověk němý nebo hluchý, vidící nebo slepý? Zdali ne já, Hospodin?
\textsuperscript{12}Nyní jdi, já sám budu s tvými ústy a budu tě učit, co máš mluvit!“
\textsuperscript{13}Ale Mojžíš odmítl: „Prosím, Panovníku, pošli si, koho chceš.“
\textsuperscript{14}Tu Hospodin vzplanul proti Mojžíšovi hněvem a řekl: „Což nemáš bratra Árona, toho lévijce? Znám ho, ten umí mluvit. Jde ti už naproti a bude se srdečně radovat, až tě uvidí.
\textsuperscript{15}Budeš k němu mluvit a vkládat mu slova do úst. Já budu s tvými ústy i s jeho ústy a budu vás poučovat, co máte činit.
\textsuperscript{16}On bude mluvit k lidu za tebe, on bude tobě ústy a ty budeš jemu Bohem.
\textsuperscript{17}A tuto hůl vezmi do ruky; budeš jí konat znamení.“
}

\subsection*{Reflexe}
Mojžíš zakouší tíhu úkolu, který před něj byl postaven, a myslí si, že se Bůh spletl. Bůh po něm žádá něco,
čeho je neschopné dosáhnout. Tak jako mnoho lidí i dnes, Mojžíš postrádá důvěru v sebe i v Hospodina,
což, jak praví Písmo, rozněcuje Boží hněv. Bůh se ale jistě nehněvá kvůli nedostatku Mojžíšovy
výmluvnosti. Spíše „Hospodin vzplanul proti Mojžíšovi hněvem,“ protože Mojžíš postrádá důvěru v Boha
a dovoluje, aby jeho naděje vymizela. Bůh nás v průběhu našeho života zkouší mnoha způsoby, přesně tak
jako Mojžíše v dnešním čtení. Samozřejmě to nedělá proto, aby se o nás něco naučil. Zkouší nás, abychom
se sami naučili něco o sobě a o Něm.

Mojžíš postrádá ctnost naděje. Poznává, že jeho nedostatky nejsou hříchem, ale nemyslí si, že je Bůh
schopný dorovnat to, co Mojžíš postrádá. Právě to je problém. Mojžíš se musí naučit, že pokud má vést
Boží lid a dosáhnout nemožného, musí se zcela spoléhat na Boha.

Těchto 90 dní Exodu je pro vás obrovskou zkouškou. Učte se od Mojžíše: selžete v tomto cvičení, pokud
opravdu nesvěříte svou naději v Boha. Tak často ve strachu převezmeme kontrolu z Božích rukou a
zkoušíme ji třímat v těch našich. Když to uděláme a uspějeme, činíme tak za velkou cenu nás, a občas i
těch, které milujeme.

Uvědomte si, že se sami nemůžete dovést ke svobodě. Poté pohleďte na kříž. Naděje v Toho, který porazil
i smrt. On vás dovede ke svobodě. On jediný vám může dát vysvobození. Doufejte v Něho.

%newday
\newpage
\section{Den 10 - JAKÝM ČLOVĚKEM SE STANETE?}
\zacatekDruhyTyden
\subsection*{Čtení na den}
\textbf{Exodus 4,18-31}
\newline
\textit{
\textsuperscript{18}Mojžíš odešel a vrátil se ke svému tchánu Jitrovi. Řekl mu: „Rád bych šel a vrátil se ke svým bratřím, kteří jsou v Egyptě, a podíval se, zda ještě žijí.“ Jitro Mojžíšovi odvětil: „Jdi v pokoji.“
\textsuperscript{19}Hospodin pak řekl Mojžíšovi ještě v Midjánu: „Jen se vrať do Egypta, neboť zemřeli všichni, kteří ti ukládali o život.“
\textsuperscript{20}Mojžíš tedy vzal svou ženu a syny, posadil je na osla a vracel se do egyptské země. A do ruky si vzal Boží hůl.
\textsuperscript{21}Hospodin dále Mojžíšovi poručil: „Až se vrátíš do Egypta, hleď, abys před faraónem udělal všechny zázraky, jimiž jsem tě pověřil. Já však zatvrdím jeho srdce a on lid nepropustí.
\textsuperscript{22}Potom faraónovi řekneš: Toto praví Hospodin: ‚Izrael je můj prvorozený syn.
\textsuperscript{23}Vzkázal jsem ti: Propusť mého syna, aby mi sloužil. Ale ty jsi jej propustit odmítl. Za to zabiji tvého prvorozeného syna.‘“
\textsuperscript{24}Když se na cestě chystali nocovat, střetl se s ním Hospodin a chtěl ho usmrtit.
\textsuperscript{25}Tu vzala Sipora kamenný nůž, obřezala předkožku svého syna, dotkla se jeho nohou a řekla: „Jsi můj ženich, je to zpečetěno krví.“
\textsuperscript{26}A Hospodin ho nechal být. Tehdy se při obřízkách říkalo: „ Jsi ženich, je to zpečetěno krví.“
\textsuperscript{27}Hospodin řekl Áronovi: „Jdi na poušť naproti Mojžíšovi.“ Áron šel, setkal se s ním u Boží hory a políbil ho.
\textsuperscript{28}Mojžíš oznámil Áronovi všechna Hospodinova slova, s nimiž ho poslal, a všechna znamení, kterými ho pověřil.
\textsuperscript{29}Pak šel Mojžíš s Áronem a shromáždili všechny izraelské starší.
\textsuperscript{30}Áron vyřídil všechna slova, která mluvil Hospodin k Mojžíšovi, a Mojžíš učinil před očima lidu ona znamení.
\textsuperscript{31}A lid uvěřil. Když slyšeli, že Hospodin navštívil Izraelce a že pohleděl na jejich ujařmení, padli na kolena a klaněli se.
}

\subsection*{Reflexe}

Uvažujme o dvou hlavních postavách knihy Exodus. První, Mojžíš, zná své postavení před Pánem. Není povýšený
ani arogantní. Zachovává si svou důstojnout a je mu dokonce dána výsada od Boha. Na druhé straně stojí faraon, tak
povýšený a arogantní, že dokonce prohlásí sebe samého za božského a jeho podřízení ho takto musí přijímat. Bůh
nakonec zničí faraonovo potomstvo a jeho dynastii a dokazuje tak faraonovi, že je pouhý člověk.

Dnešní muži často napodobují faraona. Jen málo z nich by se nestoudně prohlásilo za božské, ale mnoho z nich se
tak chová. Určují si svou vlastní pravdu, nastavují svůj vlastní směr, odmítají se spoléhat na Boha a žijí svrchovaný
život. Od Adamova pádu má člověk vzpupné srdce. Izraelci tohoto příběhu nejsou výjimkou. Budou muset být
poučeni: čtyřicet let v poušti není procházka. Bůh pošle svůj lid do poušte, aby se naučil být na Něm závislý, plně se
na Něj spolehnout. Musí pochopit, že Bůh je Bůh a že oni jsou jeho prvorozenými syny.

Člověk je schopen mnoha velkolepých činů, ale Boží syn je schopen mnohem, mnohem více. Těchto devadesát dní
nabízí skvělou příležitost, jak svůj život nasměrovat k Bohu. Vzdejte se svého vzpurného srdce. Oblečte nového
člověka, jako milovaný syn pozoruhodného Otce, který se o vás nejen postará, ale také vám pomůže stát se
spolehlivým, nezištným a svobodným mužem pro druhé.

Zvažte svůj vlastní vztah s Bohem. Vidíte Boha jako svého Otce a žijete svůj vztah k Němu jako takovému?
Obracíte se na něj často, jako se dítě obrací na svého otce? Přineste dnes tyto otázky do své svaté hodiny.

%newday
\newpage
\section{Den 11 - POKORA}
\zacatekDruhyTyden
\subsection*{Čtení na den}
\textbf{Exodus 5,1-4}
\newline
\textit{
\textsuperscript{1}Mojžíš s Áronem pak předstoupili před faraóna a řekli: „Toto praví Hospodin, Bůh Izraele: Propusť můj lid, ať mi v poušti slaví slavnost. “
\textsuperscript{2}Farao však odpověděl: „Kdo je Hospodin, že bych ho měl uposlechnout a propustit Izraele? Hospodina neznám a Izraele nepropustím!“
\textsuperscript{3}Řekli: „Potkal se s námi Bůh Hebrejů. Dovol nám nyní odejít do pouště na vzdálenost tří dnů cesty a přinést oběť Hospodinu, našemu Bohu, aby nás nenapadl morem nebo mečem.“
\textsuperscript{4}Egyptský král je okřikl: „Proč, Mojžíši a Árone, odvádíte lid od jeho prací? Jděte za svými robotami!“
}

\newpage
\subsection*{Reflexe}

Faraon arogantně ignoruje příkazy Nejvyššího Boha. Samozřejmě má své důvody. Je vládcem nad mocným
královstvím. Řídí armády a otroky. Pochází z mocné linie uctívaných mužů. Faraon je mocný. Přesto,
navzdory všem jeho vznešenostem, bude ponížen dokonce pod mouchy a žáby, protože neohne koleno před
svým Tvůrcem.

Jsme Boží synové a jsme stvořeni k Jeho obrazu a podobě. Každý z nás má mnoho silných stránek, a proto
je pýcha neustálým pokušením. Když jsme na sebe moc hrdí, začneme se čím dál více oddávat vlastním
nutkáním. Výsledkem bude vždy náš krach. Dřív nebo později se ocitneme sraženi na kolenou věcmi, které
jsou pod naši důstojnost – závislostmi, rozptýleností, nudou… Jestliže se takto necháme ovládat, pýcha nás
učiní bezmocnými.

Mějte oči upřené k nebi a pokorně si pamatujte, že je to Bůh, kdo žádá vaši svobodu – a je to On, kdo dává
vysvobození. Ve své svaté hodině dnes promluvte s Pánem o své pýše. Požádejte ho, aby vám ukázal, na
co jste příliš hrdí. Požádejte ho, aby vám ukázal pravou pokoru. Pak ho proste, aby vám dal odvahu a milost
žít dnes v opravdové pokoře.

%newday
\newpage
\section{Den 12 - ZKLAMÁNÍ Z HŘÍCHU}
\zacatekDruhyTyden
\subsection*{Čtení na den}
\textbf{Exodus 5,5-21}
\newline
\textit{
\textsuperscript{5}A farao pokračoval: „Hle, lidu země je teď mnoho, a vy chcete, aby nechali svých robot?“
\textsuperscript{6}Onoho dne přikázal farao poháněčům lidu a dozorcům:
\textsuperscript{7}„Propříště nebudete vydávat lidu slámu k výrobě cihel jako dříve. Ať si jdou slámu nasbírat sami!
\textsuperscript{8}A uložíte jim dodat stejné množství cihel, jaké vyráběli dříve. Nic jim neslevujte, jsou líní. Proto křičí: Pojďme obětovat svému Bohu.
\textsuperscript{9}Ať na ty muže těžce dolehne otročina, aby měli co dělat a nedali na lživé řeči.“
\textsuperscript{10}Poháněči lidu a dozorci vyšli a ohlásili lidu: „Toto praví farao: Nedám vám žádnou slámu.
\textsuperscript{11}Sami si jděte nabrat slámu, kde ji najdete. Ale z vaší pracovní povinnosti se nic nesleví.“
\textsuperscript{12}Lid se rozběhl po celé egyptské zemi, aby na strništích sbíral slámu.
\textsuperscript{13}Poháněči je honili: „Plňte svůj denní úkol, jako když sláma byla.“
\textsuperscript{14}Dozorci z řad Izraelců, které nad nimi ustanovili faraónovi poháněči, byli biti. Vytýkalo se jim : „Proč jste v těchto dnech nevyrobili tolik cihel jako dříve?“
\textsuperscript{15}Dozorci z řad Izraelců tedy přišli a úpěli před faraónem: „Proč se svými otroky takhle jednáš?
\textsuperscript{16}Tvým otrokům se nedodává sláma, ale pokud jde o cihly, poroučejí nám: ‚Dělejte!‘ Hle, tvoji otroci jsou biti a tvůj lid bude pykat za hřích.“
\textsuperscript{17}Farao odpověděl: „Jste lenoši líní, proto říkáte: ‚Pojďme obětovat Hospodinu.‘
\textsuperscript{18}Hned jděte dělat! Sláma vám dodávána nebude, ale dodávku cihel odvedete.“
\textsuperscript{19}Dozorci z řad Izraelců viděli, že je s nimi zle, když bylo řečeno: „Nesmíte snížit svůj denní úkol výroby cihel.“
\textsuperscript{20}Když vycházeli od faraóna, narazili na Mojžíše a Árona, kteří se s nimi chtěli setkat.
\textsuperscript{21}Vyčítali jim: „Ať se nad vámi ukáže Hospodin a rozsoudí. Vy jste pokáleli naši pověst u faraóna a jeho služebníků. Dali jste jim do ruky meč, aby nás povraždili.“
}

\subsection*{Reflexe}

Izraelci si právem stěžují na to, že by měli vyrábět stejné množství cihel, i když dostávají méně slámy. Požadavky
otrokářů se staly mnohem nedosažitelnějšími. V jejich případě je otrokářem faraon, zatímco v našem případě je
otrokářem Satan. Naše závislosti a zvyky v nás vytvářejí neuhasitelnou touhu. Jak však ukazuje dnešní čtení,
poddávat se těmto našim pokušením je hluboce neuspokojivé.

Bez ohledu na to, k čemu jsme zotročeni, jsme vždy v pokušení usilovat o rychlou a snadnou nápravu tím, že se
vzdáme naší touhy. Ironické je, že čím více se snažíme se jí vzdát, tím méně to je možné. Pokušení vytvářejí iluzi,
že budeme více naplněni, když se příště jen trochu víc dopřejeme. Po čase nás pokušení vtáhne hlouběji do otroctví
a dál od svobody, takže je pro nás těžší překonat. Každý z nás tak moc dobře ví, jak pravdivě slova svatého Pavla v
jeho dopise Římanům znějí: „Nepoznávám se ve svých skutcích; vždyť nedělám to, co chci, nýbrž to, co
nenávidím,“ (Řím 7,15).

Nikdy nezapomeňte na tuto lež, že k vám neustále mluví vaše pokušení. Udržujte ji v čele své mysli jako hnací sílu
vedoucí ke svobodě. Je to frustrace z této lži, která vás sem přivedla, a bude to frustrace z této lži, díky které budete
ochotni jít do pouště, ochotni následovat tento náročný plán svobody.

%newday
\newpage
\section{Den 13 - VHLED DÍKY SLABOSTI}
\zacatekDruhyTyden
\subsection*{Čtení na den}
\textbf{Exodus 5,22-6,12}
\newline
\textit{
\textsuperscript{22}Mojžíš se obrátil k Hospodinu a řekl: „Panovníku, proč jsi dopustil na tento lid zlo? Proč jsi mě vlastně poslal?
\textsuperscript{23}Od chvíle, kdy jsem předstoupil před faraóna, abych mluvil tvým jménem, nakládá s tímto lidem ještě hůře. A ty svůj lid stále nevysvobozuješ.“
\textsuperscript{1}Hospodin Mojžíšovi odvětil: „Nyní uvidíš, co faraónovi udělám. Donutím ho, aby je propustil; donutím ho, aby je vypudil ze své země.“
\textsuperscript{2}Bůh promluvil k Mojžíšovi a ujistil ho: „Já jsem Hospodin.
\textsuperscript{3}Ukázal jsem se Abrahamovi, Izákovi a Jákobovi jako Bůh všemohoucí. Ale své jméno Hospodin jsem jim nedal poznat.
\textsuperscript{4}Ustavil jsem s nimi také svou smlouvu, že jim dám kenaanskou zemi, zemi jejich putování, kde pobývali jako hosté.
\textsuperscript{5}Rovněž jsem uslyšel sténání Izraelců, které si Egypťané podrobili v otroctví, a rozpomenul jsem se na svou smlouvu.
\textsuperscript{6}Proto řekni Izraelcům: Já jsem Hospodin. Vyvedu vás z egyptské roboty, vysvobodím vás z vašeho otroctví a vykoupím vás vztaženou paží a velkými soudy.
\textsuperscript{7}Vezmu si vás za lid a budu vám Bohem. Poznáte, že já jsem Hospodin, váš Bůh, který vás vyvede z egyptské roboty.
\textsuperscript{8}Dovedu vás do země, kterou jsem přísežně slíbil dát Abrahamovi, Izákovi a Jákobovi. Vám ji dám do vlastnictví. Já jsem Hospodin.“
\textsuperscript{9}Mojžíš to tak Izraelcům vyhlásil, ale ti nebyli pro malomyslnost a tvrdou otročinu s to Mojžíšovi naslouchat.
\textsuperscript{10}Hospodin dále mluvil k Mojžíšovi:
\textsuperscript{11}„Předstup před faraóna, krále egyptského, a vyřiď mu, ať propustí Izraelce ze své země.“
\textsuperscript{12}Mojžíš Hospodinu namítl: „Když mi nenaslouchají Izraelci, jak by mě poslechl farao! Nejsem způsobilý mluvit.“
}

\subsection*{Reflexe}

Často je „Bůh Starého zákona“ obviněn z tvrdosti a krutosti, a dnešní Písmo zdálky potvrzuje tento stereotyp. Ale
podívej se znovu. Kdyby se Bůh unáhlil a vyřešil problémy, kterým čelí Izraelité, čeho by dosáhl? Izraelci by si
povzdechli úlevou a rychle zapomněli na své nesnáze a na to, kdo jim dává svobodu. Nenaučili by se, že Bůh jejich
otců je jejich Bůh. Nenaučili by se nic, a nic by nezískali. Izraelitům je v tíživé situaci dána možnost naučit se něco
velmi cenného: Bůh je vezme za svůj lid a bude jejich Bohem. Povede je ke svobodě a ukončí jejich zajetí.
Bezpochyby uvidí, že Bůh je Bůh a že bez Něj nemohou nic dělat.

Chvíli přemýšlejte o tom, co by byl člověk bez slabostí. Pravděpodobně by byl pyšnou a povýšenou šelmou, která si
myslí, že nepotřebuje Boha, a věří, že se dokáže vysvobodit jak v tomto světě, tak i v příštím. Naše slabost nám tak
poskytuje vhled. Pokud svou slabost necháte, naučí vás, že potřebujete Boha. Slabost vás může naučit obrátit se
k Bohu a požádat ho o svobodu, místo toho abyste věci brali sami do svých rukou.

Vaše slabost vás sem přivedla. Vaše slabost (ne váš hřích) je Božím darem pro vás. Pokud dbáte na vhled, který vám
dává (potřeba Boha), zůstanete na cestě ke svobodě. Poděkujte dnes Pánu za dar své slabosti, který vás neustále
přivádí zpět do Jeho milujícího náručí.

%newday
\newpage
\section{Den 14 - VLIV OTCE}
\zacatekDruhyTyden
\subsection*{Čtení na den}
\textbf{Exodus 6,13-27}
\newline
\textit{
\textsuperscript{13}Ale Hospodin Mojžíšovi a Áronovi domluvil a dal jim příkazy pro Izraelce i pro faraóna, krále egyptského, aby připravili odchod Izraelců z egyptské země.
\textsuperscript{14}Toto jsou představitelé otcovských rodů: Rúbenovci, potomci Izraelova prvorozeného: Chanók a Palú, Chesrón a Karmí. To jsou čeledi Rúbenovy.
\textsuperscript{15}Šimeónovci: Jemúel, Jamín, Ohad, Jakín, Sóchar a Šaul, syn Kenaanky. To jsou čeledi Šimeónovy.
\textsuperscript{16}Toto jsou jména Léviovců podle jejich rodopisu: Geršón, Kehat a Merarí. Lévi byl živ sto třicet sedm let.
\textsuperscript{17}Geršónovci: Libní a Šimeí podle svých čeledí.
\textsuperscript{18}Kehatovci: Amrám, Jishár, Chebrón a Uzíel. Kehat byl živ sto třicet tři léta.
\textsuperscript{19}Meraríovci: Machlí a Muší. To jsou lévijské čeledi podle jejich rodopisu.
\textsuperscript{20}Amrám si vzal za ženu Jókebedu, svou tetu. Ta mu porodila Árona a Mojžíše. Amrám byl živ sto třicet sedm let.
\textsuperscript{21}Synové Jishárovi: Kórach, Nefeg a Zikrí.
\textsuperscript{22}Synové Uzíelovi: Míšael, Elsáfan a Sitrí.
\textsuperscript{23}Áron si vzal za ženu Elíšebu, dceru Amínadabovu, sestru Nachšónovu. Ta mu porodila Nádaba, Abíhúa, Eleazara a Ítamara.
\textsuperscript{24}Synové Kórachovi: Asír, Elkána a Abíasaf. To jsou kórachovské čeledi.
\textsuperscript{25}Eleazar, syn Áronův, si vzal za ženu jednu z dcer Pútíelových. Ta mu porodila Pinchasa. To jsou představitelé lévijských rodů podle svých čeledí.
\textsuperscript{26}Z tohoto pokolení pocházejí ten Áron a Mojžíš, k nimž mluvil Hospodin: „Vyveďte z egyptské země Izraelce seřazené po oddílech.“
\textsuperscript{27}Oni to byli, kdo mluvili k faraónovi, králi egyptskému, že mají vyvést Izraelce z Egypta. To tedy byli Mojžíš a Áron.
}

\subsection*{Reflexe}

Při zdlouhavých biblických rodokmenech většinou ztrácíme chuť číst dál. Jména jsou cizí a význam rodinných linií je dávno
ztracen v historii. Všechno, co je obsaženo v Písmu Svatém, však ukazuje církvi a věřícím důležité pravdy. Rodokmeny nás
konkrétně spojují se sliby a smlouvami, které učinil Bůh svým lidem. Také nám připomínají – jak připomínali předkům – naši
důstojnost a náš budoucí domov.

Vaše vlastní rodinná linie vás naučí mnoho o vás samých. Síla charakteru, temperament, osobnost, vaše osoba je určena vašimi
předky. Přesněji, navzdory tomu, co nám říká naše kultura, se můžete dozvědět mnoho o sobě a o svém životě od svého otce.

Otcové, možná více, než víme, mají významný dopad na jejich potomstvo – někdy pozitivní, někdy negativní. Převážně od
našich otců se učíme sebeovládání, sebedůvěry, způsobu, jakým komunikujeme s vnějším světem. Pokud bojujete v některé z
těchto oblastí, znamená to, že vás otec selhal? To je zásadní otázka, které je každý člověk vystaven.

Tohle není místo pro vyčerpávající pojednání o otcovství. Když však pracujete, abyste si lépe porozuměli a usilovali o svobodu
sebeovládání, budete pravděpodobně považovat za přínosné přemýšlet o svém otci a vašem vztahu s ním. Vyzkoušejte toto
cvičení: pokud je to možné, požádejte svého otce, aby popsal svého vlastního otce (vašeho dědečka), a zjistíte, odkud pocházelo
mnoho rysů vašeho otce (pozitivních i negativních). Můžete také získat představu o tom, co předáváte (nebo co budete předávat)
svým vlastním dětem. Vnímejte toto cvičení pozitivně a neuchylujte se k obviňování, jde tu o pochopení. Pamatujte, že jste na
cestě k rozvoji ctnostných zvyků uprostřed bratrství. 

Tyto návyky budou darem pro vaše blízké, zejména pro vaše děti, bez
ohledu na to, jak staré jsou. Nemůžete změnit minulost, ale nyní máte příležitost pracovat s Pánem, abyste pro svou rodinu
vytvořili novou budoucnost.


% ===============================================
% ===== TRETI TYDEN
% ===============================================
%ukony
\newpage
\section*{Úkony (ukazatel cesty) pro 3. týden}

\textbf{Místo:} Egypt (jste v Egyptě)

Život se stal náročnějším. Pro Izraelity práce nepovolila. Faraon je odmítá nechat jít. I přes prvních pět ran se Mojžíši a Áronovi nedaří získat větší svobodu Izraelitům. První nadšení z bratrského exodu pro vás také možná již vyprchalo. Začínáte si uvědomovat tvrdou realitu mnoha týdnů před vámi. A co hůře, stejně jako Izraelitům, ani pro vás ještě neuběhlo tolik času na to, abyste začali pociťovat zisk svobody. Navzdory tomu všemu, komu se tento týden rozhodnete sloužit: Bohu, nebo faraonovi?

\subsection*{1. Pokračujte ve zkoumání svého dne}
Nejen, že vám pomůže projít skrz těchto 90 dní, ale udrží vás to svobodnými v 91. dnu. Dobře praktikujte noční examen, takže si vytvoříte dobrý zvyk pro 91. den. (Nevíte, jak dělat zkoumání na konci dne? Podívejte se do kapitoly Jak se modlit noční examen v Průvodci terénem.)
\subsection*{2. Přistupujte upřímně k vaší denní svaté hodině}
Čas v kontemplativní modlitbě je rozhodující při exodu. Bůh dává vysvobození. Bůh vás povede, kam potřebujete. Bůh vám bude připomínat svou lásku k vám a vaší pravou hodnotu jako Jeho prvorozeného syna. Ale neuslyšíte žádnou z těchto věcí, pokud si každý den nevyhradíte čas na ztišení své mysli a naslouchání.
\subsection*{3. Neodbývejte úkony/úkoly (don’t cut corners)}
Touto dobou už jste se dobře seznámili s asketickými disciplínami. Víte, co je na nich těžké a co jednoduché. Čím více disciplíny znáte, tím jednodušší bude je odbývat (cut corners-zkracovat si cestu). Chcete zůstat v Egyptě, nebo chcete být osvobozeni? (Pro více informací o důležitosti asketismu nahlédněte do Pilířů Exodu 90 v příručce pod pilířem Bratrství v příručce Exodu.)
\subsection*{4. Pamatujte si své \textit{proč}.}
Vzpomeňte si, proč jste začali Exodus 90. Pokud své proč zapomenete, pravděpodobně nebudete schopni/moci tuto cestu dokončit. Lákavost pohodlí je jednoduše příliš silná, než abyste ji překonali bez proč, které za to stojí. (Nezapsali jste si své proč? Vraťte se zpět do části příručky Exodu 90 nazvané Co je vaše proč a napište si své proč předtím, než budete pokračovat dál.)
\subsection*{5. Zůstaňte radostní}
Pokud budete držet své proč v čele své mysli a zůstanete pevní v modlitbě, vaše naděje na svobodu nebude jiná než vysoká. Jít exodem je náročné. Naštěstí je to Bůh, kdo dává vysvobození, které hledáme. Egyptské rány nám ukazují, že také Bůh pracuje na vaší svobodě.

\subsection*{Modlitba}
Modlete se, aby Pán osvobodil vás a vaše bratrství \newline
Modleme se za svobodu všech mužů v exodu, stejně tak, jako se oni modlí za vás.\newline
Ve jménu Otce i Syna i Ducha svatého … Otče náš… Ve jménu Otce i Syna i Ducha svatého … Amen.

\newpage


%newday
\newpage
\section{Den 15 - BŮH SI PŘEJE VAŠI SVOBODU}
\zacatekTretiTyden
\subsection*{Čtení na den}
\textbf{Exodus 6,28-7,7}
\newline
\textit{
\textsuperscript{28}To bylo tehdy, když Hospodin mluvil k Mojžíšovi v egyptské zemi.
\textsuperscript{29}Hospodin promluvil k Mojžíšovi: „Já jsem Hospodin! Řekni faraónovi, králi egyptskému, všechno, co k tobě mluvím.“
\textsuperscript{30}Mojžíš však Hospodinu namítl: „Nejsem způsobilý mluvit. Jak by mě farao poslechl?“
\textsuperscript{1}Hospodin řekl Mojžíšovi: „Pohleď, ustanovil jsem tě, abys byl pro faraóna Bohem, a Áron, tvůj bratr, bude tvým prorokem.
\textsuperscript{2}Ty mu povíš všechno, co ti přikážu, a Áron, tvůj bratr, bude mluvit s faraónem, aby propustil Izraelce ze své země.
\textsuperscript{3}Já však zatvrdím faraónovo srdce a učiním v egyptské zemi mnoho svých znamení a zázraků.
\textsuperscript{4}Farao vás neposlechne, ale já vložím na Egypt svou ruku. Vyvedu zástupy svého lidu, syny Izraele, z egyptské země, ale ji postihnu velkými soudy.
\textsuperscript{5}Egypťané poznají, že já jsem Hospodin, až vztáhnu svou ruku na Egypt a vyvedu Izraelce z jejich středu.“
\textsuperscript{6}Mojžíš a Áron učinili přesně tak, jak jim Hospodin přikázal.
\textsuperscript{7}Mojžíšovi bylo osmdesát let a Áronovi osmdesát tři léta, když mluvili s faraónem.
}

\subsection*{Reflexe}
Podívejme se na věc z této perspektivy – Bůh žádá Mojžíše, aby čelil nejmocnějšímu vládci světa v samém srdci
jeho království, kde je obklopen svými lidmi. Zde má Mojžíš říci faraonovi, co má dělat. Asi není divu, že se
Mojžíš zachová jako většina lidí, když čelí znepokojivému úkolu: couvá zpět. Mojžíš vysvětluje Bohu, že je vázán
jazykem a není způsobilý mluvit. Bůh však jeho lidskou omluvu nepřijímá.

Místo toho Bůh, vládce veškerého stvoření, říká slabému Mojžíšovi: „Ustanovil jsem tě, abys byl pro faraóna
Bohem.“ Svatý Ambrož nám říká, že Mojžíšova ctnost daleko převyšuje faraonovu moc. Mojžíšovi není dáno
nadmíru. Jeho vášně nad ním nevládnou. Je to muž, který „ostře kritizuje, že jeho tělo bylo ztělesněno autoritou,
která byla téměř královská“.Zatímco Mojžíšova sebedůvěra ochabuje, Hospodin má ve svého syna veškerou
důvěru.

Tak je to i s vámi. Vy, stejně jako Mojžíš, ovládáte své vášně a trpíte bolestmi askeze, ztělesňujete Božího syna,
královského a mocného oproti směšnému faraonovi světskosti a neřestí. Mohl jste to o sobě říci před dvaceti dny?
Způsob vašeho života se mění. Následujte Kristův plán modlitby, askeze a bratrství a budete i nadále dostávat
milost vítězit nad svými vášněmi.

Děkujte Bohu za úspěch, který jste doposud měli. Nepřetržitá vděčnost bude podporovat vaši vděčnost uprostřed
výzev tohoto duchovního cvičení.

%newday
\newpage
\section{Den 16 - SPÁSA SKRZE KŘÍŽ}
\zacatekTretiTyden
\subsection*{Čtení na den}
\textbf{Exodus 7,8-13}
\newline
\textit{
\textsuperscript{8}Hospodin dále řekl Mojžíšovi a Áronovi:
\textsuperscript{9}„Až k vám farao promluví: ‚Prokažte se nějakým zázrakem,‘ řekneš Áronovi: ‚Vezmi svou hůl a hoď ji před faraóna,‘ a stane se drakem.“
\textsuperscript{10}Mojžíš s Áronem tedy předstoupili před faraóna a učinili, jak Hospodin přikázal. Áron hodil svou hůl před faraóna i před jeho služebníky a ona se stala drakem.
\textsuperscript{11}Farao však také povolal mudrce a čaroděje, a egyptští věštci učinili svými kejklemi totéž.
\textsuperscript{12}Hodili každý svou hůl na zem a ony se staly draky. Ale Áronova hůl jejich hole pohltila.
\textsuperscript{13}Srdce faraónovo se však zatvrdilo a neposlechl je, jak Hospodin předpověděl.
}

\subsection*{Reflexe}

V dnešním čtení faraon přikazuje svým věštcům, aby zopakovali znamení, které provedl Áron.
Dosvědčuje se, že je to pouhým pokusem zmást lidi a setřít moc Božího znamení. Faraon se sanží
nabídnout alternativy k Boží cestě. V dnešní době nalézáme moderní „faraony“ všude, snažící se o to
samé.

V Písmu Svatém Áronova hůl předznamenává kříž. Jedním z příkladů je použití Áronovy hole k rozdělení
Rudého moře umožňující přechod Izraelitů ke svobodě. Tento čin předznamenává křest, kde voda, skrze
kterou přecházíme ke svobodě, dostává svou moc z kříže. Neexistuje žádná jiná cesta ke spáse než skrze
kříž, ale stejně jako faraon, i naše kultura zoufale hledá alternativu – něco jednoduššího, něco
příjemnějšího, něco, co člověk dokáže ovládat.

Kolikrát jste hledali jinou cestu ke svobodě, než jste nakonec museli přijmout, že žádná „jiná cesta“
neexistuje?„Jako Mojžíš vyvýšil hada na poušti, tak musí být vyvýšen Syn člověka,“ (J 3,14). Podívejte
se na plán před vámi: modlitba, askeze, bratrství. Nemělo by být žádným překvapením, že cesta ke
svobodě, kterou lidem poskytuje Písmo, je cestou kříže.

Pokud dokážete unést odpověď našeho Pána, podívejte se dnes na kříž a zeptejte se ho, zda existuje jiný
způsob než právě on.

%newday
\newpage
\section{Den 17 - HOJNĚ NAPLNĚNÝ}
\zacatekTretiTyden
\subsection*{Čtení na den}
\textbf{Exodus 7,14-24}
\newline
\textit{ 
\textsuperscript{14}Hospodin řekl Mojžíšovi: „Srdce faraónovo je neoblomné. Nechce lid propustit.
\textsuperscript{15}Jdi k faraónovi ráno. Až půjde k vodě, postav se naproti němu na břehu Nilu a vezmi si do ruky hůl, která se proměnila v hada.
\textsuperscript{16}Řekneš mu: Hospodin, Bůh Hebrejů, mě k tobě posílá se vzkazem: Propusť můj lid, aby mi na poušti sloužil. Ale ty jsi dosud neposlechl.
\textsuperscript{17}Toto praví Hospodin: Podle toho poznáš, že já jsem Hospodin: Holí, kterou mám v ruce, teď udeřím do vody v Nilu, a ta se promění v krev.
\textsuperscript{18}Ryby, které jsou v Nilu, leknou a Nil bude páchnout. Marně budou Egypťané usilovat, aby se mohli napít vody z Nilu.“
\textsuperscript{19}Hospodin dále řekl Mojžíšovi: „Vyzvi Árona: ‚Vezmi svou hůl a vztáhni ruku nad egyptské vody, nad průplavy, nad říční ramena, nad jezera, vůbec nad všechny nahromaděné vody.‘ Stanou se krví. V celé egyptské zemi bude krev, i ve džberech a džbánech.“
\textsuperscript{20}Mojžíš a Áron učinili, jak Hospodin přikázal. Áron pozdvihl hůl a před očima faraóna a jeho služebníků udeřil do vody v Nilu a všechna voda Nilu se proměnila v krev.
\textsuperscript{21}Ryby v Nilu lekly, Nil začal páchnout a Egypťané nemohli vodu z Nilu pít. A krev byla v celé egyptské zemi.
\textsuperscript{22}Ale totéž učinili egyptští věštci svými kejklemi. Faraónovo srdce se zatvrdilo a neposlechl je, jak Hospodin předpověděl.
\textsuperscript{23}Farao se obrátil a vešel do svého domu, a ani toto si nevzal k srdci.
\textsuperscript{24}Všichni Egypťané kopali kolem Nilu, aby přišli na pitnou vodu, protože vodu z Nilu pít nemohli.
}

\subsection*{Reflexe}

Bůh zjevuje všechny věci takové, jaké skutečně jsou. V této pasáži Bůh začíná řadu deseti ran, aby ukázal svou
moc nad faraonem a falešnými bohy Egypta. Jeho moc se ukazuje tak mocně, že se všechny vody Egypta
proměňují v krev, takže je Nil tak zkažený, že lidé nemají vodu k pití. Krev je zde symbolem tělesné existence –
hmoty lidské přirozenosti tohoto světa. Voda, zdroj lidského života, se stává zdrojem smrti a rozpadu.

V hebrejštině tento text naznačuje, že hrnce a nádoby používané pro vodu (a nyní naplněné krví) byly vyrobeny z
materiálu získaného ze stromů a kamenů, což byly shodou okolností stejné materiály, které Egypťané používali při
stavbě svých model. Jinými slovy, falešní bohové Egypťanů se nyní stali pro ně zdrojem smrti; nemohli je totiž
zachránit.

I dnes jste neustále v pokušení obrátit se k věcem těla –k sexu, moci, penězům– které vás odtrhnou od pravého
života a svobody. Nádoba vašeho života je často naplněna smrtí a úpadkem, spíše než životem. Přesto pohleďte na
jiný úryvek Písma uvedený v evangeliu sv. Jana (2. kapitola). Zde Kristus promění vodu, která je v nádobách k
čištění, na víno, v symbol života a radosti a znamení nového života, který nám nabízí v Duchu. Buďte pevní a
vězte, že pokud mu dovolíte, Ježíš také obrátí vodu vašeho života na víno. Přijďte dnes k Pánu. Požádejte ho, aby
vás tak hojně naplnil životem a radostí, že všichni lidé kolem vás budou moct vidět a vědět, co ve vás On činí.

%newday
\newpage
\section{Den 18 - ŽIJETE PRO OSTATNÍ?}
\zacatekTretiTyden
\subsection*{Čtení na den}
\textbf{Exodus 6,13-27}
\newline
\textit{ 
\textsuperscript{25}To trvalo plných sedm dní poté, co Hospodin zasáhl Nil.
\textsuperscript{26}Potom Hospodin řekl Mojžíšovi: „Předstup před faraóna a řekni mu: Toto praví Hospodin: Propusť můj lid, aby mi sloužil.
\textsuperscript{27}Budeš-li se zdráhat jej propustit, napadnu celé tvé území žábami.
\textsuperscript{28}Nil se bude žábami hemžit, vylezou a vniknou do tvého domu, do tvé ložnice a na tvé lože i do domu tvých služebníků a mezi tvůj lid, do tvých pecí a díží.
\textsuperscript{29}I po tobě, po tvém lidu a po všech tvých služebnících polezou žáby.“
\textsuperscript{1}Hospodin dále řekl Mojžíšovi: „Vyzvi Árona: ‚Vztáhni ruku se svou holí nad průplavy, nad říční ramena i nad jezera a vyveď na egyptskou zemi žáby.‘“
\textsuperscript{2}Áron vztáhl ruku nad egyptské vody a žáby vylézaly, až pokryly egyptskou zemi.
\textsuperscript{3}Ale totéž učinili věštci svými kejklemi a i oni vyvedli na egyptskou zemi žáby.
\textsuperscript{4}Tu povolal farao Mojžíše a Árona a řekl: „Proste Hospodina, aby mě i můj lid zbavil žab. Pak propustím lid, aby obětoval Hospodinu.“
\textsuperscript{5}Mojžíš faraónovi odvětil: „Rač mi sdělit, kdy mám prosit za tebe, za tvé služebníky a za tvůj lid, aby Hospodin vyhladil žáby u tebe i v tvých domech. Zůstanou jen v Nilu.“
\textsuperscript{6}Farao odpověděl: „Zítra.“ Mojžíš řekl: „ Ať je podle tvého slova, abys poznal, že nikdo není jako Hospodin, náš Bůh.
\textsuperscript{7}Žáby se stáhnou od tebe i z tvých domů, od tvých služebníků a od tvého lidu. Zůstanou jen v Nilu.“
\textsuperscript{8}Nato odešel Mojžíš s Áronem od faraóna a Mojžíš úpěnlivě volal k Hospodinu kvůli žábám, kterými faraóna postihl.
\textsuperscript{9}Hospodin učinil podle Mojžíšovy prosby a žáby v domech, ve dvorcích i na polích pošly.
\textsuperscript{10}Shrabali je na hromady a kupy a zápach z nich naplnil zemi.
\textsuperscript{11}Když však farao viděl, že nastala úleva, zůstal v srdci neoblomný a neposlechl je, jak Hospodin předpověděl.
\textsuperscript{12}Hospodin řekl Mojžíšovi: „Vyzvi Árona: ‚Vztáhni svou hůl a udeř do prachu na zemi!‘ Stanou se z něho po celé egyptské zemi komáři.“
\textsuperscript{13}I učinili tak. Áron vztáhl ruku s holí a udeřil do prachu na zemi a na lidech i na dobytku se objevili komáři. Po celé egyptské zemi se ze všeho prachu země stali komáři.
\textsuperscript{14}Když totéž chtěli učinit věštci svými kejklemi, totiž vyvést komáry, nemohli. A komáři byli na lidech i na dobytku.
\textsuperscript{15}Věštci tedy řekli faraónovi: „Je to prst Boží.“ Srdce faraónovo se však zatvrdilo a neposlechl je, jak Hospodin předpověděl.
}

\subsection*{Reflexe}

Pokud nejste osmiletý chlapec, vyhlídka na nesčetné kvákající žáby ohromující zemi je v nejlepším případě
nepříjemná. Tato slizká scéna odhaluje něco o faraonovi jako člověku. Když mu Mojžíš nabídne úlevu od
obojživelníků, Faraon souhlasí s tím, že Mojžíš by měl zasáhnout a přimluvit se u Boha, ale ne teď – „zítra“. Jeho
arogantní lhostejnost k situaci jeho poddaných odhaluje tvrdost jeho srdce a jeho narcismus. Odmítá pokleknout
před Bohem bohů a být viděn s Ním jakýmkoliv způsobemspolupracovat. Kromě toho, když je tato rána stažena,
faraon nevykazuje žádnou vděčnost Mojžíšovi a ve své nafouklé hrdosti se vůbec nezajímá o Boha.

Jednou z nejzávažnějších chyb, které děláme, je,že nebereme v úvahu nikoho jiného než sebe samého. Příliš snadno
(i když jen obrazně) přehlížíme Boha vesmíru, naše ženy, farníky, syny a dcery. Toto nerespektování a přehlížení
ostatních je opakem toho, co to znamená být mužem. Kromě toho je tato naše nevšímavost Kristova těla – což je
také přehlížení Boha samotného– jistou cestou do pekla.

Chcete dosáhnout svého skutečného potenciálu a používat svou moc od Boha, jakou vám dal při vašem stvoření?
Pak musíte myslet na ostatní. Musítesesadit sami sebe z místa Boha a pozvednout ty kolem sebe. Tak půjdete
svatou cestou. Jak teď Hospodin vidí vaše činy? Nejste si jistí? Zeptejte se ho.

%newday
\newpage
\section{Den 19 - BŮH DÁVÁ VŠECHNY POTŘEBNÉ MILOSTI}
\zacatekTretiTyden
\subsection*{Čtení na den}
\textbf{Exodus 8,16-19}
\newline
\textit{
\textsuperscript{16}Hospodin řekl Mojžíšovi: „Za časného jitra se postav před faraóna, až vyjde k vodě. Řekneš mu: Toto praví Hospodin: Propusť můj lid, aby mi sloužil!
\textsuperscript{17}Jestliže můj lid nepropustíš, pošlu na tebe, na tvé služebníky, na tvůj lid i na tvé domy mouchy. Domy Egypťanů budou plné much, i ta půda, na které žijí. 
\textsuperscript{18}Ale zemi Gošen, kde se zdržuje můj lid, v onen den podivuhodně odliším. Tam mouchy nebudou, abys poznal, že já jsem Hospodin i uprostřed této země.
\textsuperscript{19}Učiním rozdíl mezi lidem svým a lidem tvým. Toto znamení se stane zítra.“
}

\subsection*{Reflexe}

Říká se, že „Bůh nám nikdy nedává víc, než můžeme zvládnout.“ To je daleko od pravdy. Podívej se na Mojžíše.
Bůh mu dal mnohem víc, než by mohl zvládnout. Kvůli tomu má Mojžíš dvě možnosti: může utéct a snažit se utěšit
sebe sama, s vědomím, že se Bůh zeptá jiných, nebo může věřit, že Bůh zasáhne a poskytne mu milost tam, kde
jeho lidská síla nestačí.
Snažit se být dobrým křesťanem může být v dnešní kultuře velmi náročné. Mnozí z nás jsou v pokušení uvěřit lži,
že kdybychom nikdy nepoznali Boha a jeho Církev, byli bychom mnohem šťastnější, protože bychom nemuseli
snášet život mnoha křížů. Myšlenka, že naše kříže nás zotročují, je ale daleko od pravdy.

\begin{minipage}{\dimexpr\textwidth-20pt}
  Dnes se upokojte v těchto slovech svatého Františka Saleského:
\begin{quote}
  \textit{Zapamatujte si tuto jednoduchou pravdu, která je nade všechny pochybnosti: Bůh dovoluje, aby mnoho obtíží trápilo ty, kteří mu chtějí sloužit, ale nikdy je nenechá padnout pod tíhou těchto obtíží, pokud Mu stále důvěřují… Nikdy, za žádných okolností nepodléhejte pokušení znechucení, ani pod lehce uvěřitelnou záminkou pokory.}
\end{quote}
\end{minipage}


Pokud toužíte sloužit Bohu, bude od vás požadovat víc, než dokážete zvládnout. V tomto bodě duchovního cvičení
jsme si této skutečnosti dobře vědomi. Ale Bůh nikdy nedopustí, abyste se potopili pod tíhou svých břemen, dokud
Mu důvěřujete. Znovu se podívejte na své proč. Věříte, že to Bůh dokončí? Pokud ne, teď je vhodný čas o tom
s Ním mluvit. Pokud ano, věnujte čas tomu, abyste Bohu vyjádřili svou vděčnost.

%newday
\newpage
\section{Den 20 - SLUŽTE HOSPODINU}
\zacatekTretiTyden
\subsection*{Čtení na den}
\textbf{Exodus 8,20-28}
\newline
\textit{
\textsuperscript{20}A Hospodin tak učinil. Dotěrné mouchy vnikly do domu faraónova, do domu jeho služebníků a na celou egyptskou zemi. Země byla těmi mouchami zamořena.
\textsuperscript{21}Tu povolal farao Mojžíše a Árona a řekl: „Nuže, přineste oběť svému Bohu zde v zemi.“
\textsuperscript{22}Mojžíš odpověděl: „Nebylo by správné, abychom to učinili. To, co máme obětovat Hospodinu, svému Bohu, je Egypťanům ohavností. Copak by nás neukamenovali, kdybychom před nimi obětovali, co je jim ohavností?
\textsuperscript{23}Odejdeme do pouště na vzdálenost tří dnů cesty a tam budeme obětovat Hospodinu, svému Bohu, jak nám nařídil.“
\textsuperscript{24}Farao řekl: „Propustím vás tedy, abyste obětovali Hospodinu, svému Bohu, na poušti. Jenom neodcházejte příliš daleko. Proste za mne.“
\textsuperscript{25}Mojžíš odvětil: „Až od tebe odejdu, budu prosit Hospodina a zítra odletí mouchy od faraóna, od jeho služebníků i od jeho lidu. Jen ať nás opět farao neobelstí, že by nechtěl propustit lid, aby obětoval Hospodinu.“
\textsuperscript{26}Pak Mojžíš od faraóna odešel a prosil Hospodina.
\textsuperscript{27}A Hospodin učinil, jak Mojžíš řekl. Mouchy odletěly od faraóna, od jeho služebníků i od jeho lidu. Ani jediná nezůstala.
\textsuperscript{28}Ale farao zůstal v srdci neoblomný i tentokrát a lid nepropustil.
}

\subsection*{Reflexe}
Nyní jsme dospěli ke skutečného smyslu a účelu celé ságy zaznamenané v knize Exodus. Bůh po
faraonovi požaduje: „Propusť můj lid, aby mi mohl sloužit.“ Mojžíš má v úmyslu, na základě Božího
příkazu, vzít lid na třídenní cestu do pouště, kde přinesou oběti, které jsoupro Egypťany „ohavností“.
Jinými slovy vezmou zvířata, která Egypťané uctívají, a povraždí je. Tato oběť má Izraelcům dokázat, že
bohové Egypťanů jsou pouhými tvory, a ne jediným pravým Bohem.

Možná, že my, moderní muži, jsme příliš vzdělaní na to, abychom uctívali ovce a dobytek. V naší
aroganci a kultivovanosti však stále uctíváme falešné bohy či modly. Nejčastěji jsou to modly ve formě
peněz, sexu, moci, sportu a zábavy. Pokud čteme Písmo důkladně, musíme vidět, že Bůh se nejvíce stará
o Izraelity a jejich svobodu. Jeho hlavním zájmem není svoboda od otrokářů, kteří nařizují, co mají celý
den dělat. Jeho zájmem je spíše svoboda jejich duší. Zotročení si však přece nezaslouží zatracení. Ale
svobodné rozhodnutí uctívat modly namísto Boha? To už je vážné. Po 400 letech v Egyptě Izraelité
uctívají egyptské modly. Toto uctívání zotročilo jejich duše a bránilo jim od správného uctívání jediného
pravého Boha.

Povšimněte si také, že i když se faraon unavuje a dává jim povolení jít do divočiny, nechce, aby Izraelité
šli „moc daleko“. Nedovolí, aby se mu jeho pracovní síla vymkla z rukou. Musí zůstat zotročení.
Přemýšlejte o tom, kdy se vzdáváte věcí, které vás zotročují. Satan vám šeptá: „Běžte a dejte si dovolenou
od vašeho otroctví třeba na týden, po dobu postní, nebo dokonce na devadesát dní. Když se pak vrátíte
zpět, otroctví bude pro vás ještě horší.“ Svobodu nelze vyhrát a navždy ochránit ve stanoveném čase.

Těchto devadesát dní má sloužit jako skvělý začátek, čas očištění, a jako připomínka. Potrvá vám však
celý život věrnosti a spoléhání se na Boha, abyste zůstali svobodným člověkem.
Tato skutečnost by vás neměla vést ke smutku nebo zoufalství. Měla by vám přinést větší horlivost hledat
celoživotní svobodu. Pokud jste odrazeni, přineste to Pánu a dejte mu prostor, aby k vám mluvil pravdu.

Pokud jste plní nadšení, chvalte Boha za tento dar. Obraťte se na své bratry, zejména na vaši kotvu. Ne
všichni muži budou tak nadšení a horliví jako vy. Někteří mohou být dokonce v pokušení přestat. Podělte
se s nimi o svou radost a zápal.

%newday
\newpage
\section{Den 21 - ODDĚLENI PRO BOHA}
\zacatekTretiTyden
\subsection*{Čtení na den}
\textbf{Exodus 9,1-7}
\newline
\textit{
\textsuperscript{1}Hospodin řekl Mojžíšovi: „Předstup před faraóna a promluv k němu: Toto praví Hospodin, Bůh Hebrejů: Propusť můj lid, aby mi sloužil!
\textsuperscript{2}Budeš-li se zdráhat jej propustit a zatvrdíš-li se proti nim ještě víc,
\textsuperscript{3}tu na tvá stáda, která jsou na poli, na koně, na osly, na velbloudy, na skot i na brav, dolehne Hospodinova ruka velmi těžkým morem.
\textsuperscript{4}Hospodin však bude podivuhodně rozlišovat mezi stády izraelskými a stády egyptskými, takže nezajde nic z toho, co patří Izraelcům.
\textsuperscript{5}Hospodin také určil lhůtu: Zítra toto učiní Hospodin v celé zemi.“
\textsuperscript{6}A nazítří to Hospodin učinil. Všechna egyptská stáda pošla, ale z izraelských stád nepošel jediný kus. 
\textsuperscript{7}Farao si to dal zjistit, a vskutku z izraelských stád nepošel jediný kus ; přesto zůstalo srdce faraónovo neoblomné a lid nepropustil.
}

\subsection*{Reflexe}

Poselství poslané faraonovi je naprosto jasné: Bůh chce oddělit Izraelity od egyptského království. Izraelci nepatří
faraonovi, ale pouze Bohu. Bůh důrazně varuje faraóna, že bude„rozlišovat mezi stády izraelskými a stády
egyptskými, takže nezajde nic z toho, co patří Izraelcům“. Otec chrání své syny.

Stejně jako Izraelci byli odděleni, tak i vy jste byli odděleni svým křtem. Je snadné zapomenout na sílu a účinky
křtu. Když jste byly křtem ponořeni do Krista, buď jako dítě nebo jako dospělý, stali jste se syny nebeského Otce.
Stali jste se posvátnými a dostali jste všechno, co potřebujete k účasti na Božském životě.

Zkoumejte svůj život. Uvědomujete si, co to znamená, že je váš život posvátný? Kalich, který užívá kněz v
posvátné liturgii, nemůže být nikdy použit pro světskou činnost. Je posvátný a musí se s ním tak zacházet. Stejně
tak vás křest oddělil; zůstáváte uprostřed lidí, ale jste odnynějška odděleni pro službu Pánu. Váš úděl je být s
Bohem. Proto musíte odolat pokušení honbě za pozemskými věcmi, zvláště když vylučují Boží plán pro váš život.

Čím více se vyrovnáte se svou pravou identitou, tím více si uvědomujete iracionalitu honby za pozemskými věcmi,
jako je sex, moc, peníze. Proto musíme v našich srdcích často rozdmýchávat milost našeho křtu. Musíme klást
nárok na naše křestní právo a naše věčné dědictví. Musíme zůstat oddaní Bohu / Musíme zůstat
vyčleněni/odděleni/posvěceni pro Boha

Pohovořte s Bohem dnes o vaší podvátnosti a vašem oddání Jemu. Zamyslete se nad tím, jak žijete „odděleni pro
Boha“ a jak v tom selháváte. Buďte konkrétní, pozorně naslouchejte a buďte ochotni a připraveni se změnit, jak
Pán žádá. Bude to znamenat další pevný krok na cestě ke svobodě.

% ===============================================
% ===== CTVRTY TYDEN
% ===============================================
%ukony
\newpage
\section*{Úkony (ukazatel cesty) pro 4. týden}

\textbf{Místo:} Egypt (jste v Egyptě)

Egyptské rány se zhoršují. Hospodin nyní používá ty, kteří mu odporují, pro jeho vlastní užitek. Také
Izraelitům dokazuje prázdnotu egyptských model tím, že tyto modly zničí ranami. Izraelité konečně zří
pravou cenu opuštění Egypta a služby pravému Bohu. Jak putujete čtvrtým týdnem exodu, uvědomte si,
jaké falešné bohy ničí Bůh ve vašem životě skrze disciplíny tohoto duchovního cvičení. Dovolte této
skutečnosti, aby vám přinesla naději na svobodu a obrácení srdce, které potřebujete, abyste mohli nechat
Egypt za sebou a sloužit pouze Bohu.

\subsection*{1. Zavázejte se (odevzdejte se) svému bratrstvu}
Kolik bratrských setkání jste zameškali? Jste v pokušení přeskočit budoucí bratrskou svatou hodinu, mši nebo setkání? Jestli vás satan dokáže přesvědčit, že své bratry nepotřebujete, také ví, že vaše zotročení bude stejně lehké jako před exodem. Zavázejte se svému bratrstvu.
\subsection*{2. Vytvořte si cvičící plán}
Exodus 90 volá po pravidelném cvičení. Jak jste si vedli? Minulí muži Exodu zjistili, že pokud se nemohou držet vytvořeného plánu cvičení, je velmi těžké udělat si každý týden čas na cvičení. Vytvořte si plán, sepište si jej a dodržujte ho.
\subsection*{3. Uvědomte si Boží moc}
Jak se tento týden rány snáší na Egypt, uvědomte si Boží všemohoucnost. On má moc činit velké věci, včetně vysvobození z vašeho zajetí. Pamatujte: Jeho moc nezná hranic. Spoléhejte na Boha, On jediný vás dokáže vykoupit.
\subsection*{4. Jděte/vyjděte ven}
Ve světe plném obrazovek a digitálních vztahů lidé zapomněli, jak být spolu. Něktěří dokonce zapomněli, jak se říká kusu země pokryté stromy. Udělejte si tento nebo příští víkend čas na to, abyste společně vyrazili na výlet do lesa, údolí nebo pustiny. Žádné výmluvy: čím horší počasí, tím lepší pouto. Tento společný čas vám dá život a pravé pouto, které budete vy a vaše bratrstvo potřebovat pro mnoho dalších týdnů před vámi.

\subsection*{Modlitba}
Modlete se, aby Pán osvobodil vás a vaše bratrství \newline
Modleme se za svobodu všech mužů v exodu, stejně tak, jako se oni modlí za vás.\newline
Ve jménu Otce i Syna i Ducha svatého … Otče náš… Ve jménu Otce i Syna i Ducha svatého … Amen.

\newpage


%newday
\newpage
\section{Den 22 - ROZTÁT PŘED HOSPODINEM}
\zacatekCtvrtyTyden
\subsection*{Čtení na den}
\textbf{Exodus 9,8-12}
\newline
\textit{
\textsuperscript{8}Hospodin řekl Mojžíšovi a Áronovi: „Naberte si plné hrsti sazí z pece a Mojžíš ať je rozhazuje faraónovi před očima směrem k nebi.
\textsuperscript{9}Bude z nich po celé egyptské zemi poprašek, který způsobí na lidech i na dobytku po celé egyptské zemi vředy hnisavých neštovic.“
\textsuperscript{10}Nabrali tedy saze z pece, postavili se před faraóna a Mojžíš je rozhazoval směrem k nebi. Na lidech i na dobytku se objevily vředy hnisavých neštovic.
\textsuperscript{11}Ani věštci se nemohli postavit před Mojžíše pro vředy, neboť vředy byly na věštcích i na všech Egypťanech.
\textsuperscript{12}Hospodin však zatvrdil faraónovo srdce, takže je neposlechl, jak Hospodin Mojžíšovi předpověděl.
}

\subsection*{Reflexe}

Dnešní čtení představuje malý ale pozoruhodný obrat v začátku druhé řady pěti ran proti faraonovi a lidu Egypta.
Důsledky těchto ran se stanou mnohem intenzivními, kouzelníci si totiž uvědomí Boží moc a srdce faraona je
zatvrzeno uvědoměním si Boží všemohoucnosti.

Po každé z první řady ran zatvrdil faraon své srdce. V druhé řadě pěti ran je faraonovo srdce zatvrzeno Bohem. Bůh
faraonovi nedělá nic zlého, ani nepřipravuje faraonovi neúspěch. Abychom tomu porozuměli, představme si máslo
a hlínu na horkém slunci. Každá látka reaguje svým způsobem na to stejné slunce – máslo roztává a hlína tvrdne.
Slunce nedělá rozdíl nezi těmito dvěma látkami. Rozdíl je spíše v povaze každé látky, která způsobuje různou
reakci.

V knize Exodu Izraelité reagují na Boží skutky jako máslo: časem Jeho přítomnost rozpouští jejich srdce. V odezvě
se rozhodnou ctít Boží velikost. Na druhé straně se faraon dívá na stejné Boží skutky a reaguje jako hlína: Boží
přítomnost zatvrzuje jeho srdce. Faraon odpovídá pyšně a zvyšuje odpor vůči Boží vůli. Dělá to proto, že pokud je
Bůh opravdu tím, kým říká, pak je faraon podvodník. Jestli je Bůh Bohem, pak jím faraon není. Faraon může celou
svou identitu, a všechno s ní spojené, v tomto mocenském boji ztratit.

Když vstupujete každý den do své svaté hodiny, jak reagujete na Boží přítomnost? Je vaše srdce jako máslo, nebo
jako hlína? Jedním ze způsobů, jak to vyzkoušet, je zvážit vaši schopnost zůstat tichý v mysli i těle. Dokážete sedět
před všemohoucím Bohem a jen být v klidu? Nebo vás Jeho přítomnost znervózňuje?

Měli byste toužit slyšet vůli Pána a konat podle ní. Cítíte-li se ve stresu nebo v úzkosti, je to pravděpodobně proto,
že bojujete s Bohem o moc. Možná zatvrzujete své srdce v dutém pokusu zachovat si svou identitu, která je menší
než ta, ke které vás Bůh volá, být člověkem pro Církev a pro vaši rodinu. Ve vaší dnešní svaté hodině si dnes
uvědomte, jak odpovídáte na přítomnost Boha. Potom si s Ním promluvte o tom, jak a proč tak odpovídáte.

%newday
\newpage
\section{Den 23 - NECHTE SE OBRÁTIT}
\zacatekCtvrtyTyden
\subsection*{Čtení na den}
\textbf{Exodus 9,13-35}
\newline
\textit{
\textsuperscript{13}Hospodin řekl Mojžíšovi: „Za časného jitra se postav před faraóna. Řekneš mu: Toto praví Hospodin, Bůh Hebrejů: Propusť můj lid, aby mi sloužil!
\textsuperscript{14}Tentokrát zasáhnu do srdce všemi svými údery tebe i tvé služebníky a tvůj lid, abys poznal, že na celé zemi není nikdo jako já.
\textsuperscript{15}Vždyť už tehdy, když jsem vztáhl ruku, abych bil tebe i tvůj lid morem, mohl jsi být vyhlazen ze země.
\textsuperscript{16}Avšak proto jsem tě zachoval, abych na tobě ukázal svou moc a aby se po celé zemi vypravovalo o mém jménu.
\textsuperscript{17}Stále jednáš proti mému lidu zpupně a nechceš jej propustit.
\textsuperscript{18}Proto spustím zítra v tuto dobu tak hrozné krupobití, jaké v Egyptě nebylo ode dne jeho vzniku až do nynějška.
\textsuperscript{19}Nuže, dej odvést do bezpečí svá stáda a všechno, co máš na poli. Všechny lidi i dobytek, vše, co bude zastiženo na poli a nebude shromážděno do domu, potluče krupobití, takže zemřou.“
\textsuperscript{20}Kdo z faraónových služebníků se Hospodinova slova ulekl, zahnal své otroky a svá stáda do domů.
\textsuperscript{21}Kdo si slovo Hospodinovo nevzal k srdci, nechal své otroky a svá stáda na poli.
\textsuperscript{22}Hospodin řekl Mojžíšovi: „Vztáhni svou ruku k nebi. Na celou egyptskou zemi dolehne krupobití, na lidi, na dobytek i na všecky polní byliny v egyptské zemi.“
\textsuperscript{23}Když Mojžíš vztáhl svou hůl k nebi, dopustil Hospodin hromobití a krupobití. Na zemi padal oheň. Tak Hospodin spustil krupobití na egyptskou zemi.
\textsuperscript{24}Nastalo krupobití a uprostřed krupobití šlehal oheň; něco tak hrozného nebylo v celé zemi egyptské od dob, kdy se dostala do moci tohoto pronároda.
\textsuperscript{25}Krupobití potlouklo v celé egyptské zemi všechno, co bylo na poli, od lidí po dobytek; krupobití potlouklo také všechny polní byliny a polámalo všechno polní stromoví.
\textsuperscript{26}Jenom v zemi Gošenu, kde sídlili Izraelci, krupobití nebylo.
\textsuperscript{27}Tu si farao dal předvolat Mojžíše a Árona a řekl jim: „Opět jsem zhřešil. Hospodin je spravedlivý, a já i můj lid jsme svévolníci.
\textsuperscript{28}Proste Hospodina. Božího hromobití a krupobití je už dost. Propustím vás, nemusíte tu už dál zůstat.“
\textsuperscript{29}Mojžíš mu odvětil: „Jen co vyjdu z města, rozprostřu své dlaně k Hospodinu. Hromobití přestane a krupobití skončí, abys poznal, že země je Hospodinova.
\textsuperscript{30}Vím ovšem, že ty ani tvoji služebníci se stále ještě nebudete Hospodina Boha bát.“
\textsuperscript{31}Potlučen byl len a ječmen, protože ječmen byl už v klasech a len nasazoval tobolky.
\textsuperscript{32}Pšenice a špalda však potlučeny nebyly, protože jsou pozdní.
\textsuperscript{33}Mojžíš vyšel od faraóna z města a rozprostřel dlaně k Hospodinu. Hromobití a krupobití přestalo a déšť už nezaplavoval zemi.
\textsuperscript{34}Když farao viděl, že přestal déšť a krupobití i hromobití, hřešil dále. Zůstal v srdci neoblomný, on i jeho služebníci.
\textsuperscript{35}Srdce faraónovo se zatvrdilo a Izraelce nepropustil, jak Hospodin skrze Mojžíše předpověděl.
}

\subsection*{Reflexe}
Sedmou ranou krupobití ukončí dech každé bytosti, kterou potká. V Písmu vidíme, že občas Bůh jedná s lidstvem nežně a
jindy přísně. V obou případech to však je pro lidské dobro a posvěcení.
Běžná karikatura dnešní Církve je krutá a neúprosná instituce, jejíž přimární přínos lidstvu je ochromující vina a hanba.

Moderní televizní programy znázorňují spíše temný obraz katolické víry: pochmurné, zatuchlé kostely, mizerná hudba a
vysokké dřevěné krabice, ve kterých posloucháme odsuzující přednášky ze stran chladného, bezcitného celibátu s hmatatelným
pohrdáním lidstvem. Přemýšlíte někdy nad tím, jak lidé přišli na tuto myšlenku?

Možná pramení z neutuchajícího volání Církve po obrácení. Konverze může být vnímána jako násilná. Když jste ve stavu
deprese, hněvu, úzkosti, osamělosti nebo vyčerpání, volba zbavit se pohodlí může být pociťována jako násiilné vnitřní
krupobití. Může se to zdát kruté. Ale tato malá úmrtí sama sebe ukončují život těch vášní a podnětů, které brání našemu
osvobození.

Ve vaší konverzi zůstaňte blízko Pánu. Nechte své tělo i duši očistit asketickými disciplínami. Jakkoli se můžou občas zdát
kruté, očistí vás a zanechají váš život. Pokud s askezí bojujete, povězte o tom dnes Pánu. Také může posloužit přednést to
svému bratrstvu a vašemu duchovnímu vůdci. Ale nejprve to předložte Bohu.

%newday
\newpage
\section{Den 24 - ODDANÍ KONTROLE}
\zacatekCtvrtyTyden
\subsection*{Čtení na den}
\textbf{Exodus 10,1-20}
\newline
\textit{
\textsuperscript{1}Hospodin řekl Mojžíšovi: „Předstup před faraóna. Já jsem totiž učinil jeho srdce i srdce jeho služebníků neoblomné, abych mohl uprostřed nich provést tato svá znamení
\textsuperscript{2}a ty abys mohl vypravovat svým synům i vnukům o tom, co jsem v Egyptě dokázal, i o znameních, která jsem mezi nimi udělal, ať víte, že já jsem Hospodin.“
\textsuperscript{3}Mojžíš a Áron tedy předstoupili před faraóna a řekli mu: „Toto praví Hospodin, Bůh Hebrejů: Jak dlouho se budeš zdráhat pokořit se přede mnou? Propusť můj lid, aby mi sloužil.
\textsuperscript{4}Budeš-li se zdráhat propustit můj lid, pak na tvé území uvedu zítra kobylky.
\textsuperscript{5}Přikryjí povrch země, takže nebude možno zemi ani vidět, a sežerou zbytek toho, co vyvázlo, co vám zůstalo po krupobití. Ožerou také všechny stromy, které vám na polích znovu raší.
\textsuperscript{6}Naplní tvé domy, domy všech tvých služebníků i domy všech Egypťanů. Něco takového neviděli tvoji otcové ani dědové od doby, kdy začali obdělávat půdu, až dodnes.“ Nato se Mojžíš obrátil a odešel od faraóna.
\textsuperscript{7}Faraónovi služebníci řekli: „Jak dlouho nám bude tento člověk léčkou? Propusť ty muže, ať slouží Hospodinu, svému Bohu. Což jsi dosud nepoznal, že hrozí Egyptu zánik?“
\textsuperscript{8}Mojžíš a Áron byli přivedeni zpět k faraónovi. Ten jim řekl: „Nuže, služte Hospodinu, svému Bohu. Kdo všechno má jít?“
\textsuperscript{9}Mojžíš odvětil: „Půjdeme se svou mládeží i se starci, půjdeme se svými syny i dcerami, se svým bravem i skotem, neboť máme slavnost Hospodinovu.“
\textsuperscript{10}Farao jim však řekl: „To tak! Myslíte si, že Hospodin bude s vámi, když vás propustím s dětmi? To jste si zamanuli špatnou věc.
\textsuperscript{11}Kdepak! Vy muži si jděte a služte Hospodinu, když o to tak stojíte.“ A vyhnali je od faraóna.
\textsuperscript{12}Hospodin řekl Mojžíšovi: „Vztáhni nad egyptskou zemi ruku, aby přilétly na egyptskou zemi kobylky a sežraly všechny byliny země, všechno, co zůstalo po krupobití.“
\textsuperscript{13}Mojžíš tedy vztáhl nad egyptskou zemi hůl a Hospodin přihnal na zemi východní vítr. Ten vál po celý den a celou noc. Když nastalo jitro, přinesl východní vítr kobylky.
\textsuperscript{14}Kobylky přilétly na celou egyptskou zemi a spustily se na celé území Egypta v takovém množství, že tolik kobylek nebylo nikdy předtím ani potom.
\textsuperscript{15}Přikryly povrch celé země, až se na zemi zatmělo, a sežraly všechny byliny na zemi i všechno ovoce na stromech, co zbylo po krupobití. Na stromech a na polních bylinách po celé egyptské zemi nezbylo nic zeleného.
\textsuperscript{16}Farao rychle povolal Mojžíše a Árona. Řekl jim: „Zhřešil jsem proti Hospodinu, vašemu Bohu, i proti vám.
\textsuperscript{17}Sejmi prosím můj hřích ještě tentokrát a proste Hospodina, svého Boha, aby jen odvrátil ode mne tuto smrt.“
\textsuperscript{18}Mojžíš od faraóna odešel a prosil Hospodina.
\textsuperscript{19}Tu Hospodin obrátil vítr a velmi silný mořský vítr odnesl kobylky a prudce je vrhl do Rákosového moře, takže na celém egyptském území nezůstala jediná kobylka.
\textsuperscript{20}Avšak Hospodin zatvrdil faraónovo srdce, takže Izraelce nepropustil.
}

\subsection*{Reflexe}

Lidé se často ptají: „Proč by toto soužení Bůh dopustil mému příteli nebo mému příbuznému?“ Abychom odpověděli, nejprve
musíme zvážit fakt, že samotný důvod naší existence je být s Bohem a v Něm. Mimo Boha nejsme nic. Přesto se velmi často
proti Bohu bouříme. Proč se dobrým lidem stávají zlé věci? Aby se stali lepšími. Mají je povzbudit k návratu k Bohu.

V dnešním čtení se faraon již nechová moc manipulativně. Uvědomuje si, že už skoro všechno ztratil. Proto se zoufale drží
toho, co má, a snaží se udržet si své postavení, aby neztratil celé své království.

Muži hrozně bojují, aby si udrželi kontrolu. Masturbace a materialismus jsou příklady chlapeckých reakcí na stres, spoty,
zmatek, neúspěch a sklíčenost. V jakékoli z těchto situací se chlapec obrátí k nečemu, k čemukoli, co mu dá pocit kontroly – i
přesto, že ví, že je to lež. Abychom se vyhnuli stejnému pokušení jako muži, je důležité mít vždy na paměti, že tento svět nás
má připravit pro ten další.

Jako lidé dnes musíme být tím, čím mají být i Izraelité – poutníci, muži Exodu. To znamená život oddělený od věcí, pohodlí a
bezpečí. Znamená to držet se pouze Boha a věrně se k Němu přibližovat. Na co se upínáte? Co musíte opustit? Předneste tyto
věci před Pána a proste Ho, aby vám v této hodině udělil svobodu.


%newday
\newpage
\section{Den 25 - PODDAT SE NEBO VZEPŘÍT}
\zacatekCtvrtyTyden
\subsection*{Čtení na den}
\textbf{Exodus 10,21-29}
\newline
\textit{
\textsuperscript{21}Hospodin řekl Mojžíšovi: „Vztáhni svou ruku k nebi a egyptskou zemi zahalí temnota, taková temnota, že se dá nahmatat.“
\textsuperscript{22}Mojžíš vztáhl ruku k nebi. Tu nastala po celé egyptské zemi tma tmoucí a trvala po tři dny.
\textsuperscript{23}Lidé neviděli jeden druhého; po tři dny se nikdo neodvážil hnout ze svého místa. Ale všichni Izraelci měli ve svých obydlích světlo.
\textsuperscript{24}Farao povolal Mojžíše a řekl: „Odejděte! Služte Hospodinu! Zanechte tu jenom svůj brav a skot. Také vaše děti mohou jít s vámi.“
\textsuperscript{25}Mojžíš odpověděl: „Ty sám nám dáš potřebné k obětním hodům a k zápalným obětem, abychom je připravili Hospodinu, svému Bohu.
\textsuperscript{26}Půjdou s námi i naše stáda, ani pazneht tu nezůstane. Budeme z nich brát k službě Hospodinu, svému Bohu. My ještě nevíme, čím budeme Hospodinu sloužit, dokud tam nepřijdeme.“
\textsuperscript{27}Avšak Hospodin zatvrdil faraónovo srdce a on je nedovolil propustit.
\textsuperscript{28}Farao řekl: „Odejdi ode mne. Dej si pozor, ať mi už nepřijdeš na oči. Neboť v den, kdy mi přijdeš na oči, zemřeš!“
\textsuperscript{29}Mojžíš odpověděl: „Jak jsi řekl. Už ti na oči nepřijdu.“
}

\subsection*{Reflexe}
Akčoli faraonova frustrace roste, jeho srdce je stále zatvrzelé. Stále se vzpírá, i když si je vědom, že
pouze úplná poslušnost Božímu plánu je jedinou cestou k naplnění. Faraon má zjevně plán pro sebe a své
lidi, ale tento plán je jasně v rozporu s Boží vůlí. Výsledkem je to, že se on i jeho lidé potácí ve tmě.

Vezměme na vědomí, že člověk je stvořená bytost a jako taková nepatří sobě. Pouze když se vyrovnáme
s tím, že patříme Bohu samému, což vyžaduje značnou dávku pokory, můžeme jasněji vidět Boží
prozřetelnost.

Podívejme se na izraelský lid. S pokorou a poslušností Pánovu plánu se ocitnou obdarováni světlem.
Podívejte se na svůj život. Žijete ve světle nebo ve tmě? Stáváte se poslušnější k Božímu plánu pro vás
skrze dar disciplín, nebo se proti němu bouříte? Promluvte si o tom s Pánem. Proč vás Jeho dobrý plán
nutí bouřit se, malými nebo velkými způsoby? Požádejte Ho, aby vám ukázal, v čem jsou disciplíny
dobré, zejména v těch, které nejvíc nenávidíte. Požádejte Ho, aby vám ukázal, co dělá nyní
prostřednictvím těchto disciplín ve vašem životě.



%newday
\newpage
\section{Den 26 - DVEŘE SE ZAČÍNAJÍ OTVÍRAT}
\zacatekCtvrtyTyden
\subsection*{Čtení na den}
\textbf{Exodus 11,1-10}
\newline
\textit{
\textsuperscript{1}Hospodin řekl Mojžíšovi: „Ještě jednu ránu uvedu na faraóna a na Egypt. Potom vás odtud propustí, nadobro vyhostí, přímo vás odtud vyžene.
\textsuperscript{2}Vybídni lid, ať si vyžádá každý muž od svého souseda a každá žena od své sousedky stříbrné a zlaté šperky.“
\textsuperscript{3}A Hospodin zjednal lidu v očích Egypťanů přízeň. Také sám Mojžíš platil v egyptské zemi za velice významného v očích faraónových služebníků i v očích lidu.
\textsuperscript{4}Mojžíš řekl faraónovi : „Toto praví Hospodin: O půlnoci projdu Egyptem.
\textsuperscript{5}Všichni prvorození v egyptské zemi zemřou, od prvorozeného syna faraónova, který sedí na jeho trůnu, po prvorozeného syna otrokyně, která mele na mlýnku, i všechno prvorozené z dobytka.
\textsuperscript{6}Po celé egyptské zemi se bude rozléhat veliký křik, jakého nebylo a už nebude.
\textsuperscript{7}Ale na žádného Izraelce ani pes nezavrčí, ani na člověka ani na dobytče, abyste poznali, že Hospodin podivuhodně rozlišuje mezi Egyptem a Izraelem.
\textsuperscript{8}Všichni tito tvoji služebníci sestoupí ke mně, budou se mi klanět a říkat: Odejdi ty i všechen lid, který jde za tebou! Teprve potom odejdu.“ Nato Mojžíš, planoucí hněvem, od faraóna odešel.
\textsuperscript{9}Hospodin řekl Mojžíšovi: „Farao vás neposlechne, a tak mých zázraků v egyptské zemi ještě přibude.“
\textsuperscript{10}Mojžíš a Áron všechny ty zázraky před faraónem učinili, ale Hospodin zatvrdil faraónovo srdce, takže Izraelce ze své země nepropustil.  
}

\subsection*{Reflexe}
Izraelité se připravují. Bůh slíbil poslední ránu. Slíbil svobodu – ale neříká, že svoboda přijde bez
mocného boje nebo bez důvěry v Něj. Představte si, že po tolika dlouhých letech v otroctví v Egyptě se
dveře začnou otvírat. Svoboda už není pouhým snem, ale skutečnou možností.

Vidíte pro sebe stejnou možnost svobody? Teď se možná ptáte, proč jste se rozhodli pro tak radikální
duchovní cvičení. Někteří vaši bratři už možná přestali. Ale pro vás, kteří jste došli až sem, se začínají
otvírat dveře. Pokud dokážete Bohu důvěřovat hlouběji, více se spoléhat na své bratry a vytrváte, dávajíce
Bohu čas pracovat ve vašem životě, dveře se vám budou otevírat i nadále. Připravte se a proste dnes Pána
o milost vytrvalosti.


%newday
\newpage
\section{Den 27 - ŽÍT V PŘÍTOMNOSTI}
\zacatekCtvrtyTyden
\subsection*{Čtení na den}
\textbf{Exodus 12,1-20}
\newline
\textit{
\textsuperscript{1}Hospodin řekl Mojžíšovi a Áronovi v egyptské zemi:
\textsuperscript{2}„Tento měsíc bude pro vás začátkem měsíců. Bude pro vás prvním měsícem v roce.
\textsuperscript{3}Vyhlaste celé izraelské pospolitosti: Desátého dne tohoto měsíce si každý vezmete beránka podle svých rodů, beránka na rodinu.
\textsuperscript{4}Kdyby byla rodina malá a na beránka by nestačila, přibere si každý souseda, který bydlí nejblíže jeho rodiny, aby doplnil počet osob. Podle toho, kolik kdo sní, stanovíte počet na beránka.
\textsuperscript{5}Budete mít beránka bez vady, ročního samce. Vezmete jej z ovcí nebo z koz.
\textsuperscript{6}Budete jej opatrovat až do čtrnáctého dne tohoto měsíce. Navečer bude celé shromáždění izraelské pospolitosti beránky zabíjet.
\textsuperscript{7}Pak vezmou trochu krve a potřou jí obě veřeje i nadpraží u domů, v nichž jej budou jíst.
\textsuperscript{8}Tu noc budou jíst maso upečené na ohni a k němu budou jíst nekvašené chleby s hořkými bylinami.
\textsuperscript{9}Nebudete z něho jíst nic syrového ani vařeného ve vodě, nýbrž jen upečené na ohni s hlavou i s nohama a vnitřnostmi.
\textsuperscript{10}Nic z něho nenecháte do rána. Co z něho zůstane do rána, spálíte ohněm.
\textsuperscript{11}Budete jej jíst takto: Budete mít přepásaná bedra, opánky na nohou a hůl v ruce. Sníte jej ve chvatu. To bude Hospodinův hod beránka.
\textsuperscript{12}Tu noc projdu egyptskou zemí a všecko prvorozené v egyptské zemi pobiji, od lidí až po dobytek. Všechna egyptská božstva postihnu svými soudy. Já jsem Hospodin.
\textsuperscript{13}Na domech, v nichž budete, budete mít na znamení krev. Když tu krev uvidím, pominu vás a nedolehne na vás zhoubný úder, až budu bít egyptskou zemi.
\textsuperscript{14}Ten den vám bude dnem pamětním, budete jej slavit jako slavnost Hospodinovu. Budete jej slavit po všechna svá pokolení. To je provždy platné nařízení.
\textsuperscript{15}Po sedm dní budete jíst nekvašené chleby. Hned prvního dne odstraníte ze svých domů kvas. Každý, kdo by od prvního do sedmého dne jedl něco kvašeného, bude z Izraele vyobcován.
\textsuperscript{16}Prvního dne budete mít bohoslužebné shromáždění. I sedmého dne budete mít bohoslužebné shromáždění. V těch dnech se nebude konat žádné dílo. Smíte si připravit jen to, co každý potřebuje k jídlu.
\textsuperscript{17}Budete dbát na ustanovení o nekvašených chlebech, neboť právě toho dne jsem vyvedl vaše oddíly z egyptské země. Na tento den budete bedlivě dbát. To je provždy platné nařízení pro všechna vaše pokolení.
\textsuperscript{18}Od večera čtrnáctého dne prvního měsíce budete jíst nekvašené chleby až do večera jedenadvacátého dne téhož měsíce.
\textsuperscript{19}Po sedm dní se nenajde ve vašich domech kvas. Každý, kdo by jedl něco kvašeného, bude vyobcován z pospolitosti Izraele, i host a domorodec.
\textsuperscript{20}Nebudete jíst nic kvašeného. Ve všech svých obydlích budete jíst nekvašené chleby.“
}

\subsection*{Reflexe}
Je jednoduché nenávidět čas. Většina z nás neustále vyhlíží „další nejlepší věc“, která nám, jak věříme, konečně zajistí
blaženost, po které toužíme. V naší netrpělivosti máme sklon nenávidět přítomnost a toužit po lepších zítřkách. Přesto znamená
tento způsob života samotný život nenávidět. Jako křesťané jsme povoláni k životu v přítomnosti, i když je přítomnost plná
dřiny, zármutku nebo bolesti.

Dnes se Izraelité připravují na příchod úsvitu. Zároveň jsou přítomni noci, protože noc vyžaduje pozornost k detailu. Pokud
přehlédnou jednu část oběti před nimi, mohl by pro ně být úsvit mnohem tmavší, než by jinak byl.

Právě teď, ve vaší odpovědi na Boží volání po svobodě, také očekáváte úsvit. Těchto 90 dní oběti jsou cenným časem. Může
být jednoduché považovat tyto dny za obtíž nebo nesmyslnou zkoušku vůle. Možná nedočkavě vyhlížíte den, kdy tento
nesmysl přestane, a budete si moct užít života ve svobodě. Ne. Je čas zaměřit se na současné oběti před vámi. Vaše rodina
prosí, abyste byli změněni časem v modlitbě. Církev touží po tom, abyste vystoupili a byli pro její lid přínosem díky praxi
askeze. Vaši bratři na vás spoléhají, že je pozdvihnete, stejně jako oni pozdvihnou vás. Přijměte to. Žijte v přítomnosti.

Věnovali jste náležitou pozornost detailům duchovního cvičení, k němuž vás Pán povolal? Pokud ano, přijměte v této chvíli
ujištění o Boží lásce. Pokud ne, otevřete srdce milosti, kterou vám Bůh chce dát. Chce vás přivést k úsvitu. Volba spolupráce
s tímto plánem je na vás.

%newday
\newpage
\section{Den 28 - KREV BERÁNKA}
\zacatekCtvrtyTyden
\subsection*{Čtení na den}
\textbf{Exodus 12,21-28}
\newline
\textit{
\textsuperscript{21}Mojžíš svolal všechny izraelské starší a řekl jim: „Jděte si vzít kus z bravu podle vašich čeledí a zabijte velikonočního beránka.
\textsuperscript{22}Potom vezměte svazek yzopu, namočte jej v misce s krví a krví z misky potřete nadpraží a obě veřeje. Ať nikdo z vás až do rána nevychází ze dveří svého domu.
\textsuperscript{23}Až Hospodin bude procházet zemí, aby udeřil na Egypt, uvidí krev na nadpraží a na obou veřejích. Hospodin ty dveře pomine a nedopustí, aby do vašeho domu vešel zhoubce a udeřil na vás. 
\textsuperscript{24}Dbejte na toto ustanovení. To je provždy platné nařízení pro tebe i pro tvé syny.
\textsuperscript{25}Až přijdete do země, kterou vám Hospodin dá, jak přislíbil, dbejte na tuto službu.
\textsuperscript{26}Až se vás pak vaši synové budou ptát, co pro vás tato služba znamená,
\textsuperscript{27}odpovíte: ‚Je to velikonoční obětní hod Hospodinův. On v Egyptě pominul domy synů Izraele. Když udeřil na Egypt, naše domy vysvobodil.‘“ Lid padl na kolena a klaněl se.
\textsuperscript{28}Izraelci pak odešli a učinili přesně tak, jak Hospodin Mojžíšovi a Áronovi přikázal.
}

\subsection*{Reflexe}

Dnešní pasáž nás upozorňuje na moc Eucharistie. Svatý Jan Zlatoústý o této pasáži napsal svým přímým,
vyzývajícím způsobem:
\textit{Přejeme-li si porozumět moci Kristovy krve, měli bychom se vrátit ke starověké zprávě o její
předzvěsti v Egyptě. „Obětujte bezvadného beránka,“ přikazuje Mojžíš, „a jeho krví potřete
svá nadpraží a veřeje.“ Kdybychom se ho zeptali, co tím myslel a jak by krev nerozumného
zvířete mohla zachránit lid obdařený rozumem, jeho odpovědí by bylo, že zachraňující moc
leží ne v krvi samotné, ale ve znaku Pánovy krve. V těchto dnech, kdy anděl zkázy viděl krev
na dveřích, se neodvážil vstoupit. Jak se tedy přiblíží ďábel, když uvidí ne krev na veřejích, ale
pravou krev na rtech věřících, na dveřích Kristova chrámu.}

Samozřejmě má na mysli svaté přijímání, které přijímáme na rtech a do úst. Tento neuvěřitelný dar nám
poskytuje ochranu před mocí smrti a temnoty, ještě více než krev velikonočního beránka chránila děti
Izraele před andělem zkázy. Běžíte každodenně k přijímání Eucharistie? Byli jste alespoň věrní jedné mši
svaté navíc v týdnu? Krev Beránka není jen částí historické události, ale probíhající, současná realita.
Kristus nabídl svou oběť jednou pro všechny a tato oběť pokračuje každou mší svatou. Jaký to dar.

Věříte, že Tělo a Krev Kristova přítomná v Eucharistii má skutečně moc chránit vás před mocí smrti a
temnoty? Věříte, že Kristovo Tělo, Krev, Duše a Božství jsou plně přítomny v Eucharistii? Pokud ano,
chvalte Boha, že máte oči k vidění a uši k slyšení dobrých zpráv, které pro vás má. Pokud ne, dnes se
upřímně zeptejte Boha, jestli je tato dva tisíce let stará tradice opravdu pravdivá.


% ===============================================
% ===== PATY TYDEN
% ===============================================
%ukony
\newpage
\section*{Úkony (ukazatel cesty) pro 5. týden}

\textbf{Místo:} Východ z/od Egypta (východně od Egypta), útěk do pouště

Izraelité pocítili potřebu sloužit Bohu. Akčně reagovali zabitím egyptského boha (beránka) a veřejným rozmazáním jehněčí krve na jejich veřeje. Hospodin ctí odvahu, kterou prokázali, a otevírá jim bránu ke svobodě. Také vy jste prokázali odvahu a označili jste své dveře. Po měsíci odtažení od bohů a model tohoto světa jste se rozhodli zůstat s disciplínami Exodu 90. Také vaše odvaha bude vyznamenána a brána ke svobodě otevřena. Nyní vstupujete do pouště. Vyplatí se vám zůstat velmi blízko Bohu a vašemu bratrstvu. 

\subsection*{1. Vzdejte se kontroly}
Je dobré reflektovat, jaké disciplíny jste pokoušeni změnit a kde v tomto duchovním cvičení jste pokoušeni podvádět. Předneste tyto věci Bohu ve vaší dnešní svaté hodině. Přenechte kontrolu Jemu.
\subsection*{2. Zkontaktujte svou kotvu}
Už jste dnes spolu komunikovali? Byli jste v kontaktu tento týden? Pokud jste se svou kotvou dnes ještě nemluvili, teď je dobrý čas tak učinit. Počítá s vámi. Potřebuje vás. Pokud povolíte a on spadne, bude to bolestný pád, tak je to jednoduché.
\subsection*{3. Znovu si připomeňte své proč}
To je důležitý smysl Exodu. Lidé, které milujete (manželka, farnost, přátelé, děti), dychtivě touží po vašem osvobození. Stejně jako Izraelité nesmíte spouštět svůj zrak z Boha, když vstupujete do pouště k dosáhnutí svobody, o kterou usilujete. Pouze Bůh vám dokáže dát takové vysvobození.
\subsection*{4. Zvažte přečtení Průvodce terénem}
Pokud jste si stále nenašli čas na jeho přečtení, nyní si ho udělejte. Průvodce terénem rámcuje celou zkušenost Exodu 90 a pomáhá pochopit důvod každé části tohoto duchovního cvičení. (Nejdůležitější části jsou Začněte zde: co je vaše proč; Pilíře Exodu 90, a pro ženaté muže, Muž Exodu a jeho manželka.)

\subsection*{Modlitba}
Modlete se, aby Pán osvobodil vás a vaše bratrství \newline
Modleme se za svobodu všech mužů v exodu, stejně tak, jako se oni modlí za vás.\newline
Ve jménu Otce i Syna i Ducha svatého … Otče náš… Ve jménu Otce i Syna i Ducha svatého … Amen.
\newpage

%newday
\newpage
\section{Den 29 - POCHYBNOSTI O BOŽÍ DOBROTĚ}
\zacatekPatyTyden
\subsection*{Čtení na den}
\textbf{Exodus 12,29-30}
\newline
\textit{
\textsuperscript{29}Když nastala půlnoc, pobil Hospodin v egyptské zemi všechno prvorozené, od prvorozeného syna faraónova, který seděl na jeho trůnu, až po prvorozeného syna zajatce v žalářní kobce, i všechno prvorozené z dobytka.
\textsuperscript{30}Tu farao v noci vstal, i všichni jeho služebníci a všichni Egypťané, a v celém Egyptě nastal veliký křik, protože nebylo domu, kde by nebyl mrtvý.
}

\subsection*{Reflexe}

V reakci na tyto verše může být člověk na Boha rozzlobený. Může ho dokonce obviňovat za chaos a smrt,
kterou po celou dobu způlsobuje. Bůh je však Bůh. Nikdy si pro lidstvo nevybral hřích, utrpení a smrt.
V zahradě Edenu stvořil Bůh člověka v naprosté harmonii s Ním a se stvořeným světem. Člověk však,
obdarovaný svobodou, učinil volbu, která změnila téměř všechno a přinesla chaos do Božího řádu. Od té
doby Bůh pravuje na tom, aby přeměnil svět, který člověk rozházel.

Sám člověk se rozhodl jednat proti Božímu rozkazu, Bůh to dovolil z úcty k lidské svobodě. Kdyby člověk v
zahradě spolupracoval s Božím plánem, zažil by Boží lásku mnohem jinak. Podobně kdyby se faraon rozhodl
spolupracovat s Božím plánem v Egyptě, i on by mohl zažít Boží lásku úplně jinak.

Zeptejte se sami sebe: Spolupracuji s plánem, který pro mě má dnes Bůh? Zpochybňuji Boží rozhodnutí,
nebo věřím, že Jeho plán je plánem naprosté dobroty? Tyto otázky dnes stojí za to přednést ve svaté hodině.


%newday
\newpage
\section{Den 30 - USPOŘÁDAT SI SVŮJ ŽIVOT}
\zacatekPatyTyden
\subsection*{Čtení na den}
\textbf{Exodus 12,31-36}
\newline
\textit{
\textsuperscript{31}Ještě v noci povolal Mojžíše a Árona a řekl: „Seberte se a odejděte z mého lidu, vy i Izraelci. Jděte, služte Hospodinu, jak jste žádali.
\textsuperscript{32}Vezměte také svůj brav i skot, jak jste žádali, a jděte. Vyproste požehnání i pro mne.“
\textsuperscript{33}Egypťané naléhali na lid a spěchali s jeho propuštěním ze země, protože si říkali: „Všichni pomřeme!“
\textsuperscript{34}Lid tedy vzal těsto ještě nevykynuté, zabalil díže do plášťů a nesl na ramenou.
\textsuperscript{35}Izraelci jednali podle Mojžíšova rozkazu; vyžádali si též od Egypťanů stříbrné a zlaté šperky a pláště.
\textsuperscript{36}A Hospodin zjednal lidu přízeň v očích Egypťanů a oni jim vyhověli. Tak vyplenili Egypt.
}

\subsection*{Reflexe}

V souladu s tímto příběhem Exodu vyžaduje každoročně židovský svátek Pesachu užití nekvašeného chleba
v posvátném jídle. Izraelský lid měl velmi málo času na útěk z Egypta – tak málo, že nemohli ani čekat, než jim
vykyne chléb. Času bylo málo a bylo třeba jednat.

Přemýšlejte o tom v kontextu vašeho života. Poznáváte, že čas je vzácný? Víte, že je od vás požadována akce?
Pamatujete na svou smrtelnost nebo rozjímáte o dni, kdy zemřete? Den tohoto duševního probuzení nakonec jednou
přijde. Ten den vypadá pro každého člověka jinak. Může to být třeba zkušenost s autonehodou, infarkt, událost doma
nebo potyčka s násilím. Ať je to cokoli, způsobí to náhlé uvědomění, jak málo času zde na zemi máme.

Třeba se dostanete do zajetí úzkosti a strachu, když si, možná poprvé, uvědomíte, že váš život na zemi není bez konce.
Pak se, jako Izraelité, můžete ocitnout ve spěchu dát věci do pořádku, zejména ve svém osobním životě. Bůh vám dal
těchto cenných devadesát dní na to, abyste přehodnotili svůj život, prohloubili vztahy, očistili se od hříchu a přerovnali
svůj život vzhledem k Bohu. Hospodin je Bůh moudrosti a lásky. Bez ohledu na to, jakou je pro vás Exodus 90
výzvou, je dobré, že jste teď tady. Využijte tohoto času.

Dáváte stále do pořádku věci ve svém duchovním nebo rodinném životě, které vás Pán žádá, abyste urovnali? Držíte se
stranou od vychloubání se, protěžování sebe sama, od uhýbání před odpovědností? Žijete průměrně, a dokonce plýtváte
těmito devadesáti dny tím, že jim nedáváte plné úsilí nebo záměrně obcházíte pravidla? Pohovořte si o tom dnes
s Pánem v rozjímání a naslouchejte, kam vás vede.


%newday
\newpage
\section{Den 31 - VELKORYSOST S BOHEM}
\zacatekPatyTyden
\subsection*{Čtení na den}
\textbf{Exodus 12,37-42}
\newline
\textit{
\textsuperscript{37}Izraelci vytáhli z Ramesesu do Sukótu, kolem šesti set tisíc pěších mužů kromě dětí.
\textsuperscript{38}Vyšlo s nimi také mnoho přimíšeného lidu a obrovská stáda bravu a skotu.
\textsuperscript{39}Z těsta, které vynesli z Egypta, napekli nekvašené podpopelné chleby, protože ještě nevykynulo. Byli totiž z Egypta vyhnáni a nemohli otálet. Ani potravu na cestu si nestačili připravit.
\textsuperscript{40}Doba pobytu, kterou Izraelci v Egyptě strávili, byla čtyři sta třicet let.
\textsuperscript{41}Když uplynulo čtyři sta třicet let, přesně na den vyšly všechny Hospodinovy zástupy z egyptské země.
\textsuperscript{42}Byla to noc jejich bdění pro Hospodina, když je vyváděl z egyptské země. Tato noc je všem synům Izraele po všechna pokolení nocí bdění pro Hospodina.
}

\subsection*{Reflexe}
V této pasáži vidíme pozoruhodnou Boží velkorysost. Vzpomeňme si na první verše knihy Exodus. Jozef, syn Jákobův,
přišel do Egypta sám. Z tohoto muže, ke kterému se nakonec připojili jeho bratři, vzešlo 600 000 mužů (nepočítaje
ženy a děti), o kterých se dnes mluví. To nám naastavuje dvě provokativní otázky: zaprvé, Boha nikdy nikdo nepředčí
v Jeho velkorysosti. Zadruhé, je zázrak, kolik toho Bůh dokáže s málem.

Dnes se připojujete k Izraelitům, když konečně následují Hospodina z Egypta. Po celém měsíci za vámi jste vykročili
se svými bratry na další část cesty. Udělejte si chvilku na oslavu toho, co jste dokázali. Opuštění Egypta není malý
úspěch. Vzdali se jste mnoha a učinili mnoho životních změn pro to, abyste se sem dostali.

Všimněte si však, že zaslíbená zem není blízko. Ani faraon, ani Egypťané se nevzdali. Rozhodli jste se zcela opustit
svůj starý domov otroctví a modlářství, abyste se vydali na cestu ke svobodě. To je dobře. Ale cesta před vámi není
dobře dlážděná dálnice lemovaná nejlepšími restauracemi a obchody. Opustili jste civilizaci. Vkročili jste do divoké
pouště.

Teď se věci stanou těžšími. Zlý, více rozzlobený než faraon, vás bude prohnaně následovat do pouště, aby vás stíhal a
zotročoval více než předtím. Ale nebojte se, Pán vás vede, vaši bratři jsou s vámi a cíl za to stojí. Tohle vás vedlo
z Egypta a povede vás celou cestou až do země zaslíbené.

Dnes si najděte čas, abyste chválili Boha. Předneste Mu svou vděčnost za to, že vás vyvedl z Egypta a vede vás dále
směrem ke svobodě. U těch, kdo zvládli první měsíc, je mnohem pravděpodobnější, že zvládnou celou cestu.

%newday
\newpage
\section{Den 32 - EUCHARISTIE A JEDNOTA}
\zacatekPatyTyden
\subsection*{Čtení na den}
\textbf{Exodus 12,43-51}
\newline
\textit{
\textsuperscript{43}Hospodin řekl Mojžíšovi a Áronovi: „Toto je nařízení o hodu beránka: Nebude z něho jíst žádný cizinec.
\textsuperscript{44}Ale bude jej jíst každý služebník koupený za stříbro, bude-li obřezán.
\textsuperscript{45}Přistěhovalec ani nádeník jej jíst nebude.
\textsuperscript{46}Musí být sněden v témž domě. Z jeho masa nevyneseš nic z domu; žádnou jeho kost nezlámete.
\textsuperscript{47}Tak to bude dělat celá izraelská pospolitost.
\textsuperscript{48}Jestliže by u tebe pobýval host a chtěl by připravit Hospodinu hod beránka, nechť je u něho obřezán každý mužského pohlaví a potom bude smět přistoupit a tak učinit a bude jako domorodec v zemi. Ale žádný neobřezanec jej jíst nebude.
\textsuperscript{49}Stejný řád bude platit pro domorodce i pro hosta, který bude pobývat mezi vámi.“
\textsuperscript{50}Všichni Izraelci učinili přesně tak, jak Hospodin Mojžíšovi a Áronovi přikázal.
\textsuperscript{51}Právě v ten den vyvedl Hospodin Izraelce seřazené po oddílech z egyptské země.
}

\subsection*{Reflexe}
Při velikonoční večeři byl obětován bezvadný beránek. Tato oběť předznamenala bohoslužbu oběti, kde se Kristus,
neposkvrněný beránek, stal velikonoční obětí. V Exodu Bůh přikázal, aby se žádný cizinec Paschy neúčastnil, protože
by tím narušoval jednotu. Přesto nebyl cizinec vyloužen bez naděje na začlenění. Cizinec, který se chtěl spoluúčastnit
Paschy, musel učinit to, co všichni členové již učinili pro vstup do izraelské komunity: musel být obřezán. Pokud se tak
rozhodl, pak se stal členem společenství a byl přijat k účasti na Pesachu.

Tato překážka obřízky učinila rozhodnutí stát se součástí společenství poměrně vážným. Žádný dospělý muž jen tak z
nedělního rozmaru nedovolí, aby se nůž přiblížil jeho genitáliím. Můž, který byl rozhodnut se nechat obřezat, si zvolil
být plnohodnotným a aktivním účastníkem společenství. Zvolil si věrnost tomuto tělu.

Stejně tak Církev nedovolí nekatolíkům přijímat svaté přijímání z nedělního rozmaru. Církev však chce, aby měli
všichni lidé možnost přijmout svaté přijímání. Jako cizinci mezi Izraelity, stejně nemusí být nekatolíci navždy
vyloučeni z přijímání Eucharistie. Aby však mohli být přijati, musí udělat to, co se žádá po všech katolících ve
společenství: veřejně vyznat víru Církve, být pokřtěni, zpytovat svědomí, postit se, chodit na mše svaté, a přijímat
Eucharistii. Ve skutečnosti je každý zván ke svatému přijímání, ale všichni jsme povinni přijmout tuto svátost
zaslouženě a s věrností. Jste zavázáni k přijetí, nebo berete své členství v církevní komunitě za samozřejmost?

%newday
\newpage
\section{Den 33 - JSI KNĚZEM}
\zacatekPatyTyden
\subsection*{Čtení na den}
\textbf{Exodus 13,1-16}
\newline
\textit{
\textsuperscript{1}Hospodin promluvil k Mojžíšovi:
\textsuperscript{2}„Posvěť mi všechno prvorozené, co mezi Izraelci otvírá lůno, ať z lidí či z dobytka. Je to moje!“
\textsuperscript{3}Mojžíš řekl lidu: „Pamatujte na tento den, kdy jste vyšli z Egypta, z domu otroctví. Hospodin vás odtud vyvedl pevnou rukou. Proto se nesmí jíst nic kvašeného.
\textsuperscript{4}Vycházíte dnes, v měsíci ábíbu.
\textsuperscript{5}Až tě Hospodin uvede do země Kenaanců, Chetejců, Emorejců, Chivejců a Jebúsejců, o níž se zavázal tvým otcům přísahou, že ji dá tobě, zemi oplývající mlékem a medem, budeš v ní tohoto měsíce konat tuto službu:
\textsuperscript{6}Sedm dní budeš jíst nekvašené chleby. Sedmého dne bude slavnost Hospodinova.
\textsuperscript{7}Nekvašené chleby se budou jíst po sedm dní. Nespatří se u tebe nic kvašeného, na celém tvém území se u tebe nespatří žádný kvas.
\textsuperscript{8}V onen den svému synovi oznámíš: ‚To je proto, co mi prokázal Hospodin, když jsem vycházel z Egypta.‘
\textsuperscript{9}A budeš to mít jako znamení na své ruce a jako připomínku mezi svýma očima, aby v tvých ústech zůstal Hospodinův zákon, neboť pevnou rukou tě vyvedl Hospodin z Egypta.
\textsuperscript{10}Budeš dbát na toto nařízení ve stanovený čas rok co rok.
\textsuperscript{11}Až tě Hospodin uvede do země Kenaanců, jak přísežně zaslíbil tobě i tvým otcům, a až ti ji dá,
\textsuperscript{12}všechno, co otvírá lůno, odevzdáš Hospodinu. Všichni samečci, které tvůj dobytek vrhne nejprve, budou patřit Hospodinu.
\textsuperscript{13}Každého osla, který se narodil jako první, vyplatíš jehnětem. Kdybys jej nemohl vyplatit, zlomíš mu vaz. Také každého prvorozeného ze svých synů vyplatíš.
\textsuperscript{14}Až se tě tvůj syn v budoucnu zeptá, co to znamená, odpovíš mu: ‚Hospodin nás vyvedl pevnou rukou z Egypta, z domu otroctví.
\textsuperscript{15}Když se farao zatvrdil a nechtěl nás propustit, pobil Hospodin v egyptské zemi všechno prvorozené, od prvorozeného z lidí až po prvorozené z dobytka. Proto obětuji Hospodinu všechny samce, kteří otvírají lůno, a každého prvorozeného ze svých synů vyplácím.‘
\textsuperscript{16}To bude jako znamení na tvé ruce a jako pásek na čele mezi tvýma očima. Neboť pevnou rukou nás vyvedl Hospodin z Egypta.“
}

\subsection*{Reflexe}
Toto vykoupení neboli „zpětný odkup“ prvorozeného syna je důslednou připomínkou Izraelitům, že v noci, kdy odešli z Egypta,
Bůh ušetřil jejich prvorozené syny, ale nezachránil prvorozené syny Egypťanů. Izraelští prvorození synové byli služební třídou
nebo řádem kněží. Tuto výsadu by si zachovali, pokud by nebylo té nešťastné události se zlatým teletem, kdy jim byla jejich čest
odebrána a dána kmenu Levi.

V Novém zákoně Ježíš Kristus sloužil jako kněz – obzvlášť na Velký Pátek, když byl knězem i obětí zároveň. Tento čin změnil
všechno. Po vítězství Krista nad hříchem a smrtí se všichni lidé pokřtěni v Krista stanou jeho kněžskou třídou (srov. Zjev 5,10)
(ačkoli se toto kněžství liší od kněžství služebného). Proč je to důležité? Protože kněz obětuje.

To znamená, že vy, členové všeobecného kněžšství, jste povinni obětovat. Jakou oběť? Svatý Pavel nám říká: „Vybízím vás, bratří,
pro Boží milosrdenství, abyste sami sebe přinášeli jako živou, svatou, Bohu milou oběť; to ať je vaše pravá bohoslužba“ (Řím
\textsuperscript{12},1) Toto duchovní cvičení, ve kterém jste nyní zapojeni, je obětí Bohu. Plněním denních disciplín tohoto cvičení se formujete pro
dobré kněžství hodné Ježíše Krista.

Pohovořte si s Pánem o vaší službě ve všeobecném kněžství. Naslouchejte tomu, co vám chce říct o posvěcování vašeho dne a
přinášení dokonalých obětí pro ostatní z asketických disciplín, které jste se rozhodli přijmout. Modlete se, ať vám dá smysluplnější
a hlubší poznání oběti.

%newday
\newpage
\section{Den 34 - NÁSLEDUJTE BOHA VE VÍŘE}
\zacatekPatyTyden
\subsection*{Čtení na den}
\textbf{Exodus 13,17-14,9}
\newline
\textit{
\textsuperscript{17}Když farao lid propustil, nevedl je Bůh cestou směřující do země Pelištejců, i když byla kratší. Bůh totiž řekl: „Aby lid nelitoval, když uvidí, že mu hrozí válka, a nevrátil se do Egypta.“
\textsuperscript{18}Proto Bůh vedl lid oklikou, cestou přes poušť k Rákosovému moři. Izraelci vytáhli z egyptské země rozděleni do bojových útvarů.
\textsuperscript{19}Mojžíš vzal s sebou Josefovy kosti. Ten totiž zavázal Izraelce přísahou: „Až vás Bůh navštíví, vynesete odtud s sebou mé kosti.“
\textsuperscript{20}I vytáhli ze Sukótu a utábořili se v Étamu na pokraji pouště.
\textsuperscript{21}Hospodin šel před nimi ve dne v sloupu oblakovém, a tak je cestou vedl, v noci ve sloupu ohnivém, a tak jim svítil, že mohli jít ve dne i v noci.
\textsuperscript{22}Sloup oblakový se nevzdálil od lidu ve dne, ani sloup ohnivý v noci.
\textsuperscript{1}Hospodin promluvil k Mojžíšovi:
\textsuperscript{2}„Rozkaž Izraelcům, aby se obrátili a utábořili před Pí-chírotem mezi Migdólem a mořem; utáboříte se před Baal-sefónem, přímo proti němu při moři.
\textsuperscript{3}Farao si o Izraelcích řekne: Bloudí v zemi, zavřela se za nimi poušť.
\textsuperscript{4}Tu zatvrdím faraónovo srdce a on vás bude pronásledovat. Já se však na faraónovi a na všem jeho vojsku oslavím, takže Egypťané poznají, že já jsem Hospodin.“ I učinili tak.
\textsuperscript{5}Egyptskému králi bylo oznámeno, že lid uprchl. Srdce faraóna a jeho služebníků se obrátilo proti lidu. Řekli: „Co jsme to udělali, že jsme Izraele propustili z otroctví?“
\textsuperscript{6}Farao dal zapřáhnout do svého válečného vozu a vzal s sebou svůj lid.
\textsuperscript{7}Vzal též šest set vybraných vozů, totiž všechny vozy egyptské. Na všech byla tříčlenná osádka.
\textsuperscript{8}Hospodin zatvrdil srdce faraóna, krále egyptského, a ten Izraelce pronásledoval. Ale Izraelci navzdory všemu vyšli.
\textsuperscript{9}Egypťané je pronásledovali a dostihli je, když tábořili při moři, dostihli je všichni faraónovi koně, vozy, jeho jízda a vojsko, při Pí-chírotu před Baal-sefónem.
}

\subsection*{Reflexe}
Přišli jste na okraj Rudého moře. Jedním směrem se nachází krásný výhled. Druhým směrem zuří armáda
pronásledující vás, aby vás zajala nebo zabila. Tato scéna byla pro Izraelity skutečnou realitou. Pro nás dnes
také zůstává realitou, ovšem v duchovní sféře.
Kardinál Jean Daniélou napsal o konfliktu katechumenů připravujících se na křest a nepřítele, Satana.
Zevrubně popisuje, co oko nevidí a duch ještě není schopen postřehnout:

\textit{Čtyřicet postních dní v katechumenátu jsou časem soudu, čase vážného konfliktu, zatímco se
Satan a jeho pomocníci snaží ubránit si zajetí duše katechumena. To není pouhý řečnický obrat,
musí to být chápáno v doslovném smyslu skutečnosti: pohan totiž není pouhým neználkem
křesťanského zjevení, ale aktivním poddaným moci zla, a musí být vysvobozen z tohoto zajetí…
Každá konverze proto zahrnuje množství dramatických konfliktů, a všechny misionářské aktivity
zahrnují podstatu tohoto tajemství. Není to pouhé představení radostné zprávy evangelia ve
formě vhodné pro každého z různých nekřesťanských kultur, ale střet v boji s mocnostmi zla:
operace této války probíhají na nadpřirozeném poli a patří do tajemství svatosti – skrze modlitbu
a pokání je ďábel nakonec vypuzen. Ignorování tohoto aspektu věci znamená nepochopení
podstaty misionářské práce. I po Kristově vítězství zůstává lidská postata těch, kteří nejsou Jeho,
uvězněna: On rozdrtil hlavu hadu, ale svitky se stále svíjejí, aby polapily lid země. Když Satan
vidí svou kořist unikat, zdvojnásobí své síly proti katechumenovi; ale během těchto čtyřiceti dní je
Kristovo sevření také zesíleno… ale ďábel drží oběť pod tlakem celou dobu až po onu chvíli
Velikonoční vigílie, po samý okraj křtitelnice. Pouze poté se stane ona nemožná věc: moře je
rozděleno…
Vody se rozestoupily pro Izraelity a brány smrti byly otevřeny Pánu Ježíši – katechumen sestoupí
do křestní vody, učiní onen krok, nechá za sebou faraona a jeho vojska, ďábla a jeho anděly a
přejde na druhou stranu. Je zachráněn. Doslova to znamená být ztroskotancem, který přežil a byl
převezen na zem.}

Křest, nám daný pro naši spásu, je více než jen obyčejný obřad. Je to víc než pouhá pověra. Je to spásné dílo
Boží. Když dnes stojíte u Rudého moře, zvažte svou situaci. Jestliže budete následovat Pána, mohli byste se
utopit v obrovském moři – tedy pokud pro vás Hospodin nemá lepší plán, který ještě nedokážete vnímat.

Druhou možností je zastavit a otočit se zpět. Pokud zvolíte tuto možnost, v nejlepším případě budete
zotročeni, v nejhorším ztratíte sám život. Dnes je 34. den, můžete přestat. Jste volní a můžete se vrátit zpět.
Co si vyberete? Vrátíte se zpět, nebo dáte svou důvěru Bohu a budete pokračovat dál?

%newday
\newpage
\section{Den 35 - TOUHY SRDCE}
\zacatekPatyTyden
\subsection*{Čtení na den}
\textbf{Exodus 14,10-20}
\newline
\textit{
\textsuperscript{10}Když se farao přiblížil, Izraelci se rozhlédli a viděli, že Egypťané táhnou za nimi. Tu se Izraelci velmi polekali a úpěli k Hospodinu.
\textsuperscript{11}A osopili se na Mojžíše: „Což nebylo v Egyptě dost hrobů, že jsi nás odvedl, abychom zemřeli na poušti? Cos nám to udělal, že jsi nás vyvedl z Egypta?
\textsuperscript{12}Došlo na to, o čem jsme s tebou mluvili v Egyptě: Nech nás být, ať sloužíme Egyptu. Vždyť pro nás bylo lépe sloužit Egyptu než zemřít na poušti.“
\textsuperscript{13}Mojžíš řekl lidu: „Nebojte se! Vydržte a uvidíte, jak vás dnes Hospodin zachrání. Jak vidíte Egypťany dnes, tak je už nikdy neuvidíte.
\textsuperscript{14}Hospodin bude bojovat za vás a vy budete mlčky přihlížet.“
\textsuperscript{15}Hospodin řekl Mojžíšovi: „Proč ke mně úpíš? Pobídni Izraelce, ať táhnou dál.
\textsuperscript{16}Ty pak pozdvihni svou hůl, vztáhni ruku nad moře a rozpoltíš je, a tak Izraelci půjdou prostředkem moře po suchu.
\textsuperscript{17}Já zatvrdím srdce Egypťanů, takže půjdou za nimi. Oslavím se na faraónovi a na všem jeho vojsku, na jeho vozech i jízdě.
\textsuperscript{18}Egypťané poznají, že já jsem Hospodin, až budu oslaven tím, co učiním s faraónem, s jeho vozy a jízdou.“
\textsuperscript{19}Tu se zvedl Boží posel, který šel před izraelským táborem, a šel teď za nimi. Oblakový sloup se před nimi totiž zvedl, postavil se za ně
\textsuperscript{20}a vstoupil mezi tábor egyptský a izraelský. Jedněm byl oblakem a temnotou, druhým osvěcoval noc; po celou noc se jedni k druhým nepřiblížili.
}

\subsection*{Reflexe}
Tato úžasná událost je známá mnohým: rozestoupení Rudého moře. Izraelci v nemožné situaci reagují naprosto
pochybně. Bůh mluví k Mojžíšovi a říká mu něco fascinujícího: „Proč ke mně úpíš?“ To je zvláštní, protože to vypadá,
jako by Bůh reagoval na něco, co řekl Mojžíš… ale Mojžíš neřekl nic. Bůh zná touhy lidského srdce a tyto touhy
k Němu mluví. Jaká útěcha! Dokonce i když nevíme, na co se zeptat (nebo snad i když žádáme o špatnou věc), nás Bůh
zná lépe, než známe sami sebe.

I když si myslíme, že cheme dočasně ulevit svým obětem, Bůh pro nás chce to, co opravdu chceme také. Nechceme
doopravdy pomíjející potěšení z několikahodinového sledování Netflixu. Nechceme opravdu pivo. Nechceme opravdu
vysoký zisk na trhu. Chceme radost. Chceme svobodu. Chceme žít život, který stojí za žití. Vaše uši mohou být hluché
k tichému úpění vašeho srdce k Bohu: „Dej mi radost!“ Ale přesto vás Bůh slyší a ptá se: „Proč ke mně úpíš?“

Bůh vám říká: „Táhněte dál.“ Jeho slova jsou těžko slyšitelná, když se díváte na moře. Ale nebojte se. Jeho plán je
větší, než můžete vnímat. Bude s vámi skrze to všechno. Ve vaší svaté hodině si dnes udělejte čas, abyste s Bohem
hovořili o svých přáních. Poslouchejte, když vám ukazuje to, co leží pod vašimi povrchními touhami. Naslouchejte,
když vám zjevuje, co skutečně přinese radost a dobro vašemu životu, vaší rodině a vaší farnosti.

% ===============================================
% ===== SESTY TYDEN
% ===============================================
%ukony
\newpage
\section*{Úkony (ukazatel cesty) pro 6. týden}

\textbf{Místo:} Severozápadní břeh Rudého moře

Faraon a jeho armáda opustili Egypt v neúprosném pronásledování Izraelitů. Jen co si Izraelité pomysleli, že jsou zcela na svobodě, ocitli se uvězněni mezi zuřící armádou a zdánlivě neprůchodným vodním útvarem. Ztratit zde naději by znamenalo vzdát se víry v Boha, a vyústění by vedlo k ještě tvrdšímu zotročení než kdy dříve. Nacházíme se na podobném místě. Naše dřívější zvyky na nás útočí. Pokud ztratíme víru, skončíme a otočíme se nazpět, budeme znovu zotročeni. Pokud však zůstaneme oddaní naší víře, Pán nás povede skrz vody. Očistí nás a oddělí nás od našich nepřátel jako nikdy předtím. Co si tento týden zvolíte?

\subsection*{1. Udělejte si čas na dobrou zpověď}
Pokud jste naposledy byli u zpovědi na začátku vašeho exodu nebo předtím, než začal, pak je toto ideální čas vrátit se zpět. Hřích „zraňuje a oslabuje samotného hříšníka, jakož i jeho vztahy k Bohu a k bližnímu“ (KKC 1459). Proto hřích, a co je důležitější – účinek hříchu – je přímo v protikladu s cíly Exodu 90. Vyhledejte, kdy se ve vaší farnosti zpovídá, a udělejte si čas na přijetí milostí, které vám Bůh chce dát vprostřed tohoto exodu.
\subsection*{2. Držte se reflexí}
Zůstaňte na této cestě v jednotě se svýmy bratry. Nepodvádějte sebe sama v tomto exodu. Držte se denních reflexí.
\subsection*{3. Vstupte do Slova}
Kristus je ono Slovo, které čtete ve vašich reflexích Písma každý den. Nečtěte pouze slova, vstupte do Slova. To Slovo žije. Je to člověk a můžete s ním hovořit. Člověk, který mluví přímo k vám. Vstupte do Slova každý den nasloucháním, co vám ve vaší životní situaci chce Pán říci.

\subsection*{Modlitba}
Modlete se, aby Pán osvobodil vás a vaše bratrství \newline
Modleme se za svobodu všech mužů v exodu, stejně tak, jako se oni modlí za vás.\newline
Ve jménu Otce i Syna i Ducha svatého … Otče náš… Ve jménu Otce i Syna i Ducha svatého … Amen.
\newpage
%newday
\newpage
\section{Den 36 - SÍLA KŘTU }
\zacatekSestyTyden
\subsection*{Čtení na den}
\textbf{Exodus 14,21-31}
\newline
\textit{
\textsuperscript{21}Mojžíš vztáhl ruku nad moře a Hospodin hnal moře silným východním větrem, který vál po celou noc, až proměnil moře v souš. Vody byly rozpolceny.
\textsuperscript{22}Izraelci šli prostředkem moře po suchu. Vody jim byly hradbou zprava i zleva.
\textsuperscript{23}Egypťané je pronásledovali a vešli za nimi doprostřed moře, všichni faraónovi koně, vozy i jízda.
\textsuperscript{24}Za jitřního bdění vyhlédl Hospodin ze sloupu ohnivého a oblakového na egyptský tábor a vyvolal v egyptském táboře zmatek.
\textsuperscript{25}Způsobil, že se uvolnila kola jejich vozů, takže je stěží mohli ovládat. Tu si Egypťané řekli: „Utečme před Izraelem, neboť za ně bojuje proti Egyptu Hospodin.“
\textsuperscript{26}Hospodin řekl Mojžíšovi: „Vztáhni ruku nad moře! Vody se obrátí na Egypťany, na jejich vozy a jízdu.“
\textsuperscript{27}Mojžíš vztáhl ruku nad moře, a když nastávalo jitro, moře opět nabylo své moci. Egypťané utíkali proti němu a Hospodin je vehnal doprostřed moře.
\textsuperscript{28}Vody se vrátily, přikryly vozy i jízdu celého faraónova vojska, které vešlo za Izraelci do moře. Nezůstal z nich ani jediný.
\textsuperscript{29}Ale Izraelci přešli prostředkem moře po suchu a vody jim byly hradbou zprava i zleva.
\textsuperscript{30}Onoho dne zachránil Hospodin Izraele z moci Egypta. Izrael viděl na břehu moře mrtvé Egypťany.
\textsuperscript{31}Tak uviděl Izrael velikou moc, kterou osvědčil Hospodin na Egyptu. Lid se bál Hospodina a uvěřili Hospodinu i jeho služebníku Mojžíšovi.
}

\subsection*{Reflexe}

Nemělo by být překvapením, že události dnešního čtení slouží jako prvotní obraz křtu. Izraelité, povolaní mít důvěru a
víru v Boha, kráčejí k vodě. Vstupují jako lidé utlačovaní Egyptem a vycházejí na druhé straně zachránění.

V křesťanském křtu nepokřtěný, povolaný k důvěře a víře v Boha, kráří k vodě ve křtitelnici. Vstupuje jako muž
utlačovaný sevřením hříchu a smrti, a vychází z vody zachráněný. To je možné díky Duchu svatému, který se vznáší
nad vodami (tak jako když byl sám Ježíš pokřtěn) a nově nás utváří. Když se díváme na událost dnešního čtení, vidíme
něco podobného: „Hospodin hnal moře silným východním větrem, který vál po celou noc…“ Vítr zde slouží jako obraz
Ducha svatého. Stejně jako v křesťanském křtu je to Duch svatý, vznášející se nad vodami, který umožní spásu
Izraelitů.

Rozdělení Rudého moře ukazuje moc Ducha svatého, tedy moc Boží. Tato událost způsobí bázeň Izraelitů před
Bohem, a jejich víru v Něj (srov. Ex 14,31). I dnes si Židé tuto spásonosnou událost připomínají jako důkaz velkého
Božího plánu s nimi. Vzhlíží k této události a připomínají si, že jsou Božími prvorozenými syny, a On je jejich Bohem.
Nepodceňujte svůj vlastní křest. Spíše jako Izraelité si tuto spásnou událost připomínejte a vzpomeňte si, že Bůh má
pro vás také svůj velký plán. Dovolte, aby váš křest sloužil jako ustavičná připomínka toho, že jste Božím synem, a On
je vaším Bohem.

Vezměme v potaz malé kropenky ve vchodech katolických kostelů po celém světě. Jsou ty jen proto, abyste
bezdůvodně namočili konečky svých prstů a zanechali kapky na košili? Ani náhodou. Pokaždé, když si smočíte prsty
v kropence, vzpomeňte si na milosti vašeho křtu. Užívejte svěcené vody v kropenkách k tomu, abyste si připomněli, že
Bůh má pro vás svůj velký Plán, a že jste Božím synem, a On je vaším Bohem. Jaký to dar.

Předneste dnes tyto tři věci – Boží plán pro vás, vaše synovství a božskost Boha – do vaší svaté hodiny. Pozvěte Pána,
aby vám o každé sdělil více a novým způsobem.

%newday
\newpage
\section{Den 37 - VDĚČNOST PŘINÁŠÍ HOJNOU RADOST}
\zacatekSestyTyden
\subsection*{Čtení na den}
\textbf{Exodus 15,1-21}
\newline
\textit{
\textsuperscript{1}Tehdy zpíval Mojžíš a synové Izraele Hospodinu tuto píseň. Vyznávali: „Hospodinu chci zpívat, neboť se slavně vyvýšil, smetl do moře koně i s jezdcem.
\textsuperscript{2}Hospodin je má záštita a píseň, stal se mou spásou. On je můj Bůh, a já ho velebím, Bůh mého otce, a já ho vyvyšuji.
\textsuperscript{3}Hospodin je bojovný rek; Hospodin je jeho jméno.
\textsuperscript{4}Vozy faraónovy i jeho vojsko svrhl v moře, v moři Rákosovém utonul výkvět jeho vozatajstva.
\textsuperscript{5}Tůně propastné je zavalily, klesli do hlubin jak kámen.
\textsuperscript{6}Tvá pravice, Hospodine, velkolepá v síle, tvá pravice, Hospodine, zdrtí nepřítele.
\textsuperscript{7}Nesmírně vyvýšen rozmetáš útočníky, vysíláš své rozhorlení, jako oheň strniště je pozře.
\textsuperscript{8}Dechem tvého chřípí počaly se kupit vody, příboje zůstaly stát jako hráze, sesedly se tůně propastné v klín moře.
\textsuperscript{9}Nepřítel si řekl: ‚Pustím se za nimi, doženu je, rozdělím kořist, ukojím jimi svou duši, meč vytasím, podrobí si je má ruka.‘
\textsuperscript{10}Zadul jsi svým dechem a moře je zavalilo, potopili se jak olovo v nesmírných vodách.
\textsuperscript{11}Kdo je mezi bohy jako ty, Hospodine? Kdo je jako ty, tak velkolepý ve svatosti, hrozný v chvályhodných skutcích, konající divy?
\textsuperscript{12}Vztáhl jsi pravici a země je pohltila.
\textsuperscript{13}Svým milosrdenstvím jsi vedl tento lid, který jsi vykoupil, provázel jsi jej svou mocí ke své svaté nivě.
\textsuperscript{14}Uslyšely o tom národy a zmocnil se jich neklid, bolest sevřela obyvatele Pelišteje.
\textsuperscript{15}Tehdy se zhrozili edómští pohlaváři, moábské vůdce zachvátilo chvění, všichni obyvatelé Kenaanu propadli zmatku.
\textsuperscript{16}Padla na ně hrůza a strach; pro velikost tvé paže zmlknou jako kámen, dokud, Hospodine, neprojde tvůj lid, dokud neprojde ten lid, který sis získal.
\textsuperscript{17}Přivedeš a zasadíš je na hoře svého dědictví, kde jsi, Hospodine, připravil své sídlo k přebývání, kde tvé ruce, Panovníku, svatyni si přichystaly.
\textsuperscript{18}Hospodin kraluje navěky a navždy.“
\textsuperscript{19}Když totiž faraónovi koně s jeho vozy a jízdou vešli do moře, Hospodin na ně obrátil mořské vody. Ale Izraelci šli po suchu prostředkem moře.
\textsuperscript{20}Tu vzala prorokyně Mirjam, sestra Áronova, do ruky bubínek a všechny ženy vyšly za ní s bubínky v tanečním reji.
\textsuperscript{21}A Mirjam střídavě s muži prozpěvovala: „Zpívejte Hospodinu, neboť se slavně vyvýšil, smetl do moře koně i s jezdcem.“
}

\subsection*{Reflexe}
Konečně, po tak dlouhém konfliktu, byl izraelský lid vykoupen z vazeb fyzického otroctví. Ohromeni radostí propukají Mojžíš i
prorokyně Mirjam ve zpěv. Zpívají novou píseň. Vidíme to v celém Starém Zákoně, a dokonce i v knize Zjevení, následující
Kristovo vítězství na kříži: „A zpívali novou píseň: ‚Jsi hoden přijmout tu knihu a rozlomit její pečetě…‘“ (Zjev 5,9).

Zde v Exodu Mojžíš zpívá: „Kdo kromě Pána mi přinese vysvobození?“ Bůh vysvobodil svůj lid z fyzické nadvlády a vlivu mnoha
bohů Egypta. Když dnes oslavujeme něco tak pomíjivého jako sportovní mistrovství, o co víc bychom měli oslavovat něco tak
velkého a trvalého, jako je křestní svobodaTo je věčné vítězství, které stojí za to slavit.

Buďte člověkem vděčnosti. Udělejte si chvilku času, abyste poděkovali Pánu za to, že vás dovedl až sem. Poděkujte mu za dar vaší
rodiny. Poděkujte Pánu za touhu být lepším člověkem. Poděkujte Pánu za to, že můžete dýchat. Nehledě na všechno, na co si
stěžujeme, toho máme mnohem víc, za co můžeme být vděční. Zaspívejte dnes Bohu novou píseň. Naučte se s Mojžíšem a
Izraelity být člověkem vděčnosti. Neboť v životě vděčnosti najdete velkou svobodu a hojnou radost.


%newday
\newpage
\section{Den 38 - ČEKÁNÍ NA LAVIČCE – Jak sladké!}
\zacatekSestyTyden
\subsection*{Čtení na den}
\textbf{Exodus 15,22-27}
\newline
\textit{
\textsuperscript{22}Mojžíš vedl Izraele od Rákosového moře dál. Vyšli na poušť Šúr a táhli pouští po tři dny, aniž narazili na vodu.
\textsuperscript{23}Došli až do Mary, ale nemohli vodu z Mary pít, protože byla hořká. Pojmenovali ji proto Mara (to je Hořká).
\textsuperscript{24}Tu lid proti Mojžíšovi reptal: „Co budeme pít?“
\textsuperscript{25}Mojžíš úpěl k Hospodinu a Hospodin mu ukázal dřevo. Když je hodil do vody, voda zesládla. Tam dal Hospodin lidu nařízení a právní ustanovení a podrobil jej tam zkoušce.
\textsuperscript{26}Řekl: „Jestliže opravdu budeš poslouchat Hospodina, svého Boha, dělat, co je v jeho očích správné, naslouchat jeho přikázáním a dbát na všechna jeho nařízení, nepostihnu tě žádnou nemocí, kterou jsem postihl Egypt. Neboť já jsem Hospodin, já tě uzdravuji.“
\textsuperscript{27}Pak přišli do Élimu. Tam bylo dvanáct vodních pramenů a sedmdesát palem. Tam při vodách se utábořili.
}

\subsection*{Reflexe}
Izraelité, vysvobozeni ze svého fyzického otroctví, užívající si prvních dnů své svobody, se najednou nachází uprostřed
nemilosrdné pouště bez vody. Jako se stane ještě mnohokrát v budoucnu, jejich zoufalství nad Boží prozřetelností je
dovede k tlaku na Mojžíše, který má okamžitě vyřešit jejich nouzi. Přicházejí k vodě z Mary, ale nemohou ji pít,
protože je hořká. Aby Bůh tento problém vyřešil, ukazuje Mojžíši dřevo. Ano, dřevo.

Když následujete Izraelity na jejich cestě, snad jste začali poznávat, že mnoho tajemství Starého Zákona je téměř
nesrozumitelných, kromě Krista. Dnešní scénář onoho dřeva je zářným příkladem. V Novém Zákoně Kristus ukazuje
apoštolům také dřevo – kříž – a pověří je, aby jej vnášeli do životů všech národů. Tento úkon, který pokračuje již po
dvě tisíciletí, činí naše životy – se všemi břemeny a zkouškami, které se mohou zdát hořké – sladké a lehké.

Představte si sami sebe sedící venku na lavičce v nepříjemné zimě. Lavička je přišroubována k zemi na husté a rušné
ulici. Jako vždy čekáte na vašeho kolegu, aby vás vyzvedl cestou do práce. Aniž byste pomysleli na kříž, rozhodli jste
se, že budete naštvaní a cyničtí.

Nyní si sami sebe představte na podobné lavičce venku v zimě. Tentokrát je však lavička zavěšená na laně a veze vás
na vrchol hory KT-22. Sněžná bouře vámi háže. Většina lidí by to považovala za „nepříznivé počasí“, ale vy přetékáte
radostí a vděčností. Vlastně jste dokonce zaplatili za to, abyste mohli být v těchto podmínkách. Proč? Neboť pokud
vydržíte, dostanete příležitost vychutnat si skvělou lyžovačku na Squaw Valley. Jako lyžař si dokážete vychutnat jízdu
na sedačkové lanovce. Proto je i tento hořký moment vlastně sladký.
Prostřednictvím cvičení Exodu 90 se vaše čekání na lavičce stává stejně tak sladkým jako jízda lyžaře na lanovce.
Asketická cvičení vám pomůžou k tomu, abyste dokázali vidět příležitosti, které se skrývají za nepříznivým počasím.
Pouze takto dokážete vidět za sezením na studené lavičce možnost sjednotit nepohodlí a nedostatek kontroly
s utrpením kříže.

Toto utrpení pak může být vzkříšenov modlitbě skrze kříž, pro vás, vaši rodinu, ba dokonce i pro vašeho opozdilého
kolegu, stejně jako dřevo hozené do vody z Mary, které z hořké vody udělá sladkou. Také kříž, vhozený do
protivenství, činí i hořké utrpení sladkým.


%newday
\newpage
\section{Den 39 - OTROCTVÍ SRDCE}
\zacatekSestyTyden
\subsection*{Čtení na den}
\textbf{Exodus 16,1-3}
\newline
\textit{
\textsuperscript{1}Pak vytáhli z Élimu. Celá pospolitost Izraelců přišla na poušť Sín, která je mezi Élimem a Sínajem, patnáctý den druhého měsíce poté, co vyšli z egyptské země.
\textsuperscript{2}Celá pospolitost Izraelců na poušti opět reptala proti Mojžíšovi a Áronovi.
\textsuperscript{3}Izraelci jim vyčítali: „Kéž bychom byli zemřeli Hospodinovou rukou v egyptské zemi, když jsme sedávali nad hrnci masa, když jsme jídávali chléb do sytosti. Vždyť jste nás vyvedli na tuto poušť, jen abyste celé toto shromáždění umořili hladem.“
}

\subsection*{Reflexe}

Může být snadné dívat se na přední postavy Písma a cítit zvláštní pocit nadřazenosti nad nimi v jejich selháních.
Znalost těchto příběhů je karikatizuje a přináší pocit oprávněných předsudků. Adam zhřeší, odsoudí lidstvo k smrti a
my se ptáme, jak jen mohl být tak slepý. David se dopuští cizoložství a my skandalizujeme toto strašné selhání velkého
muže. Svatý Petr popírá Krista a my potupujeme jeho zbabělost.

Ve scéně z dnešního čtení je snadné se na tyto unavené, hladové, nepříjemné lidi dívat a pomyslet si: „Co je to za
rozmazlená děcka.“ Podívejme se na výsledky: Bůh je osvobodil od otroctví, rozdělil pro ně Rudé moře, utopil jejich
utlačovatele a učinil z hořké vody sladkou, aby ji mohli pít. Navzdory tomu všemu Izraelci kňučí. Horší je, že se chtějí
vrátit do otroctví, protože dostávali lepší jídlo a měli jistotu, že se někdo postará o jejich život.

V tomto 39. dni boje možná sympatizujete s požadavky Izraelitů. Status quo minulosti pro ně, a možná i pro vás,
vypadá až moc lákavě. Izraelci však ve svých srdcích jistě vědí, že status quo neznamená jen postel a pravidelnou
stravu. Také totiž znamená riziko zavraždění jejich dětí, protože se faraon cítí ohrožen. Znamená to nesvobodu
v uctívání jejich Boha. Znamená to vrátit se zpět do slávy nevlastní země, to vše na počest bohů, kterým nejsou ochotni
sloužit. Status quo je peklo.

V tomto okamžiku čelí Izraelité klíčovému rozhodnutí: vrátit se do status quo, nebo ho nechat sejít ze svých srdcí i
myslí. Svatý Jan Kassián říká: „Tělesné zřeknutí se a vyhoštění z Egypta pro nás nebude mít žádný význam, pokud
nebudeme současně schopni zřeknout se ho v srdci, což je vznešenější a přínosnější.“ Když pokračujete na této cestě a
vaše touhy vás lákají zpět ke statu quo otroctví, přijměte toto ultimátum: rozhodněte se nechat otroctví k neřestem sejít
ze svého srdce a mysli. Přizpůsobte svou vnitřní realitu vnější svobodě, ke které vás vede Bůh.

Svým plánem modlitby, askeze a bratrství vás Bůh přívedl k fyzickému oddělení od Egypta (od filmů, peněz, her atd.).
Máte však stále srdce stejné jako Izraelci? Je stále ještě v uchopení pekla, ze kterého vás Pán vyvedl?

Jsme jeden den od 40. dne. Všimněte si rozdílu mezi vaší vnější a vnitřní realitou. Důvod, proč toto cvičení netrvá 40,
ale na 90 dní, existuje. Je třeba udělat ještě více vnitřní práce. Leží před vámi větší svoboda, stejně jako pro Izraelity.
Poděkujte Pánu za vnější pokrok, který jste učinili, a následujte Jej i po 40. dni, v neúnavném úsilí o vnitřní svobodu až
do 91. dne.

Dnešek by neměl být zklamáním nad prací, kterou je ještě třeba udělat. Měl by to být den vděčnosti za dílo, které ve
vás Pán učinil. Dnes ve vaší svaté hodině přijměte tento dar radosti od Pána. Je na vás hrdý.



%newday
\newpage
\section{Den 40 - CHLÉB ŽIVOTA}
\zacatekSestyTyden
\subsection*{Čtení na den}
\textbf{Exodus 16,4-21}
\newline
\textit{
\textsuperscript{4}Hospodin řekl Mojžíšovi: „Já vám sešlu chléb jako déšť z nebe. Ať lid vychází a sbírá, co denně spotřebují. Tak je podrobím zkoušce, budou-li se řídit mým zákonem, či nikoli.
\textsuperscript{5}Když budou připravovat, co přinesou, ať je toho šestého dne dvakrát tolik, než co nasbírají každodenně.“
\textsuperscript{6}Mojžíš a Áron řekli všem Izraelcům: „Večer poznáte, že vás z egyptské země vyvedl Hospodin.
\textsuperscript{7}A ráno spatříte Hospodinovu slávu, ačkoli slyšel vaše reptání proti sobě. Co jsme my, že reptáte proti nám?“
\textsuperscript{8}Pak Mojžíš dodal: „Poznáte to podle toho, že vám Hospodin dá večer k jídlu maso a ráno k nasycení chléb, ačkoli slyšel reptání, jak jste proti němu reptali. Co jsme my? Nereptáte proti nám, ale proti Hospodinu.“
\textsuperscript{9}Áronovi Mojžíš řekl: „Vyzvi celou pospolitost Izraelců: ‚Přistupte před Hospodina, neboť slyšel vaše reptání.‘“
\textsuperscript{10}Když mluvil Áron k celé pospolitosti Izraelců, obrátili se k poušti, a vtom se ukázala v oblaku Hospodinova sláva.
\textsuperscript{11}Tu Hospodin promluvil k Mojžíšovi:
\textsuperscript{12}„Slyšel jsem reptání Izraelců. Vyhlas jim: ‚Navečer se najíte masa a ráno se nasytíte chlebem, abyste poznali, že já jsem Hospodin, váš Bůh.‘“
\textsuperscript{13}Když pak nastal večer, přiletěly křepelky a snesly se na tábor. A ráno padala kolem tábora rosa.
\textsuperscript{14}Když rosa přestala padat, hle, na povrchu pouště leželo po zemi cosi jemně šupinatého, jemného jako jíní.
\textsuperscript{15}Když to Izraelci viděli, říkali jeden druhému: „Man hú?“ (To je: „Co je to?“) Nevěděli totiž, co to je. Mojžíš jim řekl: „To je chléb, který vám dal Hospodin za pokrm.
\textsuperscript{16}Hospodin přikázal toto: Nasbírejte si ho každý, kolik potřebujete k jídlu. Každý vezmete podle počtu osob ve svém stanu ómer na hlavu.“
\textsuperscript{17}Izraelci tak učinili a nasbírali někdo více, někdo méně.
\textsuperscript{18}Pak odměřovali po ómeru. Ten, kdo nasbíral mnoho, neměl nadbytek, a kdo nasbíral málo, neměl nedostatek. Nasbírali tolik, kolik každý k jídlu potřeboval.
\textsuperscript{19}Mojžíš jim řekl: „Nikdo ať si nenechává nic do rána!“
\textsuperscript{20}Ale oni Mojžíše neposlechli a někteří si něco do rána nechali. To však zčervivělo a páchlo. Mojžíš se na ně rozlítil.
\textsuperscript{21}Sbírali to tak ráno co ráno, kolik každý k jídlu potřeboval. Když však začalo hřát slunce, rozpustilo se to.
}

\subsection*{Reflexe}
Izraelité putující po poušti, zcela závislí na Boží prozřetelnosti, k Němu volají kvůli potravě. Když Bůh naslouchá jejich modlitbě,
velkoryse reaguje manou z nebe, která se zázračně objevuje v ranní rose. Nezní to povědomě? Kolikrát jsme slyšeli
v eucharistické modlitbě při mši svaté: „Sešli rosu svého Ducha také na tyto dary a posvěť je…“?

Stejně jako Izraelité se každý z nás ocitá v poušti života. Také my voláme k Bohu za potravu. Bez výjimky nám Bůh dává všechno,
co potřebujeme. Kristus slibuje: „Já jsem chléb života. Vaši otcové jedli na poušti manu, a zemřeli. Toto je chléb, který sestupuje z
nebe: kdo z něho jí, nezemře,“ (J 6,48-50). Izraelci jedli v poušti chléb daný Bohem. Stejně tak se při mši modlíme, aby Bůh poslal
svého Ducha, aby se chléb a víno staly tělem a krví Ježíše Krista pro naši věčnou výživu.

Když chléb sestoupil z nebe, Izraelci byli ohromeni a ptali se: „Co je to?“ Také my se nacházíme ve stejném údivu. Mohl by
Kristus ve skutečnosti odevzdat sám sebe cele pro naši spásu? Bez many by Izraelité nebyli schopni přežít. Můžeme skutečně žít
bez Eucharistie? Vezměte tuto zásadní otázku do své svaté hodiny.



%newday
\newpage
\section{Den 41 - BŮH MYSLÍ SABAT VÁŽNĚ }
\zacatekSestyTyden
\subsection*{Čtení na den}
\textbf{Exodus 16,22-36}
\newline
\textit{
\textsuperscript{22}Šestého dne nasbírali toho chleba dvakrát tolik, totiž dva ómery na osobu. Tu přišli všichni předáci pospolitosti a oznámili to Mojžíšovi.
\textsuperscript{23}Ten jim řekl: „Toto praví Hospodin: Zítra je slavnost odpočinutí, Hospodinův svatý den odpočinku. Co je třeba napéci, napečte, a co je třeba uvařit, uvařte. A vše, co přebývá, uložte a opatrujte do rána.“
\textsuperscript{24}Uložili to tedy do rána, jak Mojžíš přikázal. A nezapáchalo to, ani se do toho nedali červi.
\textsuperscript{25}Mojžíš pak řekl: „Snězte to dnes, protože dnes je Hospodinův den odpočinku. Dnes nenajdete na poli nic.
\textsuperscript{26}Šest dní budete sbírat, ale sedmý den je den odpočinku. Ten den nebude nic padat.“
\textsuperscript{27}Když přesto někteří z lidu sedmého dne vyšli, aby sbírali, nic nenašli.
\textsuperscript{28}Hospodin řekl Mojžíšovi: „Jak dlouho se budete zpěčovat a nebudete dbát mých příkazů a řádů?
\textsuperscript{29}Hleďte, vždyť Hospodin vám dal den odpočinku. Proto vám dává šestého dne chléb na dva dny. Zůstaňte každý, kde jste, ať nikdo sedmého dne nevychází ze svého místa.“
\textsuperscript{30}Lid tedy sedmého dne odpočíval.
\textsuperscript{31}Dům izraelský pojmenoval ten pokrm mana. Byl jako koriandrové semeno, bílý, a chutnal jako medový koláč.
\textsuperscript{32}Mojžíš řekl: „Hospodin přikázal toto: Naplň tím ómer, aby to bylo opatrováno po všechna vaše pokolení, aby viděla chléb, kterým jsem vás na poušti živil, když jsem vás vyvedl z egyptské země.“
\textsuperscript{33}Áronovi Mojžíš řekl: „Vezmi jeden džbán, nasyp do něho plný ómer many a ulož to před Hospodinem, aby to bylo opatrováno po všechna vaše pokolení.“
\textsuperscript{34}Áron to tedy uložil před schránou svědectví, aby to bylo opatrováno, jak Hospodin Mojžíšovi přikázal.
\textsuperscript{35}Izraelci jedli manu po čtyřicet let, dokud nepřišli do země, v níž se měli usadit; jedli manu, dokud nepřišli na pokraj kenaanské země.
\textsuperscript{36}Ómer je desetina éfy.
}

\subsection*{Reflexe}
Šestého dne je Izraelitům řečeno, aby shromáždili dostatek many na dva dny, aby sedmého dne v sobotu mohli
odpočívat. Tento příkaz odráží nařízení z Genesis: „A Bůh požehnal a posvětil sedmý den, neboť v něm přestal konat
veškeré své stvořitelské dílo,“ (Gen 2,3). Také předznamenává třetí z deseti přikázání (pomni, abys den sváteční světil),
které později přijde v knize Exodus.

Nejen že stvoření dodržuje Šabat, ale činí tak i Bůh. Všimněte si, že mana v sobotu není seslána. Někteří z Izraelitů
stále chtějí nalézt manu, ale není tu žádná ke sběru. Sobora je Pánu svatá, je to jeho znamení smlouvy s Adamem a On
sám bere tento den se slavnostním odpočinkem. To nám říká, že i dnes v nedělním odpočinku spočíváme úmyslně
s Bohem.

Padlý člověk si vybudoval svět, který jej udržuje tak zaneprázdněného, že nemá žádnou chvíli na to, aby se setkal
s Bohem – nebo s kýmkoli jiným. Zvláště v neděli. Zdá se, že většina lidí si volí neděli jako den práce na zahradě nebo
nákupu potravin. Bůh si však přeje něco jiného. Čtěte toto Písmo pozorně. Bůh přikazuje Izraelitům, aby nechodili na
pole (nedělali práci na zahradě), ani nesbírali manu (nenakupovali potraviny) v sobotu. To jsou jen dva z mnoha
příkladů.

Dr. Scott Hahn poukazuje na to, že v První knize Mojžíšově byl stvořen člověk i šelmy šestého dne, ale člověk byl
stvořen pro den sedmý. Když člověk o sedmý den pracuje, padá zpět do šestého dne a není o nic lepší než šelma. Jste
zotročeni k tělesným impulzům a lidským plánům? Je aší ženě každou neděli s radostí připomínáno, že se provdala za
muže, a nebo se již dívá jen na šelmu?

Učení o odpočinku soboty je pro lidi velmi těžké slyšet. Bůh však učinil Jeho vůli pro nás jasou skrze Jeho Slovo.
Pohovořte dnes s Bohem o vlastní věrnosti odpočinku v neděli. Buďte připraveni slyšet nepříjemnou pravdu od Toho,
který chce vaši úplnou svobodu. Věřte. Bůh vám poskytne potřebnou milost, abyste mohli plnit Jeho příkazy.


%newday
\newpage
\section{Den 42 - PÍSMO VE VÁS}
\zacatekSestyTyden
\subsection*{Čtení na den}
\textbf{Exodus 17,1-7}
\newline
\textit{
\textsuperscript{1}Celá pospolitost synů Izraele táhla z pouště Sínu od stanoviště ke stanovišti podle Hospodinova rozkazu. Utábořili se v Refídimu, ale lid neměl vodu k pití.
\textsuperscript{2}Tu se lid dostal do sváru s Mojžíšem a naléhali: „Dejte nám vodu, chceme pít!“ Mojžíš se jich zeptal: „Proč se se mnou přete? Proč pokoušíte Hospodina?“
\textsuperscript{3}Lid tam žíznil po vodě a reptal proti Mojžíšovi. Vyčítali: „Proto jsi nás vyvedl z Egypta, abys nás, naše syny a stáda umořil žízní?“
\textsuperscript{4}Mojžíš úpěl k Hospodinu: „Jak se mám vůči tomuto lidu zachovat? Taktak že mě neukamenují.“
\textsuperscript{5}Hospodin Mojžíšovi řekl: „Vyjdi před lid. Vezmi s sebou některé z izraelských starších. Také hůl, kterou jsi udeřil do Nilu, si vezmi do ruky a jdi.
\textsuperscript{6}Já tam budu stát před tebou na skále na Chorébu. Udeříš do skály a vyjde z ní voda, aby lid mohl pít.“ Mojžíš to udělal před očima izraelských starších.
\textsuperscript{7}To místo pojmenoval Massa a Meriba (to je Pokušení a Svár) podle sváru Izraelců a proto, že pokoušeli Hospodina pochybováním: „Je mezi námi Hospodin, nebo není?“
}

\subsection*{Reflexe}
“Je mezi námi Hospodin, nebo není?“ Dnes se i nadále ptáme stejnou otázkou, právě jako se jí ptali lidé po celá staletí.

Tak často žijeme způsobem, jako by Bůh neexistoval. Zřídka rozjímáme Jeho přítomnost nebo moudrost, málokdy Jej
vezmeme v potaz nebo děkujeme a chválíme Jej za všechny milosti v našem životě. Poté, když něco dopadne špatně, je
až příliš jednoduché obviňovat Boha a se spravedlivým rozhořčením Jej zatratit kvůli tomu, že nevnímáme Jeho
přítomnost nebo činnost.

Dnes dojdou Izraelci tak daleko, že zkoušejí Boha, kladou na Něj nároky a jsou jen kousek toho, aby Ho prokleli za
jejich neštěstí. Aby jim prokázal svou péči a prozřetelnost, přikazuje Mojžíšovi, aby udeřil do skály Áronovou holí. Při
tom se ze skály vyleje voda. Bůh nás obdaroval stejným zázrakem. Když Ježíš Kristus visel na kříži, jeden z vojáků ho
kopím „udeřil“, probodl bok, a okamžitě vyšla krev a voda. Kristus je ona skála, ze které se nám vylila voda a krev.
Bůh je laskavý dárce. „Je mezi námi Hospodin, nebo není?“ Odpověď zní ano. Není jen mezi námi, je v nás.

Na rozdíl od Izraelitů máme my, pokřtění křesťané, v sobě Ducha Krista, který v nás doslova přebývá. Vše, co
potřebujeme, nám již bylo dáno. Kristus jasně říká: „Kdo věří ve mne, ‚proudy živé vody poplynou z jeho nitra,‘ jak
praví Písmo,“ (J 7,38). Máte žízeň? Duch Krista z vás prýští. Napijte se dnes z vody, která dává život, a začněte z ní
žít. Popovídejte si dnes s Pánem o tom, jak se k ní dostat, a jak pít a žít z této vody.



% ===============================================
% ===== SEDMY TYDEN
% ===============================================
%ukony
\newpage
\section*{Úkony (ukazatel cesty) pro 7. týden}

\textbf{Místo:} Drsná hornatá poušť, úpatí hory Sinaj

Dar znovunalezené svobody byl pro Izraelity vybojován v hlubinách Rudého moře. Nyní je Pán dovedl až
k úpatí hory Sinaj. Na vrcholu této hory promluví s Mojžíšem a dá mu návod, se kterým bude schopný udržet
sebe a Izraelity už navždy svobodné.

Vytrvali jste až do 7. týdne. To znamená, že jste v polovině tohoto duchovního cvičení. Gratulujeme. Odpoutaní
od model vašeho předchozího života jako nikdy předtím pokračujete s návodem ke svobodě v rukách: modlitba,
askeze a bratrství. Pokud tento návod nyní odmítnete, odmítnete samotný dar svobody. Stále je před námi
mnoho práce. Váš starý život může být za vámi, ale nové zvyky/návyky stále potřebují dostatek času, aby se
vytvořily.
\subsection*{1. Ctěte svou svatou hodinu}
Mnoho věcí vám chce Bůh sdělit. Touží s vámi sdílet Jeho samého. Dávejte mu stále denně čas v modlitbě. Nacházet výmluvy a říkat, že není hodný našeho času, znamená urážet samotného Boha. Zcela nás miluje a zaslouží si od nás mnohem víc než pouhé výmluvy.
\subsection*{2. Běžte ke zdroji}
„Eucharistie je zdrojem a vrcholem celého křesťanského života,“ (KKC 1324). Jak se vám vedlo v docházení na jednu další mši svatou během týdne? Pokud jste Ježíše Krista v Eucharistii nepotkávali jednou do týdně častěji, zeptejte se sami sebe, proč. Bůh sám dává vysvobození. Udělejte si dnes čas ve svém týdenním rozvrhu k navštívení další mše.
\subsection*{3. Přistupujte k vašemu nočnímu examen zodpovědně}
Pokud se vám vedlo praktikovat noční examen (zkoumání dne), uvidíte ve svém životě svobodu a úspěch, kterým vás Bůh obdařil. Pokud jste své noční examen nepraktikovali, váš osobní úspěch 91. dne vypadá bledě. Začněte zkoumat svůj den každou noc a důkladně. (Vysvětlení, jak dělat noční examen, můžete nalézt v Průvodci terénem.)
\subsection*{4. Zůstaňte radostní}
Minulý týden vás Hospodin vysvobodil z Egypta cestou Rudého moře. Tento týden se nacházíte uprostřed devadesátidenní cesty. Můžete děkovat za mnoho. Můžete mít z mnoha věcí radost. Žijte v této radosti, takže i ostatní uvidí a přijdou na to, že Bůh aktivně dává lidem vysvobození.

\subsection*{Modlitba}
Modlete se, aby Pán osvobodil vás a vaše bratrství \newline
Modleme se za svobodu všech mužů v exodu, stejně tak, jako se oni modlí za vás.\newline
Ve jménu Otce i Syna i Ducha svatého … Otče náš… Ve jménu Otce i Syna i Ducha svatého … Amen.
\newpage


%newday
\newpage
\section{Den 43 - UDĚLÁ I TO SMĚŠNÉ }
\zacatekSedmyTyden
\subsection*{Čtení na den}
\textbf{Exodus 17,8-16}
\newline
\textit{
\textsuperscript{8}Tu přitáhl Amálek, aby v Refídimu bojoval s Izraelem.
\textsuperscript{9}Mojžíš rozkázal Jozuovi: „Vyber nám muže a vyjdi do boje proti Amálekovi. Já se zítra postavím na vrchol pahorku s Hospodinovou holí v ruce.“
\textsuperscript{10}Jozue učinil, jak mu Mojžíš rozkázal, a dal se s Amálekem do boje. Mojžíš, Áron a Chúr vystoupili na vrchol pahorku.
\textsuperscript{11}Dokud Mojžíš držel ruku nahoře, vítězil Izrael, když ruku spustil, vítězil Amálek.
\textsuperscript{12}Když Mojžíšovi umdlévaly ruce, vzali kámen a podložili jej pod Mojžíše, aby se na něj posadil. Áron a Chúr, každý z jedné strany, mu podpírali ruce, takže vytrval s rukama nahoře až do západu slunce.
\textsuperscript{13}I porazil Jozue Amáleka a jeho lid ostřím meče.
\textsuperscript{14}Hospodin řekl Mojžíšovi: „Zapiš na památku do knihy a předej Jozuovi, že zcela vymažu zpod nebes památku na Amáleka.“
\textsuperscript{15}I vybudoval Mojžíš oltář a pojmenoval jej: ‚Hospodin je má korouhev.‘
\textsuperscript{16}Řekl totiž: „Je vztažena ruka nad Hospodinovým trůnem. Hospodin vyhlašuje boj proti Amálekovi do posledního pokolení.“
}

\subsection*{Reflexe}

Amálek, což doslova znamená „hříšní lidé“, odmítá Izraelitům projít směrem k jejich zaslíbené zemi. Postavili
se na odpor Boží vůli. Jsou vůči Boží vůli tvrdohlaví a čelí Izraelitům na základě jejich oprávnění a pýchy. Zde
nám svatý Augustin předkládá svou moudrost: pýcha „se stává překážkou věcem vyšším, a spojuje nás s věcmi
nižšími.“ Jak můžete toto aplikovat do svého současného života?

Vyzbrojen svatou ctností pokory Mojžíš sleduje bitvu s rukama nahoře, aby zajistil vítězství Izraelitů, zatímco
Jozue shromažďuje vojsko. Amalék jde prudce do války. Tento hříšný lid má schopnost porazit Izraelity. Ale
přesto nezvítězí. Proč? Protože Mojžíš je připraven udělat i to směšné. Má dost pokory na to, aby se poddal
Boží vůli, nehledě na to, jak bláznivě nebo slabě vypadá před ostatními. Mojžíš nechává své ruce zvednuté
k Bohu. Na druhé straně Bůh zajistí velké vítězství svého lidu.

Pokud se vyzbrojíte ctností pokory, Boží vítězství skrze vás je v podstatě nevyhnutelné. Pokud se však necháte
ochromit myšlenkou, že vás vaše síla a vůle dovedou k vítězství, váš osud bude stejný jako Amalečanů. Co
z toho, co po vás Pán žádá, je pro vás moc směšné, než abyste to učinili? Pohovořte si s Ním dnes o tom.
Možná vám ukáže něco, co jste pro svou pýchu nebyli schopni vidět.


%newday
\newpage
\section{Den 44 - PRAVDA O SOBĚ SAMOTNÉM }
\zacatekSedmyTyden
\subsection*{Čtení na den}
\textbf{Exodus 18,1-27}
\newline
\textit{
\textsuperscript{1}Jitro, midjánský kněz, Mojžíšův tchán, uslyšel o všem, co Bůh učinil Mojžíšovi a svému izraelskému lidu, že Hospodin vyvedl Izraele z Egypta.
\textsuperscript{2}Tu vzal Jitro, Mojžíšův tchán, Siporu, Mojžíšovu manželku, kterou on poslal zpět,
\textsuperscript{3}a dva její syny; první se jmenoval Geršóm (to je Hostem-tam), neboť Mojžíš řekl: „Byl jsem hostem v cizí zemi,“
\textsuperscript{4}a druhý se jmenoval Elíezer (to je Bůh-je-pomoc), neboť řekl : „Bůh mého otce je má pomoc, vysvobodil mě od faraónova meče.“
\textsuperscript{5}Jitro, Mojžíšův tchán, přišel k němu s jeho syny a manželkou na poušť, kde on tábořil, k hoře Boží.
\textsuperscript{6}Vzkázal Mojžíšovi: „Já, tvůj tchán Jitro, jsem přišel k tobě, i tvoje manželka a s ní oba její synové.“
\textsuperscript{7}Mojžíš tedy vyšel svému tchánovi vstříc, poklonil se a políbil ho. Popřáli si navzájem pokoj a vešli do stanu.
\textsuperscript{8}Mojžíš vypravoval svému tchánovi o všem, co Hospodin kvůli Izraeli učinil faraónovi a Egypťanům, a o všech útrapách, které je potkaly na cestě, a jak je Hospodin vysvobodil.
\textsuperscript{9}Jitro měl radost ze všeho dobrého, co Hospodin Izraeli prokázal, a že jej vysvobodil z moci Egypta.
\textsuperscript{10}Řekl: „Požehnán buď Hospodin, že vás vysvobodil z moci Egypta a z ruky faraónovy, že vysvobodil tento lid z područí Egypta.
\textsuperscript{11}Nyní jsem poznal, že Hospodin je větší než všichni bohové; odplatil jim podle toho, jak se vypínali nad Izraele.“
\textsuperscript{12}Jitro, tchán Mojžíšův, pak připravil Bohu zápalnou oběť a obětní hod. Áron a všichni izraelští starší přistoupili, aby s Mojžíšovým tchánem pojedli před Bohem chléb.
\textsuperscript{13}Nazítří se Mojžíš posadil, aby soudil lid. Lid musel stát před Mojžíšem od rána do večera.
\textsuperscript{14}Mojžíšův tchán se díval na celé jeho jednání s lidem a řekl: „Jakým způsobem to s lidem jednáš? Proč sám sedíš a všechen lid kolem tebe stojí od rána do večera?“
\textsuperscript{15}Mojžíš tchánovi odpověděl: „Lid ke mně přichází dotazovat se Boha.
\textsuperscript{16}Když něco mají, přijde ta záležitost přede mne a já rozsoudím mezi oběma stranami; učím je znát Boží nařízení a řády.“
\textsuperscript{17}Mojžíšův tchán mu odpověděl: „Není to dobrý způsob, jak to děláš.
\textsuperscript{18}Úplně se vyčerpáš, stejně jako tento lid, který je s tebou. Je to pro tebe příliš obtížné. Sám to nezvládneš.
\textsuperscript{19}Poslechni mě, poradím ti a Bůh bude s tebou: Ty zastupuj lid před Bohem a přednášej jejich záležitosti Bohu.
\textsuperscript{20}Budeš jim vysvětlovat nařízení a řády a učit je znát cestu, po které mají chodit, i skutky, které mají činit.
\textsuperscript{21}Vyhlédni si pak ze všeho lidu schopné muže, kteří se bojí Boha, milují pravdu a nenávidí úplatek. Dosaď je nad nimi za správce nad tisíci, sty, padesáti a deseti.
\textsuperscript{22}Oni budou soudit lid, kdykoli bude třeba. Každou důležitou záležitost přednesou tobě, každou menší záležitost rozsoudí sami. Ulehči si své břímě, ať je nesou s tebou.
\textsuperscript{23}Jestliže se podle toho zařídíš, budeš moci obstát, až ti Bůh vydá další příkazy. Také všechen tento lid dojde na své místo v pokoji.“
\textsuperscript{24}Mojžíš svého tchána uposlechl a učinil všechno, co řekl.
\textsuperscript{25}Vybral schopné muže ze všeho Izraele a ustanovil je za představitele lidu, za správce nad tisíci, sty, padesáti a deseti.
\textsuperscript{26}Ti soudili lid, kdykoli bylo třeba ; obtížné záležitosti přednášeli Mojžíšovi a všechny menší záležitosti soudili sami.
\textsuperscript{27}Potom Mojžíš svého tchána propustil a ten se ubíral do své země.
}

\subsection*{Reflexe}
Mojžíš se osvědčil a nyní se těší respektu a úcty svého lidu. Lidé ho vnímají jako prostředníka, muže, který
mluví za samotného Boha. Mojžíš si vysloužil velkou autoritu. A jako jsme již viděli, jedná s pokorou a úctou.
Když Jitro, jeho tchán, vstoupí do tábora, srdečně se pozdraví a oslavují společně svobodu Izraelitů.

Jitro, muž sloužící jako kněz svému lidu, napomíná Mojžíše, varuje ho před úskalími a radí mu, aby ustanovil
soudní systém. Mojžíš nepovažuje zásah svého tchána za hrozbu nebo útok na svou autoritu. Člověk, který
toho tolik dosáhl, by takovou radu mohl odmítnout jako urážku, ale Mojžíš přijme tuto radu a jedná podle ní.
Tím dává všem lidem příklad pokory.

Pokora je hluboce nepochopená ctnost. Skutečná pokora znamená znát pravdu o sobě. Člověk tedy může být
současně veliký i pokorný. Ten, kdo zná své silné i slabé stránky, své schopnosti i svá omezení, a obzvlášť
svou náklonnost k hříchu, je pokorný. Jedná spravedlivě a nechlubí se. Ví, že mu bylo všechno dáno jako dar,
a to dar nezasloužený. Je si vědom toho, že je zloděj, pokud sklízí uznání za jakoukoli dobrou práci, protože
se pokouší ukrást slávu Bohu.

Zvažte, kdo jste jako člověk. Jste člověk velký? Pokorný? Požádejte Pána, aby promluvil do toho, kým jste.
Zná pravdu o vás, lépe než vy samotní. Jste jeho syny.


%newday
\newpage
\section{Den 45 - SMYSL POSVÁTNA}
\zacatekSedmyTyden
\subsection*{Čtení na den}
\textbf{Exodus 19,1-15}
\newline
\textit{
\textsuperscript{1}Třetího měsíce potom, co Izraelci vyšli z egyptské země, téhož dne, přišli na Sínajskou poušť.
\textsuperscript{2}Vytáhli z Refídimu, přišli na Sínajskou poušť a utábořili se v poušti; Izrael se tam utábořil naproti hoře.
\textsuperscript{3}Mojžíš vystoupil k Bohu. Hospodin k němu zavolal z hory: „Toto povíš domu Jákobovu a oznámíš synům Izraele:
\textsuperscript{4}Vy sami jste viděli, co jsem učinil Egyptu. Nesl jsem vás na orlích křídlech a přivedl vás k sobě.
\textsuperscript{5}Nyní tedy, budete-li mě skutečně poslouchat a dodržovat mou smlouvu, budete mi zvláštním vlastnictvím jako žádný jiný lid, třebaže má je celá země.
\textsuperscript{6}Budete mi královstvím kněží, pronárodem svatým. To jsou slova, která promluvíš k synům Izraele.“
\textsuperscript{7}Mojžíš přišel, zavolal starší lidu a předložil jim všechno, co mu Hospodin přikázal.
\textsuperscript{8}Všechen lid odpověděl jednomyslně: „Budeme dělat všechno, co nám Hospodin uložil.“ Mojžíš tlumočil odpověď lidu Hospodinu.
\textsuperscript{9}Hospodin řekl Mojžíšovi: „Hle, přijdu k tobě v hustém oblaku, aby lid slyšel, až s tebou budu mluvit, a aby ti provždy věřili.“ Mojžíš totiž Hospodinu oznámil slova lidu.
\textsuperscript{10}Hospodin dále Mojžíšovi řekl: „Jdi k lidu a dnes i zítra je posvěcuj; ať si vyperou pláště
\textsuperscript{11}a ať jsou připraveni na třetí den, neboť třetího dne sestoupí Hospodin před zraky všeho lidu na horu Sínaj.
\textsuperscript{12}Vymezíš kolem lidu hranici a řekneš: Střezte se vystoupit na horu nebo i dotknout se jejího okraje. Kdokoli se hory dotkne, musí zemřít;
\textsuperscript{13}nedotkne se ho žádná ruka, bude ukamenován nebo zastřelen. Ať je to dobytče nebo člověk, nezůstane naživu. Teprve až se dlouze zatroubí na roh, smějí na horu vystoupit.“
\textsuperscript{14}Mojžíš sestoupil z hory k lidu, posvětil lid a oni si vyprali pláště.
\textsuperscript{15}Řekl také lidu: „Buďte připraveni na třetí den; nepřistupujte k ženě.“
}

\subsection*{Reflexe}
Lid Izraele se připravuje na setkání s Bohem. Považte všechny přípravy, které musí podstoupit předtím, než se setkají
s Bohem jejich otců: praní prádla, abstinence a úcta. Lidem je připomínáno, že Bůh je zcela svatý, a je jim zakázáno se
Ho dotknout nebo jen přistoupit k hoře, na které bude Mojžíš mluvit s Bohem. Hora je tak posvátná, že jen dotknutí se
jí by přineslo smrt.

Dnes jsme ztratili smysl posvátna. Lidé jen ojediněle stojí před Hospodinem v úžasu a rozjímají nad svou
bezvýznamností před Ním. Pryč jsou dny, kdy lidé vcházeli do svatyní s velkou úctou. Ty dny, kdy lidé nevcházeli do
svatyně, aniž by se řádně oblékli, duchovně připravili, a činili tak s konkrétním smyslem. Zamyslete se na chvíli nad
vaším farním kostelem. Vcházíte do něj citlivě a s úctou? Přistupujete do svatyně zbožně, a staráte se, abyste se vhodně
oblékli a připravili na vstup do ní? Procházíte se svatyní, jako by to nebylo nic jiného než průchod na jiné místo v kostele?

Bůh říká Izraelitům, aby vyprali své pláště, a připravili se tak na Hospodina. Dnes přichází dospělí lidé na mši svatou
neupravení, v žabkách a kraťasech, džínách a triku nebo mikině. „Aspoň jsem tady,“ říkají pro svou obhajobu.

Katolík, který touží po synovském vztahu s Bohem Otcem, by měl před Pána předstupovat řádně připraven – a to zahrnuje
i základní vnější úpravu. Zeptejte se dnes Pána, jak byste Jej podle Jeho přání měli uctívat při mši svaté. Buďte ochotni
naplnit Boží požadavky, stejně jako Mojžíš a Izraelité. Bůh za to stojí.


%newday
\newpage
\section{Den 46 - ÚCTA K BOHU}
\zacatekSedmyTyden
\subsection*{Čtení na den}
\textbf{Exodus 19,16-25}
\newline
\textit{
\textsuperscript{16}Když nadešel třetí den a nastalo jitro, hřmělo a blýskalo se, na hoře byl těžký oblak a zazněl velmi pronikavý zvuk polnice. Všechen lid, který byl v táboře, se třásl.
\textsuperscript{17}Mojžíš vyvedl lid z tábora vstříc Bohu a postavili se při úpatí hory.
\textsuperscript{18}Celá hora Sínaj byla zahalena kouřem, protože Hospodin na ni sestoupil v ohni. Kouř z ní stoupal jako z hutě a celá hora se silně chvěla.
\textsuperscript{19}Zvuk polnice víc a více sílil. Mojžíš mluvil a Bůh mu hlasitě odpovídal.
\textsuperscript{20}Hospodin totiž sestoupil na horu Sínaj, na vrchol hory. Zavolal Mojžíše na vrchol hory a Mojžíš tam vystoupil.
\textsuperscript{21}Hospodin Mojžíšovi řekl: „Sestup a varuj lid, aby se nikdo ne pokoušel proniknout k Hospodinu ve snaze ho uvidět. Mnoho by jich padlo.
\textsuperscript{22}Také kněží, kteří přistupují k Hospodinu, se musí posvětit, aby se na ně Hospodin neobořil.“
\textsuperscript{23}Mojžíš řekl Hospodinu: „Lid nemůže vystoupit na horu Sínaj, neboť ty sám jsi nás varoval slovy: Vymez podél hory hranici a horu posvěť.“
\textsuperscript{24}Hospodin mu řekl: „Teď sestup, potom vystoupíš spolu s Áronem; ale kněží ani lid nesmějí proniknout a vystoupit k Hospodinu, aby se na ně neobořil.“
\textsuperscript{25}Mojžíš tedy sestoupil k lidu a řekl jim to.
}

\subsection*{Reflexe}
Písmo nám říká: „Začátek moudrosti je bázeň před Hospodinem a poznat Svatého je rozumnost,“ (Přís 9,10). Co si pomyslíte, když
to slyšíte? Až moc dlouho byl Bůh zobrazován kazateli a učiteli jako květinová a mírná postava, naprosto přístupný, nenahánějící
strach. Je to opačně. Bůh je naprosto nepochopitelný a všemocný. Člověk klame sám sebe, pokud považuje Boha za neškodného.

Řádná bázeň vyvolává úctu. Zkušení horolezci mají strach ze sněhu. Sníh se většině lidí zdá neškodný, ale pro horolezce je sněhová
lavina skutečnou možností smrti. Proto horolezci respektují sílu sněhu tím, že sledují počasí, testují sněhovou pokrývku a
odpovídajícím způsobem upravují své trekové plány. Navzdory tomu, co si lidé myslí, sníh zdaleka není neškodný. Stejně – a
mnohem víc – je to i s Bohem.

V dnešním úryvku jsou Izraelité zvukem polnice upozorněni na přítomnost Boha. Zvuk polnice se v Písmu svatém objevuje ze dvou
důvodů: svolat muže k bitvě, a svolat lid k modlitbě. Není ironií, že tato svolávání sdílí stejný signál. Troubení volá muže jak do boje
v poli, tak k modlitbě před Bohem.

Zamyslete se, jak často jste se modlívali před tímto duchovním cvičením. Pokud jste nevstupovali na bojiště modlitby denně, bylo
to kvůli nedostatku času, nebo proto, že jste se vlastně řádně nebáli Boha?

Začínáte ve zkušenosti exodu vidět sílu a moc Boha dát život a vzít ho zpět? Vidíte svou potřebu Boha uctívat? Nebo Ho stále vidíte
očima městského člověka, který vidí párkrát ročně neškodný sníh pomalu se snášející na město? Polnice zazněla. Vaše volba
(ne)přistupovat denně k modlitbě již beze slov odpověděla na tuto otázku.

Pokud jste do své svaté hodiny posledních 46 dní vstupovali dobrovolně a věrně, vzdejte Bohu velkou chválu. Vaše oči dobře vidí
Jeho moc a sílu. Pokud jste se závazkem modlitby během tohoto duchovního cvičení bojovali, předneste to Pánu. Dejte Mu šanci
podělit se s vámi o věčný přínos učení se bázni a Boží úcty. On je dobrý, a rozhovor s Ním bude stát za to.


%newday
\newpage
\section{Den 47 - DAR PŘIKÁZÁNÍ }
\zacatekSedmyTyden
\subsection*{Čtení na den}
\textbf{Exodus 20,1-17}
\newline
\textit{
\textsuperscript{1}Bůh vyhlásil všechna tato přikázání:
\textsuperscript{2}„Já jsem Hospodin, tvůj Bůh; já jsem tě vyvedl z egyptské země, z domu otroctví.
\textsuperscript{3}Nebudeš mít jiného boha mimo mne.
\textsuperscript{4}Nezobrazíš si Boha zpodobením ničeho, co je nahoře na nebi, dole na zemi nebo ve vodách pod zemí.
\textsuperscript{5}Nebudeš se ničemu takovému klanět ani tomu sloužit. Já jsem Hospodin, tvůj Bůh, Bůh žárlivě milující. Stíhám vinu otců na synech do třetího i čtvrtého pokolení těch, kteří mě nenávidí,
\textsuperscript{6}ale prokazuji milosrdenství tisícům pokolení těch, kteří mě milují a má přikázání zachovávají.
\textsuperscript{7}Nezneužiješ jména Hospodina, svého Boha. Hospodin nenechá bez trestu toho, kdo by jeho jména zneužíval.
\textsuperscript{8}Pamatuj na den odpočinku, že ti má být svatý.
\textsuperscript{9}Šest dní budeš pracovat a dělat všechnu svou práci.
\textsuperscript{10}Ale sedmý den je den odpočinutí Hospodina, tvého Boha. Nebudeš dělat žádnou práci ani ty ani tvůj syn a tvá dcera ani tvůj otrok a tvá otrokyně ani tvé dobytče ani tvůj host, který žije v tvých branách.
\textsuperscript{11}V šesti dnech učinil Hospodin nebe i zemi, moře a všechno, co je v nich, a sedmého dne odpočinul. Proto požehnal Hospodin den odpočinku a oddělil jej jako svatý.
\textsuperscript{12}Cti svého otce i matku, abys byl dlouho živ na zemi, kterou ti dává Hospodin, tvůj Bůh.
\textsuperscript{13}Nezabiješ.
\textsuperscript{14}Nesesmilníš.
\textsuperscript{15}Nepokradeš.
\textsuperscript{16}Nevydáš proti svému bližnímu křivé svědectví.
\textsuperscript{17}Nebudeš dychtit po domě svého bližního. Nebudeš dychtit po ženě svého bližního ani po jeho otroku ani po jeho otrokyni ani po jeho býku ani po jeho oslu, vůbec po ničem, co patří tvému bližnímu.“
}

\subsection*{Reflexe}
Izraelité právě unikli z let služby faraonovi a zásahem zázračné a mocné ruky Hospodinovy si užívají skvělý
dar a výsadu tělesné svobody. Ale zde se najednou zdá, že je Bůh chce okovy deseti přikázání o svobodu
připravit. Proč se to děje? Nejsou snad svobodní, schopní činit svá rozhodnutí a směřovat svůj vlastní osud?
Proč se zdá, že se Bůh povyšuje nad svůj lid?

Bůh tím, že člověka stvořil a obdařil ho svobodou, velmi riskoval. Přeci to však udělal z jednoho důvodu: aby
člověk měl schopnost milovat Ho. Bez svobody není člověk schopný lásky. Kdyby nás Bůh stvořil jako roboty
a naprogramoval nás pro lásku k Němu, náš vztah s Ním by nebyl úplný. Nebyli bychom schopni Ho milovat.

Bůh do lidské svobody mnoho vložil. Vyhrál svobodu Izraelitům a také nám – na kříži. Poslední věcí, kterou
by Bůh chtěl vidět, by byla ztráta naší svobody. Deset přikázání je tedy vlastně záruka od Boha, vzor, jak
svobodu Izraelitů – a nás – zachovat.

Zamyslete se nad přikázáními: každé je navrženo tak, abychom nebyli znovuzotročeni peklem. Kdyžněkteré
z těchto přikázání porušíme, sami se připravíme o pravou svobodu. Přikázání na nás nejsou uvalena
despotickým Bohem; jsou neskutečně láskyplným darem Izraelitům – a nám. Zákony Církve, které z těchto
přikázání pramení, slouží stejnému účelu. Proto nejsou přikázání a zákony okovy, ale dary, které nám pomáhají
svobodně a správně milovat Boha.

Zvažte, které z přikázání nebo zákona Boha a Jeho Církve se vám nedaří přijmout. Přinese to Pánu a požádejte
ho, aby vám ukázal, jak vám to má pomoci milovat Boha svobodně a správně. (Pokud se vám nedaří přijmout
pokoj nad touto odpovědí, předneste to svému bratrstvu a vašemu duchovnímu vůdci pro objasnění.)


%newday
\newpage
\section{Den 48 - SCHOPEN VĚTŠÍ LÁSKY }
\zacatekSedmyTyden
\subsection*{Čtení na den}
\textbf{Exodus 20,18-26}
\newline
\textit{
\textsuperscript{18}Všechen lid pozoroval hřmění a blýskání, zvuk polnice a kouřící se horu. Lid to pozoroval, chvěl se a zůstal stát opodál.
\textsuperscript{19}Řekli Mojžíšovi: „Mluv s námi ty a budeme poslouchat. Bůh ať s námi nemluví, abychom nezemřeli.“
\textsuperscript{20}Mojžíš lidu odpověděl: „Nebojte se! Bůh přišel proto, aby vás vyzkoušel, aby bylo zřejmé, že se ho budete bát a přestanete hřešit.“
\textsuperscript{21}Lid zůstal stát opodál a Mojžíš přistoupil k mračnu, v němž byl Bůh.
\textsuperscript{22}Hospodin řekl Mojžíšovi: „Toto řekneš synům Izraele: Viděli jste, že jsem s vámi mluvil z nebe.
\textsuperscript{23}Neuděláte si mé zpodobení, neuděláte si bohy stříbrné ani zlaté.
\textsuperscript{24}Uděláš mi oltář z hlíny a budeš na něm obětovat ze svého bravu a skotu své oběti zápalné i pokojné. Na každém místě, kde určím, aby se připomínalo mé jméno, přijdu k tobě a požehnám ti.
\textsuperscript{25}Jestliže mi budeš dělat oltář z kamenů, neotesávej je; kdybys je opracoval dlátem, znesvětil bys je.
\textsuperscript{26}Nebudeš vystupovat k mému oltáři po stupních, abys u něho neodkrýval svou nahotu.“
}

\subsection*{Reflexe}
Čtěte dnes Písmo pozorně: „Bůh přišel proto, aby vás vyzkoušel, aby bylo zřejmé, že se ho budete bát a přestanete hřešit.“ Proč
Hospodin přišel, aby mluvil s Mojžíšem na hoře Sinaj? Proč je zde hřmění, blýskání, zvuk polnice, kouř z hory, a dlouhý seznam
zákonů, které začínají dneškem? Protože Hospodin chce, aby Jeho lid žil správně. Chce, aby lidé žili svobodni od hříchu. Chce, aby
Ho milovali tolik, jak jsou schopni.

Čím více víme o někom jiném, tím lépe ho můžeme milovat. Čím častěji se Izraelité setkávají s Bohem v jejich nepřízni, tím Ho více
poznávají. Z toho vyplývá, že schopnost Izraelitů milovat Boha se zvyšuje s každým setkáním. Hospodin proto volá Izraelity k takové
úrovni lásky, které jsou schopni. To je vidět v dnešním čtení, kde Bůh připomíná svému lidu: „Viděli jste, že jsem s vámi mluvil z
nebe. Neuděláte si mé zpodobení, neuděláte si bohy stříbrné ani zlaté.“ Stejná myšlenka znalosti a lásky platí i pro náš vztah k Bohu.

Během svého života se o Pánu dozvídáte víc a víc. Co jste se dozvěděli formovalo způsob, jakým jste schopni milovat Boha. Tím,
co jste se dozvěděli o Ježíši Kristu, jste mohli milovat Boha Syna. To, co jste se naučili o Eucharistii, vás zase přimělo k tomu, abyste
mohli pokleknout před Synem přítomným v chlebu a vínu. Díky tomu, co jste se naučili o mystickém Kristově těle, jste zase mohli
volat k Synu skrze členy jeho Těla, společenství svatých. S tímto požehnáním přichází zodpovědnost. Čím více víme, čeho jsme více
schopni, tím víc budeme odpovědní.

Pro příklad – pokud člověk ví, že neděle je dnem svatým, určeným Bohu, potom je odpovědný za to, že podle toho bude žít. Muž
(hlava rodiny), který to ví a bere svou rodinu v neděli na mši svatou a poté ji učí slavit památku tohoto svátku, dobře prokazuje svou
lásku k Bohu. Na druhou stranu ten, kdo nedává mši na první místo a místo toho vezme svou rodinu v neděli do kempu, obchodu
nebo na sportovní utkání, zdaleka nedosahuje své schopnosti milovat HospodinaVe skutečnosti dokonce ani dost nemiluje svou
rodinu, protože je odděluje od pramene lásky (Boha) v den, který byl učiněn pouze k službě Jemu.

Během vaší dnešní kontemplace přemýšlejte nad svou vlastní schopností milovat Pána. Zeptejte se sami sebe, jestli milujete
Hospodina tak, jak jste schopní. Pak s Ním pohovořte. Zeptejte se Ho, jak Jej můžete lépe milovat. Sepište si závěry tohoto rozhovoru
a rozhodněte se milovat Pána, jak nejvíce jste schopni.


%newday
\newpage
\section{Den 49 - LÁSKA K BLIŽNÍMU VYŽADUJE AKCI}
\zacatekSedmyTyden
\subsection*{Čtení na den}
\textbf{Exodus 21,1-11}
\newline
\textit{
\textsuperscript{1}Toto jsou právní ustanovení, která jim předložíš:
\textsuperscript{2}Když koupíš hebrejského otroka, bude sloužit šest let; sedmého roku odejde jako propuštěnec bez výkupného.
\textsuperscript{3}Jestliže přišel sám, odejde sám, měl-li ženu, odejde jeho žena s ním.
\textsuperscript{4}Jestliže mu dal jeho pán ženu, která mu porodila syny nebo dcery, zůstane žena a její děti u svého pána, a on odejde sám.
\textsuperscript{5}Prohlásí-li otrok výslovně: „Zamiloval jsem si svého pána, svou ženu a syny, nechci odejít jako propuštěnec,“
\textsuperscript{6}přivede ho jeho pán před Boha, totiž přivede ho ke dveřím nebo k veřejím, probodne mu ucho šídlem a on zůstane provždy jeho otrokem.
\textsuperscript{7}Když někdo prodá svou dceru za otrokyni, nebude s ní nakládáno jako s jinými otroky.
\textsuperscript{8}Jestliže se znelíbí svému pánu, který si ji vzal za družku, dovolí ji vyplatit, ale nemá právo prodat ji cizímu lidu a naložit s ní věrolomně.
\textsuperscript{9}Jestliže ji dal za družku svému synovi, bude s ní jednat podle práva dcer.
\textsuperscript{10}Jestliže on si vezme ještě jinou, nesmí ji zkrátit na stravě, ošacení a manželském právu.
\textsuperscript{11}Jestliže jí nezajistí tyto tři věci, smí ona odejít bez zaplacení výkupného.
}

\subsection*{Reflexe}
Pro dnešního čtenáře mohou být biblické příkazy ohledně otroků velmi znepokojující. Spíše než házet obvinění na
Izraelity (a Bohasamého) z dnešního pohledu zvažte zcela jiné podmínky a kulturu Izraelitů. Život ve starověku byl tak
neuvěřitelně těžký, že otroctví bylo často přínosem spíše pro otroka než pro jeho pána. Otrok Izraelitů měl jídlo, přístřeší
a ochranu. Boží příkazy týkající se otroctví v dnešním čtení tedy byly zárukou základních lidských práv.

Více než to nám Boží nařízení dnešního čtení poukazují na něco zásadního o Bohu. Bůh se tolik stará o způsob, jakým
spolu lidé komunikují. V Novém Zákoně nám Ježíš Kristus přikazuje „milovat Boha z celého svého srdce“ a „milovat
svého bližního jako sám sebe“ (Mk 12,30-31).

Toto pravidlo je zásadní, ale vnímáte ji ve svém každodenním životě? Chováte se ke své ženě nebo spolufarníkům
láskyplně a s úctou? Vidíte potřeby svých dětí? Ctíte svého otce i matku, zejména ve stáří? Chováte se jako gentleman
v práci, na sedadle řidiče, i v soutěžních sportech? Posloucháte svého šéfa, učitele, faráře, biskupa? Berete odpovědnost
za lidi kolem vás, ve vašem okolí či ve městě? A nakonec, podporujete aktivně své bratry v Exodu, zejména svou kotvu?
Cítí vaši bratři, že se na vás mohou spolehnout a mít ve vás oporu?

Je tu mnoho, za co se zde můžete modlit. Nechte Pána, aby přišel do vaší sebereflexe.

% ===============================================
% ===== OSMY TYDEN
% ===============================================
%ukony
\newpage
\section*{Úkony (ukazatel cesty) pro 8. týden}

\textbf{Místo:} Drsná hornatá poušť, úpatí hory Sinaj

Izraelité se utábořili na úpatí hory Sinaj, kde čekají na Boží slovo. Zůstanete zde ještě s Izraelity příští čtyři
týdny. Pro vás je to čas obnovy. Čas být v horské poušti čekající na Boží slovo. Budete-li poslouchat pozorně,
mnoho dostanete. Budete-li žít prostopášně, poputujete zpět do otroctví, ze kterého jste byli zrovna
propuštěni. Přestaňte se dívat na sebe a na dny minulosti. Upřete zrak na drsné hory před vámi. Nechte se
mocností skal a výškou vrcholů inspirovat k lepší změně.

\subsection*{1. Stále prahněte po svobodě (mějte stále touhu po vaší svobodě)}
Satan si nepřeje, abyste prahli po svobodě. Raději by byl, kdybyste toužili po pohodlí a upadli zpět do otroctví. Po tisíciletí se Satan snažil přesvědčit Boží lid, že Mojžíšovy zákony nám upírají svobodu nebo štěstí. Ať váš anděl strážný řekne Satanovi, že na tento jeho starý trik už neskočíte. Budete-li dodržovat Boží zákony, Boží zákony vás zanechají ve svobodě. Držte se plánu a zůstanete svobodní.
\subsection*{2. Držte se disciplín}
Věrně dodržované disciplíny se staly jednoduššími. Disciplíny, u kterých jste podváděli, nebo si je zlehčovali, zůstaly náročné a dráždivé. Nyní na začátku postu je čas zdvojnásobit svoje úsilí. Jestliže se vám dařilo, udržte si to. Pokud jste podváděli, řekněte o tom své kotvě a svému bratrstvu. Rozhodněte se vykonávat všechny disciplíny s láskou k Pánu a čistší touhou po svobodě.
\subsection*{3. Víte, kde je vaše kotva}
Je zlomený a ztracený v poušti sledování sportů, sociálních sítí, nebo se opírá o jídlo a pití? Nalezněte jej a v lásce si s ním promluvte. Přiveďte ho zpátky. Dodržujte s ním denní komunikaci. Teď je ta pravá chvíle se s ním zkontaktovat. Bez ohledu na to, jak brzy ráno nebo pozdě večer je, napište mu. Čte-li svá rozjímání, pochopí to.

\subsection*{Modlitba}
Modlete se, aby Pán osvobodil vás a vaše bratrství \newline
Modleme se za svobodu všech mužů v exodu, stejně tak, jako se oni modlí za vás.\newline
Ve jménu Otce i Syna i Ducha svatého … Otče náš… Ve jménu Otce i Syna i Ducha svatého … Amen.
\newpage


%newday
\newpage
\section{Den 50 - ZVOLIT SI MILOSRDENSTVÍ }
\zacatekOsmyTyden
\subsection*{Čtení na den}
\textbf{Exodus 21,12-32}
\newline
\textit{
\textsuperscript{12}Kdo někoho uhodí a ten zemře, musí zemřít.
\textsuperscript{13}Neměl-li to v úmyslu, ale Bůh dopustil, aby to jeho ruka způsobila, určím ti místo, kam se uteče.
\textsuperscript{14}Když se však někdo opováží lstivě zavraždit svého bližního, vezmeš ho i od mého oltáře, aby zemřel.
\textsuperscript{15}Kdo uhodí svého otce nebo matku, musí zemřít.
\textsuperscript{16}Kdo někoho ukradne, ať už jej prodá nebo jej u něho naleznou, musí zemřít.
\textsuperscript{17}Kdo zlořečí svému otci nebo matce, musí zemřít.
\textsuperscript{18}Když se muži dostanou do sporu a jeden druhého uhodí kamenem nebo pěstí, ale on nezemře, nýbrž je upoután na lůžko
\textsuperscript{19}a zase vstane a může vycházet o holi, bude pachatel bez viny; poskytne pouze náhradu za jeho vyřazení z práce a zajistí mu léčení.
\textsuperscript{20}Jestliže někdo uhodí svého otroka nebo otrokyni holí, takže mu zemřou pod rukou, musí být usmrcený pomstěn.
\textsuperscript{21}Jestliže však vydrží den či dva, nebude pomstěn, neboť byl jeho majetkem.
\textsuperscript{22}Když se muži budou rvát a udeří těhotnou ženu, takže potratí, ale nepřijde o život, musí pachatel zaplatit pokutu, jakou mu uloží muž té ženy; odevzdá ji prostřednictvím rozhodčích.
\textsuperscript{23}Jestliže o život přijde, dáš život za život.
\textsuperscript{24}Oko za oko, zub za zub, ruku za ruku, nohu za nohu,
\textsuperscript{25}spáleninu za spáleninu, modřinu za modřinu, jizvu za jizvu.
\textsuperscript{26}Když někdo udeří do oka svého otroka nebo otrokyni a vyrazí mu je, v náhradu za oko ho propustí na svobodu.
\textsuperscript{27}Jestliže vyrazí zub svému otroku nebo otrokyni, v náhradu za zub ho propustí na svobodu.
\textsuperscript{28}Když býk potrká muže nebo ženu, takže zemřou, musí být býk ukamenován a jeho maso se nesmí jíst; majitel býka však bude bez viny.
\textsuperscript{29}Jestliže však jde o býka trkavého již od dřívějška a jeho majitel byl varován, ale nehlídal ho, a býk usmrtí muže nebo ženu, bude býk ukamenován a také jeho majitel zemře.
\textsuperscript{30}Jestliže mu bude uloženo výkupné, dá jako výplatu za svůj život všechno, co mu bude uloženo.
\textsuperscript{31}Jestliže býk potrká syna nebo dceru, bude s ním naloženo podle téhož právního ustanovení.
\textsuperscript{32}Jestliže býk potrká otroka nebo otrokyni, dá majitel býka třicet šekelů stříbra jejich pánu a býk bude ukamenován.
}

\subsection*{Reflexe}
Po staletí byla tato pasáž, kterou znají křesťané i ateisté, používána k ospravedlnění nesčetných krutých odvetných činů. To je
ovšem hluboce chybná interpretace těchto textů. Spíše než vystupňování násilností mají tyto řádky za cíl omezit nebo zcela
zabránit násilí.

Namísto čtení tohoto úryvku z pohledu oběti si jej přečtěte tak, jak byl zamýšlen – z pohledu možného agresora. S tím si přečtěte i
tu nejznámější část: „Jestliže způsobíte svým jednáním nějakou škodu, potom dáte život za život, oko za oko, zub za zub…“ Tento
úryvek je především varováním pro všechny, kteří jsou v pokušení k agresi.

Tato pasáž poskytuje také varování oběti. Omezuje „návratnost“ za chyby a poskytuje směrnici ke spravedlivému. Tím pádem
může být tento úryvek chápán jako předzvěst Ježíšova učení, který mluví k oběti přímo: „Slyšeli jste, že bylo řečeno: ‚Oko za oko,
zub za zub.‘ Já však vám pravím, abyste se zlým nejednali jako on s vámi; ale kdo tě uhodí do pravé tváře, nastav mu i druhou; a
tomu, kdo by se chtěl s tebou soudit o košili, nech i svůj plášť. Kdo tě donutí k službě na jednu míli, jdi s ním dvě,“ (Mt 5,38-41). 
Dokonce i spravedlnost povolená v Knize Exodus je v Božích očích méně žádoucí než milosrdenství.

Zamyslete se nad tím, v jakých situacích bojujete zvolit si milosrdenství oproti spravedlnosti. Mohlo by to být v autě, ve frontě do
letadla nebo i u vás doma. Předneste to dnes Pánu. Poproste Jej, aby vás naučil, jaký má milosrdenství přínos.

%newday
\newpage
\section{Den 51 - SPRAVEDLIVÁ OMLUVA}
\zacatekOsmyTyden
\subsection*{Čtení na den}
\textbf{Exodus 21,33-22,6}
\newline
\textit{
\textsuperscript{33}Když někdo odkryje nebo vyhloubí studnu a nepřikryje ji, takže do ní spadne býk nebo osel,
\textsuperscript{1}Jestliže je zloděj přistižen při vloupání a je zbit, takže zemře, nebude lpět krev na tom, kdo ho ubil.
\textsuperscript{2}Jestliže se to však stane po východu slunce, krev na něm bude lpět. Zloděj musí poskytnout plnou náhradu. Jestliže nic nemá, bude prodán za hodnotu ukradeného.
\textsuperscript{3}Jestliže se u něho vskutku nalezne to, co ukradl, živé, ať je to býk či osel nebo ovce, poskytne dvojnásobnou náhradu.
\textsuperscript{4}Když někdo nechá spást cizí pole nebo vinici tím, že pustí svůj dobytek, aby se pásl na jiném poli, poskytne náhradu z nejlepšího, co je na jeho poli a na jeho vinici.
\textsuperscript{5}Když vypukne oheň a zachvátí trní a sežehne požaté nebo ještě stojící obilí nebo pole, poskytne ten, kdo požár zavinil, plnou náhradu.
\textsuperscript{6}Když někdo svěří svému bližnímu k opatrování stříbro nebo předměty a ty budou z domu toho muže ukradeny, poskytne zloděj, bude-li přistižen, dvojnásobnou náhradu.
}

\subsection*{Reflexe}
Spravedlnost je definována jako „vracet člověku jeho dluh“. S tímto vědomím si uvědomme dvě důležité věci: Zaprvé,
tento úryvek omezuje výši odškodnění pro osobu, která utrpěla ztrátu. Jinými slovy to znamená, že člověk, který utrpěl
nějakou škodu, nemohl požadovat nespravedlivou restituci. Zadruhé, vinná strana je povinna udělit spravedlivé
odškodnění každému, kdo byl poškozen. Pro člověka 21. století to znamená, že pokud jsme někomu nějak ublížili,
bude od nás požadováno, abychom k stížnostem přistupovali poctivě a spravedlivě.

Uprostřed chaotického a zaneprázdněného života je pravděpodobné, že aniž bychom chtěli, ublížili jsme ostatním (ať
už rodičům, sourozencům, choti, dětem, spolufarníkům, přátelům, nebo kolegům). Musíme se snažit být s nimi
spravedlivě usmířeni. Tak jako je naše smíření s Bohem skrze svátost smíření nutné pro naši úplnou svobodu, tak je to i
s našimi blízkými.

Ve vaší dnešní svaté hodině si vzpomeňte na někoho, komu dlužíte omluvu. Předneste to Pánu a hledejte Jeho vůli.
Nehledě na to, kolik odvahy budete potřebovat na splnění Boží vůle, vězte, že Bůh vám dá dost své milosti na to,
abyste Jeho vůli vykonali.

(Pokud se zdá, že žádost obdržená od Pána by mohla otevřením starých ran způsobit škodu, vyhledejte radu. Zeptejte
se svých bratří na jejich názor nebo to předneste svému duchovnímu vůdci. Rada je velkým darem, který jsme dostali
jako členové Těla Kristova. Pomáhá nám zjistit, jestli slova, která dostaneme v modlitbě, jsou opravdu od Pána.)

%newday
\newpage
\section{Den 52 - MILOVAT CIZINCE }
\zacatekOsmyTyden
\subsection*{Čtení na den}
\textbf{Exodus 22,7-23,9}
\newline
\textit{
\textsuperscript{7}Jestliže zloděj nebude přistižen, bude majitel toho domu předveden před Boha, zda sám nevztáhl ruku po výtěžku práce svého bližního.
\textsuperscript{8}Ve všech majetkových přestupcích, ať jde o býka, osla, ovci, plášť či cokoli ztraceného, o čem někdo řekne: „To je ono,“ přijde záležitost obou před Boha; koho Bůh označí jako svévolníka, ten poskytne svému bližnímu dvojnásobnou náhradu.
\textsuperscript{9}Když někdo někomu svěří do opatrování osla, býka, ovci nebo jakékoli dobytče a ono pojde nebo utrpí úraz nebo bude odehnáno, aniž to kdo viděl,
\textsuperscript{10}rozhodne mezi oběma přísaha při Hospodinu, že nevztáhl ruku po výtěžku práce svého bližního; majitel zvířete to přijme a druhý nemusí poskytnout náhradu.
\textsuperscript{11}Jestliže mu však bylo skutečně ukradeno, poskytne majiteli náhradu.
\textsuperscript{12}Jestliže bylo vskutku rozsápáno, přinese je na svědectví; za rozsápané náhradu poskytovat nebude.
\textsuperscript{13}Když si někdo vyžádá od svého bližního dobytče a ono utrpí úraz nebo uhyne, poskytne plnou náhradu, nebyl-li majitel přítomen.
\textsuperscript{14}Jestliže majitel byl přítomen, nemusí poskytnout náhradu; jde-li o námezdného dělníka, jde škoda na vrub jeho mzdy.
\textsuperscript{15}Když někdo svede pannu, která nebyla zasnoubena, a vyspí se s ní, vezme si ji za ženu a dá za ni plné věno.
\textsuperscript{16}Jestliže by se její otec rozhodně zdráhal mu ji dát, zaplatí svůdce obnos ve výši věna panen.
\textsuperscript{17}Čarodějnici nenecháš naživu.
\textsuperscript{18}Kdokoli by obcoval s dobytčetem, musí zemřít.
\textsuperscript{19}Kdo by obětoval bohům a ne samotnému Hospodinu, propadne klatbě.
\textsuperscript{20}Hostu nebudeš škodit ani ho utlačovat, neboť i vy jste byli hosty v egyptské zemi.
\textsuperscript{21}Žádnou vdovu a sirotka nebudete utiskovat.
\textsuperscript{22}Jestliže je přece budeš utiskovat a oni budou ke mně úpět, jistě jejich úpění vyslyším.
\textsuperscript{23}Vzplanu hněvem a pobiji vás mečem, takže z vašich žen budou vdovy a z vašich synů sirotci.
\textsuperscript{24}Jestliže půjčíš stříbro někomu z mého lidu, zchudlému, který je s tebou, nebudeš se k němu chovat jako lichvář, neuložíš mu úrok.
\textsuperscript{25}Jestliže se rozhodneš vzít do zástavy plášť svého bližního, do západu slunce mu jej vrátíš,
\textsuperscript{26}neboť jeho plášť, kterým si chrání tělo, je jeho jedinou přikrývkou. V čem by spal? Stane se, že bude ke mně úpět a já ho vyslyším, poněvadž jsem milostivý.
\textsuperscript{27}Nebudeš zlořečit Bohu ani nebudeš proklínat předáka ve svém lidu.
\textsuperscript{28}Neopozdíš se s dávkami z hojnosti svých úrod a vylisované šťávy svých hroznů a oliv. Dáš mi prvorozeného ze svých synů.
\textsuperscript{29}Se svým skotem a bravem naložíš tak, že zůstane sedm dní u matky, osmého dne jej dáš mně.
\textsuperscript{30}Buďte mými muži svatými. Maso zvířete rozsápaného na poli nebudete jíst, hodíte je psovi.
\textsuperscript{1}Nebudeš šířit falešnou pověst. Nespřáhneš se se svévolníkem, aby ses stal zlovolným svědkem.
\textsuperscript{2}Nepřidáš se k většině, páchá-li zlo. Nebudeš vypovídat ve sporu s ohledem na většinu a převracet právo. 
\textsuperscript{3}Ani nemajetnému nebudeš v jeho sporu nadržovat.
\textsuperscript{4}Když narazíš na býka svého nepřítele nebo na jeho zatoulaného osla, musíš mu jej vrátit.
\textsuperscript{5}Když uvidíš, že osel toho, kdo tě nenávidí, klesá pod svým břemenem, zanecháš ho snad, aniž ho vyprostíš? Spolu s ním ho vyprostíš.
\textsuperscript{6}Nebudeš převracet právo ubožáka v jeho sporu.
\textsuperscript{7}Buď dalek každého podvodu; nepřipustíš, aby byl zabit nevinný a spravedlivý, neboť svévolníka neospravedlním.
\textsuperscript{8}Nebudeš brát úplatek, neboť úplatek oslepuje i ty, kdo mají oči otevřené, a vede k překrucování záležitostí spravedlivých.
\textsuperscript{9}Nebudeš utlačovat hosta; víte přece, jak bývá hostu v duši, neboť jste byli hosty v egyptské zemi.
}

\subsection*{Reflexe}
Zákony uvedené v této části by měli vyvolat nějakou sebereflexi. Pokud vám při čtení těchto zákonů nějaký
vyvstal, přečtěte si jej znovu. Poté to předneste Pánu. Požádejte Ho, aby vám objasnil, proč vám zrovna tento
zákon vyvstává, a buďte otevřeni rozhovoru – možná dlouho odkládanému – který by mohl vzniknout.

Pokud vám žádný z těchto zákonů nijakým způsobem nevyvstává, zamyslete se nad tím posledním, týkajícím
se útlaku cizinců (hostů). Jaký máte ve svém životě postoj k útlaku „cizinců“, k lidem, kteří nejsou jako vy,
kteří vás obtěžují nebo jsou vám nepříjemní? Tento postoj k útlaku může zahrnovat aroganci, nadřazenost
nebo nepřívětivého ducha. Zaměřte se na jeden konkrétní příklad a svěřte to Pánu. Potom Ho poproste, aby
vám ukázal, jak takové lidi milovat.

%newday
\newpage
\section{Den 53 - ODPOČINOUT SI A OSLAVOVAT}
\zacatekOsmyTyden
\subsection*{Čtení na den}
\textbf{Exodus 23,10-19}
\newline
\textit{
\textsuperscript{10}Po šest let budeš osévat svou zemi a sklízet z ní úrodu.
\textsuperscript{11}Sedmého roku ji necháš ležet ladem. Nebudeš ji obdělávat, aby jedli ubožáci z tvého lidu, a co zbude, spase polní zvěř. Tak naložíš i se svou vinicí a se svým olivovím.
\textsuperscript{12}Po šest dnů budeš konat svou práci, ale sedmého dne přestaneš, aby odpočinul tvůj býk i osel a aby si mohl oddechnout syn tvé otrokyně i host.
\textsuperscript{13}Na všechno, co jsem vám řekl, budete bedlivě dbát. Jméno jiného boha nebudete připomínat; ať je není slyšet z tvých úst.
\textsuperscript{14}Třikrát v roce budeš slavit mé slavnosti :
\textsuperscript{15}Budeš zachovávat slavnost nekvašených chlebů. Po sedm dní budeš jíst nekvašené chleby, jak jsem ti přikázal, a to v určený čas měsíce ábíbu (to je měsíce klasů), neboť tehdy jsi vyšel z Egypta. Nikdo se neukáže před mou tváří s prázdnou.
\textsuperscript{16}Budeš zachovávat též slavnost žně, prvních snopků z výtěžku toho, co jsi zasel na poli, a slavnost sklizně plodin na konci roku, kdy sklízíš z pole výsledek své práce.
\textsuperscript{17}Třikrát v roce se ukáže každý, kdo je mužského pohlaví, před Pánem Hospodinem.
\textsuperscript{18}Nebudeš obětovat dobytče tak, aby krev mého obětního hodu vytekla na něco kvašeného. Nebudeš přechovávat tuk z mé slavnosti přes noc až do rána.
\textsuperscript{19}Prvotiny raných plodů své role přineseš do domu Hospodina, svého Boha. Nebudeš vařit kůzle v mléku jeho matky.
}

\subsection*{Reflexe}
Dnes Hospodin přikazuje Izraelitům (a nám) odpočívat a oslavovat. Tento příkaz nám ukazuje Boží laskavou
touhu po našem dobru. Bůh ví o naší náchylnosti k otroctví. Udělením času pro odpočinek a slavení nám
dává možnost odpočinout v Něm a chválit Ho. Dnešní čtení nám ukazuje, že Bůh se nestará pouze o to,
abychom slavili, ale říká nám také, kdy máme slavit.

Jako křesťanům nám Církev říká, abychom odpočívali od práce ve dnech přikázaných slavností a v den Páně
(viz KKC 2185). Církev praktikuje sobotní večer, čas Kristova vzkříšení, v neděli jako den Páně. Mezitím
nás svět pokouší skrze „světský šabat“, od pátka večer skrze celou sobotu. Tento alternativní čas odpočinku
není časem slavení Pána, ale nás samotných; je to doba, kdy často padáme ve zvyk oslavovat místo neděle.

Obvykle oslavujeme v pátek a sobotu, což nás mimo jiné vede k odkládání práce na neděli. Tak sepravý den
odpočinku a chvály odsune kvůli věcem, které musíme udělat.

Zastavte se a zeptejte se sami sebe:Proč tyto věci musíme udělat, když je jasná neděle? Který pán je pro nás
důležitější, ten neexistující, který říká, že tyto věci nemůžou počkat do pondělka, nebo Bůh, který nás miluje
ve své podstatě? Zeptejme se způsobem, který nemusíme chtít slyšet: Jsme ochotni hřešit (obrátit se vědomě
zády k Bohu), abychom tyto věci udělali, místo toho, abychom odpočívali a věnovali tento čas chválení
Boha? Skutečnost je často taková, že tyto věci mohou do pondělka počkat – nebo měly být udělány v pátek
nebo v sobotu, kdy jsme možná oslavovali světský šabat.

Tato nesnáze dokazuje naši potřebu dne Páně a slavností, abychom měli možnost upřednostnit Boha před
prací. Bůh nás chce svobodné. Chce, abychom byli schopni odpočívat. Zákony, které nám dává, jsou tu pro
naše dobro.

Rozčiluje vás důraz na odpočinek, slavení a volný čas v den Páně? Pokud ano, ve vaší svaté hodině dnes
popřemýšlejte nad tím, proč to může být směšné. Potom to předneste Bohu a dovolte Mu, aby k vám s láskou
mluvil pravdu.

Na druhou stranu, pokud jste nadšeni, že můžete přeměnit svou neděli a dny slavností k tomu, abyste lépe
milovali a sloužili Bohu, vzdejte Mu díky. Pak si udělejte čas, abyste pohovořili s Pánem o tom, jak pro Něj
můžete v těchto dnech žít svobodněji.

%newday
\newpage
\section{Den 54 - PŘÍTOMNOST ANDĚLŮ  }
\zacatekOsmyTyden
\subsection*{Čtení na den}
\textbf{Exodus 23,20-33}
\newline
\textit{
\textsuperscript{20}Hle, posílám před tebou posla, aby tě opatroval na cestě a aby tě uvedl na místo, které jsem připravil.
\textsuperscript{21}Měj se před ním na pozoru a poslouchej ho, nevzdoruj mu, neboť přestupky vám nepromine, poněvadž v něm je mé jméno.
\textsuperscript{22}Když jej však budeš opravdu poslouchat a činit všechno, co mluvím, stanu se nepřítelem tvých nepřátel a protivníkem tvých protivníků.
\textsuperscript{23}Můj posel půjde před tebou a uvede tě k Emorejcům, Chetejcům a Perizejcům, ke Kenaancům, Chivejcům a Jebúsejcům, a já je zničím.
\textsuperscript{24}Nebudeš se klanět jejich bohům a nebudeš jim sloužit. Nebudeš se dopouštět toho, co páchají. Úplně je rozmetáš a na kusy roztříštíš jejich posvátné sloupy.
\textsuperscript{25}Budete sloužit Hospodinu, svému Bohu, a on požehná tvému chlebu a tvé vodě. Vzdálím od tebe nemoc;
\textsuperscript{26}ve tvé zemi nebude ženy, která by potratila nebo která by byla neplodná; obdařím tě plností let.
\textsuperscript{27}Pošlu před tebou svou hrůzu a uvedu ve zmatek všechen lid, k němuž přicházíš; obrátím před tebou všechny tvé nepřátele na útěk.
\textsuperscript{28}Pošlu před tebou děsy, aby před tebou vypudili Chivejce, Kenaance a Chetejce.
\textsuperscript{29}Nevypudím je však před tebou za jeden rok, aby země nezpustla a aby se k tvé škodě nerozmnožila polní zvěř.
\textsuperscript{30}Vypudím je před tebou postupně, dokud se nerozplodíš a nepřevezmeš zemi do dědictví.
\textsuperscript{31}A určím tvé pomezí od Rákosového moře až k moři Pelištejců a od pouště až k řece Eufratu. Dám totiž do vašich rukou obyvatele země a vypudíš je před sebou.
\textsuperscript{32}Neuzavřeš smlouvu s nimi nebo s jejich bohy.
\textsuperscript{33}Nebudou v tvé zemi sídlit, aby tě nesvedli ke hříchu proti mně. Kdybys sloužil jejich bohům, stalo by se ti to léčkou.
}

\subsection*{Reflexe}
V dnešním čtení mluví Bůh o poslu, andělu, poslaném, aby vás „uvedl na místo, které jsem připravil“.

Tak jako posílá Pán anděla před Izraelity v jejich Exodu, tak posílá i anděla před vámi. Církev nám říká, že má „každý
věřící u sebe anděla jako ochránce a pastýře, aby ho vedl k životu“ (KKC 336). To proto, že se „celý život církve stejně
těší z tajemné a mocné pomoci andělů“ (KKC 334). Dnes ve vaší svaté hodině si udělejte čas na to, abyste uvažovali
nad přítomností anděla uprostřed vašeho exodu. Tento anděl, obvykle nazývaný andělem strážným, má vhled do
okolností vaší spásy. Váš strážný anděl přesně zná vaši cestu do země zaslíbené. Dovedl vás až sem a povede vás až ke
svobodě.

Máte-li ve vašem bratrstvu pět až sedm mužů ve vašem bratrstvu, pak máte pět až sedm andělů, kteří jsou přítomní a
dohlíží na vaše týdenní setkávání. Tato skutečnost je hluboká. Tak jako mluvíte s Bohem, můžete (a měli byste) mluvit
s mocnými anděli, zejména s vaším andělem strážným. Naslouchání a mluvení s vaším strážným andělem budete muset
chvíli cvičit, ale výsledek bude stát za to. Popovídejte si s vaším strážným andělem, jako byste hovořili s přítelem.
Potom naslouchejte, co má zas on pro vás.

%newday
\newpage
\section{Den 55 - SÍLA SMLOUVY }
\zacatekOsmyTyden
\subsection*{Čtení na den}
\textbf{Exodus 24,1-11}
\newline
\textit{
\textsuperscript{1}Potom Mojžíšovi řekl: „Vystup k Hospodinu, ty i Áron, Nádab a Abíhú a sedmdesát z izraelských starších. Budete se zdálky klanět.
\textsuperscript{2}K Hospodinu přistoupí jen Mojžíš. Ostatní se přibližovat nebudou. Lid nesmí vystoupit vzhůru spolu s ním.“
\textsuperscript{3}Když Mojžíš přišel nazpět, vypravoval lidu všechna slova Hospodinova a předložil mu všechna právní ustanovení. Všechen lid odpověděl jako jedněmi ústy. Řekli: „Budeme dělat všechno, o čem Hospodin mluvil.“
\textsuperscript{4}Nato Mojžíš zapsal všechna Hospodinova slova. Za časného jitra postavil pod horou oltář a dvanáct posvátných sloupů podle dvanácti izraelských kmenů.
\textsuperscript{5}Pak pověřil izraelské mládence, aby přinesli oběti zápalné a obětovali Hospodinu býčky k hodům oběti pokojné.
\textsuperscript{6}Mojžíš vzal polovinu krve a vlil ji do mís a druhou polovinou pokropil oltář.
\textsuperscript{7}Potom vzal Knihu smlouvy a předčítal lidu. Prohlásili: „Poslušně budeme dělat všechno, o čem Hospodin mluvil.“
\textsuperscript{8}Mojžíš vzal krev, pokropil lid a řekl: „Hle, krev smlouvy, kterou s vámi uzavírá Hospodin na základě všech těchto slov.“
\textsuperscript{9}Pak Mojžíš a Áron, Nádab a Abíhú a sedmdesát z izraelských starších vystoupili vzhůru.
\textsuperscript{10}Uviděli Boha Izraele. Pod jeho nohama bylo cosi jako průzračný safír, jako čisté nebe.
\textsuperscript{11}Ale nevztáhl ruku na nejpřednější z Izraelců, ačkoli uzřeli Boha; i jedli a pili.
}

\subsection*{Reflexe}
Síla oběti popsaná v dnešním čtení o Mojžíšově Smlouvě je úžasná. Akt kropení krví oltáře (znázorňující
Boha) a lidu (znázorňující je samotné) je smlouvou přivádějící obě strany. Tento bod je hlavním v dějinách
spásy a nesmí být přehlížen. Lid Izraele se zavazuje Bohu.

Ve Starém Zákoně můžeme nalézt pět klíčových smluv. Dnešní smlouva mezi Bohem a Izraelem, známá
jako Mojžíšova, je čtvrtá. Smlouva je více než jen dohoda, která může být vyjednána nebo z ní lze vystoupit.
Je to směna osob, nerozlučitelné pouto, které přivádí dvě strany blízko k sobě. To, co ve smluvním vztahu
patří jedné straně, patří i druhé a naopak. Proto je vstoupení Izraele do smlouvy s Hospodinem tak velkým
činem.

Smlouvy jsou stvrzeny krví, která znázorňuje jejich nerozlučitelnou povahu. Pro stranu, která by porušila
podmínky smlouvy, by byl osud stejný jako pro obětované zvíře. Boha nelze oklamat. Jeho závazek Izraeli je
skutečný a trvalý. Přikazuje, aby náš závazek k Němu byl stejný.

Mojžíšova smlouva spolu s dalšími čtyřmi smlouvami Starého Zákona je naplněna v Kristově umučení, smrti
a vzkříšení. Kristus zakládá Novou Smlouvu a volá všechny lidi, aby k ní přistoupili. Stejně jako se Izraelité
podíleli na krvi oběti, aby vstoupili do Mojžíšovy smlouvy s Bohem, tak se i my podílíme na krvi Kristovy
oběti, abychom vstoupili do Smlouvy Nové. Křest slouží jako náš vstup v Novou Smlouvu, a po křtu
pokaždé znovuuzavíráme naši smlouvu s Bohem, když přijímáme Kristovu krev v Eucharistii.

V Nové Smlouvě se Bůh za nás fyzicky vydává. Dává nám možnost získat všechno, co je Jeho – což, pokud
se nad tím zamyslíme, je vše, co je. Jak často přicházíme k oltáři, jako by to byl jen volný čas strávený na
mši. Přijetí Eucharistie je mnohem víc než to a váže s sebou mnohem víc odpovědnosti.

Když přicházíte k přijetí Eucharistie na mši, dáváte se plně Bohu výměnou za všechno, co je Jeho? Přineste
tuto otázku do vaší dnešní svaté hodiny. Zamyslete se nad nesmírností smlouvy, do které vstupujete.

%newday
\newpage
\section{Den 56 - OČISTIT SVÉ „PROČ“ }
\zacatekOsmyTyden
\subsection*{Čtení na den}
\textbf{Exodus 24,12-18}
\newline
\textit{
\textsuperscript{12}Hospodin řekl Mojžíšovi: „Vystup ke mně na horu a pobuď tam. Dám ti kamenné desky – zákon a přikázání, které jsem napsal, abys jim vyučoval.“
\textsuperscript{13}I povstal Mojžíš a Jozue, který mu přisluhoval, a Mojžíš vystoupil na Boží horu.
\textsuperscript{14}Starším řekl: „Zůstaňte zde, dokud se k vám nevrátíme. Budou tu s vámi Áron a Chúr. Kdo něco má, ať se obrací na ně.“
\textsuperscript{15}Mojžíš tedy vystoupil na horu a horu přikryl oblak.
\textsuperscript{16}A Hospodinova sláva přebývala na hoře Sínaji a oblak ji přikrýval po šest dní. Sedmého dne zavolal Hospodin na Mojžíše zprostřed oblaku.
\textsuperscript{17}Hospodinova sláva se jevila pohledu Izraelců jako stravující oheň na vrcholku hory.
\textsuperscript{18}Mojžíš vstoupil doprostřed oblaku. Vystoupil na horu a byl na hoře čtyřicet dní a čtyřicet nocí.
}

\subsection*{Reflexe}
V reakci na dnešní čtení z Písma říká svatý Jak Zlatoústý: „Podívejte se na příklady dobrého chování během
půstu. Mojžíš poté, co se postil po čtyřicet dní, byl schopný přijmout desky Zákona.“ Stejně jako pro
Mojžíše, i pro nás je připravena odměna za náš půst, dar od Pána. Zůstatně pozorní, běžte se svými bratry
závod tak, abyste získali cenu (1 Kor 9,24). Vytrvejte, abyste po tomto čase očisty byli schopni vaši mysl,
tělo a duši nechat ovládnout Božím zákonem.

Proč jste zde a následujete tento očistný plán k Bohu? Máte napsané své „proč“, ale co všechna ta nepsaná,
podvědomá „proč“, které s sebou vláčíte? Co bylo vlastně důvodem pro vstup do tohoto duchovního cvičení?
Bylo to ze sobeckých důvodů? Bylo to proto, že vás k tomu někdo donutil? Bylo to proto, že máte rádi velké
výzvy? Bylo to kvůli vaší pýše? Vaše důvody by již měly být očištěny. Vaše důvody by měly více souznít
s tím, který jste si na začátku napsali. Uvědomte si s vděčností k Bohu toto očištění, a tuto vděčnost pak
použijte k pohonu vpřed.


% ===============================================
% ===== DEVATY TYDEN
% ===============================================
%ukony
\newpage
\section*{Úkony (ukazatel cesty) pro 9. týden}

\textbf{Místo:} Drsná hornatá poušť, úpatí hory Sinaj

Izraelité se učí, jak postavit a vyzdobit Svatostánek, posvátné místo, kde by Bůh přebýval mezi svým lidem.
„Vaše tělo je chrámem Ducha svatého, který ve vás přebývá a jejž máte od Boha,“ (1 Kor 6,19). Jak
vyzdobujete svůj chrám? Jakým věcem dovolujete vstup na toto posvátné místo? Co děláte pro to, abyste
zajistili, že toto místo bude posvátné jako Svatostánek, aby ve vás mohl přebývat Bůh?

\subsection*{1. Držte se pravidelného cvičení}
Protože Bůh považuje naše těla za důstojný příbytek, musíme je mít v řádné úctě. Starejte se o svůj chrám. Posilujte ho. Udržujte jej aktivní a vždy v dobrém stavu, aby mohl dobře sloužit Bohu.
\subsection*{2. Dodržujte svůj závazek vůči svému bratrstvu}
Právě požehnaný zodpovědný život ve vašem bratrstvu vás vyvedl z Egypta. To skrze modlitbu, askezi a bratrství vám je dáno vysvobození. Nenechte sebe, ani své bratry, v izolaci. Držte svůj závazek bratrstvu, a obzvlášť vašim bratrským setkáním.
\subsection*{3. Udělejte si čas na vaše bratrstvo i mimo pravidelné setkání}
Muži Exodu zaznamenali, že devátý a desátý týden je největší zkouškou morálky. To z těchto týdnů dělá skvělý čas pro další výlet s bratrstvem. Bez ohledu na to, kolik lidí se může zúčastnit, udělejte si tento nebo příští týden čas na výlet. Potom jej uskutečněte, nehledě na počasí.
\subsection*{4. Dodržujte pravidelné noční examen (s pečlivostí)}
Věrnost vašemu nočnímu zkoumání dne (examen) vám pomůže k úspěchu v dalších dnech. Tento váš zvyk se nyní formuje. Držte se jej, a přinese vám mnoho milostí v 91. dnu.
\subsection*{5. Zůstaňte radostní}
Tato věta se nedá neopakovat. Pokud chybujete v radosti, je to zřejmě proto, že chybujete ve vděčnosti nebo v naději. Nedal vám snad Otec tu milost, abyste to dotáhli až do 9. týdne? Dal, a zůstane vám věrný, až do země zaslíbené. Buďte vděční a mějte naději, protože to vás udrží radostnými.
\subsection*{Modlitba}
Modlete se, aby Pán osvobodil vás a vaše bratrství \newline
Modleme se za svobodu všech mužů v exodu, stejně tak, jako se oni modlí za vás.\newline
Ve jménu Otce i Syna i Ducha svatého … Otče náš… Ve jménu Otce i Syna i Ducha svatého … Amen.
\newpage


%newday
\newpage
\section{Den 57 - ÚČAST NA SVATÉ OBĚTI }
\zacatekDevatyTyden
\subsection*{Čtení na den}
\textbf{Exodus 25,1-9}
\newline
\textit{
\textsuperscript{1}Hospodin promluvil k Mojžíšovi:
\textsuperscript{2}„Mluv k synům Izraele, ať pro mne vyberou oběť pozdvihování. Vyberete oběť pozdvihování pro mne od každého, kdo ji ze srdce dobrovolně odevzdá.
\textsuperscript{3}Toto bude oběť pozdvihování, kterou od nich vyberete: zlato, stříbro a měď;
\textsuperscript{4}látka purpurově fialová, nachová a karmínová, jemné plátno a kozí srst;
\textsuperscript{5}načerveno zbarvené beraní kůže, tachaší kůže a akáciové dřevo;
\textsuperscript{6}olej na svícení, balzámy na olej k pomazání a na kadidlo z vonných látek;
\textsuperscript{7}karneolové drahokamy a kameny pro zasazení do nárameníku a náprsníku.
\textsuperscript{8}Ať mi udělají svatyni a já budu bydlet uprostřed nich.
\textsuperscript{9}Uděláte všechno přesně podle toho, co ti ukazuji jako vzor svatého příbytku i vzor všech bohoslužebných předmětů.
}

\subsection*{Reflexe}
Jako křesťané nemusíme obětovat stejnou oběť, kterou žádá Bůh v dnešním čtení. On po nás ale žádá oběť, kterou bychom
podpořili dílo Církve. Ve skutečnosti je od křesťanů žádáno, aby dávali více než zlato a stříbro, které Bůh žádá od Izraelitů. Svatý
Pavel nám říká, že jsme žádáni, abychom dávali svá těla jako oběť Bohu a Církvi. „Vybízím vás, bratří, pro Boží milosrdenství,
abyste sami sebe přinášeli jako živou, svatou, Bohu milou oběť; to ať je vaše pravá bohoslužba,“ (Řím 12,1).

Katechismus to přesně vykládá:
> Eucharistie je také obětí Církve. Církev, jež je Kristovým Tělem, má účast na oběti své Hlavy. Je s Kristem sama celá
> obětována. Spojuje se s jeho přímluvou u Otce za všechny lidi. V Eucharistii se stává Kristova oběť také obětí údů
> jeho Těla. Život věřících, jejich chvála, jejich utrpení, jejich modlitba, jejich práce jsou spojeny s Kristovou chválou
> a modlitbou, s jeho utrpením a prací i s jeho bezvýhradnou obětí, a tím získávají novou hodnotu. Kristova oběť,
> přítomná na oltáři, poskytuje všem generacím křesťanů možnost být spojeni s jeho obětí. (KKC 1368)

Zkuste se na chvíli zamyslet nad vaší účastí na mši svaté. Jste znuděni? Má vaše mysl tendenci bloudit? Myslíte si, že je
Eucharistická modlitba jen něco pro kněze, zatímco lid prostě přihlíží? Nebo plně a aktivně vstupujete do oběti Krista na každé
mši?

Pokuste se plně vstoupit do mše tím, že v mysli přinesete duchovní oběť během Offertoria (když probíhá sbírka a přináší se dary) a
budete se účastnit Eucharistické modlitby. Když to uděláte, budete dobře připraveni na přijímání Eucharistie tím, že podstoupíte
bolest Velkého Pátku obětí na oltáři. Oslava vzkříšeného Pána (malé Velikonoce) a přijímání potom bude mnohem smysluplnější a
podstatnější. Uvažujte o tom dnes v modlitbě a rozhodněte se s Pánem lépe se účastnit oběti, kterou přinášíme při mši.


%newday
\newpage
\section{Den 58 - KDO JE TO ARCHA ÚMLUVY?}
\zacatekDevatyTyden
\subsection*{Čtení na den}
\textbf{Exodus 25,10-22}
\newline
\textit{
\textsuperscript{10}Udělají z akáciového dřeva schránu dva a půl lokte dlouhou, jeden a půl lokte širokou a jeden a půl lokte vysokou.
\textsuperscript{11}Obložíš ji čistým zlatem, uvnitř i zvnějšku ji obložíš a opatříš ji dokola zlatou obrubou.
\textsuperscript{12}Uliješ pro ni čtyři zlaté kruhy a připevníš je ke čtyřem jejím hranám: dva kruhy na jednom boku a dva kruhy na druhém.
\textsuperscript{13}Zhotovíš tyče z akáciového dřeva a potáhneš je zlatem.
\textsuperscript{14}Tyče prostrčíš skrz kruhy po stranách schrány, aby bylo možno na nich schránu nést.
\textsuperscript{15}Tyče zůstanou v kruzích, nebudou vytahovány.
\textsuperscript{16}Do schrány uložíš svědectví, které ti dám.
\textsuperscript{17}Zhotovíš příkrov z čistého zlata dlouhý dva a půl lokte a široký jeden a půl lokte.
\textsuperscript{18}Potom zhotovíš dva cheruby ze zlata; dáš je vytepat na oba konce příkrovu.
\textsuperscript{19}Jednoho cheruba uděláš na jednom konci a druhého cheruba na druhém konci. Uděláte cheruby na příkrov, na oba jeho konce.
\textsuperscript{20}Cherubové budou mít křídla rozpjatá vzhůru; svými křídly budou zastírat příkrov. Tvářemi budou obráceni k sobě; budou hledět na příkrov.
\textsuperscript{21}Příkrov dáš nahoru na schránu a do schrány uložíš svědectví, které ti dám.
\textsuperscript{22}Tam se budu s tebou setkávat a z místa nad příkrovem mezi oběma cheruby, kteří budou na schráně svědectví, budu s tebou mluvit o všem, co ti pro Izraelce přikážu.
}

\subsection*{Reflexe}

Archa úmluvy byla nejposvátnějším náboženským článkem Izraelitů. Jak jsme už viděli, různí lidé, místa a věci ve
Starém Zákoně jsou typy (předznamenání) lidí, míst a věcí Nového Zákona. Archa není žádnou vyjímkou. Otázka,
kterou bychom zde měli položit, ale nezní: „Co je to Archa úmluvy?“ Spíše bychom měli klást otázku, „Kdo je to
Archa úmluvy?“

Archa úmluvy předznamenává Pannu Marii. Zde je pár podobností, kterou tuto typologyii odhalují:
\begin{itemize}
  \item Archa byla vytvořena z nejlepšího akátového dřeva, pevného, uzavřeného dřeva, které nemohlo být poškozeno hmyzem; jako Panna Maria byla uchována od všeho hříchu, a to i od prvotního.
  \item Archa byla pokrytá zlatem, symbolem božkosti; Panna Maria byla zvolená Otcem, aby byla Boží matkou.
  \item Archa nesla Áronovu hůl (symbol kněžství), vzorek many (druh Eucharistie), a desky Smlouvy (Boží Slovo); Panna Maria nesla Ježíše Krista (nejvyššího velekněze, Eucharistii a vtělené Boží Slovo).
  \item Archa úmluvy Izraelity stále přiváděla do Boží přítomnosti; Panna Maria nás přivádí přímo k jejímu Synu, našemu Pánu.
\end{itemize}

Ale je toho víc. Když izraelská armáda poslušně následovala Archu úmluvy do boje, vyhrála. Jako křesťané budeme
také požehnáni, když budeme následovat novou Archu úmluvy, Pannu Marii, do boje. Není žádným překvapením, že
muži Exodu zjistili, že muži v bratrstvu, kteří se každodenně modlí modlitbu růžence, mají mnohem větší
pravděpodobnost, že dosáhnou svobody než ti, kteří se ji nemodlí.

Stejně jako Izraelci poznali jejich potřebu uctívat Archu úmluvy jako prostředek k Bohu, měli byste i vy rozpoznat
svou vlastní potřebu kultivovat úctu k nové Arše úmluvy. Pokud jste schopni pochopit svou potřebu úcty Panny Marie,
získáte pro sebe Matku, která může a povede vás přímo k jejímu Synu, Kristu Ježíši, v každém okamžiku, kdy k ní
zavoláte.

Zavolejte dnes k Marii ve vaší svaté hodině. Poproste ji, jako prosíte své bratry, aby se za vás přimlouvala u svého
Syna, Ježíše Krista. Naslouchejte, jak odpovídá na vaše volání.


%newday
\newpage
\section{Den 59 - JEŽÍŠ JE VDĚČNÝ ZA VÁŠ ČAS}
\zacatekDevatyTyden
\subsection*{Čtení na den}
\textbf{Exodus 25,23-30}
\newline
\textit{
\textsuperscript{23}Zhotovíš stůl z akáciového dřeva dlouhý dva lokte, široký jeden loket a vysoký jeden a půl lokte.
\textsuperscript{24}Obložíš jej čistým zlatem a opatříš jej dokola zlatou obrubou.
\textsuperscript{25}Uděláš mu také dokola na dlaň širokou lištu a k liště uděláš dokola zlatou obrubu.
\textsuperscript{26}Opatříš jej čtyřmi zlatými kruhy a připevníš je ke čtyřem rohům při jeho čtyřech nohách.
\textsuperscript{27}Kruhy budou těsně pod lištou, aby držely tyče na nošení stolu.
\textsuperscript{28}Tyče zhotovíš z akáciového dřeva a potáhneš je zlatem; na nich se stůl bude nosit.
\textsuperscript{29}Zhotovíš k němu též mísy, pánvičky, konvice a obětní misky používané k úlitbě; zhotovíš je z čistého zlata.
\textsuperscript{30}Pravidelně budeš klást přede mne na stůl předkladný chléb.
}

\subsection*{Reflexe}
Kněží Chrámu měli položit dvanáct chlebů ve dvou řadách na posvátný stůl (Lv 24,6), stejně jako zlaté nádoby používané
k obětním úlitbám, aby mohl být chléb vždy v přítomnosti před Hospodinem. Toto uspořádání je samo o sobě předznamenáním
chleba a vína, které budou jednou užívány jako svatá oběť při mši.

V písmu i v dějinách je jasné, že Hospodin touží být se svým lidem. Tak velký Král se sníží v podobě křehkého kousku chleba,
ovládaného lidskýma rukama, rozmačkaného lidskýma zubama, a vystaveného možnosti lidského znesvěcení. Proč? Aby On, ve
vší své slávě, mohl s námi strávit blízký čas, a to i s našimi nedokonalostmi. Ježíš po vás touží. Čeká na vás ve svatostánku,
v adorační kapli. Žízní po tom, být s vámi. Má stejnou touhu po všech lidech, ale víme, že pouze hrstka Mu věnuje pozornost, a
ještě méně dlí před Jeho eucharistickou přítomností. Utěšte Kristovo toužící srdce. Běžte s Ním strávit čas, za vás a za všechny,
kteří tak nikdy neučiní. Náš Pán je za váš čas s Ním tak vděčný.

Jestli se s vaším bratrstvem setkáváte každý týden, abyste spolu trávili svatou hodinu před Nejsvětější svátostí, jak navrhuje
Průvodce terénem, děláte něco svělého. Zůstaňte u toho. Pokud to neděláte, teď je ten pravý čas naplánovat alespoň jednu
bratrskou svatou hodinu za celý Exodus. Příští týdny tohoto duchovního cvičení budou nesmírně náročné jak pro vás, tak i pro vaše
bratrstvo, když budete trávit nekonečný čas s Izraelity na úpatí hory Sinaj. Přidání svaté hodiny s bratry vám dá milost navíc,
abyste tuto pouštní stálost přežili. Zkontaktujte dnes své bratry. Naplánujte si v příštích deseti dnech čas na bratrskou svatou
hodinu.

Stejně jako ve vaší svaté hodině, pokud není ve vaší lokalitě možnost eucharistické adorace, čas strávený před svatostánkem je
další skvělou alternativou. Ať se stane cokoli, udělejte si čas na Krista. Velmi touží s vámi a vaším bratrstvem strávit čas.

Dnes si udělejte čas před naším Pánem v Eucharistii, abyste Ho poprosili, aby vám ukázal, jak moc pro Něj znamená, že s Ním
trávíte čas.


%newday
\newpage
\section{Den 60 - BOŽÍ PRVNÍ VELKÁ KNIHA}
\zacatekDevatyTyden
\subsection*{Čtení na den}
\textbf{Exodus 25,31-40}
\newline
\textit{
\textsuperscript{31}Zhotovíš svícen z čistého zlata. Svícen bude mít vytepaný dřík a prut; jeho kalichy, číšky a květy budou s ním zhotoveny z jednoho kusu.
\textsuperscript{32}Z jeho stran bude vycházet šest prutů, tři pruty svícnu po jedné straně a tři pruty svícnu po druhé straně.
\textsuperscript{33}Na jednom prutu budou tři kalichy podobné mandloňovému květu: číška a květ. A tři kalichy podobné mandloňovému květu na druhém prutu: číška a květ. Tak to bude na všech šesti prutech vycházejících ze svícnu.
\textsuperscript{34}Na svícnu budou čtyři kalichy podobné mandloňovému květu s číškami a květy.
\textsuperscript{35}Jedna číška bude pod jednou dvojicí prutů, druhá číška bude pod druhou dvojicí prutů a třetí číška bude pod třetí dvojicí prutů; tak to bude u všech šesti prutů vycházejících ze svícnu.
\textsuperscript{36}Příslušné číšky a pruty budou s ním zhotoveny z jednoho kusu; všechno bude vytepané z čistého zlata.
\textsuperscript{37}Ke svícnu zhotovíš také sedm kahánků; kahánky ať jsou nasazeny tak, aby osvětlovaly prostor před ním.
\textsuperscript{38}Zhotovíš k němu z čistého zlata i nůžky na knoty a pánve na oharky.
\textsuperscript{39}Bude zhotoven se všemi těmito předměty z jednoho talentu čistého zlata.
\textsuperscript{40}Hleď, abys všechno udělal podle vzoru, který ti byl ukázán na hoře.
}

\subsection*{Reflexe}
Bůh říká Mojžíšovi, aby podle vzoru, který mu byl ukázán na svaté hoře, vyzdobil svatyni. Uvažujte o instrukcích pro
vytvoření svícnu, s obrazy mandloňových květů. Bůh si přeje, aby Jeho pozemská svatyně vypadala a byla cítit jako
vesmír, ve kterém žijeme. Svatyně se tak stává vzorem nebo mikrokosmem vesmíru, s Mojžíšovým nebeským
zjevením jako „modrotiskem“. Nakonec bude svatyně sloužit jako vzor, podle kterého bude vystavěn velký
Šalamounův Chrám. Právě tady uprostřed stvořeného světa se Bůh snaží přebývat se svým lidem.

Po tisíciletí se lidé setkávali s Bohem v přírodě. Když jdete na pěší výlet, běhat, rybařit, sedíte u táboráku, pracujete na
zahradě, zemědělčíte, nebo jen jedete do práce, všímejte si přírody kolem vás. Všimněte si východu slunce, přicházející
bouřky, jemného vánku, měnících se ročních období, divokých zvířat, meandrující řeky, nádherného stromu…
vyvolejte v mysli Boží velikost a Jeho tajemství. Připomínejte si Jeho přítomnost ve všem stvořeném, včetně lidí –
především těch, které milujeme.

Svatý Augustin říkal, že pokud by si chtěl někdo přečíst knihu o Bohu, měl by nejprve začít číst knihu stvoření – tedy
přírody. Trávíte dostatek času čtením této knihy? Oceňujete dar světa, který nám Bůh dal? Kdy se v nejbližší době
dostanete ven a půjdete na výlet do lesa, nebo alespoň na procházku v parku, možná se svým bratrstvem, se svými
dětmi, možná se všemi? Pohovořte si dnes s Pánem o vašem vztahu k přírodě, Jeho první velké knize.


%newday
\newpage
\section{Den 61 - JEDNOTA: JEDNO TĚLO, JEDNA CÍRKEV}
\zacatekDevatyTyden
\subsection*{Čtení na den}
\textbf{Exodus 26,1-14}
\newline
\textit{
\textsuperscript{1}Zhotovíš příbytek z deseti pruhů jemně tkaného plátna a z látky purpurově fialové, nachové a karmínové; zhotovíš je s umně vetkanými cheruby.
\textsuperscript{2}Jeden pruh bude dlouhý dvacet osm loket a široký čtyři lokte. Takový bude jeden pruh. Všechny pruhy budou mít stejné rozměry.
\textsuperscript{3}Pět pruhů bude navzájem po délce spojeno, a právě tak bude navzájem spojeno druhých pět.
\textsuperscript{4}Z fialového purpuru zhotovíš poutka na lemu toho pruhu, který bude na konci spojeného kusu. Stejně to uděláš na lemu krajního pruhu druhého spojeného kusu.
\textsuperscript{5}Uděláš také padesát poutek na jednom koncovém pruhu a padesát poutek uděláš na koncovém pruhu druhého spojeného kusu. Jednotlivá poutka budou naproti sobě.
\textsuperscript{6}Zhotovíš i padesát zlatých spon a sepneš jimi pruhy dohromady, takže příbytek bude spojen v jeden celek.
\textsuperscript{7}Zhotovíš též houně z kozí srsti pro stan nad příbytkem; jedenáct takových houní uděláš.
\textsuperscript{8}Jedna houně bude dlouhá třicet loket a široká čtyři lokte. Taková bude jedna houně. Všech jedenáct houní bude mít stejné rozměry.
\textsuperscript{9}Zvlášť spojíš pět houní a zvlášť šest houní. Šestou houni na přední straně stanu přeložíš.
\textsuperscript{10}Uděláš také padesát poutek na lemu jedné houně, která bude na kraji spojeného kusu, a padesát poutek uděláš na lemu koncové houně druhého spojeného kusu.
\textsuperscript{11}Zhotovíš i padesát bronzových spon, provlékneš je poutky a stan sepneš, takže bude spojen v jeden celek.
\textsuperscript{12}Z toho, co bude ze stanových houní přesahovat, bude přesahující polovina houně viset přes zadní stranu příbytku.
\textsuperscript{13}Loket z té i oné strany, o nějž budou stanové houně větší, bude tvořit na bočních stranách příbytku převis, aby byl cele přikryt.
\textsuperscript{14}Zhotovíš pro stan též přikrývku z beraních kůží zbarvených načerveno a navrch přikrývku z tachaších kůží.
}

\subsection*{Reflexe}
Dnes Bůh řídí propracovanou stavbu přenosného stanu, v němž bude umístěna Archa úmluvy a vše, co je
nezbytné pro bohoslužebné uctívání. Zde je řečeno, že stan, ačkoli vyroben z několika kusů látky, „bude
spojen v jeden celek“ (Ex 26,11). To je znamení od Otce, že je vesmír držen pohromadě podobným
způsobem a poznamenán provždy hlubokou jednotou.

Krátce před jeho umučením se Ježíš vroucně modlil: „Otče svatý, zachovej je ve svém jménu, které jsi mi
dal; nechť jsou jedno jako my“ (J 17,11). Vzhledem k povaze křesťanství jsou všichni pokřteni v jedno
tajemné Kristovo Tělo, Církev. Být oddělen nebo odpadnout od Církve by znamenalo být oddělen od
samotného těla, do kterého jste byli pokřtěni. To znamená být oddělen od Krista. Tato jednota je pro
křesťanství tak nezbytná, že Ježíš ustanovil Církev ještě dříve, než pro ni zemřel. Dokonce nám nechal
papeže, pod kterým by všichni křesťané mohli zůstat jednotní. Díky Duchu svatému je řada nástupců svatého
Petra, prvního papeže, nepřerušená, tak jak nám to slíbil Kristus (Mt 16,18).

Naneštěstí se křesťané během 2000 let rozdělili do mnoha sekt a denominací. Některé se nazývají jako jiné
organizované „církve“. Některé už nechtějí být s žádným organizovaným sborem spojovány, raději chtějí být
volně nezávislí křesťané (což jsme již vysvětlili, že není možné). Jako katolíci bychom měli uznat smutnou
skutečnost tohoto tříštění mezi křesťany. Měli bychom to považovat za výzvu k práci na sjednocení Církve,
jednoho Těla Kristova.

Satan se těší vzájemnému odporu lidí. Jako křesťané by naše jednota měla vydávat svědectví pravdy o všem,
čemu věříme. Naše jednota – včetně jednoty, kterou zažíváte se svými bratry v Exodu – by měla být
jednotou, která přečká všechny bouře. Nastal čas, abychom všichni táhli za jeden provaz, pevněji, silněji…
potřebujete, abyste se stali svobodnými a abyste svobodnými i zůstali. Proto neexistuje nic takového jako
křesťanský individualismus.

Muži Exodu by nikdy neměli být v Církvi zdrojem nejednoty nebo neshody. Pro jednotu se musíme zavázat
ke všemu, co Církev učí a věří, že je pravda. Udělejte si dnes čas pro rozhovor s Pánem o jednotě – jednotě
tajemného Kristova Těla, a jednotě Církve. Pociťujete-li bolest v něčem, co se týká Církve, nezapomeňte ji
s Pánem sdílet, a naslouchejte Jeho milující odpovědi.


%newday
\newpage
\section{Den 62 - PRÁCE S HOSPODINEM }
\zacatekDevatyTyden
\subsection*{Čtení na den}
\textbf{Exodus 26,15-30}
\newline
\textit{
\textsuperscript{15}Pro příbytek zhotovíš také desky z akáciového dřeva, aby se daly postavit.
\textsuperscript{16}Každá deska bude dlouhá deset loket a široká půldruhého lokte.
\textsuperscript{17}Každá bude mít dva čepy a jedna bude připojena k druhé; tak to provedeš na všech deskách pro příbytek.
\textsuperscript{18}Zhotovíš pro příbytek tyto desky: dvacet desek pro pravou jižní stranu.
\textsuperscript{19}Těch dvacet desek opatříš dole čtyřiceti stříbrnými patkami, po dvou patkách pod každou desku, k oběma jejím čepům.
\textsuperscript{20}I pro druhý bok příbytku, k severní straně, dvacet desek
\textsuperscript{21}a čtyřicet stříbrných patek, po dvou patkách pod každou desku.
\textsuperscript{22}Pro zadní stranu příbytku, k západu, zhotovíš šest desek.
\textsuperscript{23}Navíc zhotovíš dvě desky pro oba úhly příbytku při zadní straně.
\textsuperscript{24}Zdola budou přiloženy k sobě, navrchu budou těsně spojeny jedním kruhem. Tak tomu bude s oběma deskami v obou úhlech.
\textsuperscript{25}Bude tam tedy osm desek se stříbrnými patkami, celkem šestnáct patek, po dvou patkách pod každou deskou.
\textsuperscript{26}Zhotovíš také svlaky z akáciového dřeva, pět pro desky na jednom boku příbytku,
\textsuperscript{27}pět na druhém boku příbytku a dalších pět svlaků pro desky na zadní straně příbytku, k západu.
\textsuperscript{28}Prostřední svlak bude probíhat v poloviční výši desek od jednoho konce k druhému.
\textsuperscript{29}Desky potáhneš zlatem a zhotovíš k nim zlaté kruhy pro vsunutí svlaků; také svlaky potáhneš zlatem.
\textsuperscript{30}Příbytek postavíš podle ustanovení, jak ti bylo ukázáno na hoře.
}

\subsection*{Reflexe}
Bůh řídí každý detail stavby svatyně, ve které bude dlít. Dává pozor na každý detail (dokonce i rám a základ stanu své
přítomnosti), aby zajistil, že stan zůstane pevný a překoná jakoukoli odpor. Řídí stavbu základů a rámu, aby podpořil propracované
obklady a dům Archy úmluvy. „Nestaví-li dům Hospodin, nadarmo se namáhají stavitelé,“ (Ž 127,1). Má-li to být Boží příbytek,
musí být postaven s Bohem a v souladu s Jeho plánem.

To samé platí pro nás jako lidi: každý z nás je Božím příbytkem, a musí proto být stavěn s Bohem a v souladu s Jeho plánem.
Protože pokud Hospodin nestaví člověka, marně se lopotí ten, kdo ho staví. Až příliš často se snažíme sami sebe vybudovat.
Chybně věříme, že všechen náš úspěch a produktivita záleží pouze na našich vlastních silách. Snažíme se dosáhnout velikosti,
abychom se jednou mohli ukázat Otci a říct: „Tohle jsem vytvořil a nyní Ti to předkládám.“ To je ale velká lež.

Otec sám touží po tom, aby tě vychoval jako svého syna. Žádný dobrý otec nezplodí svého syna, aby ho poslal pryč a staral se o
sebe sám. Bůh věnuje pozornost každému detailu našeho života, vede nás, zkouší nás, podporuje nás, a těší se z našeho růstu a
postavení. Přesto všechno máme svobodu. Svobodu vložit svůj život do dobrých a zachraňujících Božích rukou, a svobodu utéct od
Jeho dobroty v marném pokusu vybudovat svůj vlastní život.

Vložili jste svou důvěru s Pána tím, že jste se dali na toto duchovní cvičení. Pokračujte v otevřené modlitbě k Bohu. Mějte v Něj
důvěru. Jste-li ochotní podrobit se plánu svého milujícího Otce, On vás postaví a vaše práce nebude marná. V dnešní modlitbě Ho
proste, jako svého Otce, aby vám ukázal směr a poradil vám, jak být mužem, jakým vás chce mít.


%newday
\newpage
\section{Den 63 - KRISTUS ROZTRHL OPONU}
\zacatekDevatyTyden
\subsection*{Čtení na den}
\textbf{Exodus 26,31-37}
\newline
\textit{
\textsuperscript{31}Zhotovíš také oponu z látky purpurově fialové, nachové a karmínové a z jemně tkaného plátna. Zhotovíš ji s umně vetkanými cheruby.
\textsuperscript{32}Zavěsíš ji na čtyřech sloupech z akáciového dřeva, potažených zlatem; háčky na nich budou ze zlata. Sloupy budou na čtyřech stříbrných patkách.
\textsuperscript{33}Zavěsíš oponu na spony a dovnitř tam za oponu vneseš schránu svědectví. Opona vám bude oddělovat svatyni od velesvatyně.
\textsuperscript{34}Ve velesvatyni vložíš na schránu svědectví příkrov.
\textsuperscript{35}Vně před oponu postavíš stůl a naproti stolu svícen při jižním boku příbytku; stůl dáš k severnímu boku.
\textsuperscript{36}Ke vchodu do stanu zhotovíš pestře vyšitý závěs z látky fialově purpurové, nachové a karmínové a z jemně tkaného plátna.
\textsuperscript{37}K závěsu zhotovíš pět sloupů z akáciového dřeva a potáhneš je zlatem; háčky na nich budou ze zlata; odliješ pro ně také pět bronzových patek.
}

\subsection*{Reflexe}
Opona chrámu byla používána pro oddělení světského od posvátného. Nikdo nemohl vstoupit za oponu, aniž by nebyl
oddělen pro posvátné povinnosti a prošel by si rituálem očištění. V Zahadě Edenu se člověk procházel s Bohem
svobodně a v jednotě. Když vstoupil hřích do světa, oddělil člověka od Boha. Hřích zničil jednotu, kterou pro nás Bůh
stvořil. Nenechte se oklamat, hřích má stále moc nás odloučit od Boha a od sebe i dnes.

Člověk si vybral nejednotu s Bohem, ale náš nebeský Otec, který zná své děti lépe než ony samy, ví, že jediná věc,
která nás naplní, je jednota s Ním. Člověk si právem zasloužil smrt kvůli jeho nevěrnosti Bohu, ale Bůh jednal
milostivě. Poslal svého jednorozeného Syna, aby na sebe vzal trest, který si člověk zasloužil, a naplnil tak starou
smlouvu. Obětovaná krev našeho ukřižovaného Pána roztrhla oponu, která dělila člověka od Boha, a dala lidem novou
důvěru, aby se mohli přiblížit k Bohu a Jeho svaté přítomnosti. „Protože Ježíš obětoval svou krev, smíme se, bratří,
odvážit vejít do svatyněcestou novou a živou, kterou nám otevřel zrušením opony – to jest obětováním svého těla (…)
přistupujme před Boha s opravdovým srdcem a v plné jistotě víry.“ (Žid 10,19-20; 10,22)

Když voják probodl Ježíšův bok, vyšla krev a voda. Kopí otevřelo Jeho bok (roztrhlo oponu) a učinilo Jeho Nejsvětější
Srdce (srdce Boha) znovu přístupné všem lidem. Bůh dovolil, aby byla opona roztržena, a otevřel vám své srdce.
Berete bytí v přítomnosti Jeho Svatého Těla a Krve na lehkou váhu? Jednáte, jako byste si zasloužili právo na
Eucharistii na základě svých vlastních zásluh nebo dokonce na základě toho, v jaké části světa žijete?

Kristova krvavá oběť, ne naše zásluhy, změnila navždy vztah člověka s Bohem. Udělejte si dnes čas, abyste si to
uvědomili a uctili oběť, kterou za vás Kristus položil. Vstupte skrze roztrženou oponu do Nejsvětějšího Srdce Božího.



% ===============================================
% ===== DESATY TYDEN
% ===============================================
%ukony
\newpage
\section*{Úkony (ukazatel cesty) pro 10. týden}

\textbf{Místo:} Drsná hornatá poušť, úpatí hory Sinaj

Jsou za vámi celé dva měsíce. Jaký to dar. Ve čteních tohoto týdne se Izraelité doslýchají Pánova slova o
kněžství a vhodných šatech pro službu ve svatostánku. Přemýšlejte, jakým způsobem svou nově nabytou
svobodu využijete, zatímco se váš nový život nyní formuje v horské poušti. Použijete ji ke službě Pánu (tak, že
hledíte na vrchol hory), nebo ke službě sami sobě (s hlavou dolů, nehledě na vaše blízké)? Jenom jedna cesta
vás zanechá svobodnými.

\subsection*{1. Zkontaktujte svou kotvu}
Po týdnech vykonávání nočního zkoumání dne (examen) jste začali dobře poznávat sami sebe. Jste v pokušení podvádět, přeskočit rozjímání, vyhnout se času s Bohem v modlitbě nebo dokonce zcela se obrátit zpět k otroctví? Zkontaktujte svou kotvu. Proto tu je. I když jste s ním již dnes mluvili, pokud jste v obavách, že byste mohli padnout, znovu mu dejte vědět. Je mnohem lepší v čase pokušení svou kotvu zkontaktovat, než padnout a muset vysvětlovat celému bratrstvu, proč jste v té době své kotvě nedali vědět.
\subsection*{2. Držte se disciplín}
Praxe askeze se stala všední a ztratila půvab, ale svoboda ne. Teď je ten čas, kdy se svaly vašich ctností opravdu zesilují. Vaše snaha, nebo její nedostatek, bude 91. dne vám i vašemu bratrstvu jasná. Budujte ctnosti. Držte se disciplín.
\subsection*{3. Stále se spoléhejte na Pána}
Dovedl vás až sem. Povede vás i dál, až do země zaslíbené. Jestli jste si začínali myslet, že jste tuto práci dokázali sami, možná vám ukáže, čeho jste schopni bez Něj. To znamená – možná vám dovolí, abyste se vrátili zpět do Egypta. Zůstaňte na úpatí hory Sinaj. Stále se spoléhejte na Pána.
\subsection*{Modlitba}
Modlete se, aby Pán osvobodil vás a vaše bratrství \newline
Modleme se za svobodu všech mužů v exodu, stejně tak, jako se oni modlí za vás.\newline
Ve jménu Otce i Syna i Ducha svatého … Otče náš… Ve jménu Otce i Syna i Ducha svatého … Amen.
\newpage


%newday
\newpage
\section{Den 64 - DUCHOVNÍ OBĚŤ}
\zacatekDesatyTyden
\subsection*{Čtení na den}
\textbf{Exodus 27,1-8}
\newline
\textit{
\textsuperscript{1}Zhotovíš oltář z akáciového dřeva pět loket dlouhý a pět loket široký. Oltář bude čtyřhranný, vysoký tři lokte.
\textsuperscript{2}Jeho čtyři úhly opatříš rohy; ty budou s ním zhotoveny z jednoho kusu. Potáhneš jej bronzem.
\textsuperscript{3}Zhotovíš k němu i hrnce na vybírání popela, lopaty a kropenky, vidlice a pánve na oheň. Všechno příslušné náčiní zhotovíš z bronzu.
\textsuperscript{4}Zhotovíš k němu také mřížový rošt z bronzu. Na čtyřech koncích mříže připevníš čtyři bronzové kruhy.
\textsuperscript{5}A zasadíš ji zdola pod obložení oltáře; mříž bude dosahovat až do poloviny oltáře.
\textsuperscript{6}Zhotovíš pro oltář i tyče; budou z akáciového dřeva a potáhneš je bronzem.
\textsuperscript{7}Tyče budou provléknuty skrze kruhy. Když bude oltář přenášen, budou tyče po obou jeho bocích.
\textsuperscript{8}Uděláš jej dutý z desek, jak ti bylo ukázáno na hoře; přesně tak jej udělají.
}

\subsection*{Reflexe}
Bůh poté, co dá Mojžíši instrukce ke stavbě stanu a Arše úmluvy, jej učí postavit oltář zápalné oběti. Tento
oltář má být centrem izraelitského uctívání, kde budou kněží obětovat beránky, kozly a býčky pro odpuštění
hříchů. V Egyptě si vzpurné děti Izraele vybíraly jiné bohy. Tím, že Izraelité obětují Hospodinu, Bůh dává
Izraeli fyzickou možnost zřeknout se falešných bohů Egypta a uctívat pouze Jeho, jediného pravého Boha.
Ačkoli je tento způsob obětování zvířat Izraelity nedokonalý a netrvalý, toto poslušné obětování je odpovědí
na Boží žádost, a pomáhá v cestě ke svobodě. Cílem totiž není obětování zvířat, ale obětování srdce Izraelitů
Bohu.

Vzdáme-li se věcí, které milujeme, budeme schopnijak zřeknout se falešných bohů (nebo věcí, které by se
falešnými bohy mohly stát), tak uctívat našeho Pána podle Jeho práva. Teď už jste ovšem zjistili, jak
jednoduché může být nechat se pohltit pýchou ve skutcích oběti a pokání. Říkáme stále všem lidem, kromě
bratrů našeho společenství, jak studené jsou ranní sprchy? Hledáme, komu bychom našimi řečmi posloužili?
Nikdy nezapomínejte, že vaše oběti jsou výrazem opravdové lásky k Bohu, ne výrazem vaší mužnosti.

Promluvte si dnes s Pánem o tom, jak může v tomto posledním měsíci dále očistit vaše srdce srkze skutky
askeze. Popovídejte si s Ním o tom, jak učinit oběti tak hlubokými na vašem duchu, jako jsou na vašem těle.





%newday
\newpage
\section{Den 65 - KULT NEFORMÁLNA }
\zacatekDesatyTyden
\subsection*{Čtení na den}
\textbf{Exodus 27,9-10,17-21}
\newline
\textit{
\textsuperscript{9}K příbytku uděláš také nádvoří. Na jižní, pravé straně nádvoří budou zástěny z jemně tkaného plátna, pro jednu stranu dlouhé sto loket.
\textsuperscript{10}Bude tam dvacet sloupů na dvaceti bronzových patkách; háčky ke sloupům a příčné tyče budou stříbrné.
\newline
\newline
\textsuperscript{17}Všechny sloupy na nádvoří budou dokola spojeny stříbrnými příčnými tyčemi; jejich háčky budou stříbrné a patky bronzové.
\textsuperscript{18}Délka nádvoří bude sto loket, šířka padesát a výška zástěn pět loket; budou z jemně tkaného plátna; patky sloupů budou bronzové.
\textsuperscript{19}Všechno náčiní příbytku pro veškerou bohoslužbu v něm i všechny jeho kolíky a všechny kolíky pro nádvoří budou z bronzu.
\textsuperscript{20}Ty pak přikážeš Izraelcům, aby ti přinášeli čistý vytlačený olivový olej ke svícení, aby bylo možno udržovat ustavičně svítící kahan.
\textsuperscript{21}Ve stanu setkávání před oponou, která bude zakrývat svědectví, bude o něj pečovat Áron se svými syny před Hospodinem od večera do rána. To je mezi Izraelci provždy platné nařízení pro všechna pokolení.
}

\subsection*{Reflexe}

Boha zajímají detaily. V posvátné liturgii na detailech záleží, protože vyjadřují Boží svatost. Když se liturgie slaví
důstojným způsobem, platně a zákonně, má schopnost nás naučit o Bohu mnohem víc, než by kdy to dokázala lidská
slova. Toto byl případ Izraelitů, a platí to i dnes. Špatně slavená liturgie nás učí špatné teologii. Proto nás Bůh učí, že si
od nás zaslouží to nejlepší ze života tím, že žádá to nejlepší, co máme, v Jeho svaté liturgii, ať už je to umění, výzdoba
nebo hudba. Zaslouží si to, protože je Bůh.

Pro laiky to znamená být velkorysí s dary, aby Církev a liturgii učinili krásnou pro Pána. To znamená nabídnout
zaplacení vyčištění varhan, kostelních zvonů nebo důstojného náboženského umění. Také to může znamenat pomáhat
našim synům vidět krásu liturgie a podněcovat je k záslužné službě u oltáře Páně.

Pro kněze a lektory či hudebníky je zodpovědnost ještě větší. Musíte studovat liturgii, rozumět, co Církev předepisuje
pro posvátná místa a pro posvátnou hudbu, a žít, co jste se naučili, s radostným přispěním Boží vůli. Bůh si zaslouží
víc než naše osobní preference; On si opravdu zaslouží to nejušlechtilejší podle předpisů Jeho nevěsty. Zaslouží si to
nejlepší.

Žijeme ve společnosti vyznávající kult neformálna. Kolik z nás nosí desetitisícové obleky do práce, a natáhne pouhé
džíny a triko na nedělní mši? Bohu záleží na detailech, a zaslouží si to nejlepší. Berte detaily vážně. Můžete začít tím,
že se na mši řádně obléknete, a naučíte svou rodinu dělat to samé. Tuto lekci si zapamatují. Pro kněze – zvažte
proškolení svých ministrantů, lektorů a ostatních, kteří se zapojují do přípravy mše svaté, o církevních dokumentech,
které se týkají důstojnosti liturgie. Mnoho z nich nemá ponětí o tom, že tyto dokumenty obsahující nauku o hudbě,
umění, architektuře a liturgii vůbec existují.

Zamyslete se, jakým způsobem vyznáváte kult neformálna ve vašem vztahu s Bohem. Popovídejte si s Ním o tom, jak
si přeje, abyste Ho více uctívali, zejména pokud jde o mši svatou.




%newday
\newpage
\section{Den 66 - VYVOLENÝ}
\zacatekDesatyTyden
\subsection*{Čtení na den}
\textbf{Exodus 28,1-4,36-38}
\newline
\textit{
\textsuperscript{1}Přikaž, aby předstoupil tvůj bratr Áron a s ním jeho synové z řad Izraelců, aby mi sloužili jako kněží: Áron a Áronovi synové Nádab, Abíhú, Eleazar a Ítamar.
\textsuperscript{2}Uděláš pro svého bratra Árona svaté roucho k slávě a ozdobě.
\textsuperscript{3}Promluvíš se všemi dovednými řemeslníky, které jsem naplnil duchem moudrosti; ti zhotoví Áronovi roucho, aby byl posvěcen a mohl mi sloužit jako kněz.
\textsuperscript{4}Toto budou roucha, která udělají: náprsník a nárameník, říza a tkaná suknice, turban a šerpa; taková svatá roucha udělají tvému bratru Áronovi a jeho synům, aby mi sloužili jako kněží.
\newline
\newline
\textsuperscript{36}Zhotovíš také květ z čistého zlata a vyryješ na něj, jako se vyrývá pečeť: ,Svatý Hospodinu‘.
\textsuperscript{37}Připevníš jej purpurově fialovou šňůrkou na turban; bude vpředu na turbanu.
\textsuperscript{38}Bude na Áronově čele. Áron bude odpovědný za nepravost při přinášení svatých věcí, které Izraelci oddělí jako svaté, za všechny jejich svaté dary. Květ bude na jeho čele trvale, aby nalezli zalíbení před Hospodinem.
}

\subsection*{Reflexe}
Dnes čteme o vyvolení Árona být knězem, svatým Hospodinu. Vy jste také byli vyvoleni být svatými Pánu, a to skrze
svátost křtu, biřmování a Eucharistie. Pokaždé, když jste se v průběhu posledních dvou měsíců praktikovali skutky
modlitby, askeze a bratrství, jste se rozhodli vystoupit z běžného života do svátostné skutečnosti, která vám byla dána.
Těmito třemi druhy činů jste vyznali svou identitu, být svatými Hospodinu, a svolili jste ke svému vyvolení.

Teď už dobře víte, že žít takovým způsobem je poměrně velkou výzvou. Být vyvoleni, snažit se o svatost, denně se
modlit, podstupovat askezi a bratrství – to vše vyžaduje nezměrné úsilí. Mnoho mužů Exodu zjistilo, že je jednodušší
žít svatě pro Pána v průběhu Exodu 90 než mimo něj. Proč? Zaprvé proto, že je Exodus 90 pojmenovaným vrcholem
hory s jasnými výzvami, které můžete vědomě zdolávat. Zadruhé, vaše bratrstvo tvoří oddanou skupinu mužů, kteří vás
nahoru potáhnou. Až pro příliš mnoho mužů minulých Exodů tyto dva aspekty s devadesátým dnem pominuly. To však
nemusí být případ vás a vašich bratrů.

Po tak namáhavém duchovním cvičení by bylo urážkou lidské identity „svatých pro Hospodina“ upadnout do
pohodlného křesťanství. Bratrstvo musí dělat po skončení Exodu 90 víc, než jen sedět a klábosit o světě. To je něco
k zamyšlení. Jak bude vypadat vaše bratrstvo v 91. dnu?

V tuto chvíli se na to ještě nemusíte příliš soustředit. Uděláte lépe, když budete se třemi dalšími týdny před sebou stále
pozornost upínat ke svému exodu. Teď už je ale vhodná doba se svými bratry mluvit o životě po Exodu. Tak daleko
byste se bez modlitby, askeze a bratrství nedostali. Život po Exodu bude požehnaný, pokud tyto tři dary udržíte naživu.

Kristus touží, abyste zůstali vyvolení, svatí Hospodinu. Nebude vás však nutit. Je to vaše každodenní volba. Žijete svůj
život jako vyvolení? Jak dnes můžete své vyvolení prožít hlouběji? Poproste Pána, aby vám pomohl odpovědět na tuto
otázku.





%newday
\newpage
\section{Den 67 - NEHODNÍ: NAŠE POTŘEBA BOHA}
\zacatekDesatyTyden
\subsection*{Čtení na den}
\textbf{Exodus 29,1-9}
\newline
\textit{
\textsuperscript{1}Toto s nimi uděláš, až je budeš světit, aby mi sloužili jako kněží: Vezmeš jednoho mladého býčka a dva berany bez vady,
\textsuperscript{2}nekvašené chleby, nekvašené bochánky zadělané olejem a nekvašené oplatky pomazané olejem; uděláš je z bílé pšeničné mouky.
\textsuperscript{3}Dáš to do jednoho koše a přineseš to v tom koši jako dar; přivedeš i býčka a oba berany.
\textsuperscript{4}Potom předvedeš Árona a jeho syny ke vchodu do stanu setkávání a omyješ je vodou.
\textsuperscript{5}Vezmeš kněžská roucha, oblékneš Áronovi suknici a řízu pod nárameník, i nárameník a náprsník. Pak ho opásáš umně utkaným nárameníkovým pásem.
\textsuperscript{6}Na hlavu mu vložíš turban a na turban připevníš svatou čelenku.
\textsuperscript{7}Nato vezmeš olej pomazání, vyleješ jej na jeho hlavu a pomažeš ho.
\textsuperscript{8}Pak předvedeš jeho syny, oblékneš jim suknice,
\textsuperscript{9}přepášeš je šerpou, Árona i jeho syny, a vstavíš jim na hlavu mitry. I stane se jim kněžství provždy platným nařízením. Tak uvedeš v úřad Árona a jeho syny.
}

\subsection*{Reflexe}
Vžijte se do pocitů Izraelity. Mojžíš má na příkaz Hospodina pomazat Árona a jeho syny na svaté kněžstvo. Při
pohledu na zmatený, ale strukturovaný proces pomazání byste se měli zamyslet nad svou lidskou bezvýznamností. Tito
muži byli vybráni, byli povoláni Hospodinem, aby mu sloužili jménem Božího hříšného lidu. Přesto sami o sobě nejsou
ani náhodou bezchybní. Také oni jsou hříšní a není jejich zásluhou, že smí sloužit před Boží tváří.

Totéž platí o mužích, kteří jsou svěceni v dnešní Církvi. Pokud jste nikdy nebyli na kněžském svěcení, zkuste nějaké
navštívit v nadcházejícím roce. Jedním z nejsilnějších momentů světících obřadů je pokorný akt prostrace. Muži, kteří
mají být vysvěceni, leží tváří k zemi. Samotnoupolohou uznávají svou bezvýznamnost pro kněžský úřad a svou potřebu
Boží podpory.

Pokud jste ženatý, rozpomeňte se na svůj svatební den. Stál jste před oltářem, tváří v tvář své nevěstě, a chystali jste se
navzájem si vyměnit své sliby. Její otec seděl jen o pár kroků dál a díval se, jak se jeho malá holčička připravuje
odevzdat se do péče jiného muže. Ten jiný muž (vy) s ní nebyl, když přicházela na svět, ani ji nezvedal ze země, když
se učila chodit, ani ji nedržel v náručí, když oplakávala prvního chlapce, který nedokázal dát patřičnou péči jejímu
srdci, což vždy její otec uměl. Ten jiný muž (vy) jí nebyl hoden. Když jste tenkrát vzhlížel ke kráse své nevěsty, stejně
jako její otec jste pociťoval svou nehodnost vzít si ji.

Bez ohledu na to, zda jste ženatí, svobodní, nebo žijete zasvěceným způsobem života, vás Bůh volá ke službě Jemu
samému a své Církvi. Dnes si máte uvědomit svou hříšnost a své nedostatky. To nemá sloužit k vašemu odsouzeníani k
ponížení, má vás to však víc než kdy dřív přimět k většímu uvědomění si vlastní potřeby Boha. Vzývejte dnes Pána.
Padněte před Ním ve svaté hodině na tvář a požádejteHo o milost, abyste mohli sloužit Jemu, Jeho Církvi a své rodině.





%newday
\newpage
\section{Den 68 - OBĚTOVAT NEUSTÁLE}
\zacatekDesatyTyden
\subsection*{Čtení na den}
\textbf{Exodus 29,38-46}
\newline
\textit{
\textsuperscript{38}Toto pak budeš přinášet na oltáři: každodenně dva jednoroční beránky.
\textsuperscript{39}Jednoho beránka přineseš ráno a druhého beránka přineseš navečer.
\textsuperscript{40}A k prvnímu beránkovi též desetinu éfy bílé mouky zadělané čtvrtinou hínu vytlačeného oleje a jako úlitbu čtvrtinu hínu vína.
\textsuperscript{41}Druhého beránka přineseš navečer; podobně jako jitřní obětní dar a příslušnou úlitbu jej přineseš v libou vůni, ohnivou oběť pro Hospodina.
\textsuperscript{42}Tuto každodenní zápalnou oběť budete přinášet u vchodu do stanu setkávání před Hospodinem po všechna vaše pokolení. Tam se s vámi budu setkávat, abych tam k tobě mluvil.
\textsuperscript{43}Budu se tam setkávat se syny Izraele a místo bude posvěceno mou slávou.
\textsuperscript{44}Posvětím stan setkávání a oltář; také Árona a jeho syny posvětím, aby mi sloužili jako kněží.
\textsuperscript{45}Budu přebývat uprostřed Izraelců a budu jejich Bohem.
\textsuperscript{46}Poznají, že já jsem Hospodin, jejich Bůh, že já jsem je vyvedl z egyptské země, abych přebýval uprostřed nich. Já jsem Hospodin, jejich Bůh.
}

\subsection*{Reflexe}
Dnešní čtení obsahuje jen nepatrný zlomek z dlouhého seznamu obětí, které měli Izraelité vykonávat. Bůh obzvláště
přikazuje, aby tyto oběti byly přinášeny “bez ustání” a “nepřetržitě”. Ačkoli to překládáme různými frázemi,
hebrejštinapoužívá jediný slovní kořen, a to tamyid, což znamená “nepřetržitě”.

Při popisu správného uctívání a služby ve svatyni se tamyid vyskytuje pro zdůraznění 34krát (od Ex 25,30 až po Num
29,38). Je nadmíru zřejmé, že Hospodin po Izraelitech chce, aby mu prokazovali chválu a úctu v každém okamžiku.
Izraelité nedostávají žádnou večerní přestávku nebo pár dní volna k tomu, aby mohli jít se svými kamarády do baru
nebo prožít trochu toho neškodného hýření. Jak je dobře, že Bůh nás lidi už nadále nevolá k takové svaté poslušnosti.

Nebo snad ano?

Jako to čteme v listu Židům: „Přinášejme tedy skrze Ježíše stále oběť chvály Bohu; naše rty nechť vyznávají jeho
jméno“ (Žd 13,15). A list dále pokračuje: „Nezapomínejme také na dobročinnost a štědrost, takové oběti se Bohu líbí“
(13,16).

Podobně jako Izraelité jsme i myvoláni neustálepřinášet Bohu oběti. Modlitba, askeze a bratrství, to nejsou jen
jednorázové požadavky. Není to jen módní trend nebo vzrušující dočasná výzva. Bůh po nich od nás touží neustále.

Jaký je váš dnešní postoj? Stále toužíte tuto cestu exodu opustit a sklouznout zpět k pohodlnému způsobu života? Nebo
vás Pán osvobodil, abyste vidělivětší dary, které vám dal, a slávu plánu, který ve vás skrze ně buduje? Učíte se modlit,
obětovat a táhnout své bratry ke svatosti. Jste formováni, abyste konali činy, o kterých si většina lidí (jsou-li k sobě
upřímní) může nechat jenom zdát. Poznáváte velikost těchto darů? Pohovořte si dnes s Pánem o touze tyto tři dary na
konci exodu využít nebo opustit. Naslouchejte Mu a nechte se utěšit láskyplným plánem, který s vámi má.





%newday
\newpage
\section{Den 69 - KADIDLO A NEBESKÁ LITURGIE }
\zacatekDesatyTyden
\subsection*{Čtení na den}
\textbf{Exodus 30,1-10}
\newline
\textit{
\textsuperscript{1}Zhotovíš také oltář k pálení kadidla. Z akáciového dřeva jej zhotovíš.
\textsuperscript{2}Bude čtyřhranný: loket dlouhý, loket široký a dva lokte vysoký. Jeho rohy budou z jednoho kusu s ním.
\textsuperscript{3}Potáhneš jej čistým zlatem, jeho vršek i jeho stěny dokola a jeho rohy, a opatříš jej dokola zlatou obrubou.
\textsuperscript{4}Zhotovíš pro něj rovněž dva zlaté kruhy, a to pod obrubou na obou jeho bocích; k oběma bočnicím je zhotovíš, aby držely tyče, na nichž bude nošen.
\textsuperscript{5}Tyče zhotovíš z akáciového dřeva a potáhneš je zlatem.
\textsuperscript{6}Oltář postavíš před oponu, která je před schránou svědectví, před příkrov, přikrývající schránu svědectví, kde se s tebou budu setkávat.
\textsuperscript{7}Na něm bude Áron pálit kadidlo z vonných látek. Bude je pálit každého rána, když bude ošetřovat kahany.
\textsuperscript{8}Bude je pálit i navečer, když bude kahany rozsvěcovat. Každodenně bude před Hospodinem pálit kadidlo po všechna vaše pokolení.
\textsuperscript{9}Nebudete na něm obětovat jiné kadidlo ani oběť zápalnou nebo přídavnou ani na něm nebudete přinášet úlitbu.
\textsuperscript{10}Jednou za rok vykoná Áron na jeho rozích smírčí obřady. Z krve smírčí oběti za hřích bude na něm vykonávat smírčí obřady jednou za rok po všechna vaše pokolení. Oltář bude velesvatý Hospodinu.“
}

\subsection*{Reflexe}
Užívání kadidla je po tisíce let běžné jak v židovské, tak křesťanské liturgii. Mnoho důvodů toto užití vysvětluje: jeho
vůně, jeho nadpozemský pohyb po svatyni, jeho schopnost pozvedat duši i oči k Bohu. Kadidlo nás propojuje s nebem.

Žalmista ve svých prosbách k Hospodinu volá: „Jako kadidlo ať míří má modlitba k tobě“ (Ž 141,2). V knize Zjevení
se píše, že „jiný anděl předstoupil se zlatou kadidelnicí před oltář; bylo mu dáno množství kadidla, aby je s modlitbami
všech posvěcených položil na zlatý oltář před trůnem. A vystoupil dým kadidla spolu s modlitbami posvěcených z ruky
anděla před Boží tvář“ (Zj 8,3). Jestli dým kadidla stoupá s modlitbami svatých k Božímu trůnu, jak důležitější je pro
nás, Církev putující, využít tak mocné svátostiny.

Rohy oltáře jsou součástí pozemské liturgie Izraelitů. V popisu nebeské liturgie v knize Zjevení čteme o hlase „od čtyř
rohů zlatého oltáře, který je před Bohem“ (Zj 9,13). Jak nebeský, tak pozemský oltář jsou ozdobeny rohy. To ukazuje,
že naše pozemská liturgie – její struktura, části, vůně, zvony, barvy, úkony – je navržena tak, aby co nejvěrněji
zrcadlila dokonalou úctu k Bohu, nebeskou liturgii. Takže až se příště při mši svaté setkáte s kadidlem, až se bude dým
rozlévat nad presbytářem, vězte, že se tím spojujete se svými předky, s anděly a svatými v jedné všeobecné oběti Bohu.

Rozjímejte dnes chvíli nad svatou liturgií. Dejte Bohu čas, aby vám o každé části řekl vše – její funkci, její plody.
Máte-li nezodpovězené otázky o určitých částech liturgie, zapište si je, abyste mohli odpovědi vyhledat. Tak jako
kadidlo, i ostatní části liturgie, jsou-li vhodně užity, nás mají něco naučit o Bohu, o nás a o našem vztahu s Ním.





%newday
\newpage
\section{Den 70 - PATŘÍME BOHU}
\zacatekDesatyTyden
\subsection*{Čtení na den}
\textbf{Exodus 30,11-21}
\newline
\textit{
\textsuperscript{11}Hospodin promluvil k Mojžíšovi:
\textsuperscript{12}„Když budeš pořizovat soupis Izraelců povolaných do služby, dá každý při sčítání výkupné Hospodinu za svůj život, aby je při sčítání nestihla nenadálá pohroma.
\textsuperscript{13}Toto dá každý, kdo přejde mezi povolané do služby: půl šekelu podle váhy určené svatyní; šekel je dvacet zrn. Tato půlka šekelu je oběť pozdvihování pro Hospodina.
\textsuperscript{14}Každý, kdo přejde mezi povolané do služby, od dvacetiletých výše, odvede Hospodinu oběť pozdvihování.
\textsuperscript{15}Bohatý nebude dávat více a nemajetný nedá méně než půl šekelu, když se bude odvádět Hospodinu oběť pozdvihování na vykonání smírčích obřadů za vaše životy.
\textsuperscript{16}Vezmeš od Izraelců obnos na smírčí oběti a věnuješ jej na službu při stanu setkávání. To bude pro Izraelce jako připomínka před Hospodinem, když se za vás budou vykonávat smírčí obřady.“
\textsuperscript{17}Hospodin dále mluvil k Mojžíšovi:
\textsuperscript{18}„Zhotovíš také bronzovou nádrž k omývání s bronzovým podstavcem a umístíš ji mezi stanem setkávání a oltářem a naleješ tam vodu.
\textsuperscript{19}Áron a jeho synové si jí budou omývat ruce a nohy.
\textsuperscript{20}Když budou přicházet ke stanu setkávání nebo když budou přistupovat k oltáři, aby konali službu a obraceli ohnivou oběť Hospodinu v obětní dým, budou se omývat vodou, aby nezemřeli.
\textsuperscript{21}Budou si omývat ruce i nohy, aby nezemřeli. To je provždy platné nařízení pro něho i pro jeho potomstvo po všechna jejich pokolení.“
}

\subsection*{Reflexe}
Bůh Mojžíšovi nařizuje, aby provedl sčítání lidu. To proto, aby lidu ukázal, že Mu náleží. Izraelský národ nejsou pouhé
izolované bytosti, nežijí sami pro sebe, ani podle svého; jsou vyvoleným lidem Božím, sjednoceni a povoláni
k věrnosti Bohu i sobě navzájem jako jeden lid Izraele.

Tento příběh má obrovský protiklad v události, která nastane o 500 let později. Král David, svedený ďáblem, se
rozhodne provést sčítání izraelského lidu (1 Pa 21,1). Výsledek jeho počínání má nedozírné následky pro všechen lid;
na Izrael byl uvalen třídenní mor a 70 tisíc mužů zemřelo (1 Pa 21,14). Jaký je rozdíl mezi sčítáním lidu krále Davida a
Mojžíše? Mojžíš sečetl Boží lid, aby poznali, že patří Bohu. Král David sečetl Boží děti, aby je prohlásil za své
vlastnictví.

Nezáleží na tom, kdo má nad námi moc, nezáleží na tom, kterému týmu fandíme nebo které zprávy odebíráme, naším
prvním a jediným vládcem je náš předobrý Otec v nebesích. Ať už jste hrdým Moravanem, Severočechem nebo
Pražanem, v první řadě mějte na paměti, že jako pokřtěným údem Těla Kristova jste především a nadevším křesťanem.
Tento titul s sebou nese nedocenitelnou skutečnost, která by vám měla přinášet radostvětší než jakékoli jiné označení.
V tomto titulu se pro nás skrývá daleko větší příslib než cokoli jiného – věčnost v nebi a dokonalou jednotu s Bohem –
především je však Bůh schopen tomuto svému slibu dostát.

Zamyslete se v modlitbě nad tím, co to znamená být křesťanem a co to znamená patřit Bohu. Vzdejte dnes Pánu chválu
s největší vděčností za jeho dar společenství s Ním a Jeho lidem, na nemž nám dal tu čest se podílet.


% ===============================================
% ===== JEDENACTY TYDEN
% ===============================================
%ukony
\newpage
\section*{Úkony (ukazatel cesty) pro 11. týden}

\textbf{Místo:} Drsná hornatá poušť, úpatí hory Sinaj

Izraelité jsou ve čteních tohoto týdne projevem lidské slabosti. Předvádějí, jak je jednoduché padnout zpět do
starých zvyků. Ukazují na potřebu stále aktivně vyhledávat Hospodina. Na druhou stranu nám tento týden
Levité dokazují, že ve zhýralém světě je možné nebojácně/odvážně důvěřovat/být věrný Hospodinu. Pán
však dokazuje něco ještě většího. Izraelitům i nám prokazuje, že navždy zůstane věrný své smlouvě, i když
ho oni – a my – zklamou. Pozorně naslouchejte slovům Písma tohoto týdne. Poslouchejte, jak Bůh odhaluje
Jeho věrnou lásku k nám.

\subsection*{1. Vzdejte se kontroly}
Tak jako mnoho Izraelitů ztrácelo trpělivost s Mojžíšem, když byl 40 dní na vrcholu hory Sinaj, tak i vy možná ztrácíte trpělivost s Exodem. Možná ztrácíte trpělivost v modlitbě (možná k vám Pán právě nemluví), s nepohodlností asketických disciplín, nebo s vašimi bratry. Přes to všechno právě tyto tři věci vás přivedly až sem, do 11. týdne. Buďte trpěliví, důvěřujte a odevzdejte kontrolu Pánu. Bůh vám dá vysvobození, které hledáte.
\subsection*{2. Vzpomeňte si na své proč}
Pravděpodobně jste již začali vidět naplnění svého proč. Možná trávíte více času se svou ženou, dětmi, farností, s Pánem. Držte se svého proč. Stále jste na cestě.
\subsection*{3. Dobře se potřetí vyzpovídejte}
Tři svátosti smíření vám možná připadají jako mnoho během jediného exodu. Přesto však biskupové radí, abychom ke zpovědi chodili nejlépe jednou za měsíc, a tak tři zpovědi za 90 dní je docela akorát. Nenechte sebe či své hříchy překazit vaši cestu ke svobodě. Dobře se tento týden vyzpovídejte.
\subsection*{4. Rozdávejte/sdílejte radost}
Jestli se vám tento týden podařilo pěstovat radost skrze vděčnost a naději, šiřte ji dál ve svém bratrstvu. Je možné, že ne všichni vaši bratři jsou na stejné úrovni radosti. Pomozte jim pěstovat vděčnost a naději. Sdílejte s nimi tento týden radost.
\subsection*{Modlitba}
Modlete se, aby Pán osvobodil vás a vaše bratrství \newline
Modleme se za svobodu všech mužů v exodu, stejně tak, jako se oni modlí za vás.\newline
Ve jménu Otce i Syna i Ducha svatého … Otče náš… Ve jménu Otce i Syna i Ducha svatého … Amen.
\newpage

%newday
\newpage
\section{Den 71 - TOUŽÍCÍ PO BOHU}
\zacatekJedenactyTyden
\subsection*{Čtení na den}
\textbf{Exodus 30,22-33}
\newline
\textit{
\textsuperscript{22}Hospodin dále mluvil k Mojžíšovi:
\textsuperscript{23}„Ty si pak vezmi nejvzácnější balzámy: pět set šekelů tekuté myrhy, poloviční množství, totiž dvě stě padesát šekelů balzámové skořice, dvě stě padesát šekelů puškvorce,
\textsuperscript{24}pět set šekelů kasie podle váhy šekelu svatyně a jeden hín olivového oleje.
\textsuperscript{25}Z toho připravíš olej svatého pomazání, vonnou mast odborně smísenou. To je olej svatého pomazání.
\textsuperscript{26}Pomažeš jím stan setkávání a schránu svědectví,
\textsuperscript{27}stůl se vším náčiním, svícen s náčiním a kadidlový oltář,
\textsuperscript{28}rovněž oltář pro zápalnou oběť se vším náčiním a nádrž s podstavcem.
\textsuperscript{29}Posvětíš je a budou velesvaté. Cokoli se jich dotkne, bude svaté.
\textsuperscript{30}Pomažeš Árona a jeho syny a posvětíš je, aby mi sloužili jako kněží.
\textsuperscript{31}Izraelcům pak vyhlásíš: To je olej svatého pomazání pro mne po všechna vaše pokolení.
\textsuperscript{32}Nesmí se vylít na tělo nepovolaného člověka. Nepřipravíte podobný olej stejného složení. Je svatý a zůstane pro vás svatý.
\textsuperscript{33}Každý, kdo namíchá podobnou mast nebo z ní dá nepovolanému, bude vyobcován ze svého lidu.“
}

\subsection*{Reflexe}
Bůh Mojžíšovi říká, že posvátný olej na pomazání nesmí být vylit na obyčejné lidi. Tento olej je vyhrazen pouze
vyvoleným mužům, těm, kteří budou sloužit Bohu a jeho lidu jako kněží. Byli jste pokřtěni v Krista jako kněz, prorok a
král. V obřadu křtu jste byli doslova pomazáni křižmem, posvátným olejem. Žijete v souladu s tímto velkým
pomazáním? Žijete život ve službě Bohu a jeho lidu? Nebo žijete jako obyčejný člověk?

Během posledních sedmdesáti dnů jste jako nikdy předtím změnili svůj vlastní způsob života a převzali dary modlitby,
askeze a bratrství. Poddali jste se Pánově vůli a denně jste požívali životodárné Boží Slovo. Ačkoli jste unaveni, jste
dobře vyživeni. Mezitím se ve vaší komunitě nachází mnoho velmi podvyživených mužů. Jejich vnitřní slabost je činí
duchovně neschopnými vést své rodiny nebo dokonce je každou neděli přivést na mši. Svět tvrdí, že jim nabízí výživu,
ale nedává nic jiného než prázdné kalorie. Co jim dáváte vy? Dokonce i zmínka o modlitbě před jídlem nebo o tom, že
jste se zúčastnili nedělní mše, jim může dát pocítit chuť sladké výživy, po které dlouho touží. A co víc, představte si,
co jim může dát pozvání na mši, společenství mužů nebo dokonce jen večeře s vaší rodinou.

Nejste obyčejní lidé. Byli jste pomazáni. Mají z vašeho pomazání vaši blízcí užitek? Udělejte si řádné zpytování
svědomí a poté si promluvte s Pánem o tom, jak chce, abyste sloužili Jeho lidu. Pokud vám Pán dá i konkrétní jména,
zapište si je. Potom se ujistěte, že jste se k těmto lidem dostali. Pán bude s vámi a požehná vám ve vaší odvaze.


%newday
\newpage
\section{Den 72 - PRACOVAT A ODPOČÍVAT}
\zacatekJedenactyTyden
\subsection*{Čtení na den}
\textbf{Exodus 31,1-6.12-17}
\newline
\textit{
\textsuperscript{1}Hospodin promluvil k Mojžíšovi:
\textsuperscript{2}„Hleď, povolal jsem jménem Besaleela, syna Urího, vnuka Chúrova, z pokolení Judova.
\textsuperscript{3}Naplnil jsem ho Božím duchem, totiž moudrostí, důvtipem a znalostí každého díla,
\textsuperscript{4}aby uměl dovedně pracovat se zlatem, stříbrem a mědí,
\textsuperscript{5}opracovávat kameny pro vsazování a obrábět dřevo ke zhotovení jakéhokoli díla.
\textsuperscript{6}Dal jsem mu také k ruce Oholíaba, syna Achísamakova z pokolení Danova. A do srdce každého dovedného řemeslníka jsem vložil moudrost, aby zhotovili všecko, co jsem ti přikázal:
\newline
\newline
\textsuperscript{12}Hospodin řekl Mojžíšovi:
\textsuperscript{13}„Promluv k Izraelcům: Dbejte na mé dny odpočinku; to je znamení mezi mnou a vámi pro všechna vaše pokolení, abyste věděli, že já Hospodin vás posvěcuji.
\textsuperscript{14}Budete dbát na den odpočinku; má být pro vás svatý. Kdo jej znesvětí, musí zemřít. Každý, kdo by v něm dělal nějakou práci, bude vyobcován ze společenství svého lidu.
\textsuperscript{15}Šest dní se bude pracovat, ale sedmého dne bude slavnost odpočinutí, Hospodinův svatý den odpočinku. Každý, kdo by dělal nějakou práci v den odpočinku, musí zemřít.
\textsuperscript{16}Ať tedy Izraelci dbají na den odpočinku a dodržují jej po všechna svá pokolení jako věčnou smlouvu.
\textsuperscript{17}To je provždy platné znamení mezi mnou a syny Izraele. V šesti dnech totiž učinil Hospodin nebe a zemi, ale sedmého dne odpočinul a oddechl si.“
}

\subsection*{Reflexe}
Bůh dává velké požehnání dvěma svým dělníkům, Besaleelovi a Oholíabovi. Tito muži dostali mnoho darů, aby vrátili
Bohu Jeho slávu ve skvostech, které vyrábějí a staví na Jeho počest.

Jak často slýcháme od nadaných lidí, že všechny jejich dovednosti a úspěchy jsou jen výsledkem jejich tvrdé práce.
Všechno, co máme, jsme dostali. Co by byl takový člověk bez vzduchu pro své plíce? Poděkujme a chvalme Dárce
všech dobrých darů za Jeho velkorysost.

Druhá část dnešního čtení znovu projednává téma odpočinku. Není náhodou, že pokyny týkající se odpočinku navazují
na instrukce o daru odborné práce. Po staletí Církev učí, že pracujeme, abychom si mohli odpočinout. To je pravý opak
současného myšlení, které tvrdí, že odpočíváme proto, abychom mohli znovu pracovat. Neděle by měla být posvátnou
a chráněnou, nejprve pro modlitbu, pak pro rodinu, přátele, čtení, výlety, rybaření, rekreaci, a hlavně pro mši svatou.

Když bude práce a kariéra zapomenuta, čas strávený s rodinou kolem večerního stolu zůstane v mysli jako radostná
vzpomínka. Podívejte se jen na poslední měsíc. Jak jste si vedli v dodržování posvátnosti neděle od vašeho rozjímání
nad její důležitostí 41. dne? Pokud jste kněz či jáhen, vztahují se na vás jiná kritéria, ale vaše povinnost odpočinku
stále trvá. Jakkoli jste povoláni, byli jste věrni Božímu volání k odpočinku? Nebo jste zůstali otrokem své práce na
úkor těch kolem vás?

Vezměte dnes tyto otázky s sebou do své svaté hodinky. Předložte své odpovědi Pánu a naslouchejte, jak vás jimi bude
vést ke větší svobodě.



%newday
\newpage
\section{Den 73 - POŠETILOST}
\zacatekJedenactyTyden
\subsection*{Čtení na den}
\textbf{Exodus 32,1-6}
\newline
\textit{
\textsuperscript{1}Když lid viděl, že Mojžíš dlouho nesestupuje z hory, shromáždil se k Áronovi a naléhali na něho: „Vstaň a udělej nám boha, který by šel před námi. Vždyť nevíme, co se stalo s Mojžíšem, s tím člověkem, který nás vyvedl z egyptské země.“
\textsuperscript{2}Áron jim řekl: „Strhněte zlaté náušnice z uší svých žen, synů a dcer a přineste je ke mně!“
\textsuperscript{3}I strhal si všechen lid z uší zlaté náušnice a přinesli je k Áronovi.
\textsuperscript{4}On je od nich vzal, připravil formu a odlil z toho sochu býčka. A oni řekli: „To je tvůj bůh, Izraeli, který tě vyvedl z egyptské země.“
\textsuperscript{5}Když to Áron viděl, vybudoval před ním oltář. Potom Áron provolal: „Zítra bude Hospodinova slavnost.“
\textsuperscript{6}Nazítří za časného jitra obětovali oběti zápalné a přinesli oběti pokojné. Pak se lid usadil k jídlu a pití. Nakonec se dali do nevázaných her.
}

\subsection*{Reflexe}
Příběh Izraelitů doposud vyprávěl o synovství, otroctví, svobodě, a především o Boží svrchovanosti. Ale přes všechno,
co pro ně Bůh udělal v jejich exodu, tito lidé stále padají do léčky sloužil falešným modlám.

Chtěli bychom s úšklebkem říct: „Pošetilci. Copak nevidí, co všechno pro ně a jejich rodiny Hospodin udělal?“ A co
soudit sami sebe? Podívejte se na svůj vlastní život. Stejně jako Izraelité jste strávili už nějaký čas v poušti. Jakým
modlám ještě stále toužíte sloužit? Nějakému seriálu, bez kterého se neobejdete, sportovnímu týmu, který je součástí
vaší identity, stravovacím návykům, díky kterým se držíte ve formě, nebo hodině, která by měla být věnována
modlitbě, ale vy si ji chcete nechat pro sebe? Abychom se vystříhali falešných model, musíme uznat, co všechno pro
nás Pán udělal. Zásadní bude naše snažení a disciplína těchto posledních bezmála tří týdnů. Cítíte se hladoví, ale
Hospodin vám dá obživu, kterou potřebujete. Otevřete svá srdce, abyste ji mohli přijmout, a důvěřujte Mu.

Pravděpodobně se setkáte se svými bratry už jen dvakrát nebo třikrát. Na nejbližším setkání si naplánujte váš život po
exodu. Pokud vy a vaše bratrstvo takový plán nemáte, brzy upadnete do stylu života jako před exodem (stejně jako
Izraelité v dnešním čtení). Budete po exodu potřebovat přestávku, a většina mužů Exodu se shodlo na tom, že dva
týdny jsou akorát – dost času pro oslavu, ale zároveň ještě netolik, abyste ztratili své nabyté návyky. Bude-li se vaše
bratrstvo setkávat i po skončení Exodu, modlitba i askeze potrvají. Pokud se však vaše bratrstvo vytratí nebo rozpadne,
tak se tak stane i s vaší modlitbou a askezí. Nebojte se, nemusíte znovu podstupovat všechny tvrdé disciplíny Exodu
90, i když se vaše bratrstvo bude nadále scházet. Stejně byste ale měli učinit nějaké rozhodnutí pro modlitbu, askezi a
bratrství. Smyslem tohoto duchovního cvičení je osvobodit muže skrze dobu očišťování. Ve Dni 91 by vaše bratrstvo
mělo zůstat pohromadě, abyste si pomáhali udržet si svobodu. Na cestě svobody je nemožné jít sám (zkuste si žít
bratrství jen sami se sebou).

Pomůžete svým bratrům zůstat svobodnými pro dobro Církve a jejich rodin? Co pro vás znamenají? Přineste dnes
odpovědi na tyto otázky Pánu. Zvnitřněte si Jeho vhled do této oblasti.



%newday
\newpage
\section{Den 74 - PRIORITY A MODLY}
\zacatekJedenactyTyden
\subsection*{Čtení na den}
\textbf{Exodus 32,7-20}
\newline
\textit{
\textsuperscript{7}I promluvil Hospodin k Mojžíšovi: „Sestup dolů. Tvůj lid, který jsi vyvedl z egyptské země, se vrhá do zkázy.
\textsuperscript{8}Brzy uhnuli z cesty, kterou jsem jim přikázal. Odlili si sochu býčka a klanějí se mu, obětují mu a říkají: ‚To je tvůj bůh, Izraeli, který tě vyvedl z egyptské země.‘“
\textsuperscript{9}Hospodin dále Mojžíšovi řekl: „Viděl jsem tento lid, je to lid tvrdé šíje.
\textsuperscript{10}Teď mě nech, ať proti nim vzplane můj hněv a skoncuji s nimi; z tebe však udělám veliký národ.“
\textsuperscript{11}Mojžíš však prosil Hospodina, svého Boha, o shovívavost: „Hospodine, proč plane tvůj hněv proti tvému lidu, který jsi vyvedl velikou silou a pevnou rukou z egyptské země?
\textsuperscript{12}Proč mají Egypťané říkat: ‚Vyvedl je se zlým úmyslem, aby je v horách povraždil a nadobro je smetl z povrchu země.‘ Upusť od svého planoucího hněvu. Dej se pohnout k lítosti nad zlem, jež proti svému lidu zamýšlíš.
\textsuperscript{13}Rozpomeň se na Abrahama, na Izáka a na Izraele, své služebníky, kterým jsi sám při sobě přísahal a vyhlásil: Rozmnožím vaše potomstvo jako nebeské hvězdy a celou tuto zemi, jak jsem řekl, dám vašemu potomstvu, aby ji navěky mělo v dědictví.“
\textsuperscript{14}A Hospodin se dal pohnout k lítosti nad zlem, o němž mluvil, že je dopustí na svůj lid.
\textsuperscript{15}Mojžíš se obrátil a sestupoval z hory s dvěma deskami svědectví v ruce. Desky byly psány po obou stranách, byly popsané po líci i po rubu.
\textsuperscript{16}Ty desky byly dílo Boží, i písmo vyryté na deskách bylo Boží.
\textsuperscript{17}Tu Jozue uslyšel, jak lid hlučí, a řekl Mojžíšovi: „V táboře je válečný ryk.“
\textsuperscript{18}Ale on odvětil: „To nezní zpěvy vítězů, to nezní zpěvy poražených, já slyším halas rozpustilých písní.“
\textsuperscript{19}Když se Mojžíš přiblížil k táboru a uviděl býčka a křepčení, vzplanul hněvem, odhodil desky a pod horou je roztříštil.
\textsuperscript{20}Pak vzal býčka, kterého udělali, spálil jej ohněm, rozemlel na prach, nasypal do vody a dal Izraelcům pít.
}

\subsection*{Reflexe}
Dnes Bůh mluví tak, jako by byl připraven utnout ruku, která svádí tělo k hříchu (Mt 18,8). Podívejte se, jak Bůh a Mojžíš mluví o
Izraelitech. Bůh říká Mojžíšovi: „Sestup dolů, protože tvůj lid, který jsi vyvedl z Egypta…“ Mojžíš odpovídá: „Proč plane tvůj
hněv proti tvému národu, který jsi vyvedl z egyptské země?“ Mojžíš vrací vlastnictví neposlušných Izraelitů Bohu. Potom Boha
žádá o věrnost tím, že Mu připomene Jeho smlouvu s Abrahamem, Izákem a Izraelem.

Účelem této diskuze není, aby Bůh změnil svůj názor, ale aby si Mojžíš připomenul moc Boží věrnosti k Jeho lidu smlouvy. Když
Mojžíš objeví zlaté tele uprostřed jeho lidu, zničí jej, stejně jako Hospodin zničil falešné modly deseti egyptskými ranami. Navrch
ještě Mojžíš ukazuje synům Izraele marnost jejich model, když je donutí vypít prach ze zlatého telete. Tato odporná skutečnost je
jen odrazem prázdné modloslužby, které se účastnili.

Jste-li věrní disciplínám Exodu 90, světské klapky vám právě padají z očí. Stále jste pokoušeni ďáblem k otroctví ve vašich
oblíbených činnostech, ale jste nyní víc než kdy předtím schopnější vidět v nich nižší prioritu. Jakou prioritu těmto věcem dáte ve
Dni 91? Rozmýšlejte nad tím, jaké konkrétní věci a činnosti to jsou, a rozhodněte se, jestli byste se jich měli nadobro úplně vzdát,
nebo je jednoduše postavit na žebříček pod Boha a rodinu. Modlářství musí být vymíceno. Dobrým koníčkům ale může být
ponechána určitý stupeň priority. Sebeovládání je klíč. Bez něj se váš koníček může stát znovu modlou, stejně jako jím byl před
tímto duchovním cvičením. Nesmíme nechat otroctví připlazit se zpátky, aby nás odvádělo od naší rodiny a od Boha.

Tento čas abstinence je nejlepší dobou pro naplánování Dne 91. Příliš pozdě bude plánovat jej v 90. dni. Popovídejte si s Pánem
dnes konkrétně o vašich oblíbených věcech. Navrhněte plán a konkrétně ho s Ním proberte. Nechte Pána, aby vedl diskuzi, a poté
sepište konkrétní závěry o tom, jakým věcem dáte přednost a jaké úplně vyškrtnete. Tyto sepsané závěry vás budou hnát
k zopdovědnosti k Božímu plánu vaší svobody ve Dni 91.



%newday
\newpage
\section{Den 75 - DĚTINSKÝ ZPŮSOB ŽIVOTA}
\zacatekJedenactyTyden
\subsection*{Čtení na den}
\textbf{Exodus 32,21-24}
\newline
\textit{
\textsuperscript{21}Áronovi Mojžíš řekl: „Co ti tento lid udělal, že jsi naň uvedl tak veliký hřích?“
\textsuperscript{22}Áron odvětil: „Nechť můj pán tolik neplane hněvem! Ty víš, že tento lid je nakloněn ke zlému.
\textsuperscript{23}Řekli mi: ‚Udělej nám boha, který by šel před námi. Vždyť nevíme, co se stalo s Mojžíšem, s tím člověkem, který nás vyvedl z egyptské země.‘
\textsuperscript{24}Řekl jsem jim: Kdo má zlaté náušnice, ať si je strhne a donese mi je. Hodil jsem to do ohně, a tak vznikl tento býček.“  
}

\subsection*{Reflexe}
Podívejte se na Áronovu dětinskou obranu proti Mojžíšově otázce. Áron se nejdřív snaží odrazit Mojžíšův hněv
svalením viny na bláznovství Izraelitů. Poté vykládá polopravdivý příběh zakončený směšnou větou: „Hodil jsem
(zlato) do ohně, a tak vznikl tento býček,“ (Ex 32,24). Je jasné, že se Áron snaží vyhnout se odpovědnosti za své činy.

Kdo je Áron? Není to jen tak nějaký Izraelita, ale Mojžíšův bratr. Byl blízko Bohu a Jeho zázračným činům od začátku
Mojžíšova vedení. Mojžíšovi byl dán, aby byl jeho ústy (Ex 4,16). I přesto tento vyvolený muž padá do velkého hříchu.
Všimněte si toho. I lidé, kteří byli Bohu blízko tak dlouho (mnohem déle než 90 dní), mohou stále ve chvíli padnout.
Mohou dokonce padnout tak hluboko, že si vytvoří modlu a lžou o tom sobě samým i ostatním. To se stává, když
přestanou čekat na Boží Slovo a začnout poslouchat kakofonii světa.

Proč si Áron a Izraelité postavili zlatého býčka? Nepřinesl jim jídlo, vodu, ani kouzelnou jízdu na létajícím koberci až
do země zaslíbené; to ani nečekali. Postavili si zlatého býčka kvůli životu, který jim umožňoval. V Písmu od 73. dne
Áron vyhlásil den oslav zlatého býčka. Výsledkem byla modloslužba stejná, jakou prováděl Egypt. Exodus 32,6
popisuje toto uctívání: lidé obětovali tomuto býčkovi (modloslužba), seděli (lenost), jedli (obžerství), pili (opilství), a
dali se do nevázaných her (v podstatě do orgií). Tento životní styl je před Pánem Bohem nemyslitelný. Avšak je to
samozřejmostí před zlatým teletem.

Zamyslete se nad modlami, které jsme si udělali v životě před Exodem 90. Žádná z nich nám nenabízela spásu, ale my
jsme je stejně uctívali. Proč? Kvůli životu, který nám umožňoval. Oblíbené sportovní týmy nám dovolují křičet na
televizi před dětmi; jídlo nám dovoluje udělat si z jakéhokoli dne den slavení; telefon nám dovoluje nikdy nemuset
čelit realitě naší zkaženosti tím, že zabere veškeré chvíle ticha; a pornografie a masturbace nám dovoluje pocítit úlevu
a potěšení od ženy bez požadavku opravdového mezilidského vztahu.

Uctíváme modly ne proto, co nám dávají, ale kvůli životu, který nám dovolují žít. Boží cesta pro nás je život ve
svobodě. Kažá jiná je cestou otroctví. Popovídejte si dnes s Pánem o vašich minulých modlách. Jaký způsob života jste
si pro sebe pěstovali? Jak vám tento způsob života zamezoval být mužem, kterým máte / by jste měl být / potřebujete
být před Bohem a vaší rodinou?



%newday
\newpage
\section{Den 76 - OČIŠŤUJÍCÍ SPRAVEDLNOST}
\zacatekJedenactyTyden
\subsection*{Čtení na den}
\textbf{Exodus 32,25-35}
\newline
\textit{
\textsuperscript{25}Mojžíš viděl, jak si lid bezuzdně počíná; to Áron jej nechal počínat si bezuzdně ke škodolibosti jejich protivníků.
\textsuperscript{26}Tu se Mojžíš postavil do brány tábora a zvolal: „Kdo je Hospodinův, ke mně!“ Seběhli se k němu všichni Léviovci.
\textsuperscript{27}Řekl jim: „Toto praví Hospodin, Bůh Izraele: Všichni si připněte k boku meč, projděte táborem tam i zpět od brány k bráně a zabijte každý svého bratra, každý svého přítele, každý svého nejbližšího.“
\textsuperscript{28}Léviovci vykonali, co jim Mojžíš rozkázal; toho dne padlo z lidu na tři tisíce mužů.
\textsuperscript{29}Potom Mojžíš řekl: „Ujměte se dnes svého úřadu pro Hospodina každý, kdo povstal proti svému synovi či bratru, aby vám dnes dal požehnání.“
\textsuperscript{30}Nazítří pak Mojžíš řekl lidu: „Dopustili jste se velikého hříchu; avšak nyní vystoupím k Hospodinu, snad jej za váš hřích usmířím.“
\textsuperscript{31}Mojžíš se tedy vrátil k Hospodinu a řekl: „Ach, tento lid se dopustil velikého hříchu, udělali si zlatého boha.
\textsuperscript{32}Můžeš jim ten hřích ještě odpustit? Ne-li, vymaž mě ze své knihy, kterou píšeš!“
\textsuperscript{33}Ale Hospodin Mojžíšovi odpověděl: „Vymažu ze své knihy toho, kdo proti mně zhřešil.
\textsuperscript{34}A ty nyní jdi a veď lid, jak jsem ti řekl. Hle, můj posel půjde před tebou. A až přijde den mého trestu, potrestám jejich hřích na nich.“
\textsuperscript{35}Hospodin udeřil na lid, protože si dali udělat býčka, kterého zhotovil Áron.
}

\subsection*{Reflexe}
Mojžíš volá Izraelity zpět do služby Bohu, a s obavami se ptá: „Kdo je Hospodinův, ke mně!“ (Ex 32,36). Ze všech
dvanácti kmenů Izraele pouze jeden, Léviovci, vystoupí vpřed. Následuje vážný čin očištění, čistka, uznání
Izraelovyněvěry v jejich smlouvě s Bohe, a vysvěcení kmenové linie kněží.

Mojžíšova reakce se může zdát prudká a bezcitná. Přesto vidíme, že jeho reakce je vlastně úplný opak, když čteme
Písmo dál. Mojžíš statečně vyhledává Boha, protože chce odčinit hříchy Izraelitů, žádá slitování, a dokonce nabízí svůj
vlastní život místo života Izraelitů. Mojžíš není bezcitný, je ztělesněním sebedarování. Jeho nabídka zde
předznamenává odčiňující oběť Ježíše Krista. Podívejte se na kříž. Stejně jako Izraelité, i my si zasluhujeme smrt a
trest za naše hříchy, ale, stejně jako Mojžíš pro Izraelity, Kristus nabízí jeho vlastní život za ty naše. Na rozdíl
od Mojžíše Bůh Kristovu nabídku přijímá.

Nakonec se zamyslete nad Boží odpovědí na hřích Izraelitů. Posílá ránu na Izraelce jako čin spravedlnosti za jejich
nevěru vůči smlouvě. Není to však Bůh, který opouští nebo zabíjí svůj lid. Je to důsledek očisty. Stejně jako Mojžíš i

Bůh navazuje v milosrdenství a neustálé věrnosti svému lidu.
Zamyslete se nad posledním týdnem svého života. Kdy jste byli nevěrní Bohu? Z lásky k vám vás Bůh stále drží při
životě. Zasloužili jste si spravedlnost, a místo toho vám byla dána milost. Jaký to dar. Chvalte Pána v dnešní modlitbě
za milosrdenství, které vám tento týden ukázal, zejména hmatatelné milosti, které vám dal srkze Jeho svátosti.



%newday
\newpage
\section{Den 77 - KRIZE IDENTITY}
\zacatekJedenactyTyden
\subsection*{Čtení na den}
\textbf{Exodus 33,1-3.12-23}
\newline
\textit{
\textsuperscript{1}Hospodin promluvil k Mojžíšovi: „Vyjdi odtud, ty i lid, který jsi vyvedl ze země egyptské, do země, kterou jsem přísežně slíbil Abrahamovi, Izákovi a Jákobovi: Dám ji tvému potomstvu.
\textsuperscript{2}Pošlu před tebou svého posla a vypudím Kenaance, Emorejce, Chetejce, Perizejce, Chivejce i Jebúsejce.
\textsuperscript{3}Půjdete do země oplývající mlékem a medem. Já však nepůjdu uprostřed vás, abych vás cestou nevyhubil, neboť jste lid tvrdošíjný.“
\newline
\newline
\textsuperscript{12}Mojžíš řekl Hospodinu: „Hleď, ty mi říkáš: Vyveď tento lid. Ale nesdělil jsi mi, koho chceš se mnou poslat, ačkoli jsi řekl: ‚Já tě znám jménem, našel jsi u mne milost.‘
\textsuperscript{13}Jestliže jsem tedy nyní u tebe našel milost, dej mi poznat svou cestu, abych poznal tebe a našel u tebe milost; pohleď, vždyť tento pronárod je tvůj lid.“
\textsuperscript{14}Odvětil: „Já sám půjdu s vámi a dám vám odpočinutí.“
\textsuperscript{15}Mojžíš mu řekl: „Kdyby s námi neměla být tvá přítomnost, pak nás odtud nevyváděj!
\textsuperscript{16}Podle čeho jiného by se poznalo, že jsem u tebe našel milost já i tvůj lid, ne-li podle toho, že s námi půjdeš; tím budeme odlišeni, já i tvůj lid, od každého lidu na tváři země.“
\textsuperscript{17}Hospodin Mojžíšovi odvětil: „Učiním i tuto věc, o které mluvíš, protože jsi u mne našel milost a já tě znám jménem.“
\textsuperscript{18}I řekl: „Dovol mi spatřit tvou slávu!“
\textsuperscript{19}Hospodin odpověděl: „Všechna má dobrota přejde před tebou a vyslovím před tebou jméno Hospodin. Smiluji se však, nad kým se smiluji, a slituji se, nad kým se slituji.“
\textsuperscript{20}Dále pravil: „Nemůžeš spatřit mou tvář, neboť člověk mě nesmí spatřit, má-li zůstat naživu.“
\textsuperscript{21}Hospodin pravil: „Hle, u mne je místo; postav se na skálu.
\textsuperscript{22}Až tudy půjde moje sláva, postavím tě do skalní rozsedliny a zakryji tě svou dlaní, dokud nepřejdu.
\textsuperscript{23}Až dlaň odtáhnu, spatříš mě zezadu, ale mou tvář nespatří nikdo.“
}

\subsection*{Reflexe}
Jako křesťané jsme povoláni k životu hrdinských ctností. Dnes nám Mojžíš dává návod, jak žít ctnostně, zejména ve víře a
statečnosti. Poté, co Bůh mluví s Mojžíšem o tom, že se od Izraele vzdálí, Mojžíš statečně odpovídá. Prosí Boha Otce, aby s ním a
Izraelity zůstal.
Kdo jsme bez našich otců? Sirotci. Když mladý chlapec vidí svého otce odjíždět na dlouhou pracovní cestu, volá: „Neodjížděj!“

Chlapec má vrozenou potřebu svého otce. Stejně tak zná svou identitu syna, a je díky tomu ochoten vyžadovat otcovu přítomnost.
Když nedokážeme volat k Bohu Otci, někdy je to pouze proto, že nemáme dostatek odvahy na to, abychom k Němu volali. Často
to však není nedostatek odvahy, ale nedostatek víry. Jak často postrádáme víru, že Bůh je dobrý otec a my jsme opravdu Jeho syny.
Naše selhání volat k Otci kvůli nedostaku víry je jen znakem naší vnitřní krize identity jako synů Božích.

Mojžíš věří v jeho synovství a Boží otcovství. Proto také, když Boha prosí o něco vskutku dobrého, dostává přesně to, oč žádá.

Mojžíš dokonce žádá víc, protože si je vědom Boží přízně. Prosí, aby mohl vidět Boží tvář. Přestože této prosbě Bůh nevyhoví,
protože Bůh ví, že by to pro Mojžíše nebylo v této době dobré, dostává něco velkého, a to výsadu vidět Bohu záda.

Pán má toho tolik, co nám chce dát. Často jsme jen moc nedůvěřiví a zbabělí se zeptat. Proste dnes v modlitbě Pána o větší víru
v Jeho otcovství a o více odvahy Ho ve svém synovství prosit o všechno, co potřebujete.


% ===============================================
% ===== DVANACTY TYDEN
% ===============================================
%ukony
\newpage
\section*{Úkony (ukazatel cesty) pro 12. týden}

\textbf{Místo:} Poušť severně až severovýchodně od hory Sinaj

Po značném kroku zpět ve čteních tohoto týdne Izraelci znovu slýchají o potřebě sloužit Hospodinu, jejich
Bohu. Stále se snažíte sloužit Hospodinu? Před vámi jsou zásadní zbývající dny exodu. Stále jste očišťováni.
Zavázali jste se k devadesáti dnům exodu, ne sedmdesáti, ne osmdesáti, ani ne osmdesáti devíti. Služte
v těchto posledních dnech Pánu, ne tím, že uberete tempo, ale že ho zrychlíte. Prodlužte krok, zvyšte
rychlost. Dejte všechno, co máte, vaší cestě k cíli, jako Kristus dal všechno, co měl, na Kalvárii.

\subsection*{1. Znovu se zavázejte k dennímu rozjímání}
V něm se vám v následujících posledních dvou týdnech mnoho objasní. Buďte oddaným bratrstvem. Rozjímání vám pomůže pochopit, kam vás Slovo vede. Také vám naznačí poslední kroky na cestě ke Dni 91. Teď by pro vás či vaše bratry bylo dost smutné sejít z cesty.
\subsection*{2. Pokračujte ve svém závazku ke Slovu}
Ačkoli se kniha Exodus pomalu uzavírá, příběh Izraelitů v poušti zůstává nedokončený. Následujte Slovo, až vás povede od knihy Exodus k dalším třem navazujícím knihám: Leviticus, Numeri a Deuteronomium.
\subsection*{3. Upevněte svůj plán pro Den 91}
Den 91 se kvapem blíží, i když jste stále ještě v těchto posledních týdnech očišťováni. Nemá-li vaše bratrstvo žádný plán pro aktivní službu vaší farnosti a rodinám ve Dni 91, nic se neuskuteční. Naplánujte si, že si jako bratrstvo odpočinete na čtrnáct dní po skončení Exodu 90. Poté si naplánujte pokračování vašich setkání. Tato časová oběť ponese mnoho ovoce pro vás, vaši rodinu, vaše bratrstvo i vaši farnost. Nezahoďte prosím tuto šanci.
\subsection*{4. Rozdávejte/sdílejte radost}
Jestli se vám tento týden podařilo pěstovat radost skrze vděčnost a naději, šiřte ji dál ve svém bratrstvu. Je možné, že ne všichni vaši bratři jsou na stejné úrovni radosti. Pomozte jim pěstovat vděčnost a naději. Sdílejte s nimi tento týden radost.
\subsection*{Modlitba}
Modlete se, aby Pán osvobodil vás a vaše bratrství \newline
Modleme se za svobodu všech mužů v exodu, stejně tak, jako se oni modlí za vás.\newline
Ve jménu Otce i Syna i Ducha svatého … Otče náš… Ve jménu Otce i Syna i Ducha svatého … Amen.
\newpage

%newday
\newpage
\section{Den 78 - PŘEDVOLAT PÁNOVU VELIKOST}
\zacatekDvanactyTyden
\subsection*{Čtení na den}
\textbf{Exodus 34,1-9}
\newline
\textit{
\textsuperscript{1}Hospodin řekl Mojžíšovi: „Vytesej si dvě kamenné desky jako ty první, já na ty desky napíšu slova, která byla na prvních deskách, jež jsi roztříštil.
\textsuperscript{2}Připrav se na ráno; vystoupíš zrána na horu Sínaj a postavíš se tam na vrcholku hory ke mně.
\textsuperscript{3}Nikdo s tebou nevystoupí, a též ať se na celé hoře nikdo neukáže, ani brav nebo skot ať se nepase poblíž té hory.“
\textsuperscript{4}Mojžíš tedy vytesal dvě kamenné desky, jako ty první. Za časného jitra vystoupil na horu Sínaj, jak mu Hospodin přikázal, a do rukou vzal obě kamenné desky.
\textsuperscript{5}Tu sestoupil Hospodin v oblaku. Mojžíš tam zůstal stát s ním a vzýval Hospodinovo jméno.
\textsuperscript{6}Když Hospodin kolem něho přecházel, zavolal: „Hospodin, Hospodin! Bůh plný slitování a milostivý, shovívavý, nejvýš milosrdný a věrný,
\textsuperscript{7}který osvědčuje milosrdenství tisícům pokolení, který odpouští vinu, přestoupení a hřích; avšak viníka nenechává bez trestu, stíhá vinu otců na synech i na vnucích do třetího a čtvrtého pokolení.“
\textsuperscript{8}Mojžíš rychle padl na kolena tváří k zemi, klaněl se
\textsuperscript{9}a řekl: „Jestliže jsem, Panovníku, nalezl u tebe milost, putuj prosím, Panovníku, uprostřed nás. Je to lid tvrdošíjný; promiň nám však vinu a hřích a přijmi nás jako dědictví.“
}

\subsection*{Reflexe}
Hospodin své jméno, svou vlastní identitu, vyhlašuje Mojžíšovi i nám. Přečtěte si tento úryvek pozorně. Každý z těchto
Božích sebepopisujícíh výroků může být pro nás pomocí, když k Pánu voláme v časech modlitby, chvály, i soužení.
Protože je Bůh neměnná Pravda sama, můžeme být klidní, že tyto výroky, sdělené před tisíciletími Mojžíšovi, pořád
platí.

Když bude vaše srdce litovat poté, co padne do hříchu, volejte k Tomu, který je „plný slitování a milostivý, shovívavý,
nejvýš milosrdný a věrný.“ Když se cítíte, jako by vás Bůh opustil, ovlejte k tomu, který je „věrnost“. V Písmu nám
Bůh sděluje Jeho jména a vlastnosti, ne abychom se jím chlubili, ale aby nás vyzbrojil více důvody a způsoby, jak
k Němu volat. Když vcházíte do Slova, zůstaňte pozorní k daru Jeho jména a používejte jej ve vašem denním životě.
Mojžíš nám to dnes ukazuje.

Když Mojžíš slyší Boha mluvit o svých vlastnostech, spěchá sklonit hlavu až k zemi. Potom Hospodina prosí, aby
odpustil nepravost a hřích Izraelitů a znovu je nazval jako své vlastní. Pán tuto pokornou a důvěrnou žádost přijímá.
Izraelité mohou znovu vstoupit do smlouvy s Hospodinem. To nám ukazuje sílu znalosti Hospodinových jmen, a jejich
použití v praxi.

Zamyslete se dnes nad mnoha jmény a vlastnosti Pána. Vyberte si jedno, buď z dnešního úryvku, nebo z jiného místa
v Písmu, a vzývejte Pána v modlitbě pomocí tohoto jména. (Jestli si nedokážete vybrat, použijte „Pán plný slitování a
věrný.“) Vězte, že Pán Bůh vaše volání uslyší, ačkoli to nepocítíte. Pokračujte v rozhovoru s Pánem o tom, jak toto
Jeho jméno ovlivňuje váš vztah s Ním.


%newday
\newpage
\section{Den 79 - NEHRAJME SI NA NEVĚSTKU}
\zacatekDvanactyTyden
\subsection*{Čtení na den}
\textbf{Exodus 34,10-21.27-28}
\newline
\textit{
\textsuperscript{10}Hospodin odpověděl: „Hle, uzavírám s vámi smlouvu. Před veškerým tvým lidem učiním podivuhodné věci, jaké nebyly stvořeny na celé zemi ani mezi všemi pronárody. Všechen lid, uprostřed něhož jsi, uvidí Hospodinovo dílo; neboť to, co já s tebou učiním, bude vzbuzovat bázeň.
\textsuperscript{11}Bedlivě dbej na to, co ti dnes přikazuji. Hle, vypudím před tebou Emorejce, Kenaance, Chetejce, Perizejce, Chivejce a Jebúsejce.
\textsuperscript{12}Dej si pozor, abys neuzavíral smlouvu s obyvateli té země, do které vejdeš, aby se nestali uprostřed tebe léčkou.
\textsuperscript{13}Proto jejich oltáře poboříte, jejich posvátné sloupy roztříštíte a jejich posvátné kůly pokácíte.
\textsuperscript{14}Nebudeš se klanět jinému bohu, protože Hospodin, jehož jméno je Žárlivý, je Bůh žárlivě milující.
\textsuperscript{15}Neuzavřeš smlouvu s obyvateli té země. Budou se svými bohy smilnit a svým bohům obětovat a tebe pozvou, abys jedl z jejich obětního hodu;
\textsuperscript{16}a ty budeš brát z jejich dcer manželky pro své syny a jejich dcery budou se svými bohy smilnit a svádět tvé syny, aby smilnili s jejich bohy.
\textsuperscript{17}Nebudeš si odlévat sochy bohů.
\textsuperscript{18}Budeš dbát na slavnost nekvašených chlebů. Sedm dní budeš jíst nekvašené chleby, jak jsem ti přikázal, ve stanovený čas v měsíci ábíbu; neboť v měsíci ábíbu jsi vyšel z Egypta.
\textsuperscript{19}Všechno, co otvírá lůno, bude patřit mně, i každý samec z prvého vrhu tvého stáda, skotu i bravu.
\textsuperscript{20}Osla, který se narodil jako první, vyplatíš ovcí; jestliže jej nevyplatíš, zlomíš mu vaz. Vyplatíš každého svého prvorozeného syna. Nikdo se neukáže před mou tváří s prázdnou.
\textsuperscript{21}Šest dní budeš pracovat, ale sedmého dne odpočineš; i při orbě a při žni odpočineš.
\newline
\newline
\textsuperscript{27}Hospodin řekl Mojžíšovi: „Napiš si tato slova, neboť podle těchto slov uzavírám s tebou a s Izraelem smlouvu.“
\textsuperscript{28}A byl tam s Hospodinem čtyřicet dní a čtyřicet nocí; chleba nepojedl a vody se nenapil, nýbrž psal na desky slova smlouvy, desatero přikázání.
}

\subsection*{Reflexe}
Existují dva typy žárlivosti. První z nich je hříšná. Je sobecká, zpravidla zlovolná, často totožná se závistí. Oproti tomu spravedlivá
žárlivost je právoplatná odezva na nedostatek něčeho, na co má člověk nárok. Žárlivost, o které Bůh v dnešním úryvku hovoří, je
spravedlivá žárlivost.

Pokud muž, který vidí svého souseda vracet se ze schůzky se supermodelkou, začne vůči svému sousedu žárlit, je to žárlivist
hříšná, neboť na tuto modelku nemá o nic větší nárok než jeho soused. Ale pokud tentýž muž uvidí souseda ze schůzky přijíždět s
jeho vlastní manželkou, má právo na to reagovat žárlivostí. Jeho žena (ačkoli není jeho majetkem) je jeho manželka, ne sousedova
manželka. Tudíž jeho spravedlivá žárlivost vyrůstá z něčeho, na co má nárok: věrnost své ženy.

Bůh mluví o svém vztahu k lidu, s nímž uzavřel smlouvu (a to jak ve Starém, tak v Novém Zákoně), jako o vztahu manželském.
Boží smlouvy s člověkem vyžadují věrnost z obou stran, jak od Boha, tak i od lidí. Bůh je samozřejmě věrný vždy, a tudíž
oprávněně pociťuje žárlivost kvůli věrnosti svého lidu. Bůh Izraelitům přikazuje strhnout oltáře a rozbít sloupoví vztyčené k poctě
modlám, aby jim pomohl vyvarovat se příležitostí, které přímo svádějí ke hříchu nevěrnosti.

Tato preventivní opatření platí i pro nás dnes. Máme v našich domech nebo na pracovním stole v práci stále ještě oltáře vystavěné
našim modlám? Vede naše modloslužba s námi žijící lidi do nevěstince s námi? Položte si dnes upřímně tyto dvě otázky.
Promluvte s nimi o Pánu ve své svaté hodince. Všímejte si věcí, nad kterými máte zodpovědnost, ve vašem domě, až pojedete
autem, budete v práci. Jestli na těchto místech máte vystavěné oltáře, pilíře nebo schrány modlám, zamyslete se nad jejich ráznou
změnou, nebo vás i ostatní budou pokoušet už dnes, nebo ve Dni 91.


%newday
\newpage
\section{Den 80 - PODOBNÍ BOHU}
\zacatekDvanactyTyden
\subsection*{Čtení na den}
\textbf{Exodus 34,29-35}
\newline
\textit{
\textsuperscript{29}Když pak Mojžíš sestupoval z hory Sínaje, měl při sestupu z hory desky svědectví v rukou. Mojžíš nevěděl, že mu od rozhovoru s Hospodinem září kůže na tváři.
\textsuperscript{30}Když Áron a všichni Izraelci uviděli, jak Mojžíšovi září kůže na tváři, báli se k němu přistoupit.
\textsuperscript{31}Ale Mojžíš je zavolal, i vrátili se k němu Áron a všichni předáci pospolitosti a Mojžíš k nim promluvil.
\textsuperscript{32}Potom přistoupili všichni Izraelci a on jim přikázal všechno, o čem s ním Hospodin mluvil na hoře Sínaji.
\textsuperscript{33}Když k nim Mojžíš přestal mluvit, dal si na tvář závoj.
\textsuperscript{34}Kdykoli Mojžíš vstupoval před Hospodina, aby s ním mluvil, odkládal závoj, dokud nevyšel. Pak vycházel, aby k Izraelcům mluvil, co mu bylo přikázáno.
\textsuperscript{35}Izraelci spatřili Mojžíšovu tvář a viděli, jak mu kůže na tváři září. Proto si Mojžíš dával na tvář závoj, pokud nešel mluvit s Hospodinem.
}

\subsection*{Reflexe}
V dnešním úryvku z Písma Mojžíš sestupuje z hory po čtyřiceti dnech a nocích modlitby a půstu před Hospodinem. Byl to druhý
čtyřicetidenní úsek Mojžíše na hoře Sinaj (první byl právě před incidentem se zlatým teletem). Mojžíš se tedy po osmdesáti dnech
modlitby a postu stává Bohu natolik podobným, že ostatní mohou v jeho tváři vidět Boží slávu.

Bytí v Pánově přítomnosti nás mění. Mojžíš strávil tolik času blízko Hospodina, že nejen jeho mysl a srdce, ale i jeho tvář vyzařuje
Boží slávu. Když lidé viděli Mojžíšovu tvář, neviděli už jen Mojžíšovo tělo – viděli v hmatatelné blízkosti i Boží slávu. Podívejte
se na životy svatých. Tyto osobnosti také strávily tolik času v Pánově blízkosti, že se staly podobnými Bohu a Jeho slávě. Když
čteme o životech svatých a slyšíme o jejich velkých činech, můžeme zahlédnout nejen kousek jejich ctnostného lidství, ale také
kousek Boží slávy.

Dnes jste již také vy strávili osmdesát dní v modlitbě a půstu před Bohem. Toto načasování není náhodou. Zeptejte se své
manželky, nejlepšího kamaráda, či kolegy, jestli jste se stali natolik podobni Bohu, že mohou vidět aspoň kousek Jeho slávy, když
na vás pohlédnou. Jestli této otázce rozumí, a pokud k vám budou upřímní, pravděpodobně odpoví „ano“.

Možná jste byli disciplínám Exodu 90 naprosto věrní. Možná jste ale podváděli, vymlouvali se, nebo jednoduše vymazali některé
disciplíny ještě na začátku. Nehledě na to, jak jste na tom, právě jste strávili osmdesát dní před Pánem, tak či onak. To není pouze
velký úspěch, ale opravdu proměňující zkušenost.

Popovídejte si dnes s Hospodinem o tom, jak proměnil vaši mysl, vaše srdce a váš způsob života, abyste se Mu více podobali.
Nebojte se tyto změny zapsat, takže budete moct i nadále žít v souladu s touto milostí. Poděkujte Pánu za tyto změny. A dnes Mu
dovolte, aby váš život stále proměňoval, abyste se Mu podobali, takže ve vás ostatní budou moci vidět Boží slávu ještě zřetelněji,
kdykoli vás uvidí.


%newday
\newpage
\section{Den 81 - DENNÍ PŘIPOMÍNKA}
\zacatekDvanactyTyden
\subsection*{Čtení na den}
\textbf{Exodus 35,1-12; 37,1-2.10.17.25.29; 38,1-2}
\newline
\textit{
\textsuperscript{1}Mojžíš shromáždil celou pospolitost Izraelců a řekl jim: „Hospodin přikázal, abyste dodržovali toto:
\textsuperscript{2}Šest dní se bude pracovat, ale sedmého dne budete mít slavnost odpočinutí, Hospodinův svatý den odpočinku; kdo by v ten den dělal nějakou práci, zemře.
\textsuperscript{3}V den odpočinku nerozděláte oheň v žádném svém obydlí.“
\textsuperscript{4}Mojžíš řekl celé pospolitosti Izraelců: „Toto přikázal Hospodin:
\textsuperscript{5}Vyberte mezi sebou pro Hospodina oběť pozdvihování. Každý ať ze srdce dobrovolně přinese jako Hospodinovu oběť pozdvihování zlato, stříbro a měď,
\textsuperscript{6}látku purpurově fialovou, nachovou nebo karmínovou, jemné plátno, kozí srst,
\textsuperscript{7}načerveno zbarvené beraní kůže, tachaší kůže, akáciové dřevo,
\textsuperscript{8}olej na svícení, balzámy na olej k pomazání a na kadidlo z vonných látek,
\textsuperscript{9}karneolové drahokamy a kameny pro zasazení do nárameníku a náprsníku.
\textsuperscript{10}A každý, kdo je mezi vámi dovedný, ať přijde a dělá vše, co Hospodin přikázal:
\textsuperscript{11}příbytek, jeho stan a přikrývku, spony a desky, svlaky, sloupy a patky,
\textsuperscript{12}schránu a tyče k ní, příkrov a vnitřní oponu,
\newline
\newline
\textsuperscript{1}Besaleel zhotovil schránu z akáciového dřeva dva a půl lokte dlouhou, jeden a půl lokte širokou a jeden a půl lokte vysokou.
\textsuperscript{2}Obložil ji uvnitř i zvnějšku čistým zlatem a opatřil ji dokola zlatou obrubou.
\newline
\newline
\textsuperscript{10}Zhotovil stůl z akáciového dřeva dva lokte dlouhý, jeden loket široký a jeden a půl lokte vysoký.
\newline
\newline
\textsuperscript{17}Zhotovil svícen z čistého zlata. Svícen měl vytepaný dřík a prut; jeho kalichy, číšky a květy byly s ním zhotoveny z jednoho kusu.
\newline
\newline
\textsuperscript{25}Zhotovil kadidlový oltář z akáciového dřeva, čtyřhranný, loket dlouhý, loket široký a dva lokte vysoký; jeho rohy byly z jednoho kusu s ním.
\newline
\newline
\textsuperscript{27}Zhotovil pro něj též dva zlaté kruhy, a to pod obrubou při jeho bocích, k oběma bočnicím, aby držely tyče, na nichž byl nošen.
\newline
\newline
\textsuperscript{29}Připravil také olej svatého pomazání a kadidlo z vonných látek, čisté, odborně smísené.
\newline
\newline
\textsuperscript{1}Zhotovil oltář pro zápalnou oběť z akáciového dřeva, čtyřhranný, pět loket dlouhý, pět loket široký a tři lokte vysoký.
\textsuperscript{2}Jeho čtyři úhly opatřil rohy; ty s ním byly zhotoveny z jednoho kusu. Potáhl jej bronzem.
}

\subsection*{Reflexe}
Mojžíš ve čteních z devátého a desátého týdne dostává instrukce, jak postavit příbytek setkávání. Nyní nastal čas stavět. Verše
dnešního čtení z Písma jsou jen úryvky z mnohem delšího záznamu Mojžíše předávajícího Boží stavitelské plány Izraelitům. Tyto
kapitoly knihy Exodus slouží jako ozvěna kapitol, které jste četli před incidentem se zlatým teletem. V onen čas Bůh předává
stavitelské plány Mojžíšovi, teď je Mojžíš předává ostatním Izraelitům.

Opakování těchto podrobností jen ukazuje na jejich důležitost pro svobodu Izraele. Avšak důležitou není jen ona jednorázová
stavba. Spíše poslušnost těmto detailům spjatá s řádnou službou ve svatostánku povede Izraelity k větší svobodě. Příbytek
setkávání bude fyzickou připomínkou jejich každodenního volání k modlitbě, askezi a bratrství, jak to Bůh předepisuje ve svých
bohoslužebných zákonech.

Dnes mnoho lidí nepovažuje velké baziliky a katedrály za nic většího než muzea. Přesto, jako původní svatostánek, slouží baziliky,
katedrály, a dokonce místní kostely jako fyzická připomínka našeho povolání denně praktikovat modlitbu, askezi a bratrství tak,
jak nám to předepsal Bůh prostřednictvím Jeho Písma a Církve.

Když míjíte kostel na vaší cestě do práce, cítíte volání k modlitbě? Když zahlédnete kříž při odchodu ze mše, rozhodnete se
nabídnout sebe jako oběť za vaši rodinu po zbytek týdne? Když vstoupíte do kostela někde daleko od domova a zahlédnete hořící
věčné světlo značící Kristovu přítomnost, cítíte se okamžitě sjednoceni s lidmi, kteří se zde před svatostánkem modlí?

Skrze podrobnosti kostelů vás Pán, stejně jako Izraelity, volá ke každodenní službě. Popovídejte si dnes s Pánem o tom, jak vás
přítomnost kostelů může volat každý den k hlubší službě.


%newday
\newpage
\section{Den 82 - BOHEM VYVOLENÍ MUŽI}
\zacatekDvanactyTyden
\subsection*{Čtení na den}
\textbf{Exodus 38,21-26}
\newline
\textit{
\textsuperscript{21}Toto jsou ti, kdo byli povoláni k službě u příbytku, u příbytku svědectví, pověření na Mojžíšův rozkaz lévijskou službou za dozoru Ítamara, syna kněze Árona:
\textsuperscript{22}Besaleel, syn Uríův, vnuk Chúrův z pokolení Judova, zhotovil všechno, co Hospodin Mojžíšovi přikázal.
\textsuperscript{23}S ním Oholíab, syn Achísamakův z pokolení Danova, řemeslník, umělec, zhotovující výšivky z látky purpurově fialové, nachové a karmínové a z jemného plátna.
\textsuperscript{24}Všechno zlato, zpracované při tom díle, při celém díle na svatyni, bylo zlato obětované podáváním, dvacet devět talentů a sedm set třicet šekelů podle váhy určené svatyní.
\textsuperscript{25}Stříbra bylo od těch, kdo byli z pospolitosti povoláni k službě, sto talentů a tisíc sedm set sedmdesát pět šekelů podle váhy určené svatyní.
\textsuperscript{26}Půl šekelu na hlavu, polovina šekelu podle váhy určené svatyní, za každého, kdo přešel mezi ty, kteří byli povoláni do služby od dvacetiletých výše, za šest set tři tisíce a pět set padesát mužů.
}

\subsection*{Reflexe}
Když pokračujeme ve čtení Exodu, do popředí vystupuje jistý vzorec: Bůh se znova a znova navrací k malé skupince
mužů. Přestože při sčítání lidu bylo napočítáno 603 550 lidí (a to jen mužů nad dvacet let), Bůh si vybírá jen několik
vyvolených. Ukazuje na každého jmenovitě, aby dokončili Jeho posvátné dílo.

Ve většině katolických farností po celém světě můžeme pozorovat, jak si Bůh volí určité lidi. Zdá se, jako by Bůh
věděl, kdo na jeho volání odpoví – a kdo ne. Používá tyto osoby, které ochotně odpovídají na eho výběr a spolupracují
na dokončení posvátných věcí, jaké po nich žádá. Musíme usilovat o to být lidmi, na které se Bůh může spolehnout.
Bůh by samozřejmě mohl dokončit vše svými silami; on člověka nepotřebuje k dokončení ničeho. Z lásky se však
rozhodl své děti do svých aktivit zapojit. Buďte připraveni a ochotni s Bohem spolupracovat.

Řekli jste své ano k následování Boha tímto devadesátidenním plánem ke svobodě. Zamyslete se teď, když máte před
sebou posledních osm dnů, nad Božím úmyslem pozvat vás k tomuto duchovnímu cvičení. Myslíte si, že Bůh bude na
konci těchto devadesáti dnů hotov s vaší přeměnou v lepšího člověka? Ani náhodou. Vševědoucí Bůh má s vámi a vaší
rodinou daleko úžasnější plán. Získání svobody je z něj pouze jedním kouskem.

Popovídejte si dnes s Pánem o tom, jak hodlá využít modlitbu, askezi a bratrství pro pokračování vaší přeměny v ještě
lepšího člověka pro vaši rodinu a farnost.


%newday
\newpage
\section{Den 83 - DOKONČILI DÍLO}
\zacatekDvanactyTyden
\subsection*{Čtení na den}
\textbf{Exodus 38,27-31; 39,33-43}
\newline
\textit{
\textsuperscript{27}Sto talentů stříbra bylo použito k odlití patek pro svatyni a patek k oponě, sto patek z jednoho sta talentů, jeden talent na jednu patku.
\textsuperscript{28}Z tisíce sedmi set sedmdesáti pěti šekelů udělal háčky ke sloupům, obložil jejich hlavice a spojil je příčkami.
\textsuperscript{29}Mědi obětované podáváním bylo sedmdesát talentů a dva tisíce čtyři sta šekelů.
\textsuperscript{30}Z toho udělal patky ke vchodu do stanu setkávání, bronzový oltář a k němu bronzový rošt a všechno náčiní k oltáři,
\textsuperscript{31}patky kolem nádvoří, patky k bráně do nádvoří i všechny kolíky pro příbytek a všechny kolíky pro nádvoří dokola.
\newline
\newline
\textsuperscript{33}Přinesli k Mojžíšovi příbytek: stan s veškerým náčiním, sponami, deskami, svlaky, sloupy a patkami;
\textsuperscript{34}přikrývku z beraních kůží, zbarvených načerveno, přikrývku z kůží tachaších a vnitřní oponu;
\textsuperscript{35}schránu svědectví s tyčemi a příkrovem;
\textsuperscript{36}stůl s veškerým náčiním a předkladné chleby;
\textsuperscript{37}svícen z čistého zlata s kahánky připravenými k nasazení a s veškerým náčiním, i olej k svícení;
\textsuperscript{38}zlatý oltář a olej k pomazání i kadidlo z vonných látek a závěs ke vchodu do stanu;
\textsuperscript{39}bronzový oltář a jeho bronzový rošt, s tyčemi a s veškerým náčiním, nádrž s podstavcem;
\textsuperscript{40}zástěny pro nádvoří, sloupy s patkami, závěs pro bránu do nádvoří s lany a kolíky i všecko náčiní pro službu v příbytku, pro stan setkávání;
\textsuperscript{41}tkaná roucha pro přisluhování ve svatyni, svatá roucha pro kněze Árona a roucha jeho synům pro kněžskou službu.
\textsuperscript{42}Všechnu práci vykonali Izraelci přesně tak, jak Hospodin Mojžíšovi přikázal.
\textsuperscript{43}Mojžíš celé dílo prohlédl; ano, vykonali je přesně tak, jak Hospodin přikázal. A Mojžíš jim požehnal.
}

\subsection*{Reflexe}
V osmdesátém třetím dni tohoto duchovního cvičení čteme o dokončení příbytku setkávání. Po zadání všech podrobností a
instrukcí vyvolení muži Izraele našli společnou cestu a dokončili toto náročné dílo. Přesto, že vy s vašimi bratry stále ještě na svém
díle pracujete, jste na cestě k podobnému konci. Připomeňte si své proč.

Po dokončení svatostánku měli vyvolení Izraelité možnost se svobodně rozhodnou pro návrat do Egypta. Tam by mohli svým
starým egyptským kamarádům povědět o jejich exodu a úžasných věcech, které pro ně Pán v poušti udělal. Možná by dokonce byli
tak nadšení z jejich svobody, že by přehlédli, že se vrátili zpět do otroctví. Byla by to příšerná volba.

Díky Bohu takto kniha Exodus nekončí. Místo toho si Izraelité vyberou odpovědět na Boží přítomnost. Zůstanou, kde zůstane On,
a jdou, kam jde On. Co budete dělat vy, až dokončíte toto duchovní cvičení? Vrátíte se zpět do Egypta? Nebo zůstanete naladěni na
Boží vlnu a budete Ho následovat, i když vás povede ještě hlouběji do Jeho očišťující lásky?

Nebojte se. Budete mít tolik potřebnou čtrnáctidenní přestávku a právoplatný čas na oslavu. Jako bylo vyvoleným Izraelitům
požehnáno po dokončení jejich stavby, tak bude i požehnáno vám na konci tohoto duchovního cvičení. Připravte se na to jako
bratrstvo. Naplánujte si čas na společnou oslavu. Bude zasloužená.

V dnešní modlitbě poproste Pána, aby vám dal milost následovat Ho ve Dni 91. Nebude vás nutit, abyste šli s Ním. Budete se pro
to muset vědomě rozhodnout. Proste Pána a On vám požehná, abyste Ho mohli dále následovat, jako vám žehnal doposud.


%newday
\newpage
\section{Den 84 - DRŽET SE BOŽÍHO PLÁNU}
\zacatekDvanactyTyden
\subsection*{Čtení na den}
\textbf{Exodus 40,1-17}
\newline
\textit{
\textsuperscript{1}Hospodin promluvil k Mojžíšovi:
\textsuperscript{2}„Prvního dne prvního měsíce postavíš příbytek stanu setkávání.
\textsuperscript{3}Tam umístíš schránu svědectví a zastřeš schránu oponou.
\textsuperscript{4}Přineseš také stůl a všechno na něm uspořádáš, přineseš i svícen a nasadíš na něj kahánky.
\textsuperscript{5}Zlatý kadidlový oltář dáš před schránu svědectví a pověsíš závěs ke vchodu do příbytku.
\textsuperscript{6}Oltář pro zápalnou oběť postavíš před vchod do příbytku stanu setkávání.
\textsuperscript{7}Mezi stan setkávání a oltář umístíš nádrž a naleješ do ní vodu.
\textsuperscript{8}Dokola postavíš nádvoří a do brány k nádvoří pověsíš závěs.
\textsuperscript{9}Potom vezmeš olej pomazání a pomažeš příbytek a všechno, co je v něm, a posvětíš jej i s veškerým náčiním, a bude svatý.
\textsuperscript{10}Pomažeš také oltář pro zápalnou oběť i s veškerým náčiním a posvětíš jej; a oltář bude velesvatý.
\textsuperscript{11}Pomažeš také nádrž s podstavcem a posvětíš ji.
\textsuperscript{12}Pak přivedeš Árona a jeho syny ke vchodu do stanu setkávání a omyješ je vodou.
\textsuperscript{13}Nato oblékneš Áronovi svatá roucha, pomažeš ho a posvětíš a bude mi sloužit jako kněz.
\textsuperscript{14}Přivedeš i jeho syny a oblékneš do suknic.
\textsuperscript{15}Pomažeš je, jako jsi pomazal jejich otce, a budou mi sloužit jako kněží. Toto pomazání je uvede v trvalé kněžství po všechna pokolení.“
\textsuperscript{16}Mojžíš učinil všechno přesně tak, jak mu Hospodin přikázal.
\textsuperscript{17}Příbytek byl postaven první den prvního měsíce druhého roku.
}

\subsection*{Reflexe}
Na začátku tohoto úryvku Bůh nařizuje, aby byl příbytek setkávání postaven „prvního dne prvního měsíce“ (Ex 40,2). Proč tak
konkretizuje toto zadání? Bylo to prvního dne prvního měsíce, co Bůh stvořil svět. Jelikož je svatostánek obrazem celého světa
(díky své pozemské výzdobě), má být postaven ve stejný den, jako byl stvořen svět; konkrétně tedy prvního dne. Boží mistrovský
plán zahrnuje krásnou pozornost pro načasování, i pro vybrání jednoho konkrétního dne a žádného jiného.

Když vás Bůh žádá, abyste udělali určitou věc v určitý čas, je to proto, že má pro vás a vaši rodinu dobrý plán. Nemusíte znát tento
plán cele, ale můžete důvěřovat, že je tento plán dobrý (Jeho plán musí být dobrý, protože Bůh je Dobro). To neznamená, že Jeho
plán není občas náročný nebo nepohodlný, to může být. Ale tento plán, i se svým přesným načasováním, bude vždycky dobrý. Bůh
vás povolal k náročnému a nepohodlnému životu, ale plody vašeho „ano“ jsou bohaté a hojné (i když je ještě nemůžete vidět
v plnosti).

Je téměř nemožné držet se Božího plánu, když Mu nedáváte každý den prostor na to, aby s vámi svůj plán sdílel. Bez modlitby, bez
Božího plánu, zůstanete na plán svůj a své rodiny sami. Jak se vám tohle dařilo v minulosti?

Po osmdesát čtyři dní jste budovali svůj zvyk každodenní modlitby. Nedávejte si od tohoto zvyku přestávku. Muži Exodu zjistili,
že pokud vaše čtrnáctidenní přestávka začínající Dnem 91 zahrnuje i přestávku od modlitby, je téměř nemožné se do tohoto zvyku
znovu dostat. Pracovali jste příliš tvrdě na to, abyste to dopustili.

Můžete si vybrat svou každodenní svatou hodinku ve Dni 91 zkrátit, ale třicet minut denně, s alespoň dvaceti minutami
kontemplativní modlitby, by mělo zůstat součástí vašeho života. Držte se tohoto zvyku a budete mnohem schopnější držet se
Božího plánu pro vás a vaši rodinu ve Dni 91.
\newpage

% ===============================================
% ===== TRINACTY TYDEN
% ===============================================
%ukony
\newpage
\section*{Úkony (ukazatel cesty) pro 13. týden}

\textbf{Místo:} Východní okraj země zaslíbené

I přesto, že bylo Izraelitům mnohokrát požehnáno, stále pochybují o Bohu (po tolika projevech milosti Izraelité
stále NEDŮVĚŘUJÍ HOSPODINU). V důsledku toho jsou právem vyloučeni ze země zaslíbené (v důsledku toho
JSOU PRO NĚ spravedlivě UZAVŘENY DVEŘE PRO VSTUP DO zaslíbené země). Tvoje budoucnost by ale měla být
úplně/naprosto jiná. V posledních třech měsících jste si vedli dobře, spoléhali jste na Boha více než kdy
předtím. Dozvěděli/učili jste se, co to znamená každý den brát na sebe svůj kříž a následovat Krista (viz L
9,23). Dozvěděli jste se o plánu svobody (naučil jsi se orientovat v mapě, která vede ke svobodě). Pán vám dal
pochopit křesťanský život mnohem hlouběji než většině lidí (Pán ti UKÁZAL JAKÝ MŮŽE BÝT křesťanský život
(DOSLOVA "SVĚŘIL TI POCHOPENÍ") DALEKO VÍCE NEŽ TO CHÁPOU OSTATNÍ MUŽI). Plán svobody/cestovní mapa je
ve Tvých rukou. Teď s ním můžete dělat, co si zvolíte, ale jste zodpovědní za následky (nyní si můžete
svobodně vybrat JAK S TÍM NALOŽÍŠ, ale nejsi OPROŠTĚN OD následků POKUD NEBUDEŠ POUŽÍVAT MAPU). Žijte tu
svobodu, sdílejte ji, a důsledkem bude vaše požehnání (A TO CO BUDE NÁSLEDOVAT BUDE
POŽEHNÁNO/POŽEHNÁNÍM).

\subsection*{1. Ohlédněte se / reflektujte na vaše uplynulé tři měsíce modlitby}
Před Exodem jste možná nebyli zvyklí se pravidelně modlit. Podívejte se, jak díky tomu váš vztah s Bohem vyrostl. Jaký plán máte pro Den 91? Jestli nemáte plán modlitby, kterého se budete držet, nebudete mít na každodenní modlitbu čas. Zapište si svůj plán.
\subsection*{2. Uvědomte si svou potřebu Boha}
Pokud přestanete Boha prosit o milost a slitování/milosrdenství, ztratíte svou svobodu. Nedovolte, aby se tato věc stala skutečností a nepříjemně překvapila vás nebo vaše bratrstvo, zejména vaši kotvu.
\subsection*{3. Uvědomte si, jak jste se v těchto posledních měsících věnovali Písmu}
Možná jste mu věnovali nejvíc z celého života. Podívejte se, kam vás zavedlo. To Slovo je Ježíš Kristus, který je živý a dovedl vás až do tohoto posledního týdne vašeho exodu. Představte si, kým byste se mohli stát, kdybyste i nadále následovali živé Slovo. Představte si, kým by se mohli stát vaši bratři, kdybyste se ponořili do Slova společně.

\subsection*{Modlitba}
Modlete se, aby Pán osvobodil vás a vaše bratrství \newline
Modleme se za svobodu všech mužů v exodu, stejně tak, jako se oni modlí za vás.\newline
Ve jménu Otce i Syna i Ducha svatého … Otče náš… Ve jménu Otce i Syna i Ducha svatého … Amen.
\newpage

%newday
\newpage
\section{Den 85 - VNĚJŠÍ JEDNÁNÍ}
\zacatekTrinactyTyden
\subsection*{Čtení na den}
\textbf{Exodus 40,34-38}
\newline
\textit{
\textsuperscript{34}Tu oblak zahalil stan setkávání a příbytek naplnila Hospodinova sláva.
\textsuperscript{35}Mojžíš nemohl přistoupit ke stanu setkávání, neboť nad ním přebýval oblak a příbytek naplňovala Hospodinova sláva.
\textsuperscript{36}Kdykoli se oblak z příbytku zvedl, vytáhli Izraelci ze všech svých stanovišť.
\textsuperscript{37}Jestliže se oblak nezvedal, nevytáhli, dokud se nezvedl.
\textsuperscript{38}Hospodinův oblak býval ve dne nad příbytkem a v noci na něm planul oheň před očima celého domu izraelského na všech jejich stanovištích.
}

\subsection*{Reflexe}
Dosáhli jsme konce knihy Exodus. Skvělá práce. Přesto však kniha Exodus nekončí v zemi zaslíbené. Příběh
Izraelitů pokračuje v dalších knihách Písma svatého, které spolu tvoří jeden příběh spásy. Také my hrajeme
roli v tomto příběhu, ale jak by mohl vypadat si necháme až na Den 91. Než se tak stane, podívejme se na
následující úryvek z dnešního čtení:

\textit{Mojžíš nemohl přistoupit ke stanu setkávání, neboť nad ním přebýval oblak a příbytek naplňovala
Hospodinova sláva. (Ex 40,35)}

Mojžíšovi není dovoleno vstoupit do Boží blízkosti. To znamená změnu oproti jeho rozhovorům s Bohem
před postavením svatostánku. Vzpomeňte si, jak Mojžíšovi zářila tvář díky bytí v Boží přítomnosti na hoře
Sinaj. Nyní, když bylo pro Izraelity postaveno místo k odčinění jejich hříchů (svatostánek), musí prokázat
svou touhu znovu se sblížit s Bohem. Dokud na to nebudou Izraelité připraveni, musí Mojžíš podstupovat
stejné odloučení od Boha, jaké si zvolili Izraelité skrze své hříchy.

Ve svém milosrdenství předkládá Bůh Mojžíšovi a Izraelitům plán, jak se s Ním znovu mohou sblížit. Otec
to dělá pro ně a pro nás tím, že nám dává svá přikázání, která nás k Němu mají krok po kroku obrátit.

Zítra se podíváme blíže na Boží plán s Izraelity v knize Leviticus. Dnes se podívejte na svůj vlastní život ve
světle dnešního úryvku z Písma. Dovoluje vám Hospodin přístup do stanu setkávání, nebo se od vás odloučil,
abyste si Jej mohli znovu navenek zvolit? Teď, když máte plán, od vás Pán očekává víc. Očekává od vás, že
budete následovat plán (modlitbu, askezi a bratrství) zpět k Němu.

Promluvte si dnes s Pánem o tom, jak váš život v modlitbě, askezi a bratrství ukazuje na vaši lásku k Němu, a
udržuje vás věrnými každým dnem.


%newday
\newpage
\section{Den 86 - PLÁN SJEDNOCENÍ}
\zacatekTrinactyTyden
\subsection*{Čtení na den}
\textbf{Leviticus 1,1; 4,27-31}
\newline
\textit{
\textsuperscript{1}I zavolal Hospodin Mojžíše a promluvil k němu ze stanu setkávání:
\newline
\newline
\textsuperscript{27}Jestliže se neúmyslně prohřeší někdo z lidu země a dopustí se proti některému příkazu Hospodinovu něčeho, co se dělat nesmí, provinil se.
\textsuperscript{28}Je-li mu oznámeno, že se dopustil hříchu, přivede jako svůj dar kozu, samici bez vady, za hřích, kterého se dopustil.
\textsuperscript{29}Vloží ruku na hlavu zvířete obětovaného za hřích a porazí oběť za hřích na místě pro zápalné oběti.
\textsuperscript{30}Pak vezme kněz trochu krve na prst a potře rohy oltáře pro zápalné oběti; všechnu ostatní krev vyleje ke spodku oltáře.
\textsuperscript{31}Všechen tuk odejme, jako bývá odňat tuk z hodu oběti pokojné, a kněz jej na oltáři obrátí v obětní dým, v libou vůni pro Hospodina. Kněz za něho vykoná smírčí obřady, a bude mu odpuštěno.
}

\subsection*{Reflexe}
Třetí kniha Písma svatého, Leviticus, pokračuje tam, kde kniha Exodus skončila. Bůh nyní promlouvá k Mojžíšovi ze stanu
setkávání. Mojžíš není povolán do přítomnosti Svatého Boha kvůli hříchům Izraele. Všimněte si však, že je Bůh stále ochoten
mluvit s Mojžíšem, i když Mojžíš není povolán do Boží důvěrné přítomnosti.

V Levitiku dostávají Izraelité dlouhé vysvětlení obřadních obětí a kajícných činů, které musí vykonat, aby dokázali svou touhu
vrátit se a udržovat si správný vztah s Bohem. Pokud Izraelité přestanou provádět tyto rituály, ztratí přízeň u Boha. Takové
rozhodnutí by bylo, jako kdyby Izraelci prohlásili: „Už nepotřebujeme vztah s Bohem, ani Jeho pomoc.“

Zamyslete se nad svým vlastním vztahem s Bohem. Již v minulosti jste se od Boha odvrátili v hříchu. Avšak i mimo stav milosti k
vám Hospodin mluví. Volá vás od vašich hříchů, abyste Ho znovu následovali k životu svobody. Nedovolte, aby vás ďábel
oklamal. Podobně jako rituály, které dal Bůh Izraelitům pro obnovení vztahu s Ním, i vám dal Bůh podobné dary: modlitbu, askezi
a bratrství. Vaše skutky askeze mohou znamenat pokání nebo činy oběti, když je sjednotíte s křížem. Kromě toho nám svátost
smíření a Eucharistie poskytují léčivou a znovusjednocující milost, po které Kristus touží pro naše duše. Vzdát se těchto darů je pro
křesťany jako říkat: „Už nepotřebujeme vztah s Bohem nebo jeho pomoc.“

Pokud je toto duchovní cvičení vaším prvním setkáním s disciplínou modlitby, askeze a bratrství, pak byl tento čas pro vás
pravděpodobně posvěcující, osvobozující, ale také poměrně vyčerpávající. Vzhlížejte ke Dni 91. Zasloužíte si přestávku. Pán slyší
a zná vaše potřeby. Po této devadesátidenní době pracného očišťování vstoupíte do dvoutýdenního období svatého odpočinku.
V něm budete mít možnost odpočinout si od některých skutků askeze. Přesto stále chraňte svou svobodu. Ptejte se sami sebe na
otázky jako: „Potřebuji skutečně ve svém životě sociální sítě, nebo by pro mou svobodu bylo nejlepší, kdybych se jich i nadále
zdržel?“ Buďte k sobě upřímní; v sázce je vaše svoboda. Udržujte pohromadě své bratrství. To také dává vaší svobodě silnou
obranu před nepřítelem.

Promluvte si dnes s Pánem o Dni 91. Zeptejte se Ho, které z asketických praktik byste si měli udržet, a které můžete odložit, i když
stále procvičovat. Buďte konkrétní a zapište si své závěry, abyste zůstali zodpovědnými ve Dni 91. Neutíkejte před touto žádostí.
Sepište je. Udělejte to pro svou svobodu; udělejte to pro svou rodinu.


%newday
\newpage
\section{Den 87 - PLÁN FUNGUJE}
\zacatekTrinactyTyden
\subsection*{Čtení na den}
\textbf{Numeri 1,1-4.17-19; 10.11-12; 11,4-6}
\newline
\textit{
\textsuperscript{1}Hospodin promluvil k Mojžíšovi na Sínajské poušti ve stanu setkávání prvního dne druhého měsíce ve druhém roce po jejich vyjití z egyptské země.
\textsuperscript{2}„Pořiďte soupis celé pospolitosti Izraelců podle čeledí otcovských rodů. Ve jmenném seznamu bude každý jednotlivec mužského pohlaví
\textsuperscript{3}od dvacetiletých výše, každý, kdo je v Izraeli schopen vycházet do boje. Ty a Áron je spočítáte po oddílech.
\textsuperscript{4}Za každé pokolení bude s vámi jeden muž, vždy představitel otcovského rodu.
\newline
\newline
\textsuperscript{17}Mojžíš a Áron přibrali tyto muže uvedené jménem.
\textsuperscript{18}Prvního dne druhého měsíce svolali celou pospolitost, aby všichni hlásili svůj původ podle čeledí otcovských rodů; do jmenného seznamu byl zapsán každý jednotlivec od dvacetiletých výše,
\textsuperscript{19}jak přikázal Hospodin Mojžíšovi. Spočítal je na Sínajské poušti.
\newline
\newline
\textsuperscript{11}Ve druhém roce, dvacátého dne druhého měsíce, se vznesl oblak od příbytku svědectví.
\textsuperscript{12}I táhli Izraelci ze Sínajské pouště dál, po jednotlivých stanovištích, až se oblak pozdržel na poušti Páranské.
\newline
\newline
\textsuperscript{4}Chátru přimíšenou mezi nimi popadla žádostivost. Rovněž Izraelci začali znovu s pláčem volat: „Kdo nám dá najíst masa?
\textsuperscript{5}Vzpomínáme na ryby, které jsme měli v Egyptě k jídlu zadarmo, na okurky a melouny, na pór, cibuli a česnek.
\textsuperscript{6}Jsme už celí seschlí, nevidíme nic jiného než tu manu.“
}

\subsection*{Reflexe}
Kniha Numeri, stejně jako kniha Leviticus, pokračuje tam, kde kniha Exodus skončila. To proto, že Leviticus
slouží spíše jako doplňková kniha většího příběhu Písma. Leviticus předkládá mnoho rituálů, které musí
Izraelité dodržovat, ale nepokrývá při tom podstatný chronologický čas. První verš knihy Numeri decentně
naznačuje neobyčejnou důležitost knihy Leviticus pro celý příběh Písma: „Hospodin promluvil k Mojžíšovi
… ve stanu setkávání“ (Num 1,1, přidaný důraz).

Jak víme, Mojžíšovi bylo zakázáno vstoupit do stanu setkávání s Bohem. Nyní je mu dovoleno znovu
vstoupit do stánku, což znamená, že rituály dané Izraelitům v knize Leviticus zapůsobily. Zamyslete se, co se
stalo. Lid Izraele si zvolil hřích před Bohem. Proto je Bůh spravedlivě odsoudil, aby nemohli vstoupit do
Jeho přítomnosti. Ve svém otcovském milosrdenství dal Bůh Izraelitům plán, který je mohl očistit a
osvobodit, aby, pokud budou plán následovat, mohli do Jeho přítomnosti znovu vstoupit.

Zní vám to povědomě?

Obřadní zákony byly pro Izraelity náročné, stejně jako jsou náročné pro nás disciplíny tohoto duchovního
cvičení. Ta krásná a důležitější věc je: Bůh je dobrý, Jeho plán je dobrý, a Jeho plán funguje.

Izraelité jsou konečně znovu v pohybu. Následujíce Pána putují k Páranské poušti. Izraelité již putují pouští
přes rok, přičemž většinu tohoto času strávili na úpatí hory Sinaj. Viděli Boha dělat pro ně úžasné věci –
egyptské rány, rozdělení Rudého moře, poskytování potravy a vody v poušti po celý rok, a poskytnutí plánu
ke svobodě poté, co se od Něj odvrátili a uctívali zlaté tele. Přesto nic z toho není pro Izraelity dost. Stále si
přejí, aby se mohli vrátit k potravě Egypta, slepí k tomu, jaké otroctví by tato volba přinesla.

Kde se v příběhu vidíte vy?

Devadesát dní končí tento týden. Byli jste svědky úžasných věcí, které ve vašem životě Pán vykonal. Stále si
stěžujete? Víte, že plán funguje, pokud se ho držíte. Stále zvažujete, že opustíte své bratry, abyste se mohli
oddat zákuskům Egypta? Jste slepí k otroství, které s takovou volbou přichází? Hleďte na kříž. Buďte dnes
vděčni. Buďte vděčni za vše, co pro vás Bůh zatím v poušti vykonal, a zvolte si Jeho následování.


%newday
\newpage
\section{Den 88 - POCHYBUJETE, ŽE TO PRO VÁS BŮH MŮŽE UDĚLAT?}
\zacatekTrinactyTyden
\subsection*{Čtení na den}
\textbf{Numeri 13,1-2.25-33; 14,1-3.26-34}
\newline
\textit{
\textsuperscript{1}Hospodin promluvil k Mojžíšovi:
\textsuperscript{2}„Pošli muže, aby prozkoumali kenaanskou zemi, kterou dávám Izraelcům. Pošlete po jednom muži z jejich otcovských pokolení, vždy jejich předáka!“
\newline
\newline
\textsuperscript{25}Po čtyřiceti dnech průzkumu země se vrátili zpět.
\textsuperscript{26}Přišli konečně k Mojžíšovi a Áronovi a k celé pospolitosti Izraelců na Páranskou poušť do Kádeše, podali jim a celé pospolitosti zprávu a ukázali jim ovoce té země.
\textsuperscript{27}Ve svém vyprávění mu řekli: „Vstoupili jsme do země, do níž jsi nás poslal. Vskutku oplývá mlékem a medem. A toto je její ovoce.
\textsuperscript{28}Jenomže lid, který v té zemi sídlí, je mocný a města jsou opevněná a nesmírně veliká. Dokonce jsme tam viděli potomky Anákovy.
\textsuperscript{29}Na jihu země sídlí Amálek, na pohoří jsou usazeni Chetejci, Jebúsejci a Emorejci, při moři a podél Jordánu Kenaanci.“
\textsuperscript{30}Káleb však uklidňoval lid bouřící se proti Mojžíšovi. Říkal: „Vzhůru! Pojďme! Obsadíme tu zemi a jistě se jí zmocníme.“
\textsuperscript{31}Ale muži, kteří šli spolu s ním, tvrdili: „Nemůžeme vytáhnout proti tomu lidu, vždyť je silnější než my.“
\textsuperscript{32}Pomluvami zhaněli Izraelcům zemi, kterou prozkoumali: „Země, kterou jsme při průzkumu prošli, je země, která požírá své obyvatele, a všechen lid, který jsme v ní spatřili, jsou muži obrovité postavy.
\textsuperscript{33}Viděli jsme tam zrůdy – Anákovci totiž patří ke zrůdám – a zdálo se nám, že jsme nepatrní jako kobylky, vskutku jsme v jejich očích byli takoví.“
\textsuperscript{1}Celá pospolitost se pozdvihla; dali se do křiku a lid tu noc proplakal.
\textsuperscript{2}Všichni Izraelci reptali proti Mojžíšovi a Áronovi a celá pospolitost jim vyčítala: „Kéž bychom byli zemřeli v egyptské zemi nebo na této poušti! Kéž bychom zemřeli!
\textsuperscript{3}Proč nás Hospodin přivedl do této země? Abychom padli mečem? Aby se naše ženy a děti staly kořistí? Nebude pro nás lépe vrátit se do Egypta?“
\newline
\newline
\textsuperscript{26}Hospodin promluvil k Mojžíšovi a Áronovi:
\textsuperscript{27}„Jak dlouho mám snášet tuto zlou pospolitost, která proti mně stále reptá? Slyšel jsem reptání Izraelců, jak proti mně reptají.
\textsuperscript{28}Vyřiď jim: Jakože jsem živ, je výrok Hospodinův, naložím s vámi tak, jak jste si o to říkali.
\textsuperscript{29}Na této poušti padnou vaše mrtvá těla, vás všech povolaných do služby, kolik je vás všech od dvacetiletých výše, kdo jste proti mně reptali.
\textsuperscript{30}Věru že nevejdete do země kromě Káleba, syna Jefunova, a Jozua, syna Núnova, ačkoliv jsem pozvedl ruku k přísaze, že v ní budete přebývat.
\textsuperscript{31}Ale vaše děti, o nichž jste tvrdili, že se stanou kořistí, do ní uvedu, takže poznají zemi, kterou jste zavrhli.
\textsuperscript{32}Avšak vy, vaše mrtvá těla padnou na této poušti
\textsuperscript{33}a vaši synové budou pastevci na poušti po čtyřicet let. Ponesou následky vašeho smilstva, dokud vaše mrtvá těla nebudou do jednoho ležet na poušti.
\textsuperscript{34}Podle počtu dnů, v nichž jste dělali průzkum země, ponesete své viny. Za každý den jeden rok, za čtyřicet dnů čtyřicet let. Tak pocítíte mou nevoli.
}

\subsection*{Reflexe}
Izraelité konečně přišli na okraj země zaslíbené. Země je přesně taková, jak ji Bůh popsal, ale není taková,
jakou Izraelité čekali. Ano, oplývá mlékem a medem, ale také je obývána silným a opevněným lidem.

Možná dnes zažíváte velkou svobodu. Nebo naopak právě dnes ne. Možná jste se zadrhli v poušti. Oddávali
jste se sladkostem, otevřeli jste si Netflix, nemohli jste se odpoutat burzovnímu trhu tak, jak byste si přáli,
spadli jste do pornografie nebo masturbace. To jsou reálné možnosti. Znamená to, že nemůžete vstoupit do
země zaslíbené?

Podívejte se na Slovo. Vprostřed svého exodu padli Izraelité do hříchů těla. Vrátili se zpět k modloslužbě a
orgiím (srov. Ex 32,1-6). Look to the Word. In the midst of their exodus, the Israelites fell into sins of the
flesh. They went back to idol worship and had an orgy (cf., Exodus 32:1–6). Odmítl Bůh tento lid a vzdal se
jejich dovedení do země zaslíbené? Ne. Dal Izraelitům plán – Zákon, který je zachován v knize Leviticus. To
bylo spravedlivé i milosrdné. Tento plán je mohl dovést k odpuštění, znovusjednocení s Bohem, a vpřed do
zaslíbené země, pokud by Izraelité činili pokání a zvolili si dodržování tohoto plánu.

Ačkoli hříchy těla Izraelity neoddálily od země zaslíbené, dnes vidíme, že to byla jejich nevíra v Boha. Pán
vás chce přivést do plné svobody. Pochybujete, že to pro vás Pán může udělat?

Dnes si o tom ve své svaté hodince s Pán upřímně popovídejte.


%newday
\newpage
\section{Den 89 - BOHU BUĎ SLÁVA}
\zacatekTrinactyTyden
\subsection*{Čtení na den}
\textbf{Numeri 20,1-13}
\newline
\textit{
\textsuperscript{1}Celá pospolitost Izraelců dorazila v prvním měsíci na poušť Sin. Lid se usadil v Kádeši. Tam zemřela Mirjam a byla tam i pochována.
\textsuperscript{2}Pospolitost neměla vody. Proto se srotili proti Mojžíšovi a Áronovi.
\textsuperscript{3}Lid se dal s Mojžíšem do sváru. Naříkali: „Kéž bychom byli také zahynuli, když zahynuli naši bratří před Hospodinem!
\textsuperscript{4}Proč jste zavedli Hospodinovo shromáždění na tuto poušť? Abychom tu pomřeli, my i náš dobytek?
\textsuperscript{5}Proč jste nás vyvedli z Egypta? Abyste nás uvedli na toto zlé místo? Na místo, kde nelze sít obilí ani pěstovat fíky nebo vinnou révu či granátová jablka, ba není tady ani voda k napití.“
\textsuperscript{6}I odešli Mojžíš a Áron od shromáždění ke vchodu do stanu setkávání a padli na tvář. Tu se jim ukázala Hospodinova sláva.
\textsuperscript{7}Hospodin promluvil k Mojžíšovi:
\textsuperscript{8}„Vezmi hůl, svolej spolu se svým bratrem Áronem pospolitost a před jejich očima promluvte ke skalisku, ať vydá vodu. Vyvedeš jim tak vodu ze skaliska a napojíš pospolitost i jejich dobytek.“
\textsuperscript{9}Mojžíš tedy vzal hůl, která byla před Hospodinem, jak mu přikázal.
\textsuperscript{10}I svolal Mojžíš s Áronem shromáždění před skalisko. Řekl jim: „Poslyšte, odbojníci! To vám z tohoto skaliska máme vyvést vodu?“
\textsuperscript{11}Nato Mojžíš pozdvihl ruku, dvakrát udeřil svou holí do skaliska a vytryskl proud vody, takže se napila pospolitost i jejich dobytek.
\textsuperscript{12}Hospodin však Mojžíšovi a Áronovi řekl: „Protože jste mi neuvěřili, když jste měli před syny Izraele dosvědčit mou svatost, neuvedete toto shromáždění do země, kterou jim dám.“
\textsuperscript{13}To jsou Vody Meriba (to je Vody sváru), protože se Izraelci přeli s Hospodinem; on však mezi nimi prokázal svou svatost.
}

\subsection*{Reflexe}
Izraelité se znovu bouří proti Mojžíšovi a Áronovi, tentokrát kvůli tělesné žízni. Tato scéna vám možná připomíná tu
ze 42. dne (Ex 17). Tyto dva příběhy jsou podobné, ale v dnešní scéně Bůh zamýšlí jít o jeden krok dál v odhalení své
slávy Izraeli.

Bůh Mojžíšovi přikazuje, aby promluvil ke skále, aby vydala vodu. Na rozdíl od předchozího případu, kdy měl Mojžíš
do skály uděřit holí, zde stačí pouhá slova. Byl by to pro Izraelity působivý pohled. Ale Mojžíšova pýcha mu vstoupí
do cesty. Místo toho, aby promluvil ke skále, jak mu bylo přikázáno, neuposlechne a užije autority svého úřadu k tomu,
aby promluvil k lidu, a tím na sebe bere slávu patřící Bohu.

Mojžíšova slova odhalují jeho hřích: „Poslyšte, odbojníci! To vám z tohoto skaliska máme vyvést vodu?“ (Num
20,10). Ano, Mojžíšův i Áronův úřad jim propůjčuje autority, ale také vyžaduje pokornou poslušnost. Pouze moc daná
od Boha může učinit, aby skála vydala vodu, přesto Mojžíš koná pyšně. Místo ke skále mluví k lidu, a směřuje pohled
Izraelitů na sebe místo na Boha. Avšak i přes Mojžíšovu neposlušnost Bůh zapezpečí svůj lid skrze Mojžíšův úřad –
ale ne bez spravedlivých následků pro Mojžíše.

Mojžíš ztrácí výsadu vstoupit do země zaslíbené. S Izraelity nyní stojíte na okraji země zaslíbené. Budete
bombardováni možnostmi, kdy se budete moci zachovat pyšně. Vaši přátelé, kolegové, manželka nebo farníci vás
budou chválit za dokončení tak intenzivního duchovního cvičení. Jaká bude vaše odpověď? Dovolíte, aby k vám
vzhlížely jejich oči, nebo je budete směřovat k Tomu, který má skutečně moc přivést muže do Dne 91 a života
svobody? Komu vzdáte čest za to, že vás přivedl až na konec tohoto duchovního cvičení? Komu patří sláva za vaše
vykoupení? Přiznejte dnes Bohu slávu za Jeho velké dílo ve vás.


%newday
\newpage
\section{Den 90 - SHEMA: ZVOLTE SI ŽIVOT }
\zacatekTrinactyTyden
\subsection*{Čtení na den}
\textbf{Deuteronomium 6,4-9; 30,11-20}
\newline
\textit{
\textsuperscript{4}Slyš, Izraeli, Hospodin je náš Bůh, Hospodin jediný.
\textsuperscript{5}Budeš milovat Hospodina, svého Boha, celým svým srdcem a celou svou duší a celou svou silou.
\textsuperscript{6}A tato slova, která ti dnes přikazuji, budeš mít v srdci.
\textsuperscript{7}Budeš je vštěpovat svým synům a budeš o nich rozmlouvat, když budeš sedět doma nebo půjdeš cestou, když budeš uléhat nebo vstávat.
\textsuperscript{8}Uvážeš si je jako znamení na ruku a budeš je mít jako pásek na čele mezi očima.
\textsuperscript{9}Napíšeš je také na veřeje svého domu a na své brány.
\newline
\newline
\textsuperscript{11}Tento příkaz, který ti dnes udílím, není pro tebe ani nepochopitelný, ani vzdálený.
\textsuperscript{12}Není v nebi, abys musel říkat: „Kdo nám vystoupí na nebe, vezme jej pro nás a ohlásí nám jej, abychom ho plnili?“
\textsuperscript{13}Ani za mořem není, abys musel říkat: „Kdo se nám přeplaví přes moře, vezme jej pro nás a ohlásí nám jej, abychom ho plnili?“
\textsuperscript{14}Vždyť to slovo je ti velmi blízko, ve tvých ústech a ve tvém srdci, abys je dodržoval.
\textsuperscript{15}Hleď, předložil jsem ti dnes život a dobro i smrt a zlo;
\textsuperscript{16}když ti dnes přikazuji, abys miloval Hospodina, svého Boha, chodil po jeho cestách a dbal na jeho přikázání, nařízení a právní ustanovení, pak budeš žít a rozmnožíš se; Hospodin, tvůj Bůh, ti bude žehnat v zemi, kterou přicházíš obsadit.
\textsuperscript{17}Jestliže se však tvé srdce odvrátí a nebudeš poslouchat, ale dáš se svést a budeš se klanět jiným bohům a sloužit jim,
\textsuperscript{18}oznamuji vám dnes, že úplně zaniknete. Nebudete dlouho živi v zemi, kam přecházíš přes Jordán, abys ji obsadil.
\textsuperscript{19}Dovolávám se dnes proti vám svědectví nebes i země: Předložil jsem ti život i smrt, požehnání i zlořečení; vyvol si tedy život, abys byl živ ty i tvé potomstvo
\textsuperscript{20}a miloval Hospodina, svého Boha, poslouchal ho a přimkl se k němu. Na něm závisí tvůj život a délka tvých dnů, abys mohl sídlit v zemi, o které přísahal Hospodin tvým otcům, Abrahamovi, Izákovi a Jákobovi, že jim ji dá.
}

\subsection*{Reflexe}
Kniha Deuteronomium zachycuje příběh exodu Izraele poté, co uplynulo čtyřicet let v poušti a generace
pochybující o Bohu vymřela. Mladší Izraelité, nová generace Izraele, jsou nyní připraveni vstoupit do země
zaslíbené. Kniha Deuteronomium sestává ze sérií projevů na rozloučenou, které dává Mojžíš před svou smrtí
nové generaci Izraele.
První část dnešního čtení se slavně nazývá Shema, což v hebrejštině znamená „slyš“. Toto slovo zahrnuje jak
poslouchání, tak i akt odpovědi. Mojžíš vyzývá tuto novou generaci Izraele k tomu, co jejich předešlá
generace neudělala. Vyzývá ji, aby byla poslušná Zákonu, a odpovídala tím, že jej bude věrně následovat.
\begin{itemize}
  \item Viděli jste, kam vede nevěrnost Bohu.
  \item Znáte detailně Boží návod pro váš život.
  \item Rozhodněte se jej následovat.
\end{itemize}

Toto Mojžíšovo opodstatněné a nekompromisní volání je tu dnes pro nás stejně jako pro novou generaci
Izraele. Plán je jasný. Denně se modlete, čiňte skutky askeze, žijte bratrství.
V druhé polovině dnešního čtení Mojžíš zdůrazňuje volbu, která před Izraelity stojí. Je to volba života nebo
smrti. Bude-li nová generace poslouchat Zákon, zvolí si život. Na druhé straně, pokud si myslí, že tvrdé
zákony nejsou pro Boha důležité nebo nestojí za to – pokud uslyší, ale nezvolí si podle toho odpovědět – pak
si bude skutečně volit smrt.
Za úsvitu zítřejšího dne budete svobodným člověkem. Budete součástí nové generace mužů připravených
sloužit svým rodinám a Církvi. Co s touto svobodou uděláte? Co uděláte s plánem, kterým vás Pán v těchto
devadesáti dnech obdaroval?
\begin{itemize}
  \item Viděli jste, kam vede nevěrnost Bohu.
  \item Znáte detailně Boží návod pro váš život.
  \item Rozhodněte se jej následovat.
\end{itemize}
Volba je na vás. Zvolte si život.


%newday
\newpage
\section{Den 91 - ZAČÁTEK}
\zacatekTrinactyTyden
\subsection*{Čtení na den}
\textbf{Genesis 1,1}
\newline
\textit{
  (Toto je jen ukázka Písma a rozjímání pro první den duchovního cvičení Dne 91.)
  \newline
  \newline
\textsuperscript{1}Na počátku stvořil Bůh nebe a zemi.
}

\subsection*{Reflexe}
Vítejte ve Dni 91 a v duchovním cvičení Dne 91. Většina z vás poprvé dokončila Exodus 90. Pro některé to byl druhý
Exodus 90. A pro pár ostatních se Exodus stal každoroční zkušeností obnovu vás a vašeho bratrstva. Bez ohledu na to,
kolikrát jste si již Exodem prošli, vám blahopřejeme k dokončení devadesáti dní očištění. Za vaši farnost, rodinu a
Církev vám děkujeme.

V průběhu četby knihy Exodus jste také, vědomě či nevědomky, začali více a hlouběji číst Písmo. Nyní, ve Dni 91,
máte možnost denně se Slovem pokračovat. Jak odcházíte z Exodu, dovolte Mu, Slovu, aby vás vedlo. Tím ochutnáte
plody pokračující formace na vaší cestě životem. Nyní se však vraťme k rozjímání nad dnešním úryvkem z Písma.

Jedním z nejlepších důkazů smyslu Písma je Písmo samo. Órigenés, teolog třetího století, napsal: „Všechno, co bylo
stvořeno, bylo stvořeno ‚na počátku‘, to znamená v Spasiteli.“ Abychom Órigenovu vhledupřidali na závažnosti,
obracíme se k prvním slovům Janova evangelia:

“Na počátku bylo Slovo, to Slovo bylo u Boha, to Slovo bylo Bůh. To bylo na počátku u Boha. Všechno povstalo skrze
ně a bez něho nepovstalo nic, co jest.“ (J 1,1-3)

Jan nám říká, že Slovo je jak u Boha, tak Bůh, a to od samotného počátku. To znamená, že nikdy nebyla doba kde by
některá osoba Trojice – Otec, Syn, nebo Duch svatý – nebyla přítomná. Vždycky existovaly. Kdo je toto Slovo? Jen o
pár veršů později nám Jan dává nápovědu: „A Slovo se stalo tělem a přebývalo mezi námi. Spatřili jsme jeho slávu,
slávu, jakou má od Otce jednorozený Syn, plný milosti a pravdy,“ (J 1,14). Jen z těchto veršů můžeme vidět, že Slovo
je samotný Ježíš Kristus, který se stal tělem, aby založil svou Církev a vysvobodil nás od smrti.

Svět, a vše stvořené, bylo stvořeno v Ježíši Kristu. Zamyslete se, co to znamená řádně slavit, reflektujte vaši zkušenost
s Exodem 90 a odhalte svou potřebu pro stálou formaci našich životů. To vše, skrze první tři kapitoly knihy Genesis,
povede ten, který je samotným Slovem. Dnes vstupte do vaší dvacetiminutové kontemplativní modlitby přinesením
hluboké vnitřní odpovědi Pánu na následující otázku.

Jste rádi, že jste dokončili Exodus 90?
\end{document}