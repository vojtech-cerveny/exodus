% ===============================================
% ===== SETTINGS FOR DOCUMENT
% ===============================================
\documentclass[11pt]{article}
\usepackage{geometry}
\geometry{
  top=3cm,
  bottom=3cm
}
\author{Exodus 90}
\usepackage[czech]{babel}
\font\myfont=cmr12 at 40pt
\setlength{\parskip}{1em plus 0.1em minus 0.2em} % Adjusts space between paragraphs
\usepackage{graphicx}
\usepackage{amssymb}
\usepackage{enumitem}
\setlist[enumerate]{itemsep=-0.4em} % Globálně nastavuje odsazení pro všechny seznamy
% Upravení prostoru okolo nečíslovaných podsekcí
\usepackage{titlesec}

% Upravení prostoru okolo nečíslovaných podsekcí
\titleformat*{\subsection}{\normalsize\bfseries}
\titlespacing*{\subsection}{0pt}{2.5ex minus 2ex}{1.3ex minus 1ex}

% ===============================================
% ===== CUSTOM COMMANDS
% ===============================================
\newcommand{\zacatekPrvniTyden}{
  \textbf{Jste v Egyptě} \newline
  Jste ochotni přijmout, že Egypt vás zotročil? „Ano“ bude vyžadovat úplnou změnu ve vašem způsobu života.

\subsection*{Úkony (ukazatelé cesty)}
\begin{enumerate}
  \item Vzdejte se kontroly
  \item Zavázejte se svému bratrstvu
  \item Najít si čas pro každodenní modlitbu
  \item Buďte radostní
  \item Každou noc zkoumejte svůj den
\end{enumerate}
Modlete se, aby Pán osvobodil vás a vaše bratrství. \newline
Modleme se za svobodu všech mužů v exodu, stejně tak, jako se oni modlí za vás.\newline
Ve jménu Otce i Syna i Ducha svatého …  Otče náš… Amen
}

\newcommand{\zacatekDruhyTyden}{
  \textbf{Jste v Egyptě} \newline
  Disciplíny duchovního cvičení zvětštily naši dříve nevědomou náklonnost k lidskému komfortu.
  Nebylo by snazší toto duchovní cvičení opustit a zůstat v pohodlí otroctví navěky?

  \subsection*{Úkony (ukazatelé cesty)}
\begin{enumerate}
  \item Dobře se vyzpovídejte
  \item Držte se denních reflexí
  \item Navštěvujte jednu mši v týdnu navíc
  \item Zvažte přečtení Průvodce terénem
  \item Uvědomte si (zjistěte), kde je vaše kotva
\end{enumerate}
Modlete se, aby Pán osvobodil vás a vaše bratrství. \newline
Modleme se za svobodu všech mužů v exodu, stejně tak, jako se oni modlí za vás.\newline
Ve jménu Otce i Syna i Ducha svatého …  Otče náš… Amen
}

\newcommand{\zacatekTretiTyden}{
  \textbf{Jste v Egyptě} \newline
  Život se stal náročnějším; Zdá se, že Mojžíši a Áronovi se nedaří získat větší svobodu. Navzdory tomu všemu,
komu se tento týden rozhodnete sloužit: Bohu, nebo faraonovi?

\subsection*{Úkony (ukazatelé cesty)}
\begin{enumerate}
  \item Pokračujte ve zkoumání svého dne
  \item Přistupujte upřímně k vaší denní svaté hodině
  \item Neodbývejte úkony/úkoly (don’t cut corners-neřežte rohy)
  \item Pamatujte si své proč
  \item Zůstaňte radostní
\end{enumerate}
Modlete se, aby Pán osvobodil vás a vaše bratrství. \newline
Modleme se za svobodu všech mužů v exodu, stejně tak, jako se oni modlí za vás.\newline
Ve jménu Otce i Syna i Ducha svatého …  Otče náš… Amen
}

\newcommand{\zacatekCtvrtyTyden}{
  \textbf{Jste v Egyptě} \newline
  Izraelité konečně zří pravou cenu opuštění Egypta. Uvědomte si, jaké falešné bohy ničí Bůh ve vašem životě.
Dovolte, aby vám tato realita přinesla změnu srdce, kterou potřebujete, abyste sloužili pouze Bohu samotnému.

\subsection*{Úkony (ukazatelé cesty)}
\begin{enumerate}
  \item Zavázejte se (odevzdejte se) svému bratrstvu
  \item Vytvořte si cvičící plán
  \item Uvědomte si Boží moc
  \item Jděte/vyjděte ven
\end{enumerate}
Modlete se, aby Pán osvobodil vás a vaše bratrství. \newline
Modleme se za svobodu všech mužů v exodu, stejně tak, jako se oni modlí za vás.\newline
Ve jménu Otce i Syna i Ducha svatého …  Otče náš… Amen
}

\newcommand{\zacatekPatyTyden}{
  \textbf{Východ z/od Egypta (východně od Egypta), útěk do pouště} \newline
  Izraelci zabili egyptského boha (beránka) a veřejně rozmazali jehněčí krev na jejich veřeje. Nyní vstupujete
  do pouště. Vyplatí se vám zůstat velmi blízko Bohu a vašemu bratrstvu.

\subsection*{Úkony (ukazatelé cesty)}
\begin{enumerate}
  \item Vzdejte se kontroly
  \item Zkontaktujte svou kotvu
  \item Znovu si připomeňte své proč
  \item Zvažte přečtení Průvodce terénem
\end{enumerate}
Modlete se, aby Pán osvobodil vás a vaše bratrství. \newline
Modleme se za svobodu všech mužů v exodu, stejně tak, jako se oni modlí za vás.\newline
Ve jménu Otce i Syna i Ducha svatého …  Otče náš… Amen
}

\newcommand{\zacatekSestyTyden}{
  \textbf{Severozápadní břeh Rudého moře} \newline 
  Faraon a jeho armáda opustili Egypt v neúprosném pronásledování Izraelitů. Jen co si Izraelité pomysleli, že jsou zcela na svobodě, ocitli se uvězněni mezi zuřící armádou a zdánlivě neprůchodným vodním útvarem. Ztratit zde naději by znamenalo vzdát se víry v Boha, a vyústění by vedlo k ještě tvrdšímu zotročení než kdy dříve. Nacházíme se na podobném místě. Naše dřívější zvyky na nás útočí. Pokud ztratíme víru, skončíme a otočíme se nazpět, budeme znovu zotročeni. Pokud však zůstaneme oddaní naší víře, Pán nás povede skrz vody. Očistí nás a oddělí nás od našich nepřátel jako nikdy předtím. Co si tento týden zvolíte?

\subsection*{Úkony (ukazatelé cesty)}
\begin{enumerate}
  \item Udělejte si čas na dobrou zpověď
  \item Držte se reflexí
  \item Vstupte do Slova
\end{enumerate}
Modlete se, aby Pán osvobodil vás a vaše bratrství. \newline
Modleme se za svobodu všech mužů v exodu, stejně tak, jako se oni modlí za vás.\newline
Ve jménu Otce i Syna i Ducha svatého …  Otče náš… Amen
}

\newcommand{\zacatekSedmyTyden}{
  \textbf{Jste v drsné hornaté poušti na úpatí hory Sinaj} \newline 
  Ve svých rukou držíte plán svobody: modlitba, askeze a bratrství. Stále je před námi mnoho práce. Nové zvyky vyžadují dostatek času (značný čas), aby se vytvořily.

  \subsection*{Úkony (ukazatelé cesty)}
\begin{enumerate}
  \item Ctěte svou svatou hodinu
  \item Běžte ke zdroji
  \item Přistupujte k vašemu nočnímu examen zodpovědně
  \item Zůstaňte radostní
\end{enumerate}
Modlete se, aby Pán osvobodil vás a vaše bratrství. \newline
Modleme se za svobodu všech mužů v exodu, stejně tak, jako se oni modlí za vás.\newline
Ve jménu Otce i Syna i Ducha svatého …  Otče náš… Amen
}

% ===============================================
% ===== DOCUMENT BEGINS
% ===============================================

\title{\myfont Exodus 90 - Česká verze}
\date{}							% Activate to display a given date or no date

\begin{document}

\maketitle
\vspace*{\fill}
Pokud preferuješ tištěnou verzi Exodusu, pak jsi zde správě. Tato verze je připravená na tisk tak, aby se dobře četla. Užij si ji a i s ní i celý Exodus 90!

% ===============================================
% ===== PRVNI TYDEN
% ===============================================
%ukony
\newpage
\section*{Úkony (ukazatel cesty) pro 1. týden}

\textbf{Místo:} Egypt (jste v Egyptě)

Tento týden se ocitnete s Izraelity v Egyptě. Egyptské prostředí se za celá staletí stalo nepřátelským a utlačujícím, a přesto jste zůstali k otroctví slepí. Naštěstí byl Mojžíš pověřen, aby vám zvěstoval pravdu. Jste ochotni uznat, že vás Egypt zotročil? Řeknete-li „ano“, bude od vás požadována kompletní změna života.

\subsection*{1. Vzdejte se kontroly}
Disciplíny Exodus 90 vám poskytují příležitost vzdát se kontroly nad svým životem a předat ji Bohu. Učte se nově svěřovat kontrolu do Božích rukou. Jako by rytíř položil svůj meč na oltář a na oplátku by obdržel Boží moc, musíte udělat totéž.

\subsection*{2. Zavázat se k vašemu bratrstvu}
Tahle cesta je těžká. Budete vyzkoušeni a testováni. Budete potřebovat své bratry a oni budou potřebovat vás. Týdenní setkání jsou nutností. Modlete se jako Izraelité, kteří byli zachráněni jako kmen (a ne jednotlivě), aby Bůh vysvobodil vaše společenství a také všechny lidi, kteří čekají na vysvobození ze závislosti, sobectví, apatie a ovládání.

\subsection*{3. Najít si čas pro každodenní modlitbu}
Strávit hodinu času v modlitbě každý den. Pokud je to nemožné, trávit co nejvíce času, jak je to možné, s minimálně dvaceti minutami tiché modlitby denně. Je dobré si naplánovat konkrétní čas během dne, nebo toho pravděpodobně zanecháte.

\subsection*{4. Buďte radostní}
Přijali jste Kristův plán svobody. Ano, bude to těžké, ale to by vás nemělo zarmoutit. Spíše se těšte z naděje na svobodu, která vás čeká. Boží pozvání do tohoto duchovního cvičení by vám mělo přinést bohatou radost.

\subsection*{5. Každou noc zkoumejte svůj den}
Exodus 90 obsahuje mnoho disciplín, na které můžete každým dnem odpovídat „ano“. Na konci každého dne před usnutím si nalezněte čas na to, abyste si prošli a zkoumali svůj den. To vám pomůže nejen úspěšně přistupovat na jednotlivé disciplíny, ale také vám pomůže vidět váš pokrok, který jste udělali na cestě ke svobodě. (Jak na to se dozvíte v příručce Exodu v sekci „Jak se modlit noční examen“).Možná verze Examenu přímo zde.

\subsection*{Modlitba}
Modlete se, aby Pán osvobodil vás a vaše bratrství.

Modleme se za svobodu všech mužů v exodu, stejně tak, jako se oni modlí za vás.

Když se učedníci zeptali Ježíše, jak se mají modlit, naučil je „Otče náš“ (viz Lukáš 11: 1–4, Matouš 6: 9–13). Připojte se ke svým bratrům Exodu po celém světě a každý den se modlete tuto mocnou modlitbu za výše uvedené úmysly.

Ve jménu Otce i Syna i Ducha svatého …  Otče náš… Ve jménu Otce i Syna i Ducha svatého … Amen.

%newday
\newpage
\section{Den 1 - JEDINĚ BŮH PŘINÁŠÍ SVOBODU}
\zacatekPrvniTyden
\subsection*{Čtení na den}
\textbf{Exodus 1, 1-7}
\newline
\textit{
\textsuperscript{1}Toto jsou jména synů Izraelových, kteří přišli do Egypta s Jákobem; každý přišel se svou rodinou:
\textsuperscript{2}Rúben, Šimeón, Lévi a Juda,
\textsuperscript{3}Isachar, Zabulón a Benjamín,
\textsuperscript{4}Dan a Neftalí, Gád a Ašer.
\textsuperscript{5}Všech, kdo vzešli z Jákobových beder, bylo sedmdesát. Josef už byl v Egyptě.
\textsuperscript{6}Potom zemřel Josef a všichni jeho bratři i celé to pokolení.
\textsuperscript{7}Ale Izraelci se rozplodili, až se to jimi hemžilo, převelice se rozmnožili a byli velice zdatní; byla jich plná země.
}
\subsection*{Reflexe}
Začínáme náročnou cestu od otroctví ke svobodě rozjímáním nad úvodním odstavcem starověké knihy Exodus - příběhu cesty Božího lidu z Egyptského otroctví ke svobodě.

Na první pohled se můžeme podivovat nad tím, jak by nám právě tento úryvek mohl být užitečný. Ale nepodceňujme Boží slovo!

Vzhledem k tomu, že kniha Exodus je příběhem cesty Izraelitů z otroctví, mělo by nám připadat zvláštní, že Izraelité byli „mimořádně silní; takže země byla jimi naplněna. “
Jak se dozvíme při další četbě, egyptský faraon se dokonce Izraelitů bál. Jak je tedy možné, že byli Izraelité zotročeni, když byli „mimořádně silní“? Proč nepovstali a neosvobodili se? Jak to, že navzdory své velké síle zůstali zotročeni tyranem?

Kniha Exodus, jak uvidíme, je také naším příběhem. Tento starodávný text není jen historií Izraelitů. Je to také určitá metafora moderního muže. Jsme-li zotročení chtíčem, technologiemi, jídlem, pitím nebo něčím jiným, nacházíme se ve stejném otroctví. Ve skutečnosti to, že jsme zotročeni, neznamená, že jsme slabí.
Ve většině případů jsou naše mozky a těla ve skutečnosti docela silné.
Jenže je více než pravděpodobné, že právě síla je tím pravým důvodem, proč jsme se nechali zotročit.

Muži jsou silní, ale když se snaží vypořádat se s životem a jeho mnoha obtížemi, vrhnou se dychtivě po čemkoliv, co jim přinese útěchu a bezpečí. V průběhu svého života (a zejména v mladém věku) objeví a začnou využívat věci nebo činnosti, o nichž si myslí, že je učiní šťastnými. Užívají těchto věcí, protože je vnímají jako prospěšné (pro svůj život).
Jenže postupem času si začínají uvědomovat, že byli podvedeni. Zjistí, že jim tyto věci nepomáhají, že jim nepřinášejí štěstí, po kterém touží. Ale i když pak velmi touží po svobodě, jejich mozek i nadále požaduje to, k čemu byl veden, že je to prospěšné. Pravdu sice objeví, ale chybí vůle.
Ale to není ostuda. Závislý muž není slabý. Může být naopak velmi silný. A stejně jako Izraelité je v této situaci schopen pochopit jednu z největších pravd Písma svatého: jedině Bůh nás může vysvobodit.
Izraelité byli mimořádně silní, ale nedokázali se osvobodit. Moderní muž to zjišťuje také.

Kolikrát jste se pokoušeli „osvobodit se“, jen abyste zjistili, že to nejde? Tisíckrát?
Když začínáme tuto cestu, nikdy (zdůrazňuji nikdy) nezapomeňte na tuto úžasnou pravdu: vy to můžete zvládnout ... ale bude to Bůh, kdo vás osvobodí.

%newday
\newpage
\section{Den 2 - KAŽDODENNÍ ÚKOLY MOHOU MUŽE ZATĚŽOVAT A OSLABOVAT}
\zacatekPrvniTyden
\subsection*{Čtení na den}
\textbf{Exodus 1, 8-14}
\newline
\textit{
\textsuperscript{8}V Egyptě však nastoupil nový král, který o Josefovi nevěděl.
\textsuperscript{9}Ten řekl svému lidu: „Hle, izraelský lid je početnější a zdatnější než my.
\textsuperscript{10}Musíme s ním nakládat moudře, aby se nerozmnožil. Kdyby došlo k válce, jistě by se připojil k těm, kdo nás nenávidí, bojoval by proti nám a odtáhl by ze země.“
\textsuperscript{11}Ustanovili tedy nad ním dráby, aby jej ujařmovali robotou. Musel stavět faraónovi města pro sklady, Pitom a Raamses.
\textsuperscript{12}Avšak jakkoli jej ujařmovali, množil se a rozmáhal dále, takže měli z Izraelců hrůzu.
\textsuperscript{13}Proto začali Egypťané Izraelce surově zotročovat.
\textsuperscript{14}Ztrpčovali jim život tvrdou otročinou při výrobě cihel a všelijakou prací na poli. Všechnu otročinu, kterou na ně uvalili, jim ještě ztěžovali surovostí.  
}\subsection*{Reflexe}
Kniha Exodus je fascinující tím, že je to skutečně příběh každého člověka, což můžeme jasně vidět v dnešním úryvku Písma.

Egypťané byli plni obav, že by se hebrejský lid mohl stát "příliš mocným", než aby ho mohli ovládat, a že by mohl "bojovat proti nim" (Egypťanům). Egypťané na to šli chytře. „Ustanovili nad nimi úkoláře (dráby), aby je trápili těžkými břemeny.“ Jinými slovy, zaměstnávali muže úkoly, mnoha úkoly.
Jak byli Izraelité stále více zatěžováni každodenní prací, přestali se zajímat o svou svobodu a moc.

Žádný člověk nemůže být hrdinou, když je tak zatížen, že ani nemá čas vzhlédnout a zvažovat svou situaci. A tak Izraelité pracovali pro svého pána ještě usilovněji, ale tím nečekaně rostla jejich moc, i když zůstávali stále zotročeni.

Zamyslete se nad tím, jaké „cihly a malty“ používá Dráb (ten Zlý) ve vašich životech, aby vás ovládal, aby vás udržel daleko od vašeho pravého synovství, abyste se nestali příliš silnými. Stejně jako faraon, závistivý k moci, je lstivý v metodách, které používá, aby nás držel.
Přemýšlejte nad „maltou a cihlami“, které vás obklopují: nekonečná práce, zběsilá činnost, neustálý tlak na to, abyste se dostali dopředu. A zamyslete se nad všemi ostatními věcmi, které ďábel používá k tomu, aby vás zotročil: alkohol, pornografie, chtíč, pýcha, strach z neúspěchu, konkurence s ostatními, peníze, sport a postavení… Všechny tyto věci způsobují, že se váš život stává hořkým a smutným.

Ale nemusíte být jimi utlačováni. Rozhodnutí o jejich odstranění vás naučí, že můžete žít bez nich a můžete uniknout jejich poutům, když vás odvádějí od důležitějších věcí.
Je jasné, že muži mají mnoho povinností. Ale většině moderních mužů by velmi prospělo zjednodušení jejich života.

Zkus dnes strávit čas přemýšlením nad oblastmi, kde by mohl být tvůj život zjednodušen. Pozvi Pána do tohoto rozhovoru a zapiš si své závěry. Jedná se o počáteční krok ke svobodě.

\textbf{Poznámka pro ženaté muže:} Jakákoli zjednodušení nebo změny ve vašem způsobu života by měly být konzultovány a prodiskutovány s vaší manželkou.

%newday
\newpage
\section{Den 3 - POSUNOUT SE VPŘED}
\zacatekPrvniTyden
\subsection*{Čtení na den}
\textbf{Exodus 1, 15-22}
\newline
\textit{
\textsuperscript{15}Egyptský král poručil hebrejským porodním bábám, z nichž jedna se jmenovala Šifra a druhá Púa:
\textsuperscript{16}„Když budete pomáhat Hebrejkám při porodu a při slehnutí zjistíte, že to je syn, usmrťte jej; bude-li to dcera, aťsi je naživu.“
\textsuperscript{17}Avšak porodní báby se bály Boha a rozkazem egyptského krále se neřídily. Nechávaly hochy naživu.
\textsuperscript{18}Egyptský král si porodní báby předvolal a řekl jim: „Co to děláte, že necháváte hochy naživu?“
\textsuperscript{19}Porodní báby faraónovi odvětily: „Hebrejky nejsou jako ženy egyptské; jsou plné života. Porodí dříve, než k nim porodní bába přijde.“
\textsuperscript{20}Bůh pak těm porodním bábám prokazoval dobrodiní a lid se množil a byl velmi zdatný.
\textsuperscript{21}Protože se porodní báby bály Boha, požehnal jejich domům.
\textsuperscript{22}Ale farao všemu svému lidu rozkázal: „Každého syna, který se jim narodí, hoďte do Nilu; každou dceru nechte naživu.“
}

\subsection*{Reflexe}

Faraon měl tak velké obavy z toho, aby se Izraelité neosvobodili ze svého zotročení, že nařídil porodním bábám, aby udělaly něco naprosto zvláštního - aby zabily každého chlapce, kterého porodí, a tím aby udusily budoucnost Izraele. Zatímco porodní báby se tomuto požadavku hrdinně vyhýbají, faraon je neúprosný a vyzývá k utopení dětí mužského pohlaví v řece Nil.

Svatý Metoděj považuje faraóna za „předobraz ďábla“. Stejně jako faraón nařídil zabití izraelských chlapců, ďábel se pokouší zabít lidskou ctnost. Vysvětluje, že vody Nilu jsou obrazem našich vášní a ten zlý chce, aby se naše duše vrhly do těchto vod, aby se utopily. Každý člověk zná bolest vnitřního udušení: osamělost pornografie, prázdnota alkoholu, nuda neprozkoumaného života a nekonečná snaha o zábavu.

Dnes je třetí den vašeho odhodlání osvobodit se od těchto věcí. To, co vás kdysi zotročovalo, je nyní prostředkem, díky němuž se stáváte svobodnými, pokud se toho snažíte zbavit.

Svatý Augustin si všímá jisté ironie v příběhu Exodus: Izraelci kráčeli vodami Rudého moře na svobodu. Ti, kteří byli zotročeni a odsouzeni k utonutí, nyní procházejí mořem na cestě ke svobodě.

Naše kultura nás obklopuje neustálým vybízením k bezduchým a ničivým požitkům. I když se od nich vzdalujeme, připadá nám, jako bychom jimi procházeli. V tom se podobáme Izraelitů, kteří procházeli Rudým mořem a měli dvě obří vodní stěny - po levici a pravici.

Ale když Boží síla otevírá cestu, naším jediným úkolem je pokročit vpřed.
Děkujte Pánu, že vám dnes otevřel cestu, a získejte odvahu jít kupředu.


%newday
\newpage
\section{Den 4 - DAR NOVÉHO ŽIVOTA}
\zacatekPrvniTyden
\subsection*{Čtení na den}
\textbf{Exodus 2, 1-10}
\newline
\textit{
\textsuperscript{1}Muž z Léviova domu šel a vzal si lévijskou dceru.
\textsuperscript{2}Žena otěhotněla a porodila syna. Když viděla, jak je půvabný, ukrývala ho po tři měsíce.
\textsuperscript{3}Ale déle už ho ukrývat nemohla. Proto pro něho připravila ze třtiny ošatku, vymazala ji asfaltem a smolou, položila do ní dítě a vložila do rákosí při břehu Nilu.
\textsuperscript{4}Jeho sestra se postavila opodál, aby zvěděla, co se s ním stane.
\textsuperscript{5}Tu sestoupila faraónova dcera, aby se omývala v Nilu, a její dívky se procházely podél Nilu. Vtom uviděla v rákosí ošatku a poslala svou otrokyni, aby ji přinesla.
\textsuperscript{6}Otevřela ji a spatřila dítě, plačícího chlapce. Bylo jí ho líto a řekla: „Je z hebrejských dětí.“
\textsuperscript{7}Jeho sestra se faraónovy dcery otázala: „Mám jít a zavolat kojnou z hebrejských žen, aby ti dítě odkojila?“
\textsuperscript{8}Faraónova dcera jí řekla: „Jdi!“ Děvče tedy šlo a zavolalo matku dítěte.
\textsuperscript{9}Faraónova dcera jí poručila: „Odnes to dítě, odkoj mi je a já ti zaplatím.“ Žena vzala dítě a odkojila je.
\textsuperscript{10}Když dítě odrostlo, přivedla je k faraónově dceři a ona je přijala za syna. Pojmenovala ho Mojžíš (to je Vytahující) . Řekla: „Vždyť jsem ho vytáhla z vody.“}

\subsection*{Reflexe}

Když se v našem životě nebo v životě Božího lidu stane něco nepředvídaného a radikálního, můžeme si být téměř jisti, že to jedná Bůh. Vidíme to i v dnešním úryvku z Písma.

Izraelité jsou zotročeni v Egyptě a nemají téměř žádnou naději, že se někdy dočkají svobody, když tu najednou Bůh vzbudí osvoboditele. Jak uvidíme v následujících dnech, narození Mojžíše a jeho povolání jako osvoboditele je předzvěstí velkého osvoboditele, který přijde: Ježíše Krista.

Již nyní můžeme začít vidět podobnosti mezi těmito dvěma postavami:
Mojžíš je zplozen nejmenovaným mužem, je to "hodné" dítě a je na tři dny umístěn v koši na řece. Připomeňme si, že u Marie bylo zjištěno, že "čeká dítě" bez pomoci svého snoubence.

Oba jsou hodné děti: o Ježíšovi čteme, že poté, co byl nalezen v chrámu, „poslouchal je (Josefa a Marii) a prospíval na duchu i na těle a byl milý Bohu i lidem.“ (Lk 2,51-52).

Všimněte si také, že obě děti byly zachráněny před bezohlednými a paranoidními vůdci, kteří se je snažili raději zabít, než aby ztratili svou moc (Mt 2,16).
Číslo tři by mělo vyvolat vzpomínku na mnoho událostí v Kristově životě: tři dny, kdy se dítě ztratí a najde v chrámu, tři dny, kdy je Kristus v hrobě, jeho veřejné působení začalo v jeho třicátém roce.

Ježíš se zjeví jako nový osvoboditel a bude hrát velkou roli v našem vlastním hledání svobody.
A konečně, a to je nejvýznamnější, Mojžíš dostává své jméno, protože faraonova dcera prohlásila: "Vytáhla jsem ho z vody." Mojžíš je tedy v tomto případě zjevením, které je v souladu se skutečností. I vy jste byli vytaženi "z vody", když jste byli pokřtěni, nejspíše jako nemluvně. Byli jste zachráněni z tyranie Zlého a bylo vám dáno vše, co je třeba k tomu, abyste byli synem Nejvyššího.

Mojžíš byl zachráněn skrze vodu, Izraelité byli zachráněni skrze vodu (Ex 14) a i vy jste byli zachráněni skrze vodu křtu.
Křest je dnes často opomíjeným přechodovým rituálem. Na milost a moc svátosti křtu se většinou zapomíná. Svatý Pavel přesto trval na svém významu: „Nevíte snad, že všichni, kteří jsme pokřtěni v Krista Ježíše, byli jsme pokřtěni v jeho smrt? Byli jsme tedy křtem spolu s ním pohřbeni ve smrt, abychom – jako Kristus byl vzkříšen z mrtvých slavnou mocí svého Otce – i my vstoupili na cestu nového života.“ (Řím 6,3-4)

Svoboda, kterou prostřednictvím tohoto exodu hledáte, pramení z nového života, který jste obdrželi při křtu.
Ježíš to řekl jasně: "Kdo uvěří a dá se pokřtít, bude spasen...". Udělali bychom tedy dobře, kdybychom si připomněli milost vlastního křtu a "oživili Boží dar, který je v tobě" (2 Tim 1,6), abyste měli vše, co je nezbytné k získání skutečné a trvalé svobody!

Připomeňte si dnes milosti svého křtu a podívejte se na toho, kdo vám tyto milosti dal. Touží ve vás ještě jednou vzbudit dar nového života. Mluvte s ním dnes otevřeně. Poděkujte mu za pozvání na toto duchovní cvičení a zeptejte se ho, proč vám tak velkoryse znovu nabízí dar nového života. Odpověď bude velká láska.

%newday
\newpage
\section{Den 5 - MUŽ PRO OSTATNÍ}
\zacatekPrvniTyden
\subsection*{Čtení na den}
\textbf{Exodus 2,11-25}
\newline
\textit{
\textsuperscript{11}V oněch dnech, když Mojžíš dospěl, vyšel ke svým bratřím a viděl jejich robotu. Spatřil nějakého Egypťana, jak ubíjí Hebreje, jednoho z jeho bratří.
\textsuperscript{12}Rozhlédl se na všechny strany, a když viděl, že tam nikdo není, ubil Egypťana a zahrabal do písku.
\textsuperscript{13}Když vyšel druhého dne, spatřil dva Hebreje, jak se rvali. Řekl tomu, který nebyl v právu: „Proč chceš ubít svého druha?“
\textsuperscript{14}Ohradil se: „Kdo tě ustanovil nad námi za velitele a soudce? Máš v úmyslu mě zavraždit, jako jsi zavraždil toho Egypťana?“ Mojžíš se ulekl a řekl si: „Jistě se o věci už ví!“
\textsuperscript{15}Farao o tom vskutku uslyšel a chtěl dát Mojžíše zavraždit. Ale Mojžíš před faraónem uprchl a usadil se v midjánské zemi; posadil se u studny.
\textsuperscript{16}Midjánský kněz měl sedm dcer. Ty přišly, vážily vodu a plnily žlaby, aby napojily stádo svého otce.
\textsuperscript{17}Tu přišli pastýři a odháněli je. Ale Mojžíš vstal, ochránil je a napojil jejich stádo.
\textsuperscript{18}Když přišly ke svému otci Reúelovi, zeptal se: „Jak to, že jste dnes přišly tak brzo?“
\textsuperscript{19}Odpověděly: „Nějaký Egypťan nás vysvobodil z rukou pastýřů. Také nám ochotně navážil vodu a napojil stádo.“
\textsuperscript{20}Reúel se zeptal svých dcer: „Kde je? Proč jste tam toho muže nechaly? Zavolejte ho, ať pojí chléb!“
\textsuperscript{21}Mojžíš se rozhodl, že u toho muže zůstane, a on mu dal svou dceru Siporu za manželku .
\textsuperscript{22}Ta porodila syna a Mojžíš mu dal jméno Geršóm (to je Hostem-tam) . Řekl: „Byl jsem hostem v cizí zemi.“
\textsuperscript{23}Po mnoha letech egyptský král zemřel, ale Izraelci vzdychali a úpěli v otročině dál . Jejich volání o pomoc vystupovalo z té otročiny k Bohu.
\textsuperscript{24}Bůh vyslyšel jejich sténání, Bůh se rozpomněl na svou smlouvu s Abrahamem, Izákem a Jákobem,
\textsuperscript{25}Bůh na syny Izraele pohleděl, Bůh se k nim přiznal.
}
\subsection*{Reflexe}

Muži jsou v tom nejlepším, když jsou skutečně „muži pro druhé“. Zralý a sebeovládající se muž, který je zformovaný Boží rukou, má velkou moc, která vychází z nového života v něm.
Ale to, co dělá muže skutečně velkým, je jeho ochota sloužit druhým, používat svou moc/sílu pro druhé - ať už je to jeho žena, děti, bratři, sousedé, církev nebo země. V moderní době jsme upadli do zlozvyku dávat na první místo své vlastní potřeby a touhy, a až na druhé, pokud vůbec, brát v úvahu potřeby ostatních.

V dnešním úryvku z Písma vidíme Mojžíše, jak je solidární s chudými a utlačovanými a jak využívá svou mladickou moc k tomu, aby zasáhl a pomohl druhým. Vidí, že dcery Reuela (Jethra) se nedokážou samy o sebe postarat a postavit se proti darebáckým pastýřům.
Když se zaměříme na péči, podporu a obranu slabších a potřebných, Bůh si nás a naši sílu může použít a použije pro dobro druhých. Brzy se o tom přesvědčíme, až Bůh udělá z Mojžíše velkého osvoboditele a soudce svého lidu.

Dnes a denně se musíme odpoutávat od současných kulturních zvyklostí a rozhodnout se (v případě potřeby každý den) překonat své vlastní potřeby ve prospěch lidí kolem nás.
Během dnešní modlitby se ptejte sami sebe: „Kdo je na vás závislý? Kdo od vás hledá ochranu nebo pomoc tváří v tvář životním nespravedlnostem a nebezpečím? Kdo se vám svěřil v naději, že mu budete pevnou oporou, mužem, na kterého je spolehnutí? Kdo vám věří, že nebudete jen přihlížet, když bude potřeba se angažovat, a to i když to pro vás bude nepohodlné?

Život má skutečně smysl a význam, když se velkoryse věnujeme druhým, nějaké věci nebo církvi.
Jste ochotni takový život žít? Jste ochotni vykročit vpřed a být mužem pro ostatní?

%newday
\newpage
\section{Den 6 - BŮH SI NÁS VOLÍ PŘEDTÍM, NEŽ SI MY ZVOLÍME JEJ}
\zacatekPrvniTyden
\subsection*{Čtení na den}
\textbf{Exodus 3, 1-6}
\newline
\textit{\textsuperscript{1}Mojžíš pásl ovce svého tchána Jitra, midjánského kněze. Jednou vedl ovce až za step a přišel k Boží hoře, k Chorébu.
\textsuperscript{2}Tu se mu ukázal Hospodinův posel v plápolajícím ohni uprostřed trnitého keře. Mojžíš viděl, jak keř v ohni hoří, ale není jím stráven.
\textsuperscript{3}Řekl si : „Zajdu se podívat na ten veliký úkaz, proč keř neshoří.“
\textsuperscript{4}Hospodin viděl, že odbočuje, aby se podíval. I zavolal na něho Bůh zprostředku keře: „Mojžíši, Mojžíši!“ Odpověděl: „Tu jsem.“
\textsuperscript{5}Řekl: „Nepřibližuj se sem! Zuj si opánky, neboť místo, na kterém stojíš, je půda svatá.“
\textsuperscript{6}A pokračoval: „Já jsem Bůh tvého otce, Bůh Abrahamův, Bůh Izákův a Bůh Jákobův.“ Mojžíš si zakryl tvář, neboť se bál na Boha pohledět.
}

\subsection*{Reflexe}

Všimněte si způsobu, jakým Bůh a Mojžíš začali své hluboké přátelství. Ne Mojžíš šel hledat a najít Boha. To se stává velmi zřídka, pokud vůbec. Ale byl to Bůh, kdo přišel s Mojžíšem jako první.  Mojžíš řeší své každodenní starosti, když se mu Bůh zjeví a dovolí mu odpovědět.  

Svatá Terezie z Avily často používala k popisu tohoto jevu příměr se slunečnicí. Když ráno vyjde slunce, jeho paprsky zalijí krajinu a slunečnice k němu otočí hlavu. Může to však udělat pouze tehdy, když na ni svítí slunce. Podobně když se duše obrací k Bohu, je to proto, že Bůh udělal první krok. To je základ duchovního života.

Může existovat tisíc důvodů, proč jste se rozhodli naplnit Exodus 90. Ale nebyli jste to vy, kdo se rozhodl to udělat - byl to Bůh, kdo vás k tomu povolal. Touto výzvou, tímto pozváním k vykonání těchto duchovních cvičení vám Bůh otevřel cestu k hlubšímu vztahu s ním. Během těchto 90 dní je Božím záměrem zjevit se vám více. Tento krok je Jeho, ale je na vás, jak na něj zareagujete. Využijte tento požehnaný čas k tomu, abyste "obrátili hlavu k Bohu" a objevili Ho tak, jak se vám dovolil zjevit.

Volejte dnes k Pánu v modlitbě. Požádejte Ho, aby se vám zjevil více než kdykoli předtím.

%newday
\newpage
\section{Den 7 - BŮH DÁVÁ ČLOVĚKU SÍLU}
\zacatekPrvniTyden
\subsection*{Čtení na den}
\textbf{}
\newline
\textit{
\textsuperscript{7}Hospodin dále řekl: „Dobře jsem viděl ujařmení svého lidu, který je v Egyptě. Slyšel jsem jeho úpění pro bezohlednost jeho poháněčů. Znám jeho bolesti.
\textsuperscript{8}Sestoupil jsem, abych jej vysvobodil z moci Egypta a vyvedl jej z oné země do země dobré a prostorné, do země oplývající mlékem a medem, na místo Kenaanců, Chetejců, Emorejců, Perizejců, Chivejců a Jebúsejců.
\textsuperscript{9}Věru, úpění Izraelců dolehlo nyní ke mně. Viděl jsem také útlak, jak je Egypťané utlačují.
\textsuperscript{10}Nuže pojď, pošlu tě k faraónovi a vyvedeš můj lid, Izraelce, z Egypta.“
\textsuperscript{11}Ale Mojžíš Bohu namítal: „Kdo jsem já, abych šel k faraónovi a vyvedl Izraelce z Egypta?“
\textsuperscript{12}Odpověděl: „Já budu s tebou! A toto ti bude znamením, že jsem tě poslal: Až vyvedeš lid z Egypta, budete sloužit Bohu na této hoře.“
\textsuperscript{13}Avšak Mojžíš Bohu namítl: „Hle, já přijdu k Izraelcům a řeknu jim: Posílá mě k vám Bůh vašich otců. Až se mě však zeptají, jaké je jeho jméno, co jim odpovím?“
\textsuperscript{14}Bůh řekl Mojžíšovi: „JSEM, KTERÝ JSEM.“ A pokračoval: „Řekni Izraelcům toto: JSEM posílá mě k vám.“
\textsuperscript{15}Bůh dále Mojžíšovi poručil: „Řekni Izraelcům toto: ‚Posílá mě k vám Hospodin, Bůh vašich otců, Bůh Abrahamův, Bůh Izákův a Bůh Jákobův.‘ To je navěky mé jméno, jím si mě budou připomínat od pokolení do pokolení.
\textsuperscript{16}Jdi, shromažď izraelské starší a pověz jim: ,Ukázal se mi Hospodin, Bůh vašich otců, Bůh Abrahamův, Izákův a Jákobův, a řekl: Rozhodl jsem se vás navštívit, vím , jak s vámi v Egyptě nakládají,
\textsuperscript{17}a prohlásil jsem: Vyvedu vás z egyptského ujařmení do země Kenaanců, Chetejců, Emorejců, Perizejců, Chivejců a Jebúsejců, do země oplývající mlékem a medem.‘
\textsuperscript{18}Až tě vyslechnou, půjdeš ty a izraelští starší k egyptskému králi a řeknete mu: ‚Potkal se s námi Hospodin, Bůh Hebrejů. Dovol nám nyní odejít do pouště na vzdálenost tří dnů cesty a přinést oběť Hospodinu, našemu Bohu.‘
\textsuperscript{19}Vím, že vám egyptský král nedovolí jít, leda z donucení.
\textsuperscript{20}Proto vztáhnu ruku a budu bít Egypt všemožnými svými divy, které učiním uprostřed něho. Potom vás propustí.
\textsuperscript{21}Zjednám tomuto lidu u Egypťanů přízeň. Až budete odcházet, nepůjdete s prázdnou.
\textsuperscript{22}Každá žena si vyžádá od sousedky a spolubydlící stříbrné a zlaté ozdoby a pláště. Vložíte je na své syny a dcery. Tak vypleníte Egypt.“
}

\subsection*{Reflexe}

Bůh dává vysvobození. Vidíme to na příběhu Izraelitů. V dnešním čtení říká Bůh Mojžíšovi, že nejprve propustí
Izraelity z ruky Egypťanů. Poté je dovede do „zaslíbené země“, která je ovšem obývána Kennaanci, Chetejci,
Emorejsi, Perizejsi, Chivejci a Jebúsejci – všemi nepřáteli Izraele. Dokážete si představit, co si Mojžíš mohl myslet?
„Chceš nás osvobodit od našich otrokářů (což nemůže dopadnout dobře) jenom proto, abys nás mohl dovést
doprostřed našich napřátel?“ Copak je divu, že chtěl být Mojžíš sám? Ale Bůh, aby dal Mojžíšovi odvahu, dělá něco
naprosto nemyslitelného. Zjevuje mu svoje svaté jméno: „Jsem, který jsem.“

V moderním světě jsme zapomněli na důležitost teologie jmen. Ve starověku vědět něčí jméno znamenalo mít nad
ním nějakou moc. Proto Adam pojmenoval všechna zvířata v zahradě Edenu. Prohlašoval tím svou nadřazenost nad
zvířecí říší (Gen 2,20). Když dává Bůh poznat své jméno Mojžíšovi, také mu tím propůjčuje svou božskou moc. Jak
by mohl teď Mojžíš pochybovat o příslibu jemu a jeho lidu? Pouze potřebuje jít za Bohem v důvěře.

Také si povšimněte významnosti úkolu danému Mojžíšovi a zprávy, která mu je dána k předání Izraelitům: „Jsem
mě poslal k vám.“ Znovu Mojžíš předznamenává Ježíše Krista. Tak jako byl Mojžíš poslán Bohem k Izraelitům,
Kristus byl poslán Otcem k osvobození nás všech. „Neboj se“, říká Ježíš, „jen věř,“ (Lk 8,50). Když procházíte
disciplínami Exodu 90, zapíráte se a celou dobu bojujete, mějte oči upřeny na Ježíše. Byl poslán, aby vás vykoupil
od vás samotných, vašich hříchů, zotročení. Když bojujete za svobodu v dennodenním boji, Bůh tam bojuje s vámi
a pro vás.

Zavolejte na Boha. Čeká, aby vám dal svou moc a sílu.


% ===============================================
% ===== DRUHY TYDEN
% ===============================================
%ukony
\newpage
\section*{Úkony (ukazatel cesty) pro 2. týden}

\textbf{Místo:} Egypt (jste v Egyptě)

Izraelité jsou daleko od svobody. Mojžíšova a Áronova poslušnost Bohu pouze zhoršila jejich situaci a zvýraznila jejich otroctví více než kdy dříve. Na tomto místě našeho exodu se také naše otroctví stalo viditelnějším. Disciplíny duchovního cvičení zvětštily naši dříve nevědomou náklonnost k lidskému komfortu. Tato rutina nás formuje, ale nejsme ještě vůbec blízko svobody. Proč si stěžujeme život poslušností Bohu? Nebylo by snazší toto duchovní cvičení opustit a zůstat v pohodlí otroctví navěky? Této úvaze čelí Izraelité i my tento týden.

\subsection*{1. Dobře se vyzpovídejte}
Pokoušet se začít Exodus 90, aniž bychom šli nejdříve ke zpovědi, je jako pokoušet se vylézt na vrchol nebezpečné americké sopky Mount Rainier s pětadevadesáti kily kamení v krosně. Proveditelné, ale pošetilé. Kříž, který na sebe musíte brát každý den je dost těžký tak, jak je. Nechte Boha vyndat ono kamení z vaší krosny. Běžte ke zpovědi, abyste mohli zdolat tuto horu a být opravdu svobodní.
\subsection*{2. Držte se denních reflexí}
Pokud denní čtení a reflexe neprovádíte, neděláte vůbec Exodus 90. Denní čtení Písma a rozjímání nad ním dovoluje Ježíši Kristu, Slovu, aby vás vedl na cestě vaším exodem. Neskončíte se studenou sprchou jen proto, že vám nevyhovuje, tak neskončujte ani se čtením Písma jen proto, že se vám nechce. Držte se tohoto rozjímání. Udrží vás a vaše bratrstvo jednotné po dobu exodu.
\subsection*{3. Navštěvujte jednu mši v týdnu navíc}
Na otázku, co by vyzvalo lidi k tomu, aby více rostli ve víře, odpověděl kněz Augustinského Institutu takto: „Ať chodí na další mši svatou během dní v týdnu.“ Nyní nastal ten čas. Zvolte si den v týdnu a konkrétní čas mše, na kterou budete chodit každý týden, vedle povinných mší svatých v neděli a o svátcích. (Pro více informací o účincích navštěvování více mší svatých v týdnu a způsobu, jak toho využívala různá bratrstva v minulosti, nahlédněte do sekce \textit{Posílit své bratrství} pod pilířem Bratrství v příručce Exodu.)
\subsection*{4. Zvažte přečtení \textit{Průvodce terénem}}
Pokud jste si nenašli čas na přečtení \textit{Průvodce terénem} Exodu 90 předtím, než jste Exodus začali, zvažte jeho přečtení dnes, nebo tuto neděli. Tento průvodce rámcuje celou zkušenost Exodu 90 a pomůže vám pochopit důvod každé části vašeho exodu. Porozumění těmto důvodům často pomáhá k tomu, abyste se zavázali k disciplínám s větší radostí a lehčím srdcem. (Nejdůležitější části \textit{Průvodce terénem} jsou: \textit{Začněte zde: Co je vaše proč}, \textit{Pilíře Exodu 90}, a pro ženaté muže \textit{Muž Exodu a jeho manželka.})
\subsection*{5. Uvědomte si (zjistěte), kde je vaše kotva}
Zavázali jste se k denní komunikaci s vaší kotvou. On na vás spoléhá v tom, že budete naplňovat váš závazek. Jestli jste tak činili, skvěle. Jestli ne, teď je čas začít. Brzy budete potřebovat vaši kotvu stejně tak, jako on potřebuje vás. (Myslíte si, že každodenní komunikace s vaší kotvou není důležitá? Přečtěte si sekci \textit{Kotva (Nepřeskakujte: Smrt je pravděpodobná)} pod pilířem Bratrství v \textit{Pilířích Exodu 90.})

\subsection*{Modlitba}
Modlete se, aby Pán osvobodil vás a vaše bratrství \newline
Modleme se za svobodu všech mužů v exodu, stejně tak, jako se oni modlí za vás.\newline
Ve jménu Otce i Syna i Ducha svatého … Otče náš… Ve jménu Otce i Syna i Ducha svatého … Amen.

%newday
\newpage
\section{Den 8 - HOSPODIN JE JEDINÝ BŮH}
\zacatekDruhyTyden
\subsection*{Čtení na den}
\textbf{Exodus 4,1-9}
\newline
\textit{
\textsuperscript{1}Mojžíš však znovu namítal: „Nikoli, neuvěří mi a neuposlechnou mě, ale řeknou: Hospodin se ti neukázal.“
\textsuperscript{2}Hospodin mu řekl: „Co to máš v ruce?“ Odpověděl: „Hůl.“
\textsuperscript{3}Hospodin řekl: „Hoď ji na zem.“ Hodil ji na zem a stal se z ní had. Mojžíš se dal před ním na útěk.
\textsuperscript{4}Ale Hospodin Mojžíšovi poručil: „Vztáhni ruku a chyť ho za ocas.“ Vztáhl tedy ruku, uchopil ho a v dlani se mu z něho stala hůl.
\textsuperscript{5}„Aby uvěřili, že se ti ukázal Hospodin, Bůh jejich otců, Bůh Abrahamův, Bůh Izákův a Bůh Jákobův.“
\textsuperscript{6}Dále mu Hospodin řekl: „Vlož si ruku za ňadra.“ Vložil tedy ruku za ňadra. Když ruku vytáhl, byla malomocná, bílá jako sníh.
\textsuperscript{7}Tu poručil: „Dej ruku zpět za ňadra.“ Dal ruku zpět za ňadra. Když ji ze záňadří vytáhl, byla opět jako ostatní tělo.
\textsuperscript{8}„A tak jestliže ti neuvěří a nedají na první znamení, uvěří druhému znamení.
\textsuperscript{9}Jestliže však neuvěří ani těmto dvěma znamením a neuposlechnou tě, nabereš vodu z Nilu a vyleješ ji na suchou zemi. Z vody, kterou nabereš z Nilu, se stane na suché zemi krev.“
}

\subsection*{Reflexe}
Skrze Starý Zákon pracuje Bůh nepřetržitě, aby svému lidu ukázal, že je jediný Bůh. Ale jeho lid bojuje
s tím, aby uvěřil. Znamení po znamení ukazuje Bůh svou moc svému lidu a ostatním národům jako Egyptu.
Ti, kteří vidí tato znamení a uvěří, zakusí Boží lásku. Ti, kteří vidí a neuvěří, jdou směrem ke své vlastní
zkáze.

Věříte, že Pán je jediný Bůh? Žijete tak, jako byste tomu věřili? Dobrým způsobem, jak to vyzkoušet, je
podívat se na vaši neděli a první a poslední věc, co uděláte každý den. Co je nejdůležitější částí vaší neděle?
Sportovní utkání? Práce na zahradě? Co je první věcí, co uděláte každé ráno, když se probudíte, a každý
večer předtím, než usnete? Zkontrolovat svůj telefon? Zapnout zprávy? Bůh vás hledá. Chce, abyste věděli,
že On je jediný Bůh. Kontrolování telefonu vám nezaručí vysvobození, ale klečení na kolenou každé ráno
a každou noc vedle své postele před tím, kdo má moc vás vysvobodit, ano. Skrze tyto činy vám může dát
Bůh vysvobození.

Když pokračujete se čtením úžasných věcí, které Bůh učinil pro jeho lid Izrael, obraťte pozornost také na
svůj vlastní život. Vidíte ty úžasné věci, které pro vás, ve vašem životě a na oltáři, udělal? Pohleďte na tyto
věci a posilte dnes svou víru v to, že Pán je Bůh. Ano, Hospodin je jediný Bůh.


%newday
\newpage
\section{Den 9 - DŮVĚŘUJ BOHU}
\zacatekDruhyTyden
\subsection*{Čtení na den}
\textbf{Exodus 4,10-17}
\newline
\textit{
\textsuperscript{10}Ale Mojžíš Hospodinu namítal: „Prosím, Panovníku, nejsem člověk výmluvný; nebyl jsem dříve, nejsem ani nyní, když ke svému služebníku mluvíš. Mám neobratná ústa a neobratný jazyk.“
\textsuperscript{11}Hospodin mu však řekl: „Kdo dal člověku ústa? Kdo působí, že je člověk němý nebo hluchý, vidící nebo slepý? Zdali ne já, Hospodin?
\textsuperscript{12}Nyní jdi, já sám budu s tvými ústy a budu tě učit, co máš mluvit!“
\textsuperscript{13}Ale Mojžíš odmítl: „Prosím, Panovníku, pošli si, koho chceš.“
\textsuperscript{14}Tu Hospodin vzplanul proti Mojžíšovi hněvem a řekl: „Což nemáš bratra Árona, toho lévijce? Znám ho , ten umí mluvit. Jde ti už naproti a bude se srdečně radovat, až tě uvidí.
\textsuperscript{15}Budeš k němu mluvit a vkládat mu slova do úst. Já budu s tvými ústy i s jeho ústy a budu vás poučovat, co máte činit.
\textsuperscript{16}On bude mluvit k lidu za tebe, on bude tobě ústy a ty budeš jemu Bohem.
\textsuperscript{17}A tuto hůl vezmi do ruky; budeš jí konat znamení.“
}

\subsection*{Reflexe}
Mojžíš zakouší tíhu úkolu, který před něj byl postaven, a myslí si, že se Bůh spletl. Bůh po něm žádá něco,
čeho je neschopné dosáhnout. Tak jako mnoho lidí i dnes, Mojžíš postrádá důvěru v sebe i v Hospodina,
což, jak praví Písmo, rozněcuje Boží hněv. Bůh se ale jistě nehněvá kvůli nedostatku Mojžíšovy
výmluvnosti. Spíše „Hospodin vzplanul proti Mojžíšovi hněvem,“ protože Mojžíš postrádá důvěru v Boha
a dovoluje, aby jeho naděje vymizela. Bůh nás v průběhu našeho života zkouší mnoha způsoby, přesně tak
jako Mojžíše v dnešním čtení. Samozřejmě to nedělá proto, aby se o nás něco naučil. Zkouší nás, abychom
se sami naučili něco o sobě a o Něm.

Mojžíš postrádá ctnost naděje. Poznává, že jeho nedostatky nejsou hříchem, ale nemyslí si, že je Bůh
schopný dorovnat to, co Mojžíš postrádá. Právě to je problém. Mojžíš se musí naučit, že pokud má vést
Boží lid a dosáhnout nemožného, musí se zcela spoléhat na Boha.

Těchto 90 dní Exodu je pro vás obrovskou zkouškou. Učte se od Mojžíše: selžete v tomto cvičení, pokud
opravdu nesvěříte svou naději v Boha. Tak často ve strachu převezmeme kontrolu z Božích rukou a
zkoušíme ji třímat v těch našich. Když to uděláme a uspějeme, činíme tak za velkou cenu nás, a občas i
těch, které milujeme.

Uvědomte si, že se sami nemůžete dovést ke svobodě. Poté pohleďte na kříž. Naděje v Toho, který porazil
i smrt. On vás dovede ke svobodě. On jediný vám může dát vysvobození. Doufejte v Něho.

%newday
\newpage
\section{Den 10 - JAKÝM ČLOVĚKEM SE STANETE?}
\zacatekDruhyTyden
\subsection*{Čtení na den}
\textbf{Exodus 4,18-31}
\newline
\textit{
\textsuperscript{18}Mojžíš odešel a vrátil se ke svému tchánu Jitrovi. Řekl mu: „Rád bych šel a vrátil se ke svým bratřím, kteří jsou v Egyptě, a podíval se, zda ještě žijí.“ Jitro Mojžíšovi odvětil: „Jdi v pokoji.“
\textsuperscript{19}Hospodin pak řekl Mojžíšovi ještě v Midjánu: „Jen se vrať do Egypta, neboť zemřeli všichni, kteří ti ukládali o život.“
\textsuperscript{20}Mojžíš tedy vzal svou ženu a syny, posadil je na osla a vracel se do egyptské země. A do ruky si vzal Boží hůl.
\textsuperscript{21}Hospodin dále Mojžíšovi poručil: „Až se vrátíš do Egypta, hleď, abys před faraónem udělal všechny zázraky, jimiž jsem tě pověřil. Já však zatvrdím jeho srdce a on lid nepropustí.
\textsuperscript{22}Potom faraónovi řekneš: Toto praví Hospodin: ‚Izrael je můj prvorozený syn.
\textsuperscript{23}Vzkázal jsem ti: Propusť mého syna, aby mi sloužil. Ale ty jsi jej propustit odmítl. Za to zabiji tvého prvorozeného syna.‘“
\textsuperscript{24}Když se na cestě chystali nocovat, střetl se s ním Hospodin a chtěl ho usmrtit.
\textsuperscript{25}Tu vzala Sipora kamenný nůž, obřezala předkožku svého syna, dotkla se jeho nohou a řekla: „Jsi můj ženich, je to zpečetěno krví.“
\textsuperscript{26}A Hospodin ho nechal být. Tehdy se při obřízkách říkalo: „ Jsi ženich, je to zpečetěno krví.“
\textsuperscript{27}Hospodin řekl Áronovi: „Jdi na poušť naproti Mojžíšovi.“ Áron šel, setkal se s ním u Boží hory a políbil ho.
\textsuperscript{28}Mojžíš oznámil Áronovi všechna Hospodinova slova, s nimiž ho poslal, a všechna znamení, kterými ho pověřil.
\textsuperscript{29}Pak šel Mojžíš s Áronem a shromáždili všechny izraelské starší.
\textsuperscript{30}Áron vyřídil všechna slova, která mluvil Hospodin k Mojžíšovi, a Mojžíš učinil před očima lidu ona znamení.
\textsuperscript{31}A lid uvěřil. Když slyšeli, že Hospodin navštívil Izraelce a že pohleděl na jejich ujařmení, padli na kolena a klaněli se.
}

\subsection*{Reflexe}

Uvažujme o dvou hlavních postavách knihy Exodus. První, Mojžíš, zná své postavení před Pánem. Není povýšený
ani arogantní. Zachovává si svou důstojnout a je mu dokonce dána výsada od Boha. Na druhé straně stojí faraon, tak
povýšený a arogantní, že dokonce prohlásí sebe samého za božského a jeho podřízení ho takto musí přijímat. Bůh
nakonec zničí faraonovo potomstvo a jeho dynastii a dokazuje tak faraonovi, že je pouhý člověk.

Dnešní muži často napodobují faraona. Jen málo z nich by se nestoudně prohlásilo za božské, ale mnoho z nich se
tak chová. Určují si svou vlastní pravdu, nastavují svůj vlastní směr, odmítají se spoléhat na Boha a žijí svrchovaný
život. Od Adamova pádu má člověk vzpupné srdce. Izraelci tohoto příběhu nejsou výjimkou. Budou muset být
poučeni: čtyřicet let v poušti není procházka. Bůh pošle svůj lid do poušte, aby se naučil být na Něm závislý, plně se
na Něj spolehnout. Musí pochopit, že Bůh je Bůh a že oni jsou jeho prvorozenými syny.

Člověk je schopen mnoha velkolepých činů, ale Boží syn je schopen mnohem, mnohem více. Těchto devadesát dní
nabízí skvělou příležitost, jak svůj život nasměrovat k Bohu. Vzdejte se svého vzpurného srdce. Oblečte nového
člověka, jako milovaný syn pozoruhodného Otce, který se o vás nejen postará, ale také vám pomůže stát se
spolehlivým, nezištným a svobodným mužem pro druhé.

Zvažte svůj vlastní vztah s Bohem. Vidíte Boha jako svého Otce a žijete svůj vztah k Němu jako takovému?
Obracíte se na něj často, jako se dítě obrací na svého otce? Přineste dnes tyto otázky do své svaté hodiny.

%newday
\newpage
\section{Den 11 - POKORA}
\zacatekDruhyTyden
\subsection*{Čtení na den}
\textbf{Exodus 5,1-4}
\newline
\textit{
\textsuperscript{1}Mojžíš s Áronem pak předstoupili před faraóna a řekli: „Toto praví Hospodin, Bůh Izraele: Propusť můj lid, ať mi v poušti slaví slavnost .“
\textsuperscript{2}Farao však odpověděl: „Kdo je Hospodin, že bych ho měl uposlechnout a propustit Izraele? Hospodina neznám a Izraele nepropustím!“
\textsuperscript{3}Řekli: „Potkal se s námi Bůh Hebrejů. Dovol nám nyní odejít do pouště na vzdálenost tří dnů cesty a přinést oběť Hospodinu, našemu Bohu, aby nás nenapadl morem nebo mečem.“
\textsuperscript{4}Egyptský král je okřikl: „Proč, Mojžíši a Árone, odvádíte lid od jeho prací? Jděte za svými robotami!“
}

\newpage
\subsection*{Reflexe}

Faraon arogantně ignoruje příkazy Nejvyššího Boha. Samozřejmě má své důvody. Je vládcem nad mocným
královstvím. Řídí armády a otroky. Pochází z mocné linie uctívaných mužů. Faraon je mocný. Přesto,
navzdory všem jeho vznešenostem, bude ponížen dokonce pod mouchy a žáby, protože neohne koleno před
svým Tvůrcem.

Jsme Boží synové a jsme stvořeni k Jeho obrazu a podobě. Každý z nás má mnoho silných stránek, a proto
je pýcha neustálým pokušením. Když jsme na sebe moc hrdí, začneme se čím dál více oddávat vlastním
nutkáním. Výsledkem bude vždy náš krach. Dřív nebo později se ocitneme sraženi na kolenou věcmi, které
jsou pod naši důstojnost – závislostmi, rozptýleností, nudou… Jestliže se takto necháme ovládat, pýcha nás
učiní bezmocnými.

Mějte oči upřené k nebi a pokorně si pamatujte, že je to Bůh, kdo žádá vaši svobodu – a je to On, kdo dává
vysvobození. Ve své svaté hodině dnes promluvte s Pánem o své pýše. Požádejte ho, aby vám ukázal, na
co jste příliš hrdí. Požádejte ho, aby vám ukázal pravou pokoru. Pak ho proste, aby vám dal odvahu a milost
žít dnes v opravdové pokoře.

%newday
\newpage
\section{Den 12 - ZKLAMÁNÍ Z HŘÍCHU}
\zacatekDruhyTyden
\subsection*{Čtení na den}
\textbf{Exodus 5,5-21}
\newline
\textit{
\textsuperscript{5}A farao pokračoval: „Hle, lidu země je teď mnoho, a vy chcete, aby nechali svých robot?“
\textsuperscript{6}Onoho dne přikázal farao poháněčům lidu a dozorcům:
\textsuperscript{7}„Propříště nebudete vydávat lidu slámu k výrobě cihel jako dříve. Ať si jdou slámu nasbírat sami!
\textsuperscript{8}A uložíte jim dodat stejné množství cihel, jaké vyráběli dříve. Nic jim neslevujte, jsou líní. Proto křičí: Pojďme obětovat svému Bohu.
\textsuperscript{9}Ať na ty muže těžce dolehne otročina, aby měli co dělat a nedali na lživé řeči.“
\textsuperscript{10}Poháněči lidu a dozorci vyšli a ohlásili lidu: „Toto praví farao: Nedám vám žádnou slámu.
\textsuperscript{11}Sami si jděte nabrat slámu, kde ji najdete. Ale z vaší pracovní povinnosti se nic nesleví.“
\textsuperscript{12}Lid se rozběhl po celé egyptské zemi, aby na strništích sbíral slámu.
\textsuperscript{13}Poháněči je honili: „Plňte svůj denní úkol, jako když sláma byla.“
\textsuperscript{14}Dozorci z řad Izraelců, které nad nimi ustanovili faraónovi poháněči, byli biti. Vytýkalo se jim : „Proč jste v těchto dnech nevyrobili tolik cihel jako dříve?“
\textsuperscript{15}Dozorci z řad Izraelců tedy přišli a úpěli před faraónem: „Proč se svými otroky takhle jednáš?
\textsuperscript{16}Tvým otrokům se nedodává sláma, ale pokud jde o cihly, poroučejí nám: ‚Dělejte!‘ Hle, tvoji otroci jsou biti a tvůj lid bude pykat za hřích.“
\textsuperscript{17}Farao odpověděl: „Jste lenoši líní, proto říkáte: ‚Pojďme obětovat Hospodinu.‘
\textsuperscript{18}Hned jděte dělat! Sláma vám dodávána nebude, ale dodávku cihel odvedete.“
\textsuperscript{19}Dozorci z řad Izraelců viděli, že je s nimi zle, když bylo řečeno: „Nesmíte snížit svůj denní úkol výroby cihel.“
\textsuperscript{20}Když vycházeli od faraóna, narazili na Mojžíše a Árona, kteří se s nimi chtěli setkat.
\textsuperscript{21}Vyčítali jim: „Ať se nad vámi ukáže Hospodin a rozsoudí. Vy jste pokáleli naši pověst u faraóna a jeho služebníků. Dali jste jim do ruky meč, aby nás povraždili.“
}

\subsection*{Reflexe}

Izraelci si právem stěžují na to, že by měli vyrábět stejné množství cihel, i když dostávají méně slámy. Požadavky
otrokářů se staly mnohem nedosažitelnějšími. V jejich případě je otrokářem faraon, zatímco v našem případě je
otrokářem Satan. Naše závislosti a zvyky v nás vytvářejí neuhasitelnou touhu. Jak však ukazuje dnešní čtení,
poddávat se těmto našim pokušením je hluboce neuspokojivé.

Bez ohledu na to, k čemu jsme zotročeni, jsme vždy v pokušení usilovat o rychlou a snadnou nápravu tím, že se
vzdáme naší touhy. Ironické je, že čím více se snažíme se jí vzdát, tím méně to je možné. Pokušení vytvářejí iluzi,
že budeme více naplněni, když se příště jen trochu víc dopřejeme. Po čase nás pokušení vtáhne hlouběji do otroctví
a dál od svobody, takže je pro nás těžší překonat. Každý z nás tak moc dobře ví, jak pravdivě slova svatého Pavla v
jeho dopise Římanům znějí: „Nepoznávám se ve svých skutcích; vždyť nedělám to, co chci, nýbrž to, co
nenávidím,“ (Řím 7,15).

Nikdy nezapomeňte na tuto lež, že k vám neustále mluví vaše pokušení. Udržujte ji v čele své mysli jako hnací sílu
vedoucí ke svobodě. Je to frustrace z této lži, která vás sem přivedla, a bude to frustrace z této lži, díky které budete
ochotni jít do pouště, ochotni následovat tento náročný plán svobody.

%newday
\newpage
\section{Den 13 - VHLED DÍKY SLABOSTI}
\zacatekDruhyTyden
\subsection*{Čtení na den}
\textbf{Exodus 5,22-6,12}
\newline
\textit{
\textsuperscript{22}Mojžíš se obrátil k Hospodinu a řekl: „Panovníku, proč jsi dopustil na tento lid zlo? Proč jsi mě vlastně poslal?
\textsuperscript{23}Od chvíle, kdy jsem předstoupil před faraóna, abych mluvil tvým jménem, nakládá s tímto lidem ještě hůře. A ty svůj lid stále nevysvobozuješ.“
\textsuperscript{1}Hospodin Mojžíšovi odvětil: „Nyní uvidíš, co faraónovi udělám. Donutím ho , aby je propustil; donutím ho , aby je vypudil ze své země.“
\textsuperscript{2}Bůh promluvil k Mojžíšovi a ujistil ho: „Já jsem Hospodin.
\textsuperscript{3}Ukázal jsem se Abrahamovi, Izákovi a Jákobovi jako Bůh všemohoucí. Ale své jméno Hospodin jsem jim nedal poznat.
\textsuperscript{4}Ustavil jsem s nimi také svou smlouvu, že jim dám kenaanskou zemi, zemi jejich putování, kde pobývali jako hosté.
\textsuperscript{5}Rovněž jsem uslyšel sténání Izraelců, které si Egypťané podrobili v otroctví, a rozpomenul jsem se na svou smlouvu.
\textsuperscript{6}Proto řekni Izraelcům: Já jsem Hospodin. Vyvedu vás z egyptské roboty, vysvobodím vás z vašeho otroctví a vykoupím vás vztaženou paží a velkými soudy.
\textsuperscript{7}Vezmu si vás za lid a budu vám Bohem. Poznáte, že já jsem Hospodin, váš Bůh, který vás vyvede z egyptské roboty.
\textsuperscript{8}Dovedu vás do země, kterou jsem přísežně slíbil dát Abrahamovi, Izákovi a Jákobovi. Vám ji dám do vlastnictví. Já jsem Hospodin.“
\textsuperscript{9}Mojžíš to tak Izraelcům vyhlásil, ale ti nebyli pro malomyslnost a tvrdou otročinu s to Mojžíšovi naslouchat.
\textsuperscript{10}Hospodin dále mluvil k Mojžíšovi:
\textsuperscript{11}„Předstup před faraóna, krále egyptského, a vyřiď mu , ať propustí Izraelce ze své země.“
\textsuperscript{12}Mojžíš Hospodinu namítl: „Když mi nenaslouchají Izraelci, jak by mě poslechl farao! Nejsem způsobilý mluvit.“
}

\subsection*{Reflexe}

Často je „Bůh Starého zákona“ obviněn z tvrdosti a krutosti, a dnešní Písmo zdálky potvrzuje tento stereotyp. Ale
podívej se znovu. Kdyby se Bůh unáhlil a vyřešil problémy, kterým čelí Izraelité, čeho by dosáhl? Izraelci by si
povzdechli úlevou a rychle zapomněli na své nesnáze a na to, kdo jim dává svobodu. Nenaučili by se, že Bůh jejich
otců je jejich Bůh. Nenaučili by se nic, a nic by nezískali. Izraelitům je v tíživé situaci dána možnost naučit se něco
velmi cenného: Bůh je vezme za svůj lid a bude jejich Bohem. Povede je ke svobodě a ukončí jejich zajetí.
Bezpochyby uvidí, že Bůh je Bůh a že bez Něj nemohou nic dělat.

Chvíli přemýšlejte o tom, co by byl člověk bez slabostí. Pravděpodobně by byl pyšnou a povýšenou šelmou, která si
myslí, že nepotřebuje Boha, a věří, že se dokáže vysvobodit jak v tomto světě, tak i v příštím. Naše slabost nám tak
poskytuje vhled. Pokud svou slabost necháte, naučí vás, že potřebujete Boha. Slabost vás může naučit obrátit se
k Bohu a požádat ho o svobodu, místo toho abyste věci brali sami do svých rukou.

Vaše slabost vás sem přivedla. Vaše slabost (ne váš hřích) je Božím darem pro vás. Pokud dbáte na vhled, který vám
dává (potřeba Boha), zůstanete na cestě ke svobodě. Poděkujte dnes Pánu za dar své slabosti, který vás neustále
přivádí zpět do Jeho milujícího náručí.

%newday
\newpage
\section{Den 14 - VLIV OTCE}
\zacatekDruhyTyden
\subsection*{Čtení na den}
\textbf{Exodus 6,13-27}
\newline
\textit{
\textsuperscript{13}Ale Hospodin Mojžíšovi a Áronovi domluvil a dal jim příkazy pro Izraelce i pro faraóna, krále egyptského, aby připravili odchod Izraelců z egyptské země.
\textsuperscript{14}Toto jsou představitelé otcovských rodů: Rúbenovci, potomci Izraelova prvorozeného: Chanók a Palú, Chesrón a Karmí. To jsou čeledi Rúbenovy.
\textsuperscript{15}Šimeónovci: Jemúel, Jamín, Ohad, Jakín, Sóchar a Šaul, syn Kenaanky. To jsou čeledi Šimeónovy.
\textsuperscript{16}Toto jsou jména Léviovců podle jejich rodopisu: Geršón, Kehat a Merarí. Lévi byl živ sto třicet sedm let.
\textsuperscript{17}Geršónovci: Libní a Šimeí podle svých čeledí.
\textsuperscript{18}Kehatovci: Amrám, Jishár, Chebrón a Uzíel. Kehat byl živ sto třicet tři léta.
\textsuperscript{19}Meraríovci: Machlí a Muší. To jsou lévijské čeledi podle jejich rodopisu.
\textsuperscript{20}Amrám si vzal za ženu Jókebedu, svou tetu. Ta mu porodila Árona a Mojžíše. Amrám byl živ sto třicet sedm let.
\textsuperscript{21}Synové Jishárovi: Kórach, Nefeg a Zikrí.
\textsuperscript{22}Synové Uzíelovi: Míšael, Elsáfan a Sitrí.
\textsuperscript{23}Áron si vzal za ženu Elíšebu, dceru Amínadabovu, sestru Nachšónovu. Ta mu porodila Nádaba, Abíhúa, Eleazara a Ítamara.
\textsuperscript{24}Synové Kórachovi: Asír, Elkána a Abíasaf. To jsou kórachovské čeledi.
\textsuperscript{25}Eleazar, syn Áronův, si vzal za ženu jednu z dcer Pútíelových. Ta mu porodila Pinchasa. To jsou představitelé lévijských rodů podle svých čeledí.
\textsuperscript{26}Z tohoto pokolení pocházejí ten Áron a Mojžíš, k nimž mluvil Hospodin: „Vyveďte z egyptské země Izraelce seřazené po oddílech.“
\textsuperscript{27}Oni to byli, kdo mluvili k faraónovi, králi egyptskému, že mají vyvést Izraelce z Egypta. To tedy byli Mojžíš a Áron.
}

\subsection*{Reflexe}

Při zdlouhavých biblických rodokmenech většinou ztrácíme chuť číst dál. Jména jsou cizí a význam rodinných linií je dávno
ztracen v historii. Všechno, co je obsaženo v Písmu Svatém, však ukazuje církvi a věřícím důležité pravdy. Rodokmeny nás
konkrétně spojují se sliby a smlouvami, které učinil Bůh svým lidem. Také nám připomínají – jak připomínali předkům – naši
důstojnost a náš budoucí domov.

Vaše vlastní rodinná linie vás naučí mnoho o vás samých. Síla charakteru, temperament, osobnost, vaše osoba je určena vašimi
předky. Přesněji, navzdory tomu, co nám říká naše kultura, se můžete dozvědět mnoho o sobě a o svém životě od svého otce.

Otcové, možná více, než víme, mají významný dopad na jejich potomstvo – někdy pozitivní, někdy negativní. Převážně od
našich otců se učíme sebeovládání, sebedůvěry, způsobu, jakým komunikujeme s vnějším světem. Pokud bojujete v některé z
těchto oblastí, znamená to, že vás otec selhal? To je zásadní otázka, které je každý člověk vystaven.

Tohle není místo pro vyčerpávající pojednání o otcovství. Když však pracujete, abyste si lépe porozuměli a usilovali o svobodu
sebeovládání, budete pravděpodobně považovat za přínosné přemýšlet o svém otci a vašem vztahu s ním. Vyzkoušejte toto
cvičení: pokud je to možné, požádejte svého otce, aby popsal svého vlastního otce (vašeho dědečka), a zjistíte, odkud pocházelo
mnoho rysů vašeho otce (pozitivních i negativních). Můžete také získat představu o tom, co předáváte (nebo co budete předávat)
svým vlastním dětem. Vnímejte toto cvičení pozitivně a neuchylujte se k obviňování, jde tu o pochopení. Pamatujte, že jste na
cestě k rozvoji ctnostných zvyků uprostřed bratrství. 

Tyto návyky budou darem pro vaše blízké, zejména pro vaše děti, bez
ohledu na to, jak staré jsou. Nemůžete změnit minulost, ale nyní máte příležitost pracovat s Pánem, abyste pro svou rodinu
vytvořili novou budoucnost.


% ===============================================
% ===== TRETI TYDEN
% ===============================================
%ukony
\newpage
\section*{Úkony (ukazatel cesty) pro 3. týden}

\textbf{Místo:} Egypt (jste v Egyptě)

Život se stal náročnějším. Pro Izraelity práce nepovolila. Faraon je odmítá nechat jít. I přes prvních pět ran se Mojžíši a Áronovi nedaří získat větší svobodu Izraelitům. První nadšení z bratrského exodu pro vás také možná již vyprchalo. Začínáte si uvědomovat tvrdou realitu mnoha týdnů před vámi. A co hůře, stejně jako Izraelitům, ani pro vás ještě neuběhlo tolik času na to, abyste začali pociťovat zisk svobody. Navzdory tomu všemu, komu se tento týden rozhodnete sloužit: Bohu, nebo faraonovi?

\subsection*{1. Pokračujte ve zkoumání svého dne}
Nejen, že vám pomůže projít skrz těchto 90 dní, ale udrží vás to svobodnými v 91. dnu. Dobře praktikujte noční examen, takže si vytvoříte dobrý zvyk pro 91. den. (Nevíte, jak dělat zkoumání na konci dne? Podívejte se do kapitoly Jak se modlit noční examen v Průvodci terénem.)
\subsection*{2. Přistupujte upřímně k vaší denní svaté hodině}
Čas v kontemplativní modlitbě je rozhodující při exodu. Bůh dává vysvobození. Bůh vás povede, kam potřebujete. Bůh vám bude připomínat svou lásku k vám a vaší pravou hodnotu jako Jeho prvorozeného syna. Ale neuslyšíte žádnou z těchto věcí, pokud si každý den nevyhradíte čas na ztišení své mysli a naslouchání.
\subsection*{3. Neodbývejte úkony/úkoly (don’t cut corners)}
Touto dobou už jste se dobře seznámili s asketickými disciplínami. Víte, co je na nich těžké a co jednoduché. Čím více disciplíny znáte, tím jednodušší bude je odbývat (cut corners-zkracovat si cestu). Chcete zůstat v Egyptě, nebo chcete být osvobozeni? (Pro více informací o důležitosti asketismu nahlédněte do Pilířů Exodu 90 v příručce pod pilířem Bratrství v příručce Exodu.)
\subsection*{4. Pamatujte si své \textit{proč}.}
Vzpomeňte si, proč jste začali Exodus 90. Pokud své proč zapomenete, pravděpodobně nebudete schopni/moci tuto cestu dokončit. Lákavost pohodlí je jednoduše příliš silná, než abyste ji překonali bez proč, které za to stojí. (Nezapsali jste si své proč? Vraťte se zpět do části příručky Exodu 90 nazvané Co je vaše proč a napište si své proč předtím, než budete pokračovat dál.)
\subsection*{5. Zůstaňte radostní}
Pokud budete držet své proč v čele své mysli a zůstanete pevní v modlitbě, vaše naděje na svobodu nebude jiná než vysoká. Jít exodem je náročné. Naštěstí je to Bůh, kdo dává vysvobození, které hledáme. Egyptské rány nám ukazují, že také Bůh pracuje na vaší svobodě.

\subsection*{Modlitba}
Modlete se, aby Pán osvobodil vás a vaše bratrství \newline
Modleme se za svobodu všech mužů v exodu, stejně tak, jako se oni modlí za vás.\newline
Ve jménu Otce i Syna i Ducha svatého … Otče náš… Ve jménu Otce i Syna i Ducha svatého … Amen.

\newpage


%newday
\newpage
\section{Den 15 - BŮH SI PŘEJE VAŠI SVOBODU}
\zacatekTretiTyden
\subsection*{Čtení na den}
\textbf{Exodus 6,28-7,7}
\newline
\textit{
\textsuperscript{28}To bylo tehdy, když Hospodin mluvil k Mojžíšovi v egyptské zemi.
\textsuperscript{29}Hospodin promluvil k Mojžíšovi: „Já jsem Hospodin! Řekni faraónovi, králi egyptskému, všechno, co k tobě mluvím.“
\textsuperscript{30}Mojžíš však Hospodinu namítl: „Nejsem způsobilý mluvit. Jak by mě farao poslechl?“
\textsuperscript{1}Hospodin řekl Mojžíšovi: „Pohleď, ustanovil jsem tě, abys byl pro faraóna Bohem, a Áron, tvůj bratr, bude tvým prorokem.
\textsuperscript{2}Ty mu povíš všechno, co ti přikážu, a Áron, tvůj bratr, bude mluvit s faraónem, aby propustil Izraelce ze své země.
\textsuperscript{3}Já však zatvrdím faraónovo srdce a učiním v egyptské zemi mnoho svých znamení a zázraků.
\textsuperscript{4}Farao vás neposlechne, ale já vložím na Egypt svou ruku. Vyvedu zástupy svého lidu, syny Izraele, z egyptské země, ale ji postihnu velkými soudy.
\textsuperscript{5}Egypťané poznají, že já jsem Hospodin, až vztáhnu svou ruku na Egypt a vyvedu Izraelce z jejich středu.“
\textsuperscript{6}Mojžíš a Áron učinili přesně tak, jak jim Hospodin přikázal.
\textsuperscript{7}Mojžíšovi bylo osmdesát let a Áronovi osmdesát tři léta, když mluvili s faraónem.
}

\subsection*{Reflexe}
Podívejme se na věc z této perspektivy – Bůh žádá Mojžíše, aby čelil nejmocnějšímu vládci světa v samém srdci
jeho království, kde je obklopen svými lidmi. Zde má Mojžíš říci faraonovi, co má dělat. Asi není divu, že se
Mojžíš zachová jako většina lidí, když čelí znepokojivému úkolu: couvá zpět. Mojžíš vysvětluje Bohu, že je vázán
jazykem a není způsobilý mluvit. Bůh však jeho lidskou omluvu nepřijímá.

Místo toho Bůh, vládce veškerého stvoření, říká slabému Mojžíšovi: „Ustanovil jsem tě, abys byl pro faraóna
Bohem.“ Svatý Ambrož nám říká, že Mojžíšova ctnost daleko převyšuje faraonovu moc. Mojžíšovi není dáno
nadmíru. Jeho vášně nad ním nevládnou. Je to muž, který „ostře kritizuje, že jeho tělo bylo ztělesněno autoritou,
která byla téměř královská“.Zatímco Mojžíšova sebedůvěra ochabuje, Hospodin má ve svého syna veškerou
důvěru.

Tak je to i s vámi. Vy, stejně jako Mojžíš, ovládáte své vášně a trpíte bolestmi askeze, ztělesňujete Božího syna,
královského a mocného oproti směšnému faraonovi světskosti a neřestí. Mohl jste to o sobě říci před dvaceti dny?
Způsob vašeho života se mění. Následujte Kristův plán modlitby, askeze a bratrství a budete i nadále dostávat
milost vítězit nad svými vášněmi.

Děkujte Bohu za úspěch, který jste doposud měli. Nepřetržitá vděčnost bude podporovat vaši vděčnost uprostřed
výzev tohoto duchovního cvičení.

%newday
\newpage
\section{Den 16 - SPÁSA SKRZE KŘÍŽ}
\zacatekTretiTyden
\subsection*{Čtení na den}
\textbf{Exodus 7,8-13}
\newline
\textit{
\textsuperscript{8}Hospodin dále řekl Mojžíšovi a Áronovi:
\textsuperscript{9}„Až k vám farao promluví: ‚Prokažte se nějakým zázrakem,‘ řekneš Áronovi: ‚Vezmi svou hůl a hoď ji před faraóna,‘ a stane se drakem.“
\textsuperscript{10}Mojžíš s Áronem tedy předstoupili před faraóna a učinili, jak Hospodin přikázal. Áron hodil svou hůl před faraóna i před jeho služebníky a ona se stala drakem.
\textsuperscript{11}Farao však také povolal mudrce a čaroděje, a egyptští věštci učinili svými kejklemi totéž.
\textsuperscript{12}Hodili každý svou hůl na zem a ony se staly draky. Ale Áronova hůl jejich hole pohltila.
\textsuperscript{13}Srdce faraónovo se však zatvrdilo a neposlechl je, jak Hospodin předpověděl.
}

\subsection*{Reflexe}

V dnešním čtení faraon přikazuje svým věštcům, aby zopakovali znamení, které provedl Áron.
Dosvědčuje se, že je to pouhým pokusem zmást lidi a setřít moc Božího znamení. Faraon se sanží
nabídnout alternativy k Boží cestě. V dnešní době nalézáme moderní „faraony“ všude, snažící se o to
samé.

V Písmu Svatém Áronova hůl předznamenává kříž. Jedním z příkladů je použití Áronovy hole k rozdělení
Rudého moře umožňující přechod Izraelitů ke svobodě. Tento čin předznamenává křest, kde voda, skrze
kterou přecházíme ke svobodě, dostává svou moc z kříže. Neexistuje žádná jiná cesta ke spáse než skrze
kříž, ale stejně jako faraon, i naše kultura zoufale hledá alternativu – něco jednoduššího, něco
příjemnějšího, něco, co člověk dokáže ovládat.

Kolikrát jste hledali jinou cestu ke svobodě, než jste nakonec museli přijmout, že žádná „jiná cesta“
neexistuje?„Jako Mojžíš vyvýšil hada na poušti, tak musí být vyvýšen Syn člověka,“ (J 3,14). Podívejte
se na plán před vámi: modlitba, askeze, bratrství. Nemělo by být žádným překvapením, že cesta ke
svobodě, kterou lidem poskytuje Písmo, je cestou kříže.

Pokud dokážete unést odpověď našeho Pána, podívejte se dnes na kříž a zeptejte se ho, zda existuje jiný
způsob než právě on.

%newday
\newpage
\section{Den 17 - HOJNĚ NAPLNĚNÝ}
\zacatekTretiTyden
\subsection*{Čtení na den}
\textbf{Exodus 7,14-24}
\newline
\textit{ 
\textsuperscript{14}Hospodin řekl Mojžíšovi: „Srdce faraónovo je neoblomné. Nechce lid propustit.
\textsuperscript{15}Jdi k faraónovi ráno. Až půjde k vodě, postav se naproti němu na břehu Nilu a vezmi si do ruky hůl, která se proměnila v hada.
\textsuperscript{16}Řekneš mu: Hospodin, Bůh Hebrejů, mě k tobě posílá se vzkazem: Propusť můj lid, aby mi na poušti sloužil. Ale ty jsi dosud neposlechl.
\textsuperscript{17}Toto praví Hospodin: Podle toho poznáš, že já jsem Hospodin: Holí, kterou mám v ruce, teď udeřím do vody v Nilu, a ta se promění v krev.
\textsuperscript{18}Ryby, které jsou v Nilu, leknou a Nil bude páchnout. Marně budou Egypťané usilovat, aby se mohli napít vody z Nilu.“
\textsuperscript{19}Hospodin dále řekl Mojžíšovi: „Vyzvi Árona: ‚Vezmi svou hůl a vztáhni ruku nad egyptské vody, nad průplavy, nad říční ramena, nad jezera, vůbec nad všechny nahromaděné vody.‘ Stanou se krví. V celé egyptské zemi bude krev, i ve džberech a džbánech.“
\textsuperscript{20}Mojžíš a Áron učinili, jak Hospodin přikázal. Áron pozdvihl hůl a před očima faraóna a jeho služebníků udeřil do vody v Nilu a všechna voda Nilu se proměnila v krev.
\textsuperscript{21}Ryby v Nilu lekly, Nil začal páchnout a Egypťané nemohli vodu z Nilu pít. A krev byla v celé egyptské zemi.
\textsuperscript{22}Ale totéž učinili egyptští věštci svými kejklemi. Faraónovo srdce se zatvrdilo a neposlechl je, jak Hospodin předpověděl.
\textsuperscript{23}Farao se obrátil a vešel do svého domu, a ani toto si nevzal k srdci.
\textsuperscript{24}Všichni Egypťané kopali kolem Nilu, aby přišli na pitnou vodu, protože vodu z Nilu pít nemohli.
}

\subsection*{Reflexe}

Bůh zjevuje všechny věci takové, jaké skutečně jsou. V této pasáži Bůh začíná řadu deseti ran, aby ukázal svou
moc nad faraonem a falešnými bohy Egypta. Jeho moc se ukazuje tak mocně, že se všechny vody Egypta
proměňují v krev, takže je Nil tak zkažený, že lidé nemají vodu k pití. Krev je zde symbolem tělesné existence –
hmoty lidské přirozenosti tohoto světa. Voda, zdroj lidského života, se stává zdrojem smrti a rozpadu.

V hebrejštině tento text naznačuje, že hrnce a nádoby používané pro vodu (a nyní naplněné krví) byly vyrobeny z
materiálu získaného ze stromů a kamenů, což byly shodou okolností stejné materiály, které Egypťané používali při
stavbě svých model. Jinými slovy, falešní bohové Egypťanů se nyní stali pro ně zdrojem smrti; nemohli je totiž
zachránit.

I dnes jste neustále v pokušení obrátit se k věcem těla –k sexu, moci, penězům– které vás odtrhnou od pravého
života a svobody. Nádoba vašeho života je často naplněna smrtí a úpadkem, spíše než životem. Přesto pohleďte na
jiný úryvek Písma uvedený v evangeliu sv. Jana (2. kapitola). Zde Kristus promění vodu, která je v nádobách k
čištění, na víno, v symbol života a radosti a znamení nového života, který nám nabízí v Duchu. Buďte pevní a
vězte, že pokud mu dovolíte, Ježíš také obrátí vodu vašeho života na víno. Přijďte dnes k Pánu. Požádejte ho, aby
vás tak hojně naplnil životem a radostí, že všichni lidé kolem vás budou moct vidět a vědět, co ve vás On činí.

%newday
\newpage
\section{Den 18 - ŽIJETE PRO OSTATNÍ?}
\zacatekTretiTyden
\subsection*{Čtení na den}
\textbf{Exodus 6,13-27}
\newline
\textit{ 
\textsuperscript{25}To trvalo plných sedm dní poté, co Hospodin zasáhl Nil.
\textsuperscript{26}Potom Hospodin řekl Mojžíšovi: „Předstup před faraóna a řekni mu: Toto praví Hospodin: Propusť můj lid, aby mi sloužil.
\textsuperscript{27}Budeš-li se zdráhat jej propustit, napadnu celé tvé území žábami.
\textsuperscript{28}Nil se bude žábami hemžit, vylezou a vniknou do tvého domu, do tvé ložnice a na tvé lože i do domu tvých služebníků a mezi tvůj lid, do tvých pecí a díží.
\textsuperscript{29}I po tobě, po tvém lidu a po všech tvých služebnících polezou žáby.“
\textsuperscript{1}Hospodin dále řekl Mojžíšovi: „Vyzvi Árona: ‚Vztáhni ruku se svou holí nad průplavy, nad říční ramena i nad jezera a vyveď na egyptskou zemi žáby.‘“
\textsuperscript{2}Áron vztáhl ruku nad egyptské vody a žáby vylézaly, až pokryly egyptskou zemi.
\textsuperscript{3}Ale totéž učinili věštci svými kejklemi a i oni vyvedli na egyptskou zemi žáby.
\textsuperscript{4}Tu povolal farao Mojžíše a Árona a řekl: „Proste Hospodina, aby mě i můj lid zbavil žab. Pak propustím lid, aby obětoval Hospodinu.“
\textsuperscript{5}Mojžíš faraónovi odvětil: „Rač mi sdělit , kdy mám prosit za tebe, za tvé služebníky a za tvůj lid, aby Hospodin vyhladil žáby u tebe i v tvých domech. Zůstanou jen v Nilu.“
\textsuperscript{6}Farao odpověděl: „Zítra.“ Mojžíš řekl: „ Ať je podle tvého slova, abys poznal, že nikdo není jako Hospodin, náš Bůh.
\textsuperscript{7}Žáby se stáhnou od tebe i z tvých domů, od tvých služebníků a od tvého lidu. Zůstanou jen v Nilu.“
\textsuperscript{8}Nato odešel Mojžíš s Áronem od faraóna a Mojžíš úpěnlivě volal k Hospodinu kvůli žábám, kterými faraóna postihl.
\textsuperscript{9}Hospodin učinil podle Mojžíšovy prosby a žáby v domech, ve dvorcích i na polích pošly.
\textsuperscript{10}Shrabali je na hromady a kupy a zápach z nich naplnil zemi.
\textsuperscript{11}Když však farao viděl, že nastala úleva, zůstal v srdci neoblomný a neposlechl je, jak Hospodin předpověděl.
\textsuperscript{12}Hospodin řekl Mojžíšovi: „Vyzvi Árona: ‚Vztáhni svou hůl a udeř do prachu na zemi!‘ Stanou se z něho po celé egyptské zemi komáři.“
\textsuperscript{13}I učinili tak. Áron vztáhl ruku s holí a udeřil do prachu na zemi a na lidech i na dobytku se objevili komáři. Po celé egyptské zemi se ze všeho prachu země stali komáři.
\textsuperscript{14}Když totéž chtěli učinit věštci svými kejklemi, totiž vyvést komáry, nemohli. A komáři byli na lidech i na dobytku.
\textsuperscript{15}Věštci tedy řekli faraónovi: „Je to prst Boží.“ Srdce faraónovo se však zatvrdilo a neposlechl je, jak Hospodin předpověděl.
}

\subsection*{Reflexe}

Pokud nejste osmiletý chlapec, vyhlídka na nesčetné kvákající žáby ohromující zemi je v nejlepším případě
nepříjemná. Tato slizká scéna odhaluje něco o faraonovi jako člověku. Když mu Mojžíš nabídne úlevu od
obojživelníků, Faraon souhlasí s tím, že Mojžíš by měl zasáhnout a přimluvit se u Boha, ale ne teď – „zítra“. Jeho
arogantní lhostejnost k situaci jeho poddaných odhaluje tvrdost jeho srdce a jeho narcismus. Odmítá pokleknout
před Bohem bohů a být viděn s Ním jakýmkoliv způsobemspolupracovat. Kromě toho, když je tato rána stažena,
faraon nevykazuje žádnou vděčnost Mojžíšovi a ve své nafouklé hrdosti se vůbec nezajímá o Boha.

Jednou z nejzávažnějších chyb, které děláme, je,že nebereme v úvahu nikoho jiného než sebe samého. Příliš snadno
(i když jen obrazně) přehlížíme Boha vesmíru, naše ženy, farníky, syny a dcery. Toto nerespektování a přehlížení
ostatních je opakem toho, co to znamená být mužem. Kromě toho je tato naše nevšímavost Kristova těla – což je
také přehlížení Boha samotného– jistou cestou do pekla.

Chcete dosáhnout svého skutečného potenciálu a používat svou moc od Boha, jakou vám dal při vašem stvoření?
Pak musíte myslet na ostatní. Musítesesadit sami sebe z místa Boha a pozvednout ty kolem sebe. Tak půjdete
svatou cestou. Jak teď Hospodin vidí vaše činy? Nejste si jistí? Zeptejte se ho.

%newday
\newpage
\section{Den 19 - BŮH DÁVÁ VŠECHNY POTŘEBNÉ MILOSTI}
\zacatekTretiTyden
\subsection*{Čtení na den}
\textbf{Exodus 8,16-19}
\newline
\textit{
\textsuperscript{16}Hospodin řekl Mojžíšovi: „Za časného jitra se postav před faraóna, až vyjde k vodě. Řekneš mu: Toto praví Hospodin: Propusť můj lid, aby mi sloužil!
\textsuperscript{17}Jestliže můj lid nepropustíš, pošlu na tebe, na tvé služebníky, na tvůj lid i na tvé domy mouchy. Domy Egypťanů budou plné much, i ta půda, na které žijí .
\textsuperscript{18}Ale zemi Gošen, kde se zdržuje můj lid, v onen den podivuhodně odliším. Tam mouchy nebudou, abys poznal, že já jsem Hospodin i uprostřed této země.
\textsuperscript{19}Učiním rozdíl mezi lidem svým a lidem tvým. Toto znamení se stane zítra.“
}

\subsection*{Reflexe}

Říká se, že „Bůh nám nikdy nedává víc, než můžeme zvládnout.“ To je daleko od pravdy. Podívej se na Mojžíše.
Bůh mu dal mnohem víc, než by mohl zvládnout. Kvůli tomu má Mojžíš dvě možnosti: může utéct a snažit se utěšit
sebe sama, s vědomím, že se Bůh zeptá jiných, nebo může věřit, že Bůh zasáhne a poskytne mu milost tam, kde
jeho lidská síla nestačí.
Snažit se být dobrým křesťanem může být v dnešní kultuře velmi náročné. Mnozí z nás jsou v pokušení uvěřit lži,
že kdybychom nikdy nepoznali Boha a jeho Církev, byli bychom mnohem šťastnější, protože bychom nemuseli
snášet život mnoha křížů. Myšlenka, že naše kříže nás zotročují, je ale daleko od pravdy.

\begin{minipage}{\dimexpr\textwidth-20pt}
  Dnes se upokojte v těchto slovech svatého Františka Saleského:
\begin{quote}
  \textit{Zapamatujte si tuto jednoduchou pravdu, která je nade všechny pochybnosti: Bůh dovoluje, aby mnoho obtíží trápilo ty, kteří mu chtějí sloužit, ale nikdy je nenechá padnout pod tíhou těchto obtíží, pokud Mu stále důvěřují… Nikdy, za žádných okolností nepodléhejte pokušení znechucení, ani pod lehce uvěřitelnou záminkou pokory.}
\end{quote}
\end{minipage}


Pokud toužíte sloužit Bohu, bude od vás požadovat víc, než dokážete zvládnout. V tomto bodě duchovního cvičení
jsme si této skutečnosti dobře vědomi. Ale Bůh nikdy nedopustí, abyste se potopili pod tíhou svých břemen, dokud
Mu důvěřujete. Znovu se podívejte na své proč. Věříte, že to Bůh dokončí? Pokud ne, teď je vhodný čas o tom
s Ním mluvit. Pokud ano, věnujte čas tomu, abyste Bohu vyjádřili svou vděčnost.

%newday
\newpage
\section{Den 20 - SLUŽTE HOSPODINU}
\zacatekTretiTyden
\subsection*{Čtení na den}
\textbf{Exodus 8,20-28}
\newline
\textit{
\textsuperscript{20}A Hospodin tak učinil. Dotěrné mouchy vnikly do domu faraónova, do domu jeho služebníků a na celou egyptskou zemi. Země byla těmi mouchami zamořena.
\textsuperscript{21}Tu povolal farao Mojžíše a Árona a řekl: „Nuže, přineste oběť svému Bohu zde v zemi.“
\textsuperscript{22}Mojžíš odpověděl: „Nebylo by správné, abychom to učinili. To, co máme obětovat Hospodinu, svému Bohu, je Egypťanům ohavností. Copak by nás neukamenovali, kdybychom před nimi obětovali, co je jim ohavností?
\textsuperscript{23}Odejdeme do pouště na vzdálenost tří dnů cesty a tam budeme obětovat Hospodinu, svému Bohu, jak nám nařídil.“
\textsuperscript{24}Farao řekl: „Propustím vás tedy, abyste obětovali Hospodinu, svému Bohu, na poušti. Jenom neodcházejte příliš daleko. Proste za mne.“
\textsuperscript{25}Mojžíš odvětil: „Až od tebe odejdu, budu prosit Hospodina a zítra odletí mouchy od faraóna, od jeho služebníků i od jeho lidu. Jen ať nás opět farao neobelstí, že by nechtěl propustit lid, aby obětoval Hospodinu.“
\textsuperscript{26}Pak Mojžíš od faraóna odešel a prosil Hospodina.
\textsuperscript{27}A Hospodin učinil, jak Mojžíš řekl. Mouchy odletěly od faraóna, od jeho služebníků i od jeho lidu. Ani jediná nezůstala.
\textsuperscript{28}Ale farao zůstal v srdci neoblomný i tentokrát a lid nepropustil.
}

\subsection*{Reflexe}
Nyní jsme dospěli ke skutečného smyslu a účelu celé ságy zaznamenané v knize Exodus. Bůh po
faraonovi požaduje: „Propusť můj lid, aby mi mohl sloužit.“ Mojžíš má v úmyslu, na základě Božího
příkazu, vzít lid na třídenní cestu do pouště, kde přinesou oběti, které jsoupro Egypťany „ohavností“.
Jinými slovy vezmou zvířata, která Egypťané uctívají, a povraždí je. Tato oběť má Izraelcům dokázat, že
bohové Egypťanů jsou pouhými tvory, a ne jediným pravým Bohem.

Možná, že my, moderní muži, jsme příliš vzdělaní na to, abychom uctívali ovce a dobytek. V naší
aroganci a kultivovanosti však stále uctíváme falešné bohy či modly. Nejčastěji jsou to modly ve formě
peněz, sexu, moci, sportu a zábavy. Pokud čteme Písmo důkladně, musíme vidět, že Bůh se nejvíce stará
o Izraelity a jejich svobodu. Jeho hlavním zájmem není svoboda od otrokářů, kteří nařizují, co mají celý
den dělat. Jeho zájmem je spíše svoboda jejich duší. Zotročení si však přece nezaslouží zatracení. Ale
svobodné rozhodnutí uctívat modly namísto Boha? To už je vážné. Po 400 letech v Egyptě Izraelité
uctívají egyptské modly. Toto uctívání zotročilo jejich duše a bránilo jim od správného uctívání jediného
pravého Boha.

Povšimněte si také, že i když se faraon unavuje a dává jim povolení jít do divočiny, nechce, aby Izraelité
šli „moc daleko“. Nedovolí, aby se mu jeho pracovní síla vymkla z rukou. Musí zůstat zotročení.
Přemýšlejte o tom, kdy se vzdáváte věcí, které vás zotročují. Satan vám šeptá: „Běžte a dejte si dovolenou
od vašeho otroctví třeba na týden, po dobu postní, nebo dokonce na devadesát dní. Když se pak vrátíte
zpět, otroctví bude pro vás ještě horší.“ Svobodu nelze vyhrát a navždy ochránit ve stanoveném čase.

Těchto devadesát dní má sloužit jako skvělý začátek, čas očištění, a jako připomínka. Potrvá vám však
celý život věrnosti a spoléhání se na Boha, abyste zůstali svobodným člověkem.
Tato skutečnost by vás neměla vést ke smutku nebo zoufalství. Měla by vám přinést větší horlivost hledat
celoživotní svobodu. Pokud jste odrazeni, přineste to Pánu a dejte mu prostor, aby k vám mluvil pravdu.

Pokud jste plní nadšení, chvalte Boha za tento dar. Obraťte se na své bratry, zejména na vaši kotvu. Ne
všichni muži budou tak nadšení a horliví jako vy. Někteří mohou být dokonce v pokušení přestat. Podělte
se s nimi o svou radost a zápal.

%newday
\newpage
\section{Den 21 - ODDĚLENI PRO BOHA}
\zacatekTretiTyden
\subsection*{Čtení na den}
\textbf{Exodus 9,1-7}
\newline
\textit{
\textsuperscript{1}Hospodin řekl Mojžíšovi: „Předstup před faraóna a promluv k němu: Toto praví Hospodin, Bůh Hebrejů: Propusť můj lid, aby mi sloužil!
\textsuperscript{2}Budeš-li se zdráhat jej propustit a zatvrdíš-li se proti nim ještě víc,
\textsuperscript{3}tu na tvá stáda, která jsou na poli, na koně, na osly, na velbloudy, na skot i na brav, dolehne Hospodinova ruka velmi těžkým morem.
\textsuperscript{4}Hospodin však bude podivuhodně rozlišovat mezi stády izraelskými a stády egyptskými, takže nezajde nic z toho, co patří Izraelcům.
\textsuperscript{5}Hospodin také určil lhůtu: Zítra toto učiní Hospodin v celé zemi.“
\textsuperscript{6}A nazítří to Hospodin učinil. Všechna egyptská stáda pošla, ale z izraelských stád nepošel jediný kus .
\textsuperscript{7}Farao si to dal zjistit, a vskutku z izraelských stád nepošel jediný kus ; přesto zůstalo srdce faraónovo neoblomné a lid nepropustil.
}

\subsection*{Reflexe}

Poselství poslané faraonovi je naprosto jasné: Bůh chce oddělit Izraelity od egyptského království. Izraelci nepatří
faraonovi, ale pouze Bohu. Bůh důrazně varuje faraóna, že bude„rozlišovat mezi stády izraelskými a stády
egyptskými, takže nezajde nic z toho, co patří Izraelcům“. Otec chrání své syny.

Stejně jako Izraelci byli odděleni, tak i vy jste byli odděleni svým křtem. Je snadné zapomenout na sílu a účinky
křtu. Když jste byly křtem ponořeni do Krista, buď jako dítě nebo jako dospělý, stali jste se syny nebeského Otce.
Stali jste se posvátnými a dostali jste všechno, co potřebujete k účasti na Božském životě.

Zkoumejte svůj život. Uvědomujete si, co to znamená, že je váš život posvátný? Kalich, který užívá kněz v
posvátné liturgii, nemůže být nikdy použit pro světskou činnost. Je posvátný a musí se s ním tak zacházet. Stejně
tak vás křest oddělil; zůstáváte uprostřed lidí, ale jste odnynějška odděleni pro službu Pánu. Váš úděl je být s
Bohem. Proto musíte odolat pokušení honbě za pozemskými věcmi, zvláště když vylučují Boží plán pro váš život.

Čím více se vyrovnáte se svou pravou identitou, tím více si uvědomujete iracionalitu honby za pozemskými věcmi,
jako je sex, moc, peníze. Proto musíme v našich srdcích často rozdmýchávat milost našeho křtu. Musíme klást
nárok na naše křestní právo a naše věčné dědictví. Musíme zůstat oddaní Bohu / Musíme zůstat
vyčleněni/odděleni/posvěceni pro Boha

Pohovořte s Bohem dnes o vaší podvátnosti a vašem oddání Jemu. Zamyslete se nad tím, jak žijete „odděleni pro
Boha“ a jak v tom selháváte. Buďte konkrétní, pozorně naslouchejte a buďte ochotni a připraveni se změnit, jak
Pán žádá. Bude to znamenat další pevný krok na cestě ke svobodě.

% ===============================================
% ===== CTVRTY TYDEN
% ===============================================
%ukony
\newpage
\section*{Úkony (ukazatel cesty) pro 4. týden}

\textbf{Místo:} Egypt (jste v Egyptě)

Egyptské rány se zhoršují. Hospodin nyní používá ty, kteří mu odporují, pro jeho vlastní užitek. Také
Izraelitům dokazuje prázdnotu egyptských model tím, že tyto modly zničí ranami. Izraelité konečně zří
pravou cenu opuštění Egypta a služby pravému Bohu. Jak putujete čtvrtým týdnem exodu, uvědomte si,
jaké falešné bohy ničí Bůh ve vašem životě skrze disciplíny tohoto duchovního cvičení. Dovolte této
skutečnosti, aby vám přinesla naději na svobodu a obrácení srdce, které potřebujete, abyste mohli nechat
Egypt za sebou a sloužit pouze Bohu.

\subsection*{1. Zavázejte se (odevzdejte se) svému bratrstvu}
Kolik bratrských setkání jste zameškali? Jste v pokušení přeskočit budoucí bratrskou svatou hodinu, mši nebo setkání? Jestli vás satan dokáže přesvědčit, že své bratry nepotřebujete, také ví, že vaše zotročení bude stejně lehké jako před exodem. Zavázejte se svému bratrstvu.
\subsection*{2. Vytvořte si cvičící plán}
Exodus 90 volá po pravidelném cvičení. Jak jste si vedli? Minulí muži Exodu zjistili, že pokud se nemohou držet vytvořeného plánu cvičení, je velmi těžké udělat si každý týden čas na cvičení. Vytvořte si plán, sepište si jej a dodržujte ho.
\subsection*{3. Uvědomte si Boží moc}
Jak se tento týden rány snáší na Egypt, uvědomte si Boží všemohoucnost. On má moc činit velké věci, včetně vysvobození z vašeho zajetí. Pamatujte: Jeho moc nezná hranic. Spoléhejte na Boha, On jediný vás dokáže vykoupit.
\subsection*{4. Jděte/vyjděte ven}
Ve světe plném obrazovek a digitálních vztahů lidé zapomněli, jak být spolu. Něktěří dokonce zapomněli, jak se říká kusu země pokryté stromy. Udělejte si tento nebo příští víkend čas na to, abyste společně vyrazili na výlet do lesa, údolí nebo pustiny. Žádné výmluvy: čím horší počasí, tím lepší pouto. Tento společný čas vám dá život a pravé pouto, které budete vy a vaše bratrstvo potřebovat pro mnoho dalších týdnů před vámi.

\subsection*{Modlitba}
Modlete se, aby Pán osvobodil vás a vaše bratrství \newline
Modleme se za svobodu všech mužů v exodu, stejně tak, jako se oni modlí za vás.\newline
Ve jménu Otce i Syna i Ducha svatého … Otče náš… Ve jménu Otce i Syna i Ducha svatého … Amen.

\newpage


%newday
\newpage
\section{Den 22 - ROZTÁT PŘED HOSPODINEM}
\zacatekCtvrtyTyden
\subsection*{Čtení na den}
\textbf{Exodus 9,8-12}
\newline
\textit{
\textsuperscript{8}Hospodin řekl Mojžíšovi a Áronovi: „Naberte si plné hrsti sazí z pece a Mojžíš ať je rozhazuje faraónovi před očima směrem k nebi.
\textsuperscript{9}Bude z nich po celé egyptské zemi poprašek, který způsobí na lidech i na dobytku po celé egyptské zemi vředy hnisavých neštovic.“
\textsuperscript{10}Nabrali tedy saze z pece, postavili se před faraóna a Mojžíš je rozhazoval směrem k nebi. Na lidech i na dobytku se objevily vředy hnisavých neštovic.
\textsuperscript{11}Ani věštci se nemohli postavit před Mojžíše pro vředy, neboť vředy byly na věštcích i na všech Egypťanech.
\textsuperscript{12}Hospodin však zatvrdil faraónovo srdce, takže je neposlechl, jak Hospodin Mojžíšovi předpověděl.
}

\subsection*{Reflexe}

Dnešní čtení představuje malý ale pozoruhodný obrat v začátku druhé řady pěti ran proti faraonovi a lidu Egypta.
Důsledky těchto ran se stanou mnohem intenzivními, kouzelníci si totiž uvědomí Boží moc a srdce faraona je
zatvrzeno uvědoměním si Boží všemohoucnosti.

Po každé z první řady ran zatvrdil faraon své srdce. V druhé řadě pěti ran je faraonovo srdce zatvrzeno Bohem. Bůh
faraonovi nedělá nic zlého, ani nepřipravuje faraonovi neúspěch. Abychom tomu porozuměli, představme si máslo
a hlínu na horkém slunci. Každá látka reaguje svým způsobem na to stejné slunce – máslo roztává a hlína tvrdne.
Slunce nedělá rozdíl nezi těmito dvěma látkami. Rozdíl je spíše v povaze každé látky, která způsobuje různou
reakci.

V knize Exodu Izraelité reagují na Boží skutky jako máslo: časem Jeho přítomnost rozpouští jejich srdce. V odezvě
se rozhodnou ctít Boží velikost. Na druhé straně se faraon dívá na stejné Boží skutky a reaguje jako hlína: Boží
přítomnost zatvrzuje jeho srdce. Faraon odpovídá pyšně a zvyšuje odpor vůči Boží vůli. Dělá to proto, že pokud je
Bůh opravdu tím, kým říká, pak je faraon podvodník. Jestli je Bůh Bohem, pak jím faraon není. Faraon může celou
svou identitu, a všechno s ní spojené, v tomto mocenském boji ztratit.

Když vstupujete každý den do své svaté hodiny, jak reagujete na Boží přítomnost? Je vaše srdce jako máslo, nebo
jako hlína? Jedním ze způsobů, jak to vyzkoušet, je zvážit vaši schopnost zůstat tichý v mysli i těle. Dokážete sedět
před všemohoucím Bohem a jen být v klidu? Nebo vás Jeho přítomnost znervózňuje?

Měli byste toužit slyšet vůli Pána a konat podle ní. Cítíte-li se ve stresu nebo v úzkosti, je to pravděpodobně proto,
že bojujete s Bohem o moc. Možná zatvrzujete své srdce v dutém pokusu zachovat si svou identitu, která je menší
než ta, ke které vás Bůh volá, být člověkem pro Církev a pro vaši rodinu. Ve vaší dnešní svaté hodině si dnes
uvědomte, jak odpovídáte na přítomnost Boha. Potom si s Ním promluvte o tom, jak a proč tak odpovídáte.

%newday
\newpage
\section{Den 23 - NECHTE SE OBRÁTIT}
\zacatekCtvrtyTyden
\subsection*{Čtení na den}
\textbf{Exodus 9,13-35}
\newline
\textit{
\textsuperscript{13}Hospodin řekl Mojžíšovi: „Za časného jitra se postav před faraóna. Řekneš mu: Toto praví Hospodin, Bůh Hebrejů: Propusť můj lid, aby mi sloužil!
\textsuperscript{14}Tentokrát zasáhnu do srdce všemi svými údery tebe i tvé služebníky a tvůj lid, abys poznal, že na celé zemi není nikdo jako já.
\textsuperscript{15}Vždyť už tehdy, když jsem vztáhl ruku, abych bil tebe i tvůj lid morem, mohl jsi být vyhlazen ze země.
\textsuperscript{16}Avšak proto jsem tě zachoval, abych na tobě ukázal svou moc a aby se po celé zemi vypravovalo o mém jménu.
\textsuperscript{17}Stále jednáš proti mému lidu zpupně a nechceš jej propustit.
\textsuperscript{18}Proto spustím zítra v tuto dobu tak hrozné krupobití, jaké v Egyptě nebylo ode dne jeho vzniku až do nynějška.
\textsuperscript{19}Nuže, dej odvést do bezpečí svá stáda a všechno, co máš na poli. Všechny lidi i dobytek, vše , co bude zastiženo na poli a nebude shromážděno do domu, potluče krupobití, takže zemřou.“
\textsuperscript{20}Kdo z faraónových služebníků se Hospodinova slova ulekl, zahnal své otroky a svá stáda do domů.
\textsuperscript{21}Kdo si slovo Hospodinovo nevzal k srdci, nechal své otroky a svá stáda na poli.
\textsuperscript{22}Hospodin řekl Mojžíšovi: „Vztáhni svou ruku k nebi. Na celou egyptskou zemi dolehne krupobití, na lidi, na dobytek i na všecky polní byliny v egyptské zemi.“
\textsuperscript{23}Když Mojžíš vztáhl svou hůl k nebi, dopustil Hospodin hromobití a krupobití. Na zemi padal oheň. Tak Hospodin spustil krupobití na egyptskou zemi.
\textsuperscript{24}Nastalo krupobití a uprostřed krupobití šlehal oheň; něco tak hrozného nebylo v celé zemi egyptské od dob, kdy se dostala do moci tohoto pronároda.
\textsuperscript{25}Krupobití potlouklo v celé egyptské zemi všechno, co bylo na poli, od lidí po dobytek; krupobití potlouklo také všechny polní byliny a polámalo všechno polní stromoví.
\textsuperscript{26}Jenom v zemi Gošenu, kde sídlili Izraelci, krupobití nebylo.
\textsuperscript{27}Tu si farao dal předvolat Mojžíše a Árona a řekl jim: „Opět jsem zhřešil. Hospodin je spravedlivý, a já i můj lid jsme svévolníci.
\textsuperscript{28}Proste Hospodina. Božího hromobití a krupobití je už dost. Propustím vás, nemusíte tu už dál zůstat.“
\textsuperscript{29}Mojžíš mu odvětil: „Jen co vyjdu z města, rozprostřu své dlaně k Hospodinu. Hromobití přestane a krupobití skončí, abys poznal, že země je Hospodinova.
\textsuperscript{30}Vím ovšem, že ty ani tvoji služebníci se stále ještě nebudete Hospodina Boha bát.“
\textsuperscript{31}Potlučen byl len a ječmen, protože ječmen byl už v klasech a len nasazoval tobolky.
\textsuperscript{32}Pšenice a špalda však potlučeny nebyly, protože jsou pozdní.
\textsuperscript{33}Mojžíš vyšel od faraóna z města a rozprostřel dlaně k Hospodinu. Hromobití a krupobití přestalo a déšť už nezaplavoval zemi.
\textsuperscript{34}Když farao viděl, že přestal déšť a krupobití i hromobití, hřešil dále. Zůstal v srdci neoblomný, on i jeho služebníci.
\textsuperscript{35}Srdce faraónovo se zatvrdilo a Izraelce nepropustil, jak Hospodin skrze Mojžíše předpověděl.
}

\subsection*{Reflexe}
Sedmou ranou krupobití ukončí dech každé bytosti, kterou potká. V Písmu vidíme, že občas Bůh jedná s lidstvem nežně a
jindy přísně. V obou případech to však je pro lidské dobro a posvěcení.
Běžná karikatura dnešní Církve je krutá a neúprosná instituce, jejíž přimární přínos lidstvu je ochromující vina a hanba.

Moderní televizní programy znázorňují spíše temný obraz katolické víry: pochmurné, zatuchlé kostely, mizerná hudba a
vysokké dřevěné krabice, ve kterých posloucháme odsuzující přednášky ze stran chladného, bezcitného celibátu s hmatatelným
pohrdáním lidstvem. Přemýšlíte někdy nad tím, jak lidé přišli na tuto myšlenku?

Možná pramení z neutuchajícího volání Církve po obrácení. Konverze může být vnímána jako násilná. Když jste ve stavu
deprese, hněvu, úzkosti, osamělosti nebo vyčerpání, volba zbavit se pohodlí může být pociťována jako násiilné vnitřní
krupobití. Může se to zdát kruté. Ale tato malá úmrtí sama sebe ukončují život těch vášní a podnětů, které brání našemu
osvobození.

Ve vaší konverzi zůstaňte blízko Pánu. Nechte své tělo i duši očistit asketickými disciplínami. Jakkoli se můžou občas zdát
kruté, očistí vás a zanechají váš život. Pokud s askezí bojujete, povězte o tom dnes Pánu. Také může posloužit přednést to
svému bratrstvu a vašemu duchovnímu vůdci. Ale nejprve to předložte Bohu.

%newday
\newpage
\section{Den 24 - ODDANÍ KONTROLE}
\zacatekCtvrtyTyden
\subsection*{Čtení na den}
\textbf{Exodus 10,1-20}
\newline
\textit{
\textsuperscript{1}Hospodin řekl Mojžíšovi: „Předstup před faraóna. Já jsem totiž učinil jeho srdce i srdce jeho služebníků neoblomné, abych mohl uprostřed nich provést tato svá znamení
\textsuperscript{2}a ty abys mohl vypravovat svým synům i vnukům o tom, co jsem v Egyptě dokázal, i o znameních, která jsem mezi nimi udělal, ať víte, že já jsem Hospodin.“
\textsuperscript{3}Mojžíš a Áron tedy předstoupili před faraóna a řekli mu: „Toto praví Hospodin, Bůh Hebrejů: Jak dlouho se budeš zdráhat pokořit se přede mnou? Propusť můj lid, aby mi sloužil.
\textsuperscript{4}Budeš-li se zdráhat propustit můj lid, pak na tvé území uvedu zítra kobylky.
\textsuperscript{5}Přikryjí povrch země, takže nebude možno zemi ani vidět, a sežerou zbytek toho, co vyvázlo, co vám zůstalo po krupobití. Ožerou také všechny stromy, které vám na polích znovu raší.
\textsuperscript{6}Naplní tvé domy, domy všech tvých služebníků i domy všech Egypťanů. Něco takového neviděli tvoji otcové ani dědové od doby, kdy začali obdělávat půdu, až dodnes.“ Nato se Mojžíš obrátil a odešel od faraóna.
\textsuperscript{7}Faraónovi služebníci řekli: „Jak dlouho nám bude tento člověk léčkou? Propusť ty muže, ať slouží Hospodinu, svému Bohu. Což jsi dosud nepoznal, že hrozí Egyptu zánik?“
\textsuperscript{8}Mojžíš a Áron byli přivedeni zpět k faraónovi. Ten jim řekl: „Nuže, služte Hospodinu, svému Bohu. Kdo všechno má jít?“
\textsuperscript{9}Mojžíš odvětil: „Půjdeme se svou mládeží i se starci, půjdeme se svými syny i dcerami, se svým bravem i skotem, neboť máme slavnost Hospodinovu.“
\textsuperscript{10}Farao jim však řekl: „To tak! Myslíte si, že Hospodin bude s vámi, když vás propustím s dětmi? To jste si zamanuli špatnou věc.
\textsuperscript{11}Kdepak! Vy muži si jděte a služte Hospodinu, když o to tak stojíte.“ A vyhnali je od faraóna.
\textsuperscript{12}Hospodin řekl Mojžíšovi: „Vztáhni nad egyptskou zemi ruku, aby přilétly na egyptskou zemi kobylky a sežraly všechny byliny země, všechno, co zůstalo po krupobití.“
\textsuperscript{13}Mojžíš tedy vztáhl nad egyptskou zemi hůl a Hospodin přihnal na zemi východní vítr. Ten vál po celý den a celou noc. Když nastalo jitro, přinesl východní vítr kobylky.
\textsuperscript{14}Kobylky přilétly na celou egyptskou zemi a spustily se na celé území Egypta v takovém množství, že tolik kobylek nebylo nikdy předtím ani potom.
\textsuperscript{15}Přikryly povrch celé země, až se na zemi zatmělo, a sežraly všechny byliny na zemi i všechno ovoce na stromech, co zbylo po krupobití. Na stromech a na polních bylinách po celé egyptské zemi nezbylo nic zeleného.
\textsuperscript{16}Farao rychle povolal Mojžíše a Árona. Řekl jim: „Zhřešil jsem proti Hospodinu, vašemu Bohu, i proti vám.
\textsuperscript{17}Sejmi prosím můj hřích ještě tentokrát a proste Hospodina, svého Boha, aby jen odvrátil ode mne tuto smrt.“
\textsuperscript{18}Mojžíš od faraóna odešel a prosil Hospodina.
\textsuperscript{19}Tu Hospodin obrátil vítr a velmi silný mořský vítr odnesl kobylky a prudce je vrhl do Rákosového moře, takže na celém egyptském území nezůstala jediná kobylka.
\textsuperscript{20}Avšak Hospodin zatvrdil faraónovo srdce, takže Izraelce nepropustil.
}

\subsection*{Reflexe}

Lidé se často ptají: „Proč by toto soužení Bůh dopustil mému příteli nebo mému příbuznému?“ Abychom odpověděli, nejprve
musíme zvážit fakt, že samotný důvod naší existence je být s Bohem a v Něm. Mimo Boha nejsme nic. Přesto se velmi často
proti Bohu bouříme. Proč se dobrým lidem stávají zlé věci? Aby se stali lepšími. Mají je povzbudit k návratu k Bohu.

V dnešním čtení se faraon již nechová moc manipulativně. Uvědomuje si, že už skoro všechno ztratil. Proto se zoufale drží
toho, co má, a snaží se udržet si své postavení, aby neztratil celé své království.

Muži hrozně bojují, aby si udrželi kontrolu. Masturbace a materialismus jsou příklady chlapeckých reakcí na stres, spoty,
zmatek, neúspěch a sklíčenost. V jakékoli z těchto situací se chlapec obrátí k nečemu, k čemukoli, co mu dá pocit kontroly – i
přesto, že ví, že je to lež. Abychom se vyhnuli stejnému pokušení jako muži, je důležité mít vždy na paměti, že tento svět nás
má připravit pro ten další.

Jako lidé dnes musíme být tím, čím mají být i Izraelité – poutníci, muži Exodu. To znamená život oddělený od věcí, pohodlí a
bezpečí. Znamená to držet se pouze Boha a věrně se k Němu přibližovat. Na co se upínáte? Co musíte opustit? Předneste tyto
věci před Pána a proste Ho, aby vám v této hodině udělil svobodu.


%newday
\newpage
\section{Den 25 - PODDAT SE NEBO VZEPŘÍT}
\zacatekCtvrtyTyden
\subsection*{Čtení na den}
\textbf{Exodus 10,21-29}
\newline
\textit{
\textsuperscript{21}Hospodin řekl Mojžíšovi: „Vztáhni svou ruku k nebi a egyptskou zemi zahalí temnota, taková temnota, že se dá nahmatat.“
\textsuperscript{22}Mojžíš vztáhl ruku k nebi. Tu nastala po celé egyptské zemi tma tmoucí a trvala po tři dny.
\textsuperscript{23}Lidé neviděli jeden druhého; po tři dny se nikdo neodvážil hnout ze svého místa. Ale všichni Izraelci měli ve svých obydlích světlo.
\textsuperscript{24}Farao povolal Mojžíše a řekl: „Odejděte! Služte Hospodinu! Zanechte tu jenom svůj brav a skot. Také vaše děti mohou jít s vámi.“
\textsuperscript{25}Mojžíš odpověděl: „Ty sám nám dáš potřebné k obětním hodům a k zápalným obětem, abychom je připravili Hospodinu, svému Bohu.
\textsuperscript{26}Půjdou s námi i naše stáda, ani pazneht tu nezůstane. Budeme z nich brát k službě Hospodinu, svému Bohu. My ještě nevíme, čím budeme Hospodinu sloužit, dokud tam nepřijdeme.“
\textsuperscript{27}Avšak Hospodin zatvrdil faraónovo srdce a on je nedovolil propustit.
\textsuperscript{28}Farao řekl: „Odejdi ode mne. Dej si pozor, ať mi už nepřijdeš na oči. Neboť v den, kdy mi přijdeš na oči, zemřeš!“
\textsuperscript{29}Mojžíš odpověděl: „Jak jsi řekl. Už ti na oči nepřijdu.“
}

\subsection*{Reflexe}
Akčoli faraonova frustrace roste, jeho srdce je stále zatvrzelé. Stále se vzpírá, i když si je vědom, že
pouze úplná poslušnost Božímu plánu je jedinou cestou k naplnění. Faraon má zjevně plán pro sebe a své
lidi, ale tento plán je jasně v rozporu s Boží vůlí. Výsledkem je to, že se on i jeho lidé potácí ve tmě.

Vezměme na vědomí, že člověk je stvořená bytost a jako taková nepatří sobě. Pouze když se vyrovnáme
s tím, že patříme Bohu samému, což vyžaduje značnou dávku pokory, můžeme jasněji vidět Boží
prozřetelnost.

Podívejme se na izraelský lid. S pokorou a poslušností Pánovu plánu se ocitnou obdarováni světlem.
Podívejte se na svůj život. Žijete ve světle nebo ve tmě? Stáváte se poslušnější k Božímu plánu pro vás
skrze dar disciplín, nebo se proti němu bouříte? Promluvte si o tom s Pánem. Proč vás Jeho dobrý plán
nutí bouřit se, malými nebo velkými způsoby? Požádejte Ho, aby vám ukázal, v čem jsou disciplíny
dobré, zejména v těch, které nejvíc nenávidíte. Požádejte Ho, aby vám ukázal, co dělá nyní
prostřednictvím těchto disciplín ve vašem životě.



%newday
\newpage
\section{Den 26 - DVEŘE SE ZAČÍNAJÍ OTVÍRAT}
\zacatekCtvrtyTyden
\subsection*{Čtení na den}
\textbf{Exodus 11,1-10}
\newline
\textit{
\textsuperscript{1}Hospodin řekl Mojžíšovi: „Ještě jednu ránu uvedu na faraóna a na Egypt. Potom vás odtud propustí, nadobro vyhostí, přímo vás odtud vyžene.
\textsuperscript{2}Vybídni lid, ať si vyžádá každý muž od svého souseda a každá žena od své sousedky stříbrné a zlaté šperky.“
\textsuperscript{3}A Hospodin zjednal lidu v očích Egypťanů přízeň. Také sám Mojžíš platil v egyptské zemi za velice významného v očích faraónových služebníků i v očích lidu.
\textsuperscript{4}Mojžíš řekl faraónovi : „Toto praví Hospodin: O půlnoci projdu Egyptem.
\textsuperscript{5}Všichni prvorození v egyptské zemi zemřou, od prvorozeného syna faraónova, který sedí na jeho trůnu, po prvorozeného syna otrokyně, která mele na mlýnku, i všechno prvorozené z dobytka.
\textsuperscript{6}Po celé egyptské zemi se bude rozléhat veliký křik, jakého nebylo a už nebude.
\textsuperscript{7}Ale na žádného Izraelce ani pes nezavrčí, ani na člověka ani na dobytče, abyste poznali, že Hospodin podivuhodně rozlišuje mezi Egyptem a Izraelem.
\textsuperscript{8}Všichni tito tvoji služebníci sestoupí ke mně, budou se mi klanět a říkat: Odejdi ty i všechen lid, který jde za tebou! Teprve potom odejdu.“ Nato Mojžíš, planoucí hněvem, od faraóna odešel.
\textsuperscript{9}Hospodin řekl Mojžíšovi: „Farao vás neposlechne, a tak mých zázraků v egyptské zemi ještě přibude.“
\textsuperscript{10}Mojžíš a Áron všechny ty zázraky před faraónem učinili, ale Hospodin zatvrdil faraónovo srdce, takže Izraelce ze své země nepropustil.  
}

\subsection*{Reflexe}
Izraelité se připravují. Bůh slíbil poslední ránu. Slíbil svobodu – ale neříká, že svoboda přijde bez
mocného boje nebo bez důvěry v Něj. Představte si, že po tolika dlouhých letech v otroctví v Egyptě se
dveře začnou otvírat. Svoboda už není pouhým snem, ale skutečnou možností.

Vidíte pro sebe stejnou možnost svobody? Teď se možná ptáte, proč jste se rozhodli pro tak radikální
duchovní cvičení. Někteří vaši bratři už možná přestali. Ale pro vás, kteří jste došli až sem, se začínají
otvírat dveře. Pokud dokážete Bohu důvěřovat hlouběji, více se spoléhat na své bratry a vytrváte, dávajíce
Bohu čas pracovat ve vašem životě, dveře se vám budou otevírat i nadále. Připravte se a proste dnes Pána
o milost vytrvalosti.


%newday
\newpage
\section{Den 27 - ŽÍT V PŘÍTOMNOSTI}
\zacatekCtvrtyTyden
\subsection*{Čtení na den}
\textbf{Exodus 12,1-20}
\newline
\textit{
\textsuperscript{1}Hospodin řekl Mojžíšovi a Áronovi v egyptské zemi:
\textsuperscript{2}„Tento měsíc bude pro vás začátkem měsíců. Bude pro vás prvním měsícem v roce.
\textsuperscript{3}Vyhlaste celé izraelské pospolitosti: Desátého dne tohoto měsíce si každý vezmete beránka podle svých rodů, beránka na rodinu.
\textsuperscript{4}Kdyby byla rodina malá a na beránka by nestačila, přibere si každý souseda, který bydlí nejblíže jeho rodiny, aby doplnil počet osob. Podle toho, kolik kdo sní, stanovíte počet na beránka.
\textsuperscript{5}Budete mít beránka bez vady, ročního samce. Vezmete jej z ovcí nebo z koz.
\textsuperscript{6}Budete jej opatrovat až do čtrnáctého dne tohoto měsíce. Navečer bude celé shromáždění izraelské pospolitosti beránky zabíjet.
\textsuperscript{7}Pak vezmou trochu krve a potřou jí obě veřeje i nadpraží u domů, v nichž jej budou jíst.
\textsuperscript{8}Tu noc budou jíst maso upečené na ohni a k němu budou jíst nekvašené chleby s hořkými bylinami.
\textsuperscript{9}Nebudete z něho jíst nic syrového ani vařeného ve vodě, nýbrž jen upečené na ohni s hlavou i s nohama a vnitřnostmi.
\textsuperscript{10}Nic z něho nenecháte do rána. Co z něho zůstane do rána, spálíte ohněm.
\textsuperscript{11}Budete jej jíst takto: Budete mít přepásaná bedra, opánky na nohou a hůl v ruce. Sníte jej ve chvatu. To bude Hospodinův hod beránka.
\textsuperscript{12}Tu noc projdu egyptskou zemí a všecko prvorozené v egyptské zemi pobiji, od lidí až po dobytek. Všechna egyptská božstva postihnu svými soudy. Já jsem Hospodin.
\textsuperscript{13}Na domech, v nichž budete, budete mít na znamení krev. Když tu krev uvidím, pominu vás a nedolehne na vás zhoubný úder, až budu bít egyptskou zemi.
\textsuperscript{14}Ten den vám bude dnem pamětním, budete jej slavit jako slavnost Hospodinovu. Budete jej slavit po všechna svá pokolení. To je provždy platné nařízení.
\textsuperscript{15}Po sedm dní budete jíst nekvašené chleby. Hned prvního dne odstraníte ze svých domů kvas. Každý, kdo by od prvního do sedmého dne jedl něco kvašeného, bude z Izraele vyobcován.
\textsuperscript{16}Prvního dne budete mít bohoslužebné shromáždění. I sedmého dne budete mít bohoslužebné shromáždění. V těch dnech se nebude konat žádné dílo. Smíte si připravit jen to, co každý potřebuje k jídlu.
\textsuperscript{17}Budete dbát na ustanovení o nekvašených chlebech, neboť právě toho dne jsem vyvedl vaše oddíly z egyptské země. Na tento den budete bedlivě dbát. To je provždy platné nařízení pro všechna vaše pokolení.
\textsuperscript{18}Od večera čtrnáctého dne prvního měsíce budete jíst nekvašené chleby až do večera jedenadvacátého dne téhož měsíce.
\textsuperscript{19}Po sedm dní se nenajde ve vašich domech kvas. Každý, kdo by jedl něco kvašeného, bude vyobcován z pospolitosti Izraele, i host a domorodec.
\textsuperscript{20}Nebudete jíst nic kvašeného. Ve všech svých obydlích budete jíst nekvašené chleby.“
}

\subsection*{Reflexe}
Je jednoduché nenávidět čas. Většina z nás neustále vyhlíží „další nejlepší věc“, která nám, jak věříme, konečně zajistí
blaženost, po které toužíme. V naší netrpělivosti máme sklon nenávidět přítomnost a toužit po lepších zítřkách. Přesto znamená
tento způsob života samotný život nenávidět. Jako křesťané jsme povoláni k životu v přítomnosti, i když je přítomnost plná
dřiny, zármutku nebo bolesti.

Dnes se Izraelité připravují na příchod úsvitu. Zároveň jsou přítomni noci, protože noc vyžaduje pozornost k detailu. Pokud
přehlédnou jednu část oběti před nimi, mohl by pro ně být úsvit mnohem tmavší, než by jinak byl.

Právě teď, ve vaší odpovědi na Boží volání po svobodě, také očekáváte úsvit. Těchto 90 dní oběti jsou cenným časem. Může
být jednoduché považovat tyto dny za obtíž nebo nesmyslnou zkoušku vůle. Možná nedočkavě vyhlížíte den, kdy tento
nesmysl přestane, a budete si moct užít života ve svobodě. Ne. Je čas zaměřit se na současné oběti před vámi. Vaše rodina
prosí, abyste byli změněni časem v modlitbě. Církev touží po tom, abyste vystoupili a byli pro její lid přínosem díky praxi
askeze. Vaši bratři na vás spoléhají, že je pozdvihnete, stejně jako oni pozdvihnou vás. Přijměte to. Žijte v přítomnosti.

Věnovali jste náležitou pozornost detailům duchovního cvičení, k němuž vás Pán povolal? Pokud ano, přijměte v této chvíli
ujištění o Boží lásce. Pokud ne, otevřete srdce milosti, kterou vám Bůh chce dát. Chce vás přivést k úsvitu. Volba spolupráce
s tímto plánem je na vás.

%newday
\newpage
\section{Den 28 - KREV BERÁNKA}
\zacatekCtvrtyTyden
\subsection*{Čtení na den}
\textbf{Exodus 12,21-28}
\newline
\textit{
\textsuperscript{21}Mojžíš svolal všechny izraelské starší a řekl jim: „Jděte si vzít kus z bravu podle vašich čeledí a zabijte velikonočního beránka.
\textsuperscript{22}Potom vezměte svazek yzopu, namočte jej v misce s krví a krví z misky potřete nadpraží a obě veřeje. Ať nikdo z vás až do rána nevychází ze dveří svého domu.
\textsuperscript{23}Až Hospodin bude procházet zemí , aby udeřil na Egypt, uvidí krev na nadpraží a na obou veřejích. Hospodin ty dveře pomine a nedopustí, aby do vašeho domu vešel zhoubce a udeřil na vás .
\textsuperscript{24}Dbejte na toto ustanovení. To je provždy platné nařízení pro tebe i pro tvé syny.
\textsuperscript{25}Až přijdete do země, kterou vám Hospodin dá, jak přislíbil, dbejte na tuto službu.
\textsuperscript{26}Až se vás pak vaši synové budou ptát, co pro vás tato služba znamená,
\textsuperscript{27}odpovíte: ‚Je to velikonoční obětní hod Hospodinův. On v Egyptě pominul domy synů Izraele. Když udeřil na Egypt, naše domy vysvobodil.‘“ Lid padl na kolena a klaněl se.
\textsuperscript{28}Izraelci pak odešli a učinili přesně tak, jak Hospodin Mojžíšovi a Áronovi přikázal.
}

\subsection*{Reflexe}

Dnešní pasáž nás upozorňuje na moc Eucharistie. Svatý Jan Zlatoústý o této pasáži napsal svým přímým,
vyzývajícím způsobem:
\textit{Přejeme-li si porozumět moci Kristovy krve, měli bychom se vrátit ke starověké zprávě o její
předzvěsti v Egyptě. „Obětujte bezvadného beránka,“ přikazuje Mojžíš, „a jeho krví potřete
svá nadpraží a veřeje.“ Kdybychom se ho zeptali, co tím myslel a jak by krev nerozumného
zvířete mohla zachránit lid obdařený rozumem, jeho odpovědí by bylo, že zachraňující moc
leží ne v krvi samotné, ale ve znaku Pánovy krve. V těchto dnech, kdy anděl zkázy viděl krev
na dveřích, se neodvážil vstoupit. Jak se tedy přiblíží ďábel, když uvidí ne krev na veřejích, ale
pravou krev na rtech věřících, na dveřích Kristova chrámu.}

Samozřejmě má na mysli svaté přijímání, které přijímáme na rtech a do úst. Tento neuvěřitelný dar nám
poskytuje ochranu před mocí smrti a temnoty, ještě více než krev velikonočního beránka chránila děti
Izraele před andělem zkázy. Běžíte každodenně k přijímání Eucharistie? Byli jste alespoň věrní jedné mši
svaté navíc v týdnu? Krev Beránka není jen částí historické události, ale probíhající, současná realita.
Kristus nabídl svou oběť jednou pro všechny a tato oběť pokračuje každou mší svatou. Jaký to dar.

Věříte, že Tělo a Krev Kristova přítomná v Eucharistii má skutečně moc chránit vás před mocí smrti a
temnoty? Věříte, že Kristovo Tělo, Krev, Duše a Božství jsou plně přítomny v Eucharistii? Pokud ano,
chvalte Boha, že máte oči k vidění a uši k slyšení dobrých zpráv, které pro vás má. Pokud ne, dnes se
upřímně zeptejte Boha, jestli je tato dva tisíce let stará tradice opravdu pravdivá.


% ===============================================
% ===== PATY TYDEN
% ===============================================
%ukony
\newpage
\section*{Úkony (ukazatel cesty) pro 5. týden}

\textbf{Místo:} Východ z/od Egypta (východně od Egypta), útěk do pouště

Izraelité pocítili potřebu sloužit Bohu. Akčně reagovali zabitím egyptského boha (beránka) a veřejným rozmazáním jehněčí krve na jejich veřeje. Hospodin ctí odvahu, kterou prokázali, a otevírá jim bránu ke svobodě. Také vy jste prokázali odvahu a označili jste své dveře. Po měsíci odtažení od bohů a model tohoto světa jste se rozhodli zůstat s disciplínami Exodu 90. Také vaše odvaha bude vyznamenána a brána ke svobodě otevřena. Nyní vstupujete do pouště. Vyplatí se vám zůstat velmi blízko Bohu a vašemu bratrstvu. 

\subsection*{1. Vzdejte se kontroly}
Je dobré reflektovat, jaké disciplíny jste pokoušeni změnit a kde v tomto duchovním cvičení jste pokoušeni podvádět. Předneste tyto věci Bohu ve vaší dnešní svaté hodině. Přenechte kontrolu Jemu.
\subsection*{2. Zkontaktujte svou kotvu}
Už jste dnes spolu komunikovali? Byli jste v kontaktu tento týden? Pokud jste se svou kotvou dnes ještě nemluvili, teď je dobrý čas tak učinit. Počítá s vámi. Potřebuje vás. Pokud povolíte a on spadne, bude to bolestný pád, tak je to jednoduché.
\subsection*{3. Znovu si připomeňte své proč}
To je důležitý smysl Exodu. Lidé, které milujete (manželka, farnost, přátelé, děti), dychtivě touží po vašem osvobození. Stejně jako Izraelité nesmíte spouštět svůj zrak z Boha, když vstupujete do pouště k dosáhnutí svobody, o kterou usilujete. Pouze Bůh vám dokáže dát takové vysvobození.
\subsection*{4. Zvažte přečtení Průvodce terénem}
Pokud jste si stále nenašli čas na jeho přečtení, nyní si ho udělejte. Průvodce terénem rámcuje celou zkušenost Exodu 90 a pomáhá pochopit důvod každé části tohoto duchovního cvičení. (Nejdůležitější části jsou Začněte zde: co je vaše proč; Pilíře Exodu 90, a pro ženaté muže, Muž Exodu a jeho manželka.)

\subsection*{Modlitba}
Modlete se, aby Pán osvobodil vás a vaše bratrství \newline
Modleme se za svobodu všech mužů v exodu, stejně tak, jako se oni modlí za vás.\newline
Ve jménu Otce i Syna i Ducha svatého … Otče náš… Ve jménu Otce i Syna i Ducha svatého … Amen.
\newpage

%newday
\newpage
\section{Den 29 - POCHYBNOSTI O BOŽÍ DOBROTĚ}
\zacatekPatyTyden
\subsection*{Čtení na den}
\textbf{Exodus 12,29-30}
\newline
\textit{
\textsuperscript{29}Když nastala půlnoc, pobil Hospodin v egyptské zemi všechno prvorozené, od prvorozeného syna faraónova, který seděl na jeho trůnu, až po prvorozeného syna zajatce v žalářní kobce, i všechno prvorozené z dobytka.
\textsuperscript{30}Tu farao v noci vstal, i všichni jeho služebníci a všichni Egypťané, a v celém Egyptě nastal veliký křik, protože nebylo domu, kde by nebyl mrtvý.
}

\subsection*{Reflexe}

V reakci na tyto verše může být člověk na Boha rozzlobený. Může ho dokonce obviňovat za chaos a smrt,
kterou po celou dobu způlsobuje. Bůh je však Bůh. Nikdy si pro lidstvo nevybral hřích, utrpení a smrt.
V zahradě Edenu stvořil Bůh člověka v naprosté harmonii s Ním a se stvořeným světem. Člověk však,
obdarovaný svobodou, učinil volbu, která změnila téměř všechno a přinesla chaos do Božího řádu. Od té
doby Bůh pravuje na tom, aby přeměnil svět, který člověk rozházel.

Sám člověk se rozhodl jednat proti Božímu rozkazu, Bůh to dovolil z úcty k lidské svobodě. Kdyby člověk v
zahradě spolupracoval s Božím plánem, zažil by Boží lásku mnohem jinak. Podobně kdyby se faraon rozhodl
spolupracovat s Božím plánem v Egyptě, i on by mohl zažít Boží lásku úplně jinak.

Zeptejte se sami sebe: Spolupracuji s plánem, který pro mě má dnes Bůh? Zpochybňuji Boží rozhodnutí,
nebo věřím, že Jeho plán je plánem naprosté dobroty? Tyto otázky dnes stojí za to přednést ve svaté hodině.


%newday
\newpage
\section{Den 30 - USPOŘÁDAT SI SVŮJ ŽIVOT}
\zacatekPatyTyden
\subsection*{Čtení na den}
\textbf{Exodus 12,31-36}
\newline
\textit{
\textsuperscript{31}Ještě v noci povolal Mojžíše a Árona a řekl: „Seberte se a odejděte z mého lidu, vy i Izraelci. Jděte, služte Hospodinu, jak jste žádali.
\textsuperscript{32}Vezměte také svůj brav i skot, jak jste žádali, a jděte. Vyproste požehnání i pro mne.“
\textsuperscript{33}Egypťané naléhali na lid a spěchali s jeho propuštěním ze země, protože si říkali: „Všichni pomřeme!“
\textsuperscript{34}Lid tedy vzal těsto ještě nevykynuté, zabalil díže do plášťů a nesl na ramenou.
\textsuperscript{35}Izraelci jednali podle Mojžíšova rozkazu; vyžádali si též od Egypťanů stříbrné a zlaté šperky a pláště.
\textsuperscript{36}A Hospodin zjednal lidu přízeň v očích Egypťanů a oni jim vyhověli. Tak vyplenili Egypt.
}

\subsection*{Reflexe}

V souladu s tímto příběhem Exodu vyžaduje každoročně židovský svátek Pesachu užití nekvašeného chleba
v posvátném jídle. Izraelský lid měl velmi málo času na útěk z Egypta – tak málo, že nemohli ani čekat, než jim
vykyne chléb. Času bylo málo a bylo třeba jednat.

Přemýšlejte o tom v kontextu vašeho života. Poznáváte, že čas je vzácný? Víte, že je od vás požadována akce?
Pamatujete na svou smrtelnost nebo rozjímáte o dni, kdy zemřete? Den tohoto duševního probuzení nakonec jednou
přijde. Ten den vypadá pro každého člověka jinak. Může to být třeba zkušenost s autonehodou, infarkt, událost doma
nebo potyčka s násilím. Ať je to cokoli, způsobí to náhlé uvědomění, jak málo času zde na zemi máme.

Třeba se dostanete do zajetí úzkosti a strachu, když si, možná poprvé, uvědomíte, že váš život na zemi není bez konce.
Pak se, jako Izraelité, můžete ocitnout ve spěchu dát věci do pořádku, zejména ve svém osobním životě. Bůh vám dal
těchto cenných devadesát dní na to, abyste přehodnotili svůj život, prohloubili vztahy, očistili se od hříchu a přerovnali
svůj život vzhledem k Bohu. Hospodin je Bůh moudrosti a lásky. Bez ohledu na to, jakou je pro vás Exodus 90
výzvou, je dobré, že jste teď tady. Využijte tohoto času.

Dáváte stále do pořádku věci ve svém duchovním nebo rodinném životě, které vás Pán žádá, abyste urovnali? Držíte se
stranou od vychloubání se, protěžování sebe sama, od uhýbání před odpovědností? Žijete průměrně, a dokonce plýtváte
těmito devadesáti dny tím, že jim nedáváte plné úsilí nebo záměrně obcházíte pravidla? Pohovořte si o tom dnes
s Pánem v rozjímání a naslouchejte, kam vás vede.


%newday
\newpage
\section{Den 31 - VELKORYSOST S BOHEM}
\zacatekPatyTyden
\subsection*{Čtení na den}
\textbf{Exodus 12,37-42}
\newline
\textit{
\textsuperscript{37}Izraelci vytáhli z Ramesesu do Sukótu, kolem šesti set tisíc pěších mužů kromě dětí.
\textsuperscript{38}Vyšlo s nimi také mnoho přimíšeného lidu a obrovská stáda bravu a skotu.
\textsuperscript{39}Z těsta, které vynesli z Egypta, napekli nekvašené podpopelné chleby, protože ještě nevykynulo. Byli totiž z Egypta vyhnáni a nemohli otálet. Ani potravu na cestu si nestačili připravit.
\textsuperscript{40}Doba pobytu, kterou Izraelci v Egyptě strávili, byla čtyři sta třicet let.
\textsuperscript{41}Když uplynulo čtyři sta třicet let, přesně na den vyšly všechny Hospodinovy zástupy z egyptské země.
\textsuperscript{42}Byla to noc jejich bdění pro Hospodina, když je vyváděl z egyptské země. Tato noc je všem synům Izraele po všechna pokolení nocí bdění pro Hospodina.
}

\subsection*{Reflexe}
V této pasáži vidíme pozoruhodnou Boží velkorysost. Vzpomeňme si na první verše knihy Exodus. Jozef, syn Jákobův,
přišel do Egypta sám. Z tohoto muže, ke kterému se nakonec připojili jeho bratři, vzešlo 600 000 mužů (nepočítaje
ženy a děti), o kterých se dnes mluví. To nám naastavuje dvě provokativní otázky: zaprvé, Boha nikdy nikdo nepředčí
v Jeho velkorysosti. Zadruhé, je zázrak, kolik toho Bůh dokáže s málem.

Dnes se připojujete k Izraelitům, když konečně následují Hospodina z Egypta. Po celém měsíci za vámi jste vykročili
se svými bratry na další část cesty. Udělejte si chvilku na oslavu toho, co jste dokázali. Opuštění Egypta není malý
úspěch. Vzdali se jste mnoha a učinili mnoho životních změn pro to, abyste se sem dostali.

Všimněte si však, že zaslíbená zem není blízko. Ani faraon, ani Egypťané se nevzdali. Rozhodli jste se zcela opustit
svůj starý domov otroctví a modlářství, abyste se vydali na cestu ke svobodě. To je dobře. Ale cesta před vámi není
dobře dlážděná dálnice lemovaná nejlepšími restauracemi a obchody. Opustili jste civilizaci. Vkročili jste do divoké
pouště.

Teď se věci stanou těžšími. Zlý, více rozzlobený než faraon, vás bude prohnaně následovat do pouště, aby vás stíhal a
zotročoval více než předtím. Ale nebojte se, Pán vás vede, vaši bratři jsou s vámi a cíl za to stojí. Tohle vás vedlo
z Egypta a povede vás celou cestou až do země zaslíbené.

Dnes si najděte čas, abyste chválili Boha. Předneste Mu svou vděčnost za to, že vás vyvedl z Egypta a vede vás dále
směrem ke svobodě. U těch, kdo zvládli první měsíc, je mnohem pravděpodobnější, že zvládnou celou cestu.

%newday
\newpage
\section{Den 32 - EUCHARISTIE A JEDNOTA}
\zacatekPatyTyden
\subsection*{Čtení na den}
\textbf{Exodus 12,43-51}
\newline
\textit{
\textsuperscript{43}Hospodin řekl Mojžíšovi a Áronovi: „Toto je nařízení o hodu beránka: Nebude z něho jíst žádný cizinec.
\textsuperscript{44}Ale bude jej jíst každý služebník koupený za stříbro, bude-li obřezán.
\textsuperscript{45}Přistěhovalec ani nádeník jej jíst nebude.
\textsuperscript{46}Musí být sněden v témž domě. Z jeho masa nevyneseš nic z domu; žádnou jeho kost nezlámete.
\textsuperscript{47}Tak to bude dělat celá izraelská pospolitost.
\textsuperscript{48}Jestliže by u tebe pobýval host a chtěl by připravit Hospodinu hod beránka, nechť je u něho obřezán každý mužského pohlaví a potom bude smět přistoupit a tak učinit a bude jako domorodec v zemi. Ale žádný neobřezanec jej jíst nebude.
\textsuperscript{49}Stejný řád bude platit pro domorodce i pro hosta, který bude pobývat mezi vámi.“
\textsuperscript{50}Všichni Izraelci učinili přesně tak, jak Hospodin Mojžíšovi a Áronovi přikázal.
\textsuperscript{51}Právě v ten den vyvedl Hospodin Izraelce seřazené po oddílech z egyptské země.
}

\subsection*{Reflexe}
Při velikonoční večeři byl obětován bezvadný beránek. Tato oběť předznamenala bohoslužbu oběti, kde se Kristus,
neposkvrněný beránek, stal velikonoční obětí. V Exodu Bůh přikázal, aby se žádný cizinec Paschy neúčastnil, protože
by tím narušoval jednotu. Přesto nebyl cizinec vyloužen bez naděje na začlenění. Cizinec, který se chtěl spoluúčastnit
Paschy, musel učinit to, co všichni členové již učinili pro vstup do izraelské komunity: musel být obřezán. Pokud se tak
rozhodl, pak se stal členem společenství a byl přijat k účasti na Pesachu.

Tato překážka obřízky učinila rozhodnutí stát se součástí společenství poměrně vážným. Žádný dospělý muž jen tak z
nedělního rozmaru nedovolí, aby se nůž přiblížil jeho genitáliím. Můž, který byl rozhodnut se nechat obřezat, si zvolil
být plnohodnotným a aktivním účastníkem společenství. Zvolil si věrnost tomuto tělu.

Stejně tak Církev nedovolí nekatolíkům přijímat svaté přijímání z nedělního rozmaru. Církev však chce, aby měli
všichni lidé možnost přijmout svaté přijímání. Jako cizinci mezi Izraelity, stejně nemusí být nekatolíci navždy
vyloučeni z přijímání Eucharistie. Aby však mohli být přijati, musí udělat to, co se žádá po všech katolících ve
společenství: veřejně vyznat víru Církve, být pokřtěni, zpytovat svědomí, postit se, chodit na mše svaté, a přijímat
Eucharistii. Ve skutečnosti je každý zván ke svatému přijímání, ale všichni jsme povinni přijmout tuto svátost
zaslouženě a s věrností. Jste zavázáni k přijetí, nebo berete své členství v církevní komunitě za samozřejmost?

%newday
\newpage
\section{Den 33 - JSI KNĚZEM}
\zacatekPatyTyden
\subsection*{Čtení na den}
\textbf{Exodus 13,1-16}
\newline
\textit{
\textsuperscript{1}Hospodin promluvil k Mojžíšovi:
\textsuperscript{2}„Posvěť mi všechno prvorozené, co mezi Izraelci otvírá lůno, ať z lidí či z dobytka. Je to moje!“
\textsuperscript{3}Mojžíš řekl lidu: „Pamatujte na tento den, kdy jste vyšli z Egypta, z domu otroctví. Hospodin vás odtud vyvedl pevnou rukou. Proto se nesmí jíst nic kvašeného.
\textsuperscript{4}Vycházíte dnes, v měsíci ábíbu.
\textsuperscript{5}Až tě Hospodin uvede do země Kenaanců, Chetejců, Emorejců, Chivejců a Jebúsejců, o níž se zavázal tvým otcům přísahou, že ji dá tobě, zemi oplývající mlékem a medem, budeš v ní tohoto měsíce konat tuto službu:
\textsuperscript{6}Sedm dní budeš jíst nekvašené chleby. Sedmého dne bude slavnost Hospodinova.
\textsuperscript{7}Nekvašené chleby se budou jíst po sedm dní. Nespatří se u tebe nic kvašeného, na celém tvém území se u tebe nespatří žádný kvas.
\textsuperscript{8}V onen den svému synovi oznámíš: ‚To je proto, co mi prokázal Hospodin, když jsem vycházel z Egypta.‘
\textsuperscript{9}A budeš to mít jako znamení na své ruce a jako připomínku mezi svýma očima, aby v tvých ústech zůstal Hospodinův zákon, neboť pevnou rukou tě vyvedl Hospodin z Egypta.
\textsuperscript{10}Budeš dbát na toto nařízení ve stanovený čas rok co rok.
\textsuperscript{11}Až tě Hospodin uvede do země Kenaanců, jak přísežně zaslíbil tobě i tvým otcům, a až ti ji dá,
\textsuperscript{12}všechno, co otvírá lůno, odevzdáš Hospodinu. Všichni samečci, které tvůj dobytek vrhne nejprve, budou patřit Hospodinu.
\textsuperscript{13}Každého osla, který se narodil jako první, vyplatíš jehnětem. Kdybys jej nemohl vyplatit, zlomíš mu vaz. Také každého prvorozeného ze svých synů vyplatíš.
\textsuperscript{14}Až se tě tvůj syn v budoucnu zeptá, co to znamená, odpovíš mu: ‚Hospodin nás vyvedl pevnou rukou z Egypta, z domu otroctví.
\textsuperscript{15}Když se farao zatvrdil a nechtěl nás propustit, pobil Hospodin v egyptské zemi všechno prvorozené, od prvorozeného z lidí až po prvorozené z dobytka. Proto obětuji Hospodinu všechny samce, kteří otvírají lůno, a každého prvorozeného ze svých synů vyplácím.‘
\textsuperscript{16}To bude jako znamení na tvé ruce a jako pásek na čele mezi tvýma očima. Neboť pevnou rukou nás vyvedl Hospodin z Egypta.“
}

\subsection*{Reflexe}
Toto vykoupení neboli „zpětný odkup“ prvorozeného syna je důslednou připomínkou Izraelitům, že v noci, kdy odešli z Egypta,
Bůh ušetřil jejich prvorozené syny, ale nezachránil prvorozené syny Egypťanů. Izraelští prvorození synové byli služební třídou
nebo řádem kněží. Tuto výsadu by si zachovali, pokud by nebylo té nešťastné události se zlatým teletem, kdy jim byla jejich čest
odebrána a dána kmenu Levi.

V Novém zákoně Ježíš Kristus sloužil jako kněz – obzvlášť na Velký Pátek, když byl knězem i obětí zároveň. Tento čin změnil
všechno. Po vítězství Krista nad hříchem a smrtí se všichni lidé pokřtěni v Krista stanou jeho kněžskou třídou (srov. Zjev 5,10)
(ačkoli se toto kněžství liší od kněžství služebného). Proč je to důležité? Protože kněz obětuje.

To znamená, že vy, členové všeobecného kněžšství, jste povinni obětovat. Jakou oběť? Svatý Pavel nám říká: „Vybízím vás, bratří,
pro Boží milosrdenství, abyste sami sebe přinášeli jako živou, svatou, Bohu milou oběť; to ať je vaše pravá bohoslužba“ (Řím
\textsuperscript{12},1) Toto duchovní cvičení, ve kterém jste nyní zapojeni, je obětí Bohu. Plněním denních disciplín tohoto cvičení se formujete pro
dobré kněžství hodné Ježíše Krista.

Pohovořte si s Pánem o vaší službě ve všeobecném kněžství. Naslouchejte tomu, co vám chce říct o posvěcování vašeho dne a
přinášení dokonalých obětí pro ostatní z asketických disciplín, které jste se rozhodli přijmout. Modlete se, ať vám dá smysluplnější
a hlubší poznání oběti.

%newday
\newpage
\section{Den 34 - NÁSLEDUJTE BOHA VE VÍŘE}
\zacatekPatyTyden
\subsection*{Čtení na den}
\textbf{Exodus 13,17-14,9}
\newline
\textit{
\textsuperscript{17}Když farao lid propustil, nevedl je Bůh cestou směřující do země Pelištejců, i když byla kratší. Bůh totiž řekl: „Aby lid nelitoval, když uvidí, že mu hrozí válka, a nevrátil se do Egypta.“
\textsuperscript{18}Proto Bůh vedl lid oklikou, cestou přes poušť k Rákosovému moři. Izraelci vytáhli z egyptské země rozděleni do bojových útvarů.
\textsuperscript{19}Mojžíš vzal s sebou Josefovy kosti. Ten totiž zavázal Izraelce přísahou: „Až vás Bůh navštíví, vynesete odtud s sebou mé kosti.“
\textsuperscript{20}I vytáhli ze Sukótu a utábořili se v Étamu na pokraji pouště.
\textsuperscript{21}Hospodin šel před nimi ve dne v sloupu oblakovém, a tak je cestou vedl, v noci ve sloupu ohnivém, a tak jim svítil, že mohli jít ve dne i v noci.
\textsuperscript{22}Sloup oblakový se nevzdálil od lidu ve dne, ani sloup ohnivý v noci.
\textsuperscript{1}Hospodin promluvil k Mojžíšovi:
\textsuperscript{2}„Rozkaž Izraelcům, aby se obrátili a utábořili před Pí-chírotem mezi Migdólem a mořem; utáboříte se před Baal-sefónem, přímo proti němu při moři.
\textsuperscript{3}Farao si o Izraelcích řekne: Bloudí v zemi, zavřela se za nimi poušť.
\textsuperscript{4}Tu zatvrdím faraónovo srdce a on vás bude pronásledovat. Já se však na faraónovi a na všem jeho vojsku oslavím, takže Egypťané poznají, že já jsem Hospodin.“ I učinili tak.
\textsuperscript{5}Egyptskému králi bylo oznámeno, že lid uprchl. Srdce faraóna a jeho služebníků se obrátilo proti lidu. Řekli: „Co jsme to udělali, že jsme Izraele propustili z otroctví?“
\textsuperscript{6}Farao dal zapřáhnout do svého válečného vozu a vzal s sebou svůj lid.
\textsuperscript{7}Vzal též šest set vybraných vozů, totiž všechny vozy egyptské. Na všech byla tříčlenná osádka.
\textsuperscript{8}Hospodin zatvrdil srdce faraóna, krále egyptského, a ten Izraelce pronásledoval. Ale Izraelci navzdory všemu vyšli.
\textsuperscript{9}Egypťané je pronásledovali a dostihli je, když tábořili při moři, dostihli je všichni faraónovi koně, vozy, jeho jízda a vojsko, při Pí-chírotu před Baal-sefónem.
}

\subsection*{Reflexe}
Přišli jste na okraj Rudého moře. Jedním směrem se nachází krásný výhled. Druhým směrem zuří armáda
pronásledující vás, aby vás zajala nebo zabila. Tato scéna byla pro Izraelity skutečnou realitou. Pro nás dnes
také zůstává realitou, ovšem v duchovní sféře.
Kardinál Jean Daniélou napsal o konfliktu katechumenů připravujících se na křest a nepřítele, Satana.
Zevrubně popisuje, co oko nevidí a duch ještě není schopen postřehnout:

\textit{Čtyřicet postních dní v katechumenátu jsou časem soudu, čase vážného konfliktu, zatímco se
Satan a jeho pomocníci snaží ubránit si zajetí duše katechumena. To není pouhý řečnický obrat,
musí to být chápáno v doslovném smyslu skutečnosti: pohan totiž není pouhým neználkem
křesťanského zjevení, ale aktivním poddaným moci zla, a musí být vysvobozen z tohoto zajetí…
Každá konverze proto zahrnuje množství dramatických konfliktů, a všechny misionářské aktivity
zahrnují podstatu tohoto tajemství. Není to pouhé představení radostné zprávy evangelia ve
formě vhodné pro každého z různých nekřesťanských kultur, ale střet v boji s mocnostmi zla:
operace této války probíhají na nadpřirozeném poli a patří do tajemství svatosti – skrze modlitbu
a pokání je ďábel nakonec vypuzen. Ignorování tohoto aspektu věci znamená nepochopení
podstaty misionářské práce. I po Kristově vítězství zůstává lidská postata těch, kteří nejsou Jeho,
uvězněna: On rozdrtil hlavu hadu, ale svitky se stále svíjejí, aby polapily lid země. Když Satan
vidí svou kořist unikat, zdvojnásobí své síly proti katechumenovi; ale během těchto čtyřiceti dní je
Kristovo sevření také zesíleno… ale ďábel drží oběť pod tlakem celou dobu až po onu chvíli
Velikonoční vigílie, po samý okraj křtitelnice. Pouze poté se stane ona nemožná věc: moře je
rozděleno…
Vody se rozestoupily pro Izraelity a brány smrti byly otevřeny Pánu Ježíši – katechumen sestoupí
do křestní vody, učiní onen krok, nechá za sebou faraona a jeho vojska, ďábla a jeho anděly a
přejde na druhou stranu. Je zachráněn. Doslova to znamená být ztroskotancem, který přežil a byl
převezen na zem.}

Křest, nám daný pro naši spásu, je více než jen obyčejný obřad. Je to víc než pouhá pověra. Je to spásné dílo
Boží. Když dnes stojíte u Rudého moře, zvažte svou situaci. Jestliže budete následovat Pána, mohli byste se
utopit v obrovském moři – tedy pokud pro vás Hospodin nemá lepší plán, který ještě nedokážete vnímat.

Druhou možností je zastavit a otočit se zpět. Pokud zvolíte tuto možnost, v nejlepším případě budete
zotročeni, v nejhorším ztratíte sám život. Dnes je 34. den, můžete přestat. Jste volní a můžete se vrátit zpět.
Co si vyberete? Vrátíte se zpět, nebo dáte svou důvěru Bohu a budete pokračovat dál?

%newday
\newpage
\section{Den 35 - TOUHY SRDCE}
\zacatekPatyTyden
\subsection*{Čtení na den}
\textbf{Exodus 14,10-20}
\newline
\textit{
\textsuperscript{10}Když se farao přiblížil, Izraelci se rozhlédli a viděli, že Egypťané táhnou za nimi. Tu se Izraelci velmi polekali a úpěli k Hospodinu.
\textsuperscript{11}A osopili se na Mojžíše: „Což nebylo v Egyptě dost hrobů, že jsi nás odvedl, abychom zemřeli na poušti? Cos nám to udělal, že jsi nás vyvedl z Egypta?
\textsuperscript{12}Došlo na to, o čem jsme s tebou mluvili v Egyptě: Nech nás být, ať sloužíme Egyptu. Vždyť pro nás bylo lépe sloužit Egyptu než zemřít na poušti.“
\textsuperscript{13}Mojžíš řekl lidu: „Nebojte se! Vydržte a uvidíte, jak vás dnes Hospodin zachrání. Jak vidíte Egypťany dnes, tak je už nikdy neuvidíte.
\textsuperscript{14}Hospodin bude bojovat za vás a vy budete mlčky přihlížet .“
\textsuperscript{15}Hospodin řekl Mojžíšovi: „Proč ke mně úpíš? Pobídni Izraelce, ať táhnou dál .
\textsuperscript{16}Ty pak pozdvihni svou hůl, vztáhni ruku nad moře a rozpoltíš je, a tak Izraelci půjdou prostředkem moře po suchu.
\textsuperscript{17}Já zatvrdím srdce Egypťanů, takže půjdou za nimi. Oslavím se na faraónovi a na všem jeho vojsku, na jeho vozech i jízdě.
\textsuperscript{18}Egypťané poznají, že já jsem Hospodin, až budu oslaven tím , co učiním s faraónem, s jeho vozy a jízdou.“
\textsuperscript{19}Tu se zvedl Boží posel, který šel před izraelským táborem, a šel teď za nimi. Oblakový sloup se před nimi totiž zvedl, postavil se za ně
\textsuperscript{20}a vstoupil mezi tábor egyptský a izraelský. Jedněm byl oblakem a temnotou, druhým osvěcoval noc; po celou noc se jedni k druhým nepřiblížili.
}

\subsection*{Reflexe}
Tato úžasná událost je známá mnohým: rozestoupení Rudého moře. Izraelci v nemožné situaci reagují naprosto
pochybně. Bůh mluví k Mojžíšovi a říká mu něco fascinujícího: „Proč ke mně úpíš?“ To je zvláštní, protože to vypadá,
jako by Bůh reagoval na něco, co řekl Mojžíš… ale Mojžíš neřekl nic. Bůh zná touhy lidského srdce a tyto touhy
k Němu mluví. Jaká útěcha! Dokonce i když nevíme, na co se zeptat (nebo snad i když žádáme o špatnou věc), nás Bůh
zná lépe, než známe sami sebe.

I když si myslíme, že cheme dočasně ulevit svým obětem, Bůh pro nás chce to, co opravdu chceme také. Nechceme
doopravdy pomíjející potěšení z několikahodinového sledování Netflixu. Nechceme opravdu pivo. Nechceme opravdu
vysoký zisk na trhu. Chceme radost. Chceme svobodu. Chceme žít život, který stojí za žití. Vaše uši mohou být hluché
k tichému úpění vašeho srdce k Bohu: „Dej mi radost!“ Ale přesto vás Bůh slyší a ptá se: „Proč ke mně úpíš?“

Bůh vám říká: „Táhněte dál.“ Jeho slova jsou těžko slyšitelná, když se díváte na moře. Ale nebojte se. Jeho plán je
větší, než můžete vnímat. Bude s vámi skrze to všechno. Ve vaší svaté hodině si dnes udělejte čas, abyste s Bohem
hovořili o svých přáních. Poslouchejte, když vám ukazuje to, co leží pod vašimi povrchními touhami. Naslouchejte,
když vám zjevuje, co skutečně přinese radost a dobro vašemu životu, vaší rodině a vaší farnosti.

% ===============================================
% ===== SESTY TYDEN
% ===============================================
%ukony
\newpage
\section*{Úkony (ukazatel cesty) pro 6. týden}

\textbf{Místo:} Severozápadní břeh Rudého moře

Faraon a jeho armáda opustili Egypt v neúprosném pronásledování Izraelitů. Jen co si Izraelité pomysleli, že jsou zcela na svobodě, ocitli se uvězněni mezi zuřící armádou a zdánlivě neprůchodným vodním útvarem. Ztratit zde naději by znamenalo vzdát se víry v Boha, a vyústění by vedlo k ještě tvrdšímu zotročení než kdy dříve. Nacházíme se na podobném místě. Naše dřívější zvyky na nás útočí. Pokud ztratíme víru, skončíme a otočíme se nazpět, budeme znovu zotročeni. Pokud však zůstaneme oddaní naší víře, Pán nás povede skrz vody. Očistí nás a oddělí nás od našich nepřátel jako nikdy předtím. Co si tento týden zvolíte?

\subsection*{1. Udělejte si čas na dobrou zpověď}
Pokud jste naposledy byli u zpovědi na začátku vašeho exodu nebo předtím, než začal, pak je toto ideální čas vrátit se zpět. Hřích „zraňuje a oslabuje samotného hříšníka, jakož i jeho vztahy k Bohu a k bližnímu“ (KKC 1459). Proto hřích, a co je důležitější – účinek hříchu – je přímo v protikladu s cíly Exodu 90. Vyhledejte, kdy se ve vaší farnosti zpovídá, a udělejte si čas na přijetí milostí, které vám Bůh chce dát vprostřed tohoto exodu.
\subsection*{2. Držte se reflexí}
Zůstaňte na této cestě v jednotě se svýmy bratry. Nepodvádějte sebe sama v tomto exodu. Držte se denních reflexí.
\subsection*{3. Vstupte do Slova}
Kristus je ono Slovo, které čtete ve vašich reflexích Písma každý den. Nečtěte pouze slova, vstupte do Slova. To Slovo žije. Je to člověk a můžete s ním hovořit. Člověk, který mluví přímo k vám. Vstupte do Slova každý den nasloucháním, co vám ve vaší životní situaci chce Pán říci.

\subsection*{Modlitba}
Modlete se, aby Pán osvobodil vás a vaše bratrství \newline
Modleme se za svobodu všech mužů v exodu, stejně tak, jako se oni modlí za vás.\newline
Ve jménu Otce i Syna i Ducha svatého … Otče náš… Ve jménu Otce i Syna i Ducha svatého … Amen.
\newpage
%newday
\newpage
\section{Den 36 - SÍLA KŘTU }
\zacatekSestyTyden
\subsection*{Čtení na den}
\textbf{Exodus 14,21-31}
\newline
\textit{
\textsuperscript{21}Mojžíš vztáhl ruku nad moře a Hospodin hnal moře silným východním větrem, který vál po celou noc, až proměnil moře v souš. Vody byly rozpolceny.
\textsuperscript{22}Izraelci šli prostředkem moře po suchu. Vody jim byly hradbou zprava i zleva.
\textsuperscript{23}Egypťané je pronásledovali a vešli za nimi doprostřed moře, všichni faraónovi koně, vozy i jízda.
\textsuperscript{24}Za jitřního bdění vyhlédl Hospodin ze sloupu ohnivého a oblakového na egyptský tábor a vyvolal v egyptském táboře zmatek.
\textsuperscript{25}Způsobil, že se uvolnila kola jejich vozů, takže je stěží mohli ovládat. Tu si Egypťané řekli: „Utečme před Izraelem, neboť za ně bojuje proti Egyptu Hospodin.“
\textsuperscript{26}Hospodin řekl Mojžíšovi: „Vztáhni ruku nad moře! Vody se obrátí na Egypťany, na jejich vozy a jízdu.“
\textsuperscript{27}Mojžíš vztáhl ruku nad moře, a když nastávalo jitro, moře opět nabylo své moci. Egypťané utíkali proti němu a Hospodin je vehnal doprostřed moře.
\textsuperscript{28}Vody se vrátily, přikryly vozy i jízdu celého faraónova vojska, které vešlo za Izraelci do moře. Nezůstal z nich ani jediný.
\textsuperscript{29}Ale Izraelci přešli prostředkem moře po suchu a vody jim byly hradbou zprava i zleva.
\textsuperscript{30}Onoho dne zachránil Hospodin Izraele z moci Egypta. Izrael viděl na břehu moře mrtvé Egypťany.
\textsuperscript{31}Tak uviděl Izrael velikou moc, kterou osvědčil Hospodin na Egyptu. Lid se bál Hospodina a uvěřili Hospodinu i jeho služebníku Mojžíšovi.
}

\subsection*{Reflexe}

Nemělo by být překvapením, že události dnešního čtení slouží jako prvotní obraz křtu. Izraelité, povolaní mít důvěru a
víru v Boha, kráčejí k vodě. Vstupují jako lidé utlačovaní Egyptem a vycházejí na druhé straně zachránění.

V křesťanském křtu nepokřtěný, povolaný k důvěře a víře v Boha, kráří k vodě ve křtitelnici. Vstupuje jako muž
utlačovaný sevřením hříchu a smrti, a vychází z vody zachráněný. To je možné díky Duchu svatému, který se vznáší
nad vodami (tak jako když byl sám Ježíš pokřtěn) a nově nás utváří. Když se díváme na událost dnešního čtení, vidíme
něco podobného: „Hospodin hnal moře silným východním větrem, který vál po celou noc…“ Vítr zde slouží jako obraz
Ducha svatého. Stejně jako v křesťanském křtu je to Duch svatý, vznášející se nad vodami, který umožní spásu
Izraelitů.

Rozdělení Rudého moře ukazuje moc Ducha svatého, tedy moc Boží. Tato událost způsobí bázeň Izraelitů před
Bohem, a jejich víru v Něj (srov. Ex 14,31). I dnes si Židé tuto spásonosnou událost připomínají jako důkaz velkého
Božího plánu s nimi. Vzhlíží k této události a připomínají si, že jsou Božími prvorozenými syny, a On je jejich Bohem.
Nepodceňujte svůj vlastní křest. Spíše jako Izraelité si tuto spásnou událost připomínejte a vzpomeňte si, že Bůh má
pro vás také svůj velký plán. Dovolte, aby váš křest sloužil jako ustavičná připomínka toho, že jste Božím synem, a On
je vaším Bohem.

Vezměme v potaz malé kropenky ve vchodech katolických kostelů po celém světě. Jsou ty jen proto, abyste
bezdůvodně namočili konečky svých prstů a zanechali kapky na košili? Ani náhodou. Pokaždé, když si smočíte prsty
v kropence, vzpomeňte si na milosti vašeho křtu. Užívejte svěcené vody v kropenkách k tomu, abyste si připomněli, že
Bůh má pro vás svůj velký Plán, a že jste Božím synem, a On je vaším Bohem. Jaký to dar.

Předneste dnes tyto tři věci – Boží plán pro vás, vaše synovství a božskost Boha – do vaší svaté hodiny. Pozvěte Pána,
aby vám o každé sdělil více a novým způsobem.

%newday
\newpage
\section{Den 37 - VDĚČNOST PŘINÁŠÍ HOJNOU RADOST}
\zacatekSestyTyden
\subsection*{Čtení na den}
\textbf{Exodus 15,1-21}
\newline
\textit{
\textsuperscript{1}Tehdy zpíval Mojžíš a synové Izraele Hospodinu tuto píseň. Vyznávali: „Hospodinu chci zpívat, neboť se slavně vyvýšil, smetl do moře koně i s jezdcem.
\textsuperscript{2}Hospodin je má záštita a píseň, stal se mou spásou. On je můj Bůh, a já ho velebím, Bůh mého otce, a já ho vyvyšuji.
\textsuperscript{3}Hospodin je bojovný rek; Hospodin je jeho jméno.
\textsuperscript{4}Vozy faraónovy i jeho vojsko svrhl v moře, v moři Rákosovém utonul výkvět jeho vozatajstva.
\textsuperscript{5}Tůně propastné je zavalily, klesli do hlubin jak kámen.
\textsuperscript{6}Tvá pravice, Hospodine, velkolepá v síle, tvá pravice, Hospodine, zdrtí nepřítele.
\textsuperscript{7}Nesmírně vyvýšen rozmetáš útočníky, vysíláš své rozhorlení, jako oheň strniště je pozře.
\textsuperscript{8}Dechem tvého chřípí počaly se kupit vody, příboje zůstaly stát jako hráze, sesedly se tůně propastné v klín moře.
\textsuperscript{9}Nepřítel si řekl: ‚Pustím se za nimi , doženu je , rozdělím kořist, ukojím jimi svou duši, meč vytasím, podrobí si je má ruka.‘
\textsuperscript{10}Zadul jsi svým dechem a moře je zavalilo, potopili se jak olovo v nesmírných vodách.
\textsuperscript{11}Kdo je mezi bohy jako ty, Hospodine? Kdo je jako ty, tak velkolepý ve svatosti, hrozný v chvályhodných skutcích , konající divy?
\textsuperscript{12}Vztáhl jsi pravici a země je pohltila.
\textsuperscript{13}Svým milosrdenstvím jsi vedl tento lid, který jsi vykoupil, provázel jsi jej svou mocí ke své svaté nivě.
\textsuperscript{14}Uslyšely o tom národy a zmocnil se jich neklid, bolest sevřela obyvatele Pelišteje.
\textsuperscript{15}Tehdy se zhrozili edómští pohlaváři, moábské vůdce zachvátilo chvění, všichni obyvatelé Kenaanu propadli zmatku.
\textsuperscript{16}Padla na ně hrůza a strach; pro velikost tvé paže zmlknou jako kámen, dokud, Hospodine, neprojde tvůj lid, dokud neprojde ten lid, který sis získal.
\textsuperscript{17}Přivedeš a zasadíš je na hoře svého dědictví, kde jsi, Hospodine, připravil své sídlo k přebývání, kde tvé ruce, Panovníku, svatyni si přichystaly.
\textsuperscript{18}Hospodin kraluje navěky a navždy.“
\textsuperscript{19}Když totiž faraónovi koně s jeho vozy a jízdou vešli do moře, Hospodin na ně obrátil mořské vody. Ale Izraelci šli po suchu prostředkem moře.
\textsuperscript{20}Tu vzala prorokyně Mirjam, sestra Áronova, do ruky bubínek a všechny ženy vyšly za ní s bubínky v tanečním reji.
\textsuperscript{21}A Mirjam střídavě s muži prozpěvovala: „Zpívejte Hospodinu, neboť se slavně vyvýšil, smetl do moře koně i s jezdcem.“
}

\subsection*{Reflexe}
Konečně, po tak dlouhém konfliktu, byl izraelský lid vykoupen z vazeb fyzického otroctví. Ohromeni radostí propukají Mojžíš i
prorokyně Mirjam ve zpěv. Zpívají novou píseň. Vidíme to v celém Starém Zákoně, a dokonce i v knize Zjevení, následující
Kristovo vítězství na kříži: „A zpívali novou píseň: ‚Jsi hoden přijmout tu knihu a rozlomit její pečetě…‘“ (Zjev 5,9).

Zde v Exodu Mojžíš zpívá: „Kdo kromě Pána mi přinese vysvobození?“ Bůh vysvobodil svůj lid z fyzické nadvlády a vlivu mnoha
bohů Egypta. Když dnes oslavujeme něco tak pomíjivého jako sportovní mistrovství, o co víc bychom měli oslavovat něco tak
velkého a trvalého, jako je křestní svobodaTo je věčné vítězství, které stojí za to slavit.

Buďte člověkem vděčnosti. Udělejte si chvilku času, abyste poděkovali Pánu za to, že vás dovedl až sem. Poděkujte mu za dar vaší
rodiny. Poděkujte Pánu za touhu být lepším člověkem. Poděkujte Pánu za to, že můžete dýchat. Nehledě na všechno, na co si
stěžujeme, toho máme mnohem víc, za co můžeme být vděční. Zaspívejte dnes Bohu novou píseň. Naučte se s Mojžíšem a
Izraelity být člověkem vděčnosti. Neboť v životě vděčnosti najdete velkou svobodu a hojnou radost.


%newday
\newpage
\section{Den 38 - ČEKÁNÍ NA LAVIČCE – Jak sladké!}
\zacatekSestyTyden
\subsection*{Čtení na den}
\textbf{Exodus 15,22-27}
\newline
\textit{
\textsuperscript{22}Mojžíš vedl Izraele od Rákosového moře dál. Vyšli na poušť Šúr a táhli pouští po tři dny, aniž narazili na vodu.
\textsuperscript{23}Došli až do Mary, ale nemohli vodu z Mary pít, protože byla hořká. Pojmenovali ji proto Mara (to je Hořká) .
\textsuperscript{24}Tu lid proti Mojžíšovi reptal: „Co budeme pít?“
\textsuperscript{25}Mojžíš úpěl k Hospodinu a Hospodin mu ukázal dřevo. Když je hodil do vody, voda zesládla. Tam dal Hospodin lidu nařízení a právní ustanovení a podrobil jej tam zkoušce.
\textsuperscript{26}Řekl: „Jestliže opravdu budeš poslouchat Hospodina, svého Boha, dělat, co je v jeho očích správné, naslouchat jeho přikázáním a dbát na všechna jeho nařízení, nepostihnu tě žádnou nemocí, kterou jsem postihl Egypt. Neboť já jsem Hospodin, já tě uzdravuji.“
\textsuperscript{27}Pak přišli do Élimu. Tam bylo dvanáct vodních pramenů a sedmdesát palem. Tam při vodách se utábořili.
}

\subsection*{Reflexe}
Izraelité, vysvobozeni ze svého fyzického otroctví, užívající si prvních dnů své svobody, se najednou nachází uprostřed
nemilosrdné pouště bez vody. Jako se stane ještě mnohokrát v budoucnu, jejich zoufalství nad Boží prozřetelností je
dovede k tlaku na Mojžíše, který má okamžitě vyřešit jejich nouzi. Přicházejí k vodě z Mary, ale nemohou ji pít,
protože je hořká. Aby Bůh tento problém vyřešil, ukazuje Mojžíši dřevo. Ano, dřevo.

Když následujete Izraelity na jejich cestě, snad jste začali poznávat, že mnoho tajemství Starého Zákona je téměř
nesrozumitelných, kromě Krista. Dnešní scénář onoho dřeva je zářným příkladem. V Novém Zákoně Kristus ukazuje
apoštolům také dřevo – kříž – a pověří je, aby jej vnášeli do životů všech národů. Tento úkon, který pokračuje již po
dvě tisíciletí, činí naše životy – se všemi břemeny a zkouškami, které se mohou zdát hořké – sladké a lehké.

Představte si sami sebe sedící venku na lavičce v nepříjemné zimě. Lavička je přišroubována k zemi na husté a rušné
ulici. Jako vždy čekáte na vašeho kolegu, aby vás vyzvedl cestou do práce. Aniž byste pomysleli na kříž, rozhodli jste
se, že budete naštvaní a cyničtí.

Nyní si sami sebe představte na podobné lavičce venku v zimě. Tentokrát je však lavička zavěšená na laně a veze vás
na vrchol hory KT-22. Sněžná bouře vámi háže. Většina lidí by to považovala za „nepříznivé počasí“, ale vy přetékáte
radostí a vděčností. Vlastně jste dokonce zaplatili za to, abyste mohli být v těchto podmínkách. Proč? Neboť pokud
vydržíte, dostanete příležitost vychutnat si skvělou lyžovačku na Squaw Valley. Jako lyžař si dokážete vychutnat jízdu
na sedačkové lanovce. Proto je i tento hořký moment vlastně sladký.
Prostřednictvím cvičení Exodu 90 se vaše čekání na lavičce stává stejně tak sladkým jako jízda lyžaře na lanovce.
Asketická cvičení vám pomůžou k tomu, abyste dokázali vidět příležitosti, které se skrývají za nepříznivým počasím.
Pouze takto dokážete vidět za sezením na studené lavičce možnost sjednotit nepohodlí a nedostatek kontroly
s utrpením kříže.

Toto utrpení pak může být vzkříšenov modlitbě skrze kříž, pro vás, vaši rodinu, ba dokonce i pro vašeho opozdilého
kolegu, stejně jako dřevo hozené do vody z Mary, které z hořké vody udělá sladkou. Také kříž, vhozený do
protivenství, činí i hořké utrpení sladkým.


%newday
\newpage
\section{Den 39 - OTROCTVÍ SRDCE}
\zacatekSestyTyden
\subsection*{Čtení na den}
\textbf{Exodus 16,1-3}
\newline
\textit{
\textsuperscript{1}Pak vytáhli z Élimu. Celá pospolitost Izraelců přišla na poušť Sín, která je mezi Élimem a Sínajem, patnáctý den druhého měsíce poté, co vyšli z egyptské země.
\textsuperscript{2}Celá pospolitost Izraelců na poušti opět reptala proti Mojžíšovi a Áronovi.
\textsuperscript{3}Izraelci jim vyčítali: „Kéž bychom byli zemřeli Hospodinovou rukou v egyptské zemi, když jsme sedávali nad hrnci masa, když jsme jídávali chléb do sytosti. Vždyť jste nás vyvedli na tuto poušť, jen abyste celé toto shromáždění umořili hladem.“
}

\subsection*{Reflexe}

Může být snadné dívat se na přední postavy Písma a cítit zvláštní pocit nadřazenosti nad nimi v jejich selháních.
Znalost těchto příběhů je karikatizuje a přináší pocit oprávněných předsudků. Adam zhřeší, odsoudí lidstvo k smrti a
my se ptáme, jak jen mohl být tak slepý. David se dopuští cizoložství a my skandalizujeme toto strašné selhání velkého
muže. Svatý Petr popírá Krista a my potupujeme jeho zbabělost.

Ve scéně z dnešního čtení je snadné se na tyto unavené, hladové, nepříjemné lidi dívat a pomyslet si: „Co je to za
rozmazlená děcka.“ Podívejme se na výsledky: Bůh je osvobodil od otroctví, rozdělil pro ně Rudé moře, utopil jejich
utlačovatele a učinil z hořké vody sladkou, aby ji mohli pít. Navzdory tomu všemu Izraelci kňučí. Horší je, že se chtějí
vrátit do otroctví, protože dostávali lepší jídlo a měli jistotu, že se někdo postará o jejich život.

V tomto 39. dni boje možná sympatizujete s požadavky Izraelitů. Status quo minulosti pro ně, a možná i pro vás,
vypadá až moc lákavě. Izraelci však ve svých srdcích jistě vědí, že status quo neznamená jen postel a pravidelnou
stravu. Také totiž znamená riziko zavraždění jejich dětí, protože se faraon cítí ohrožen. Znamená to nesvobodu
v uctívání jejich Boha. Znamená to vrátit se zpět do slávy nevlastní země, to vše na počest bohů, kterým nejsou ochotni
sloužit. Status quo je peklo.

V tomto okamžiku čelí Izraelité klíčovému rozhodnutí: vrátit se do status quo, nebo ho nechat sejít ze svých srdcí i
myslí. Svatý Jan Kassián říká: „Tělesné zřeknutí se a vyhoštění z Egypta pro nás nebude mít žádný význam, pokud
nebudeme současně schopni zřeknout se ho v srdci, což je vznešenější a přínosnější.“ Když pokračujete na této cestě a
vaše touhy vás lákají zpět ke statu quo otroctví, přijměte toto ultimátum: rozhodněte se nechat otroctví k neřestem sejít
ze svého srdce a mysli. Přizpůsobte svou vnitřní realitu vnější svobodě, ke které vás vede Bůh.

Svým plánem modlitby, askeze a bratrství vás Bůh přívedl k fyzickému oddělení od Egypta (od filmů, peněz, her atd.).
Máte však stále srdce stejné jako Izraelci? Je stále ještě v uchopení pekla, ze kterého vás Pán vyvedl?

Jsme jeden den od 40. dne. Všimněte si rozdílu mezi vaší vnější a vnitřní realitou. Důvod, proč toto cvičení netrvá 40,
ale na 90 dní, existuje. Je třeba udělat ještě více vnitřní práce. Leží před vámi větší svoboda, stejně jako pro Izraelity.
Poděkujte Pánu za vnější pokrok, který jste učinili, a následujte Jej i po 40. dni, v neúnavném úsilí o vnitřní svobodu až
do 91. dne.

Dnešek by neměl být zklamáním nad prací, kterou je ještě třeba udělat. Měl by to být den vděčnosti za dílo, které ve
vás Pán učinil. Dnes ve vaší svaté hodině přijměte tento dar radosti od Pána. Je na vás hrdý.



%newday
\newpage
\section{Den 40 - CHLÉB ŽIVOTA}
\zacatekSestyTyden
\subsection*{Čtení na den}
\textbf{Exodus 16,4-21}
\newline
\textit{
\textsuperscript{4}Hospodin řekl Mojžíšovi: „Já vám sešlu chléb jako déšť z nebe. Ať lid vychází a sbírá, co denně spotřebují. Tak je podrobím zkoušce, budou-li se řídit mým zákonem, či nikoli.
\textsuperscript{5}Když budou připravovat, co přinesou, ať je toho šestého dne dvakrát tolik, než co nasbírají každodenně.“
\textsuperscript{6}Mojžíš a Áron řekli všem Izraelcům: „Večer poznáte, že vás z egyptské země vyvedl Hospodin.
\textsuperscript{7}A ráno spatříte Hospodinovu slávu, ačkoli slyšel vaše reptání proti sobě. Co jsme my, že reptáte proti nám?“
\textsuperscript{8}Pak Mojžíš dodal: „Poznáte to podle toho, že vám Hospodin dá večer k jídlu maso a ráno k nasycení chléb, ačkoli slyšel reptání, jak jste proti němu reptali. Co jsme my? Nereptáte proti nám, ale proti Hospodinu.“
\textsuperscript{9}Áronovi Mojžíš řekl: „Vyzvi celou pospolitost Izraelců: ‚Přistupte před Hospodina, neboť slyšel vaše reptání.‘“
\textsuperscript{10}Když mluvil Áron k celé pospolitosti Izraelců, obrátili se k poušti, a vtom se ukázala v oblaku Hospodinova sláva.
\textsuperscript{11}Tu Hospodin promluvil k Mojžíšovi:
\textsuperscript{12}„Slyšel jsem reptání Izraelců. Vyhlas jim: ‚Navečer se najíte masa a ráno se nasytíte chlebem, abyste poznali, že já jsem Hospodin, váš Bůh.‘“
\textsuperscript{13}Když pak nastal večer, přiletěly křepelky a snesly se na tábor. A ráno padala kolem tábora rosa.
\textsuperscript{14}Když rosa přestala padat, hle, na povrchu pouště leželo po zemi cosi jemně šupinatého, jemného jako jíní.
\textsuperscript{15}Když to Izraelci viděli, říkali jeden druhému: „Man hú?“ (To je: „Co je to?“) Nevěděli totiž, co to je. Mojžíš jim řekl: „To je chléb, který vám dal Hospodin za pokrm.
\textsuperscript{16}Hospodin přikázal toto: Nasbírejte si ho každý, kolik potřebujete k jídlu. Každý vezmete podle počtu osob ve svém stanu ómer na hlavu.“
\textsuperscript{17}Izraelci tak učinili a nasbírali někdo více, někdo méně.
\textsuperscript{18}Pak odměřovali po ómeru. Ten, kdo nasbíral mnoho, neměl nadbytek, a kdo nasbíral málo, neměl nedostatek. Nasbírali tolik, kolik každý k jídlu potřeboval.
\textsuperscript{19}Mojžíš jim řekl: „Nikdo ať si nenechává nic do rána!“
\textsuperscript{20}Ale oni Mojžíše neposlechli a někteří si něco do rána nechali. To však zčervivělo a páchlo. Mojžíš se na ně rozlítil.
\textsuperscript{21}Sbírali to tak ráno co ráno, kolik každý k jídlu potřeboval. Když však začalo hřát slunce, rozpustilo se to.
}

\subsection*{Reflexe}
Izraelité putující po poušti, zcela závislí na Boží prozřetelnosti, k Němu volají kvůli potravě. Když Bůh naslouchá jejich modlitbě,
velkoryse reaguje manou z nebe, která se zázračně objevuje v ranní rose. Nezní to povědomě? Kolikrát jsme slyšeli
v eucharistické modlitbě při mši svaté: „Sešli rosu svého Ducha také na tyto dary a posvěť je…“?

Stejně jako Izraelité se každý z nás ocitá v poušti života. Také my voláme k Bohu za potravu. Bez výjimky nám Bůh dává všechno,
co potřebujeme. Kristus slibuje: „Já jsem chléb života. Vaši otcové jedli na poušti manu, a zemřeli. Toto je chléb, který sestupuje z
nebe: kdo z něho jí, nezemře,“ (J 6,48-50). Izraelci jedli v poušti chléb daný Bohem. Stejně tak se při mši modlíme, aby Bůh poslal
svého Ducha, aby se chléb a víno staly tělem a krví Ježíše Krista pro naši věčnou výživu.

Když chléb sestoupil z nebe, Izraelci byli ohromeni a ptali se: „Co je to?“ Také my se nacházíme ve stejném údivu. Mohl by
Kristus ve skutečnosti odevzdat sám sebe cele pro naši spásu? Bez many by Izraelité nebyli schopni přežít. Můžeme skutečně žít
bez Eucharistie? Vezměte tuto zásadní otázku do své svaté hodiny.



%newday
\newpage
\section{Den 41 - BŮH MYSLÍ SABAT VÁŽNĚ }
\zacatekSestyTyden
\subsection*{Čtení na den}
\textbf{Exodus 16,22-36}
\newline
\textit{
\textsuperscript{22}Šestého dne nasbírali toho chleba dvakrát tolik, totiž dva ómery na osobu. Tu přišli všichni předáci pospolitosti a oznámili to Mojžíšovi.
\textsuperscript{23}Ten jim řekl: „Toto praví Hospodin: Zítra je slavnost odpočinutí, Hospodinův svatý den odpočinku. Co je třeba napéci, napečte, a co je třeba uvařit, uvařte. A vše, co přebývá, uložte a opatrujte do rána.“
\textsuperscript{24}Uložili to tedy do rána, jak Mojžíš přikázal. A nezapáchalo to, ani se do toho nedali červi.
\textsuperscript{25}Mojžíš pak řekl: „Snězte to dnes, protože dnes je Hospodinův den odpočinku. Dnes nenajdete na poli nic.
\textsuperscript{26}Šest dní budete sbírat, ale sedmý den je den odpočinku. Ten den nebude nic padat .“
\textsuperscript{27}Když přesto někteří z lidu sedmého dne vyšli, aby sbírali, nic nenašli.
\textsuperscript{28}Hospodin řekl Mojžíšovi: „Jak dlouho se budete zpěčovat a nebudete dbát mých příkazů a řádů?
\textsuperscript{29}Hleďte, vždyť Hospodin vám dal den odpočinku. Proto vám dává šestého dne chléb na dva dny. Zůstaňte každý, kde jste, ať nikdo sedmého dne nevychází ze svého místa.“
\textsuperscript{30}Lid tedy sedmého dne odpočíval.
\textsuperscript{31}Dům izraelský pojmenoval ten pokrm mana. Byl jako koriandrové semeno, bílý, a chutnal jako medový koláč.
\textsuperscript{32}Mojžíš řekl: „Hospodin přikázal toto: Naplň tím ómer, aby to bylo opatrováno po všechna vaše pokolení, aby viděla chléb, kterým jsem vás na poušti živil, když jsem vás vyvedl z egyptské země.“
\textsuperscript{33}Áronovi Mojžíš řekl: „Vezmi jeden džbán, nasyp do něho plný ómer many a ulož to před Hospodinem, aby to bylo opatrováno po všechna vaše pokolení.“
\textsuperscript{34}Áron to tedy uložil před schránou svědectví, aby to bylo opatrováno, jak Hospodin Mojžíšovi přikázal.
\textsuperscript{35}Izraelci jedli manu po čtyřicet let, dokud nepřišli do země, v níž se měli usadit; jedli manu, dokud nepřišli na pokraj kenaanské země.
\textsuperscript{36}Ómer je desetina éfy.
}

\subsection*{Reflexe}
Šestého dne je Izraelitům řečeno, aby shromáždili dostatek many na dva dny, aby sedmého dne v sobotu mohli
odpočívat. Tento příkaz odráží nařízení z Genesis: „A Bůh požehnal a posvětil sedmý den, neboť v něm přestal konat
veškeré své stvořitelské dílo,“ (Gen 2,3). Také předznamenává třetí z deseti přikázání (pomni, abys den sváteční světil),
které později přijde v knize Exodus.

Nejen že stvoření dodržuje Šabat, ale činí tak i Bůh. Všimněte si, že mana v sobotu není seslána. Někteří z Izraelitů
stále chtějí nalézt manu, ale není tu žádná ke sběru. Sobora je Pánu svatá, je to jeho znamení smlouvy s Adamem a On
sám bere tento den se slavnostním odpočinkem. To nám říká, že i dnes v nedělním odpočinku spočíváme úmyslně
s Bohem.

Padlý člověk si vybudoval svět, který jej udržuje tak zaneprázdněného, že nemá žádnou chvíli na to, aby se setkal
s Bohem – nebo s kýmkoli jiným. Zvláště v neděli. Zdá se, že většina lidí si volí neděli jako den práce na zahradě nebo
nákupu potravin. Bůh si však přeje něco jiného. Čtěte toto Písmo pozorně. Bůh přikazuje Izraelitům, aby nechodili na
pole (nedělali práci na zahradě), ani nesbírali manu (nenakupovali potraviny) v sobotu. To jsou jen dva z mnoha
příkladů.

Dr. Scott Hahn poukazuje na to, že v První knize Mojžíšově byl stvořen člověk i šelmy šestého dne, ale člověk byl
stvořen pro den sedmý. Když člověk o sedmý den pracuje, padá zpět do šestého dne a není o nic lepší než šelma. Jste
zotročeni k tělesným impulzům a lidským plánům? Je aší ženě každou neděli s radostí připomínáno, že se provdala za
muže, a nebo se již dívá jen na šelmu?

Učení o odpočinku soboty je pro lidi velmi těžké slyšet. Bůh však učinil Jeho vůli pro nás jasou skrze Jeho Slovo.
Pohovořte dnes s Bohem o vlastní věrnosti odpočinku v neděli. Buďte připraveni slyšet nepříjemnou pravdu od Toho,
který chce vaši úplnou svobodu. Věřte. Bůh vám poskytne potřebnou milost, abyste mohli plnit Jeho příkazy.


%newday
\newpage
\section{Den 42 - PÍSMO VE VÁS}
\zacatekSestyTyden
\subsection*{Čtení na den}
\textbf{Exodus 17,1-7}
\newline
\textit{
\textsuperscript{1}Celá pospolitost synů Izraele táhla z pouště Sínu od stanoviště ke stanovišti podle Hospodinova rozkazu. Utábořili se v Refídimu, ale lid neměl vodu k pití.
\textsuperscript{2}Tu se lid dostal do sváru s Mojžíšem a naléhali: „Dejte nám vodu, chceme pít!“ Mojžíš se jich zeptal: „Proč se se mnou přete? Proč pokoušíte Hospodina?“
\textsuperscript{3}Lid tam žíznil po vodě a reptal proti Mojžíšovi. Vyčítali: „Proto jsi nás vyvedl z Egypta, abys nás, naše syny a stáda umořil žízní?“
\textsuperscript{4}Mojžíš úpěl k Hospodinu: „Jak se mám vůči tomuto lidu zachovat? Taktak že mě neukamenují.“
\textsuperscript{5}Hospodin Mojžíšovi řekl: „Vyjdi před lid. Vezmi s sebou některé z izraelských starších. Také hůl, kterou jsi udeřil do Nilu, si vezmi do ruky a jdi.
\textsuperscript{6}Já tam budu stát před tebou na skále na Chorébu. Udeříš do skály a vyjde z ní voda, aby lid mohl pít.“ Mojžíš to udělal před očima izraelských starších.
\textsuperscript{7}To místo pojmenoval Massa a Meriba (to je Pokušení a Svár) podle sváru Izraelců a proto, že pokoušeli Hospodina pochybováním: „Je mezi námi Hospodin, nebo není?“
}

\subsection*{Reflexe}
“Je mezi námi Hospodin, nebo není?“ Dnes se i nadále ptáme stejnou otázkou, právě jako se jí ptali lidé po celá staletí.

Tak často žijeme způsobem, jako by Bůh neexistoval. Zřídka rozjímáme Jeho přítomnost nebo moudrost, málokdy Jej
vezmeme v potaz nebo děkujeme a chválíme Jej za všechny milosti v našem životě. Poté, když něco dopadne špatně, je
až příliš jednoduché obviňovat Boha a se spravedlivým rozhořčením Jej zatratit kvůli tomu, že nevnímáme Jeho
přítomnost nebo činnost.

Dnes dojdou Izraelci tak daleko, že zkoušejí Boha, kladou na Něj nároky a jsou jen kousek toho, aby Ho prokleli za
jejich neštěstí. Aby jim prokázal svou péči a prozřetelnost, přikazuje Mojžíšovi, aby udeřil do skály Áronovou holí. Při
tom se ze skály vyleje voda. Bůh nás obdaroval stejným zázrakem. Když Ježíš Kristus visel na kříži, jeden z vojáků ho
kopím „udeřil“, probodl bok, a okamžitě vyšla krev a voda. Kristus je ona skála, ze které se nám vylila voda a krev.
Bůh je laskavý dárce. „Je mezi námi Hospodin, nebo není?“ Odpověď zní ano. Není jen mezi námi, je v nás.

Na rozdíl od Izraelitů máme my, pokřtění křesťané, v sobě Ducha Krista, který v nás doslova přebývá. Vše, co
potřebujeme, nám již bylo dáno. Kristus jasně říká: „Kdo věří ve mne, ‚proudy živé vody poplynou z jeho nitra,‘ jak
praví Písmo,“ (J 7,38). Máte žízeň? Duch Krista z vás prýští. Napijte se dnes z vody, která dává život, a začněte z ní
žít. Popovídejte si dnes s Pánem o tom, jak se k ní dostat, a jak pít a žít z této vody.



% ===============================================
% ===== SEDMY TYDEN
% ===============================================
%ukony
\newpage
\section*{Úkony (ukazatel cesty) pro 7. týden}

\textbf{Místo:} Drsná hornatá poušť, úpatí hory Sinaj

Dar znovunalezené svobody byl pro Izraelity vybojován v hlubinách Rudého moře. Nyní je Pán dovedl až
k úpatí hory Sinaj. Na vrcholu této hory promluví s Mojžíšem a dá mu návod, se kterým bude schopný udržet
sebe a Izraelity už navždy svobodné.

Vytrvali jste až do 7. týdne. To znamená, že jste v polovině tohoto duchovního cvičení. Gratulujeme. Odpoutaní
od model vašeho předchozího života jako nikdy předtím pokračujete s návodem ke svobodě v rukách: modlitba,
askeze a bratrství. Pokud tento návod nyní odmítnete, odmítnete samotný dar svobody. Stále je před námi
mnoho práce. Váš starý život může být za vámi, ale nové zvyky/návyky stále potřebují dostatek času, aby se
vytvořily.
\subsection*{1. Ctěte svou svatou hodinu}
Mnoho věcí vám chce Bůh sdělit. Touží s vámi sdílet Jeho samého. Dávejte mu stále denně čas v modlitbě. Nacházet výmluvy a říkat, že není hodný našeho času, znamená urážet samotného Boha. Zcela nás miluje a zaslouží si od nás mnohem víc než pouhé výmluvy.
\subsection*{2. Běžte ke zdroji}
„Eucharistie je zdrojem a vrcholem celého křesťanského života,“ (KKC 1324). Jak se vám vedlo v docházení na jednu další mši svatou během týdne? Pokud jste Ježíše Krista v Eucharistii nepotkávali jednou do týdně častěji, zeptejte se sami sebe, proč. Bůh sám dává vysvobození. Udělejte si dnes čas ve svém týdenním rozvrhu k navštívení další mše.
\subsection*{3. Přistupujte k vašemu nočnímu examen zodpovědně}
Pokud se vám vedlo praktikovat noční examen (zkoumání dne), uvidíte ve svém životě svobodu a úspěch, kterým vás Bůh obdařil. Pokud jste své noční examen nepraktikovali, váš osobní úspěch 91. dne vypadá bledě. Začněte zkoumat svůj den každou noc a důkladně. (Vysvětlení, jak dělat noční examen, můžete nalézt v Průvodci terénem.)
\subsection*{4. Zůstaňte radostní}
Minulý týden vás Hospodin vysvobodil z Egypta cestou Rudého moře. Tento týden se nacházíte uprostřed devadesátidenní cesty. Můžete děkovat za mnoho. Můžete mít z mnoha věcí radost. Žijte v této radosti, takže i ostatní uvidí a přijdou na to, že Bůh aktivně dává lidem vysvobození.

\subsection*{Modlitba}
Modlete se, aby Pán osvobodil vás a vaše bratrství \newline
Modleme se za svobodu všech mužů v exodu, stejně tak, jako se oni modlí za vás.\newline
Ve jménu Otce i Syna i Ducha svatého … Otče náš… Ve jménu Otce i Syna i Ducha svatého … Amen.
\newpage


%newday
\newpage
\section{Den 43 - UDĚLÁ I TO SMĚŠNÉ }
\zacatekSedmyTyden
\subsection*{Čtení na den}
\textbf{Exodus 17,8-16}
\newline
\textit{
\textsuperscript{8}Tu přitáhl Amálek, aby v Refídimu bojoval s Izraelem.
\textsuperscript{9}Mojžíš rozkázal Jozuovi: „Vyber nám muže a vyjdi do boje proti Amálekovi. Já se zítra postavím na vrchol pahorku s Hospodinovou holí v ruce.“
\textsuperscript{10}Jozue učinil, jak mu Mojžíš rozkázal, a dal se s Amálekem do boje. Mojžíš, Áron a Chúr vystoupili na vrchol pahorku.
\textsuperscript{11}Dokud Mojžíš držel ruku nahoře, vítězil Izrael, když ruku spustil, vítězil Amálek.
\textsuperscript{12}Když Mojžíšovi umdlévaly ruce, vzali kámen a podložili jej pod Mojžíše , aby se na něj posadil. Áron a Chúr, každý z jedné strany, mu podpírali ruce, takže vytrval s rukama nahoře až do západu slunce.
\textsuperscript{13}I porazil Jozue Amáleka a jeho lid ostřím meče.
\textsuperscript{14}Hospodin řekl Mojžíšovi: „Zapiš na památku do knihy a předej Jozuovi, že zcela vymažu zpod nebes památku na Amáleka.“
\textsuperscript{15}I vybudoval Mojžíš oltář a pojmenoval jej: ‚Hospodin je má korouhev.‘
\textsuperscript{16}Řekl totiž: „Je vztažena ruka nad Hospodinovým trůnem. Hospodin vyhlašuje boj proti Amálekovi do posledního pokolení.“
}

\subsection*{Reflexe}

Amálek, což doslova znamená „hříšní lidé“, odmítá Izraelitům projít směrem k jejich zaslíbené zemi. Postavili
se na odpor Boží vůli. Jsou vůči Boží vůli tvrdohlaví a čelí Izraelitům na základě jejich oprávnění a pýchy. Zde
nám svatý Augustin předkládá svou moudrost: pýcha „se stává překážkou věcem vyšším, a spojuje nás s věcmi
nižšími.“ Jak můžete toto aplikovat do svého současného života?

Vyzbrojen svatou ctností pokory Mojžíš sleduje bitvu s rukama nahoře, aby zajistil vítězství Izraelitů, zatímco
Jozue shromažďuje vojsko. Amalék jde prudce do války. Tento hříšný lid má schopnost porazit Izraelity. Ale
přesto nezvítězí. Proč? Protože Mojžíš je připraven udělat i to směšné. Má dost pokory na to, aby se poddal
Boží vůli, nehledě na to, jak bláznivě nebo slabě vypadá před ostatními. Mojžíš nechává své ruce zvednuté
k Bohu. Na druhé straně Bůh zajistí velké vítězství svého lidu.

Pokud se vyzbrojíte ctností pokory, Boží vítězství skrze vás je v podstatě nevyhnutelné. Pokud se však necháte
ochromit myšlenkou, že vás vaše síla a vůle dovedou k vítězství, váš osud bude stejný jako Amalečanů. Co
z toho, co po vás Pán žádá, je pro vás moc směšné, než abyste to učinili? Pohovořte si s Ním dnes o tom.
Možná vám ukáže něco, co jste pro svou pýchu nebyli schopni vidět.


%newday
\newpage
\section{Den 44 - PRAVDA O SOBĚ SAMOTNÉM }
\zacatekSedmyTyden
\subsection*{Čtení na den}
\textbf{Exodus 18,1-27}
\newline
\textit{
\textsuperscript{1}Jitro, midjánský kněz, Mojžíšův tchán, uslyšel o všem, co Bůh učinil Mojžíšovi a svému izraelskému lidu, že Hospodin vyvedl Izraele z Egypta.
\textsuperscript{2}Tu vzal Jitro, Mojžíšův tchán, Siporu, Mojžíšovu manželku, kterou on poslal zpět ,
\textsuperscript{3}a dva její syny; první se jmenoval Geršóm (to je Hostem-tam) , neboť Mojžíš řekl: „Byl jsem hostem v cizí zemi,“
\textsuperscript{4}a druhý se jmenoval Elíezer (to je Bůh-je-pomoc) , neboť řekl : „Bůh mého otce je má pomoc, vysvobodil mě od faraónova meče.“
\textsuperscript{5}Jitro, Mojžíšův tchán, přišel k němu s jeho syny a manželkou na poušť, kde on tábořil, k hoře Boží.
\textsuperscript{6}Vzkázal Mojžíšovi: „Já, tvůj tchán Jitro, jsem přišel k tobě, i tvoje manželka a s ní oba její synové.“
\textsuperscript{7}Mojžíš tedy vyšel svému tchánovi vstříc, poklonil se a políbil ho. Popřáli si navzájem pokoj a vešli do stanu.
\textsuperscript{8}Mojžíš vypravoval svému tchánovi o všem, co Hospodin kvůli Izraeli učinil faraónovi a Egypťanům, a o všech útrapách, které je potkaly na cestě, a jak je Hospodin vysvobodil.
\textsuperscript{9}Jitro měl radost ze všeho dobrého, co Hospodin Izraeli prokázal, a že jej vysvobodil z moci Egypta.
\textsuperscript{10}Řekl: „Požehnán buď Hospodin, že vás vysvobodil z moci Egypta a z ruky faraónovy, že vysvobodil tento lid z područí Egypta.
\textsuperscript{11}Nyní jsem poznal, že Hospodin je větší než všichni bohové; odplatil jim podle toho, jak se vypínali nad Izraele.“
\textsuperscript{12}Jitro, tchán Mojžíšův, pak připravil Bohu zápalnou oběť a obětní hod. Áron a všichni izraelští starší přistoupili, aby s Mojžíšovým tchánem pojedli před Bohem chléb.
\textsuperscript{13}Nazítří se Mojžíš posadil, aby soudil lid. Lid musel stát před Mojžíšem od rána do večera.
\textsuperscript{14}Mojžíšův tchán se díval na celé jeho jednání s lidem a řekl: „Jakým způsobem to s lidem jednáš? Proč sám sedíš a všechen lid kolem tebe stojí od rána do večera?“
\textsuperscript{15}Mojžíš tchánovi odpověděl: „Lid ke mně přichází dotazovat se Boha.
\textsuperscript{16}Když něco mají, přijde ta záležitost přede mne a já rozsoudím mezi oběma stranami; učím je znát Boží nařízení a řády.“
\textsuperscript{17}Mojžíšův tchán mu odpověděl: „Není to dobrý způsob, jak to děláš.
\textsuperscript{18}Úplně se vyčerpáš, stejně jako tento lid, který je s tebou. Je to pro tebe příliš obtížné. Sám to nezvládneš.
\textsuperscript{19}Poslechni mě, poradím ti a Bůh bude s tebou: Ty zastupuj lid před Bohem a přednášej jejich záležitosti Bohu.
\textsuperscript{20}Budeš jim vysvětlovat nařízení a řády a učit je znát cestu, po které mají chodit, i skutky, které mají činit.
\textsuperscript{21}Vyhlédni si pak ze všeho lidu schopné muže, kteří se bojí Boha, milují pravdu a nenávidí úplatek. Dosaď je nad nimi za správce nad tisíci, sty, padesáti a deseti.
\textsuperscript{22}Oni budou soudit lid, kdykoli bude třeba . Každou důležitou záležitost přednesou tobě, každou menší záležitost rozsoudí sami. Ulehči si své břímě, ať je nesou s tebou.
\textsuperscript{23}Jestliže se podle toho zařídíš, budeš moci obstát, až ti Bůh vydá další příkazy. Také všechen tento lid dojde na své místo v pokoji.“
\textsuperscript{24}Mojžíš svého tchána uposlechl a učinil všechno, co řekl.
\textsuperscript{25}Vybral schopné muže ze všeho Izraele a ustanovil je za představitele lidu, za správce nad tisíci, sty, padesáti a deseti.
\textsuperscript{26}Ti soudili lid, kdykoli bylo třeba ; obtížné záležitosti přednášeli Mojžíšovi a všechny menší záležitosti soudili sami.
\textsuperscript{27}Potom Mojžíš svého tchána propustil a ten se ubíral do své země.
}

\subsection*{Reflexe}
Mojžíš se osvědčil a nyní se těší respektu a úcty svého lidu. Lidé ho vnímají jako prostředníka, muže, který
mluví za samotného Boha. Mojžíš si vysloužil velkou autoritu. A jako jsme již viděli, jedná s pokorou a úctou.
Když Jitro, jeho tchán, vstoupí do tábora, srdečně se pozdraví a oslavují společně svobodu Izraelitů.

Jitro, muž sloužící jako kněz svému lidu, napomíná Mojžíše, varuje ho před úskalími a radí mu, aby ustanovil
soudní systém. Mojžíš nepovažuje zásah svého tchána za hrozbu nebo útok na svou autoritu. Člověk, který
toho tolik dosáhl, by takovou radu mohl odmítnout jako urážku, ale Mojžíš přijme tuto radu a jedná podle ní.
Tím dává všem lidem příklad pokory.

Pokora je hluboce nepochopená ctnost. Skutečná pokora znamená znát pravdu o sobě. Člověk tedy může být
současně veliký i pokorný. Ten, kdo zná své silné i slabé stránky, své schopnosti i svá omezení, a obzvlášť
svou náklonnost k hříchu, je pokorný. Jedná spravedlivě a nechlubí se. Ví, že mu bylo všechno dáno jako dar,
a to dar nezasloužený. Je si vědom toho, že je zloděj, pokud sklízí uznání za jakoukoli dobrou práci, protože
se pokouší ukrást slávu Bohu.

Zvažte, kdo jste jako člověk. Jste člověk velký? Pokorný? Požádejte Pána, aby promluvil do toho, kým jste.
Zná pravdu o vás, lépe než vy samotní. Jste jeho syny.


%newday
\newpage
\section{Den 45 - SMYSL POSVÁTNA}
\zacatekSedmyTyden
\subsection*{Čtení na den}
\textbf{Exodus 19,1-15}
\newline
\textit{
\textsuperscript{1}Třetího měsíce potom, co Izraelci vyšli z egyptské země, téhož dne, přišli na Sínajskou poušť.
\textsuperscript{2}Vytáhli z Refídimu, přišli na Sínajskou poušť a utábořili se v poušti; Izrael se tam utábořil naproti hoře.
\textsuperscript{3}Mojžíš vystoupil k Bohu. Hospodin k němu zavolal z hory: „Toto povíš domu Jákobovu a oznámíš synům Izraele:
\textsuperscript{4}Vy sami jste viděli, co jsem učinil Egyptu. Nesl jsem vás na orlích křídlech a přivedl vás k sobě.
\textsuperscript{5}Nyní tedy, budete-li mě skutečně poslouchat a dodržovat mou smlouvu, budete mi zvláštním vlastnictvím jako žádný jiný lid, třebaže má je celá země.
\textsuperscript{6}Budete mi královstvím kněží, pronárodem svatým. To jsou slova, která promluvíš k synům Izraele.“
\textsuperscript{7}Mojžíš přišel, zavolal starší lidu a předložil jim všechno, co mu Hospodin přikázal.
\textsuperscript{8}Všechen lid odpověděl jednomyslně: „Budeme dělat všechno, co nám Hospodin uložil.“ Mojžíš tlumočil odpověď lidu Hospodinu.
\textsuperscript{9}Hospodin řekl Mojžíšovi: „Hle, přijdu k tobě v hustém oblaku, aby lid slyšel, až s tebou budu mluvit, a aby ti provždy věřili.“ Mojžíš totiž Hospodinu oznámil slova lidu.
\textsuperscript{10}Hospodin dále Mojžíšovi řekl: „Jdi k lidu a dnes i zítra je posvěcuj; ať si vyperou pláště
\textsuperscript{11}a ať jsou připraveni na třetí den, neboť třetího dne sestoupí Hospodin před zraky všeho lidu na horu Sínaj.
\textsuperscript{12}Vymezíš kolem lidu hranici a řekneš: Střezte se vystoupit na horu nebo i dotknout se jejího okraje. Kdokoli se hory dotkne, musí zemřít;
\textsuperscript{13}nedotkne se ho žádná ruka, bude ukamenován nebo zastřelen. Ať je to dobytče nebo člověk, nezůstane naživu. Teprve až se dlouze zatroubí na roh, smějí na horu vystoupit.“
\textsuperscript{14}Mojžíš sestoupil z hory k lidu, posvětil lid a oni si vyprali pláště.
\textsuperscript{15}Řekl také lidu: „Buďte připraveni na třetí den; nepřistupujte k ženě.“
}

\subsection*{Reflexe}
Lid Izraele se připravuje na setkání s Bohem. Považte všechny přípravy, které musí podstoupit předtím, než se setkají
s Bohem jejich otců: praní prádla, abstinence a úcta. Lidem je připomínáno, že Bůh je zcela svatý, a je jim zakázáno se
Ho dotknout nebo jen přistoupit k hoře, na které bude Mojžíš mluvit s Bohem. Hora je tak posvátná, že jen dotknutí se
jí by přineslo smrt.

Dnes jsme ztratili smysl posvátna. Lidé jen ojediněle stojí před Hospodinem v úžasu a rozjímají nad svou
bezvýznamností před Ním. Pryč jsou dny, kdy lidé vcházeli do svatyní s velkou úctou. Ty dny, kdy lidé nevcházeli do
svatyně, aniž by se řádně oblékli, duchovně připravili, a činili tak s konkrétním smyslem. Zamyslete se na chvíli nad
vaším farním kostelem. Vcházíte do něj citlivě a s úctou? Přistupujete do svatyně zbožně, a staráte se, abyste se vhodně
oblékli a připravili na vstup do ní? Procházíte se svatyní, jako by to nebylo nic jiného než průchod na jiné místo v kostele?

Bůh říká Izraelitům, aby vyprali své pláště, a připravili se tak na Hospodina. Dnes přichází dospělí lidé na mši svatou
neupravení, v žabkách a kraťasech, džínách a triku nebo mikině. „Aspoň jsem tady,“ říkají pro svou obhajobu.

Katolík, který touží po synovském vztahu s Bohem Otcem, by měl před Pána předstupovat řádně připraven – a to zahrnuje
i základní vnější úpravu. Zeptejte se dnes Pána, jak byste Jej podle Jeho přání měli uctívat při mši svaté. Buďte ochotni
naplnit Boží požadavky, stejně jako Mojžíš a Izraelité. Bůh za to stojí.


%newday
\newpage
\section{Den 46 - ÚCTA K BOHU}
\zacatekSedmyTyden
\subsection*{Čtení na den}
\textbf{Exodus 19,16-25}
\newline
\textit{
\textsuperscript{16}Když nadešel třetí den a nastalo jitro, hřmělo a blýskalo se, na hoře byl těžký oblak a zazněl velmi pronikavý zvuk polnice. Všechen lid, který byl v táboře, se třásl.
\textsuperscript{17}Mojžíš vyvedl lid z tábora vstříc Bohu a postavili se při úpatí hory.
\textsuperscript{18}Celá hora Sínaj byla zahalena kouřem, protože Hospodin na ni sestoupil v ohni. Kouř z ní stoupal jako z hutě a celá hora se silně chvěla.
\textsuperscript{19}Zvuk polnice víc a více sílil. Mojžíš mluvil a Bůh mu hlasitě odpovídal.
\textsuperscript{20}Hospodin totiž sestoupil na horu Sínaj, na vrchol hory. Zavolal Mojžíše na vrchol hory a Mojžíš tam vystoupil.
\textsuperscript{21}Hospodin Mojžíšovi řekl: „Sestup a varuj lid, aby se nikdo ne pokoušel proniknout k Hospodinu ve snaze ho uvidět. Mnoho by jich padlo.
\textsuperscript{22}Také kněží, kteří přistupují k Hospodinu, se musí posvětit, aby se na ně Hospodin neobořil.“
\textsuperscript{23}Mojžíš řekl Hospodinu: „Lid nemůže vystoupit na horu Sínaj, neboť ty sám jsi nás varoval slovy: Vymez podél hory hranici a horu posvěť.“
\textsuperscript{24}Hospodin mu řekl: „Teď sestup, potom vystoupíš spolu s Áronem; ale kněží ani lid nesmějí proniknout a vystoupit k Hospodinu, aby se na ně neobořil.“
\textsuperscript{25}Mojžíš tedy sestoupil k lidu a řekl jim to.
}

\subsection*{Reflexe}
Písmo nám říká: „Začátek moudrosti je bázeň před Hospodinem a poznat Svatého je rozumnost,“ (Přís 9,10). Co si pomyslíte, když
to slyšíte? Až moc dlouho byl Bůh zobrazován kazateli a učiteli jako květinová a mírná postava, naprosto přístupný, nenahánějící
strach. Je to opačně. Bůh je naprosto nepochopitelný a všemocný. Člověk klame sám sebe, pokud považuje Boha za neškodného.

Řádná bázeň vyvolává úctu. Zkušení horolezci mají strach ze sněhu. Sníh se většině lidí zdá neškodný, ale pro horolezce je sněhová
lavina skutečnou možností smrti. Proto horolezci respektují sílu sněhu tím, že sledují počasí, testují sněhovou pokrývku a
odpovídajícím způsobem upravují své trekové plány. Navzdory tomu, co si lidé myslí, sníh zdaleka není neškodný. Stejně – a
mnohem víc – je to i s Bohem.

V dnešním úryvku jsou Izraelité zvukem polnice upozorněni na přítomnost Boha. Zvuk polnice se v Písmu svatém objevuje ze dvou
důvodů: svolat muže k bitvě, a svolat lid k modlitbě. Není ironií, že tato svolávání sdílí stejný signál. Troubení volá muže jak do boje
v poli, tak k modlitbě před Bohem.

Zamyslete se, jak často jste se modlívali před tímto duchovním cvičením. Pokud jste nevstupovali na bojiště modlitby denně, bylo
to kvůli nedostatku času, nebo proto, že jste se vlastně řádně nebáli Boha?

Začínáte ve zkušenosti exodu vidět sílu a moc Boha dát život a vzít ho zpět? Vidíte svou potřebu Boha uctívat? Nebo Ho stále vidíte
očima městského člověka, který vidí párkrát ročně neškodný sníh pomalu se snášející na město? Polnice zazněla. Vaše volba
(ne)přistupovat denně k modlitbě již beze slov odpověděla na tuto otázku.

Pokud jste do své svaté hodiny posledních 46 dní vstupovali dobrovolně a věrně, vzdejte Bohu velkou chválu. Vaše oči dobře vidí
Jeho moc a sílu. Pokud jste se závazkem modlitby během tohoto duchovního cvičení bojovali, předneste to Pánu. Dejte Mu šanci
podělit se s vámi o věčný přínos učení se bázni a Boží úcty. On je dobrý, a rozhovor s Ním bude stát za to.


%newday
\newpage
\section{Den 47 - DAR PŘIKÁZÁNÍ }
\zacatekSedmyTyden
\subsection*{Čtení na den}
\textbf{Exodus 20,1-17}
\newline
\textit{
\textsuperscript{1}Bůh vyhlásil všechna tato přikázání:
\textsuperscript{2}„Já jsem Hospodin, tvůj Bůh; já jsem tě vyvedl z egyptské země, z domu otroctví.
\textsuperscript{3}Nebudeš mít jiného boha mimo mne.
\textsuperscript{4}Nezobrazíš si Boha zpodobením ničeho, co je nahoře na nebi, dole na zemi nebo ve vodách pod zemí.
\textsuperscript{5}Nebudeš se ničemu takovému klanět ani tomu sloužit. Já jsem Hospodin, tvůj Bůh, Bůh žárlivě milující. Stíhám vinu otců na synech do třetího i čtvrtého pokolení těch, kteří mě nenávidí,
\textsuperscript{6}ale prokazuji milosrdenství tisícům pokolení těch, kteří mě milují a má přikázání zachovávají.
\textsuperscript{7}Nezneužiješ jména Hospodina, svého Boha. Hospodin nenechá bez trestu toho, kdo by jeho jména zneužíval.
\textsuperscript{8}Pamatuj na den odpočinku, že ti má být svatý.
\textsuperscript{9}Šest dní budeš pracovat a dělat všechnu svou práci.
\textsuperscript{10}Ale sedmý den je den odpočinutí Hospodina, tvého Boha. Nebudeš dělat žádnou práci ani ty ani tvůj syn a tvá dcera ani tvůj otrok a tvá otrokyně ani tvé dobytče ani tvůj host, který žije v tvých branách.
\textsuperscript{11}V šesti dnech učinil Hospodin nebe i zemi, moře a všechno, co je v nich, a sedmého dne odpočinul. Proto požehnal Hospodin den odpočinku a oddělil jej jako svatý.
\textsuperscript{12}Cti svého otce i matku, abys byl dlouho živ na zemi, kterou ti dává Hospodin, tvůj Bůh.
\textsuperscript{13}Nezabiješ.
\textsuperscript{14}Nesesmilníš.
\textsuperscript{15}Nepokradeš.
\textsuperscript{16}Nevydáš proti svému bližnímu křivé svědectví.
\textsuperscript{17}Nebudeš dychtit po domě svého bližního. Nebudeš dychtit po ženě svého bližního ani po jeho otroku ani po jeho otrokyni ani po jeho býku ani po jeho oslu, vůbec po ničem, co patří tvému bližnímu.“
}

\subsection*{Reflexe}
Izraelité právě unikli z let služby faraonovi a zásahem zázračné a mocné ruky Hospodinovy si užívají skvělý
dar a výsadu tělesné svobody. Ale zde se najednou zdá, že je Bůh chce okovy deseti přikázání o svobodu
připravit. Proč se to děje? Nejsou snad svobodní, schopní činit svá rozhodnutí a směřovat svůj vlastní osud?
Proč se zdá, že se Bůh povyšuje nad svůj lid?

Bůh tím, že člověka stvořil a obdařil ho svobodou, velmi riskoval. Přeci to však udělal z jednoho důvodu: aby
člověk měl schopnost milovat Ho. Bez svobody není člověk schopný lásky. Kdyby nás Bůh stvořil jako roboty
a naprogramoval nás pro lásku k Němu, náš vztah s Ním by nebyl úplný. Nebyli bychom schopni Ho milovat.

Bůh do lidské svobody mnoho vložil. Vyhrál svobodu Izraelitům a také nám – na kříži. Poslední věcí, kterou
by Bůh chtěl vidět, by byla ztráta naší svobody. Deset přikázání je tedy vlastně záruka od Boha, vzor, jak
svobodu Izraelitů – a nás – zachovat.

Zamyslete se nad přikázáními: každé je navrženo tak, abychom nebyli znovuzotročeni peklem. Kdyžněkteré
z těchto přikázání porušíme, sami se připravíme o pravou svobodu. Přikázání na nás nejsou uvalena
despotickým Bohem; jsou neskutečně láskyplným darem Izraelitům – a nám. Zákony Církve, které z těchto
přikázání pramení, slouží stejnému účelu. Proto nejsou přikázání a zákony okovy, ale dary, které nám pomáhají
svobodně a správně milovat Boha.

Zvažte, které z přikázání nebo zákona Boha a Jeho Církve se vám nedaří přijmout. Přinese to Pánu a požádejte
ho, aby vám ukázal, jak vám to má pomoci milovat Boha svobodně a správně. (Pokud se vám nedaří přijmout
pokoj nad touto odpovědí, předneste to svému bratrstvu a vašemu duchovnímu vůdci pro objasnění.)


%newday
\newpage
\section{Den 48 - SCHOPEN VĚTŠÍ LÁSKY }
\zacatekSedmyTyden
\subsection*{Čtení na den}
\textbf{Exodus 20,18-26}
\newline
\textit{
\textsuperscript{18}Všechen lid pozoroval hřmění a blýskání, zvuk polnice a kouřící se horu. Lid to pozoroval, chvěl se a zůstal stát opodál.
\textsuperscript{19}Řekli Mojžíšovi: „Mluv s námi ty a budeme poslouchat. Bůh ať s námi nemluví, abychom nezemřeli.“
\textsuperscript{20}Mojžíš lidu odpověděl: „Nebojte se! Bůh přišel proto, aby vás vyzkoušel, aby bylo zřejmé, že se ho budete bát a přestanete hřešit.“
\textsuperscript{21}Lid zůstal stát opodál a Mojžíš přistoupil k mračnu, v němž byl Bůh.
\textsuperscript{22}Hospodin řekl Mojžíšovi: „Toto řekneš synům Izraele: Viděli jste, že jsem s vámi mluvil z nebe.
\textsuperscript{23}Neuděláte si mé zpodobení , neuděláte si bohy stříbrné ani zlaté.
\textsuperscript{24}Uděláš mi oltář z hlíny a budeš na něm obětovat ze svého bravu a skotu své oběti zápalné i pokojné. Na každém místě, kde určím, aby se připomínalo mé jméno, přijdu k tobě a požehnám ti.
\textsuperscript{25}Jestliže mi budeš dělat oltář z kamenů, neotesávej je; kdybys je opracoval dlátem, znesvětil bys je.
\textsuperscript{26}Nebudeš vystupovat k mému oltáři po stupních, abys u něho neodkrýval svou nahotu.“
}

\subsection*{Reflexe}
Čtěte dnes Písmo pozorně: „Bůh přišel proto, aby vás vyzkoušel, aby bylo zřejmé, že se ho budete bát a přestanete hřešit.“ Proč
Hospodin přišel, aby mluvil s Mojžíšem na hoře Sinaj? Proč je zde hřmění, blýskání, zvuk polnice, kouř z hory, a dlouhý seznam
zákonů, které začínají dneškem? Protože Hospodin chce, aby Jeho lid žil správně. Chce, aby lidé žili svobodni od hříchu. Chce, aby
Ho milovali tolik, jak jsou schopni.

Čím více víme o někom jiném, tím lépe ho můžeme milovat. Čím častěji se Izraelité setkávají s Bohem v jejich nepřízni, tím Ho více
poznávají. Z toho vyplývá, že schopnost Izraelitů milovat Boha se zvyšuje s každým setkáním. Hospodin proto volá Izraelity k takové
úrovni lásky, které jsou schopni. To je vidět v dnešním čtení, kde Bůh připomíná svému lidu: „Viděli jste, že jsem s vámi mluvil z
nebe. Neuděláte si mé zpodobení, neuděláte si bohy stříbrné ani zlaté.“ Stejná myšlenka znalosti a lásky platí i pro náš vztah k Bohu.

Během svého života se o Pánu dozvídáte víc a víc. Co jste se dozvěděli formovalo způsob, jakým jste schopni milovat Boha. Tím,
co jste se dozvěděli o Ježíši Kristu, jste mohli milovat Boha Syna. To, co jste se naučili o Eucharistii, vás zase přimělo k tomu, abyste
mohli pokleknout před Synem přítomným v chlebu a vínu. Díky tomu, co jste se naučili o mystickém Kristově těle, jste zase mohli
volat k Synu skrze členy jeho Těla, společenství svatých. S tímto požehnáním přichází zodpovědnost. Čím více víme, čeho jsme více
schopni, tím víc budeme odpovědní.

Pro příklad – pokud člověk ví, že neděle je dnem svatým, určeným Bohu, potom je odpovědný za to, že podle toho bude žít. Muž
(hlava rodiny), který to ví a bere svou rodinu v neděli na mši svatou a poté ji učí slavit památku tohoto svátku, dobře prokazuje svou
lásku k Bohu. Na druhou stranu ten, kdo nedává mši na první místo a místo toho vezme svou rodinu v neděli do kempu, obchodu
nebo na sportovní utkání, zdaleka nedosahuje své schopnosti milovat HospodinaVe skutečnosti dokonce ani dost nemiluje svou
rodinu, protože je odděluje od pramene lásky (Boha) v den, který byl učiněn pouze k službě Jemu.

Během vaší dnešní kontemplace přemýšlejte nad svou vlastní schopností milovat Pána. Zeptejte se sami sebe, jestli milujete
Hospodina tak, jak jste schopní. Pak s Ním pohovořte. Zeptejte se Ho, jak Jej můžete lépe milovat. Sepište si závěry tohoto rozhovoru
a rozhodněte se milovat Pána, jak nejvíce jste schopni.


%newday
\newpage
\section{Den 49 - LÁSKA K BLIŽNÍMU VYŽADUJE AKCI}
\zacatekSedmyTyden
\subsection*{Čtení na den}
\textbf{Exodus 21,1-11}
\newline
\textit{
\textsuperscript{1}Toto jsou právní ustanovení, která jim předložíš:
\textsuperscript{2}Když koupíš hebrejského otroka, bude sloužit šest let; sedmého roku odejde jako propuštěnec bez výkupného.
\textsuperscript{3}Jestliže přišel sám, odejde sám, měl-li ženu, odejde jeho žena s ním.
\textsuperscript{4}Jestliže mu dal jeho pán ženu, která mu porodila syny nebo dcery, zůstane žena a její děti u svého pána, a on odejde sám.
\textsuperscript{5}Prohlásí-li otrok výslovně: „Zamiloval jsem si svého pána, svou ženu a syny, nechci odejít jako propuštěnec,“
\textsuperscript{6}přivede ho jeho pán před Boha, totiž přivede ho ke dveřím nebo k veřejím, probodne mu ucho šídlem a on zůstane provždy jeho otrokem.
\textsuperscript{7}Když někdo prodá svou dceru za otrokyni, nebude s ní nakládáno jako s jinými otroky.
\textsuperscript{8}Jestliže se znelíbí svému pánu, který si ji vzal za družku, dovolí ji vyplatit, ale nemá právo prodat ji cizímu lidu a naložit s ní věrolomně.
\textsuperscript{9}Jestliže ji dal za družku svému synovi, bude s ní jednat podle práva dcer.
\textsuperscript{10}Jestliže on si vezme ještě jinou, nesmí ji zkrátit na stravě, ošacení a manželském právu.
\textsuperscript{11}Jestliže jí nezajistí tyto tři věci, smí ona odejít bez zaplacení výkupného.
}

\subsection*{Reflexe}
Pro dnešního čtenáře mohou být biblické příkazy ohledně otroků velmi znepokojující. Spíše než házet obvinění na
Izraelity (a Bohasamého) z dnešního pohledu zvažte zcela jiné podmínky a kulturu Izraelitů. Život ve starověku byl tak
neuvěřitelně těžký, že otroctví bylo často přínosem spíše pro otroka než pro jeho pána. Otrok Izraelitů měl jídlo, přístřeší
a ochranu. Boží příkazy týkající se otroctví v dnešním čtení tedy byly zárukou základních lidských práv.

Více než to nám Boží nařízení dnešního čtení poukazují na něco zásadního o Bohu. Bůh se tolik stará o způsob, jakým
spolu lidé komunikují. V Novém Zákoně nám Ježíš Kristus přikazuje „milovat Boha z celého svého srdce“ a „milovat
svého bližního jako sám sebe“ (Mk 12,30-31).

Toto pravidlo je zásadní, ale vnímáte ji ve svém každodenním životě? Chováte se ke své ženě nebo spolufarníkům
láskyplně a s úctou? Vidíte potřeby svých dětí? Ctíte svého otce i matku, zejména ve stáří? Chováte se jako gentleman
v práci, na sedadle řidiče, i v soutěžních sportech? Posloucháte svého šéfa, učitele, faráře, biskupa? Berete odpovědnost
za lidi kolem vás, ve vašem okolí či ve městě? A nakonec, podporujete aktivně své bratry v Exodu, zejména svou kotvu?
Cítí vaši bratři, že se na vás mohou spolehnout a mít ve vás oporu?

Je tu mnoho, za co se zde můžete modlit. Nechte Pána, aby přišel do vaší sebereflexe.


\end{document}