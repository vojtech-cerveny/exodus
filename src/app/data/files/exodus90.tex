% ===============================================
% ===== SETTINGS FOR DOCUMENT
% ===============================================
\documentclass[11pt]{article}
\usepackage{geometry}
\geometry{
  top=3cm,
  bottom=3cm
}
\author{Exodus 90}
\usepackage[czech]{babel}
\font\myfont=cmr12 at 40pt
\setlength{\parskip}{1em plus 0.1em minus 0.2em} % Adjusts space between paragraphs
\usepackage{graphicx}
\usepackage{amssymb}
\usepackage{enumitem}
\setlist[enumerate]{itemsep=-0.4em} % Globálně nastavuje odsazení pro všechny seznamy
% Upravení prostoru okolo nečíslovaných podsekcí
\usepackage{titlesec}

% Upravení prostoru okolo nečíslovaných podsekcí
\titleformat*{\subsection}{\normalsize\bfseries}
\titlespacing*{\subsection}{0pt}{2.5ex minus 2ex}{1.3ex minus 1ex}

% ===============================================
% ===== CUSTOM COMMANDS
% ===============================================
\newcommand{\zacatekPrvniTyden}{
  Jste v Egyptě \newline
  Jste ochotni přijmout, že Egypt vás zotročil? „Ano“ bude vyžadovat úplnou změnu ve vašem způsobu života.

\subsection*{Úkony (ukazatelé cesty)}
\begin{enumerate}
  \item Vzdejte se kontroly
  \item Zavázejte se svému bratrstvu
  \item Najít si čas pro každodenní modlitbu
  \item Buďte radostní
  \item Každou noc zkoumejte svůj den
\end{enumerate}
Modlete se, aby Pán osvobodil vás a vaše bratrství. \newline
Modleme se za svobodu všech mužů v exodu, stejně tak, jako se oni modlí za vás.\newline
Ve jménu Otce i Syna i Ducha svatého …  Otče náš… Amen
}

\newcommand{\zacatekDruhyTyden}{
  Jste v Egyptě \newline
  Disciplíny duchovního cvičení zvětštily naši dříve nevědomou náklonnost k lidskému komfortu.
  Nebylo by snazší toto duchovní cvičení opustit a zůstat v pohodlí otroctví navěky?

  \subsection*{Úkony (ukazatelé cesty)}
\begin{enumerate}
  \item Dobře se vyzpovídejte
  \item Držte se denních reflexí
  \item Navštěvujte jednu mši v týdnu navíc
  \item Zvažte přečtení Průvodce terénem
  \item Uvědomte si (zjistěte), kde je vaše kotva
\end{enumerate}
Modlete se, aby Pán osvobodil vás a vaše bratrství. \newline
Modleme se za svobodu všech mužů v exodu, stejně tak, jako se oni modlí za vás.\newline
Ve jménu Otce i Syna i Ducha svatého …  Otče náš… Amen
}

\newcommand{\zacatekTretiTyden}{
  Jste v Egyptě \newline
  Život se stal náročnějším; Zdá se, že Mojžíši a Áronovi se nedaří získat větší svobodu. Navzdory tomu všemu,
komu se tento týden rozhodnete sloužit: Bohu, nebo faraonovi?

\subsection*{Úkony (ukazatelé cesty)}
\begin{enumerate}
  \item Pokračujte ve zkoumání svého dne
  \item Přistupujte upřímně k vaší denní svaté hodině
  \item Neodbývejte úkony/úkoly (don’t cut corners-neřežte rohy)
  \item Pamatujte si své proč
  \item Zůstaňte radostní
\end{enumerate}
Modlete se, aby Pán osvobodil vás a vaše bratrství. \newline
Modleme se za svobodu všech mužů v exodu, stejně tak, jako se oni modlí za vás.\newline
Ve jménu Otce i Syna i Ducha svatého …  Otče náš… Amen
}

% ===============================================
% ===== DOCUMENT BEGINS
% ===============================================

\title{\myfont Exodus 90 - Česká verze}
\date{}							% Activate to display a given date or no date

\begin{document}

\maketitle
\vspace*{\fill}
Pokud preferuješ tištěnou verzi Exodusu, pak jsi zde správě. Tato verze je připravená na tisk tak, aby se dobře četla. Užij si ji a i s ní i celý Exodus 90!

% ===============================================
% ===== PRVNI TYDEN
% ===============================================
%ukony
\newpage
\section*{Úkony (ukazatel cesty) pro 1. týden}

\textbf{Místo:} Egypt (jste v Egyptě)

Tento týden se ocitnete s Izraelity v Egyptě. Egyptské prostředí se za celá staletí stalo nepřátelským a utlačujícím, a přesto jste zůstali k otroctví slepí. Naštěstí byl Mojžíš pověřen, aby vám zvěstoval pravdu. Jste ochotni uznat, že vás Egypt zotročil? Řeknete-li „ano“, bude od vás požadována kompletní změna života.

\subsection*{1. Vzdejte se kontroly}
Disciplíny Exodus 90 vám poskytují příležitost vzdát se kontroly nad svým životem a předat ji Bohu. Učte se nově svěřovat kontrolu do Božích rukou. Jako by rytíř položil svůj meč na oltář a na oplátku by obdržel Boží moc, musíte udělat totéž.

\subsection*{2. Zavázat se k vašemu bratrstvu}
Tahle cesta je těžká. Budete vyzkoušeni a testováni. Budete potřebovat své bratry a oni budou potřebovat vás. Týdenní setkání jsou nutností. Modlete se jako Izraelité, kteří byli zachráněni jako kmen (a ne jednotlivě), aby Bůh vysvobodil vaše společenství a také všechny lidi, kteří čekají na vysvobození ze závislosti, sobectví, apatie a ovládání.

\subsection*{3. Najít si čas pro každodenní modlitbu}
Strávit hodinu času v modlitbě každý den. Pokud je to nemožné, trávit co nejvíce času, jak je to možné, s minimálně dvaceti minutami tiché modlitby denně. Je dobré si naplánovat konkrétní čas během dne, nebo toho pravděpodobně zanecháte.

\subsection*{4. Buďte radostní}
Přijali jste Kristův plán svobody. Ano, bude to těžké, ale to by vás nemělo zarmoutit. Spíše se těšte z naděje na svobodu, která vás čeká. Boží pozvání do tohoto duchovního cvičení by vám mělo přinést bohatou radost.

\subsection*{5. Každou noc zkoumejte svůj den}
Exodus 90 obsahuje mnoho disciplín, na které můžete každým dnem odpovídat „ano“. Na konci každého dne před usnutím si nalezněte čas na to, abyste si prošli a zkoumali svůj den. To vám pomůže nejen úspěšně přistupovat na jednotlivé disciplíny, ale také vám pomůže vidět váš pokrok, který jste udělali na cestě ke svobodě. (Jak na to se dozvíte v příručce Exodu v sekci „Jak se modlit noční examen“).Možná verze Examenu přímo zde.

\subsection*{Modlitba}
Modlete se, aby Pán osvobodil vás a vaše bratrství.

Modleme se za svobodu všech mužů v exodu, stejně tak, jako se oni modlí za vás.

Když se učedníci zeptali Ježíše, jak se mají modlit, naučil je „Otče náš“ (viz Lukáš 11: 1–4, Matouš 6: 9–13). Připojte se ke svým bratrům Exodu po celém světě a každý den se modlete tuto mocnou modlitbu za výše uvedené úmysly.

Ve jménu Otce i Syna i Ducha svatého …  Otče náš… Ve jménu Otce i Syna i Ducha svatého … Amen.

%newday
\newpage
\section{Den 1 - JEDINĚ BŮH PŘINÁŠÍ SVOBODU}
\zacatekPrvniTyden
\subsection*{Čtení na den}
\textbf{Exodus 1, 1-7}
\newline
\textit{
\textsuperscript{1}Toto jsou jména synů Izraelových, kteří přišli do Egypta s Jákobem; každý přišel se svou rodinou:
\textsuperscript{2}Rúben, Šimeón, Lévi a Juda,
\textsuperscript{3}Isachar, Zabulón a Benjamín,
\textsuperscript{4}Dan a Neftalí, Gád a Ašer.
\textsuperscript{5}Všech, kdo vzešli z Jákobových beder, bylo sedmdesát. Josef už byl v Egyptě.
\textsuperscript{6}Potom zemřel Josef a všichni jeho bratři i celé to pokolení.
\textsuperscript{7}Ale Izraelci se rozplodili, až se to jimi hemžilo, převelice se rozmnožili a byli velice zdatní; byla jich plná země.
}
\subsection*{Reflexe}
Začínáme náročnou cestu od otroctví ke svobodě rozjímáním nad úvodním odstavcem starověké knihy Exodus - příběhu cesty Božího lidu z Egyptského otroctví ke svobodě.

Na první pohled se můžeme podivovat nad tím, jak by nám právě tento úryvek mohl být užitečný. Ale nepodceňujme Boží slovo!

Vzhledem k tomu, že kniha Exodus je příběhem cesty Izraelitů z otroctví, mělo by nám připadat zvláštní, že Izraelité byli „mimořádně silní; takže země byla jimi naplněna. “
Jak se dozvíme při další četbě, egyptský faraon se dokonce Izraelitů bál. Jak je tedy možné, že byli Izraelité zotročeni, když byli „mimořádně silní“? Proč nepovstali a neosvobodili se? Jak to, že navzdory své velké síle zůstali zotročeni tyranem?

Kniha Exodus, jak uvidíme, je také naším příběhem. Tento starodávný text není jen historií Izraelitů. Je to také určitá metafora moderního muže. Jsme-li zotročení chtíčem, technologiemi, jídlem, pitím nebo něčím jiným, nacházíme se ve stejném otroctví. Ve skutečnosti to, že jsme zotročeni, neznamená, že jsme slabí.
Ve většině případů jsou naše mozky a těla ve skutečnosti docela silné.
Jenže je více než pravděpodobné, že právě síla je tím pravým důvodem, proč jsme se nechali zotročit.

Muži jsou silní, ale když se snaží vypořádat se s životem a jeho mnoha obtížemi, vrhnou se dychtivě po čemkoliv, co jim přinese útěchu a bezpečí. V průběhu svého života (a zejména v mladém věku) objeví a začnou využívat věci nebo činnosti, o nichž si myslí, že je učiní šťastnými. Užívají těchto věcí, protože je vnímají jako prospěšné (pro svůj život).
Jenže postupem času si začínají uvědomovat, že byli podvedeni. Zjistí, že jim tyto věci nepomáhají, že jim nepřinášejí štěstí, po kterém touží. Ale i když pak velmi touží po svobodě, jejich mozek i nadále požaduje to, k čemu byl veden, že je to prospěšné. Pravdu sice objeví, ale chybí vůle.
Ale to není ostuda. Závislý muž není slabý. Může být naopak velmi silný. A stejně jako Izraelité je v této situaci schopen pochopit jednu z největších pravd Písma svatého: jedině Bůh nás může vysvobodit.
Izraelité byli mimořádně silní, ale nedokázali se osvobodit. Moderní muž to zjišťuje také.

Kolikrát jste se pokoušeli „osvobodit se“, jen abyste zjistili, že to nejde? Tisíckrát?
Když začínáme tuto cestu, nikdy (zdůrazňuji nikdy) nezapomeňte na tuto úžasnou pravdu: vy to můžete zvládnout ... ale bude to Bůh, kdo vás osvobodí.

%newday
\newpage
\section{Den 2 - KAŽDODENNÍ ÚKOLY MOHOU MUŽE ZATĚŽOVAT A OSLABOVAT}
\zacatekPrvniTyden
\subsection*{Čtení na den}
\textbf{Exodus 1, 8-14}
\newline
\textit{
\textsuperscript{8}V Egyptě však nastoupil nový král, který o Josefovi nevěděl.
\textsuperscript{9}Ten řekl svému lidu: „Hle, izraelský lid je početnější a zdatnější než my.
\textsuperscript{10}Musíme s ním nakládat moudře, aby se nerozmnožil. Kdyby došlo k válce, jistě by se připojil k těm, kdo nás nenávidí, bojoval by proti nám a odtáhl by ze země.“
\textsuperscript{11}Ustanovili tedy nad ním dráby, aby jej ujařmovali robotou. Musel stavět faraónovi města pro sklady, Pitom a Raamses.
\textsuperscript{12}Avšak jakkoli jej ujařmovali, množil se a rozmáhal dále, takže měli z Izraelců hrůzu.
\textsuperscript{13}Proto začali Egypťané Izraelce surově zotročovat.
\textsuperscript{14}Ztrpčovali jim život tvrdou otročinou při výrobě cihel a všelijakou prací na poli. Všechnu otročinu, kterou na ně uvalili, jim ještě ztěžovali surovostí.  
}\subsection*{Reflexe}
Kniha Exodus je fascinující tím, že je to skutečně příběh každého člověka, což můžeme jasně vidět v dnešním úryvku Písma.

Egypťané byli plni obav, že by se hebrejský lid mohl stát "příliš mocným", než aby ho mohli ovládat, a že by mohl "bojovat proti nim" (Egypťanům). Egypťané na to šli chytře. „Ustanovili nad nimi úkoláře (dráby), aby je trápili těžkými břemeny.“ Jinými slovy, zaměstnávali muže úkoly, mnoha úkoly.
Jak byli Izraelité stále více zatěžováni každodenní prací, přestali se zajímat o svou svobodu a moc.

Žádný člověk nemůže být hrdinou, když je tak zatížen, že ani nemá čas vzhlédnout a zvažovat svou situaci. A tak Izraelité pracovali pro svého pána ještě usilovněji, ale tím nečekaně rostla jejich moc, i když zůstávali stále zotročeni.

Zamyslete se nad tím, jaké „cihly a malty“ používá Dráb (ten Zlý) ve vašich životech, aby vás ovládal, aby vás udržel daleko od vašeho pravého synovství, abyste se nestali příliš silnými. Stejně jako faraon, závistivý k moci, je lstivý v metodách, které používá, aby nás držel.
Přemýšlejte nad „maltou a cihlami“, které vás obklopují: nekonečná práce, zběsilá činnost, neustálý tlak na to, abyste se dostali dopředu. A zamyslete se nad všemi ostatními věcmi, které ďábel používá k tomu, aby vás zotročil: alkohol, pornografie, chtíč, pýcha, strach z neúspěchu, konkurence s ostatními, peníze, sport a postavení… Všechny tyto věci způsobují, že se váš život stává hořkým a smutným.

Ale nemusíte být jimi utlačováni. Rozhodnutí o jejich odstranění vás naučí, že můžete žít bez nich a můžete uniknout jejich poutům, když vás odvádějí od důležitějších věcí.
Je jasné, že muži mají mnoho povinností. Ale většině moderních mužů by velmi prospělo zjednodušení jejich života.

Zkus dnes strávit čas přemýšlením nad oblastmi, kde by mohl být tvůj život zjednodušen. Pozvi Pána do tohoto rozhovoru a zapiš si své závěry. Jedná se o počáteční krok ke svobodě.

\textbf{Poznámka pro ženaté muže:} Jakákoli zjednodušení nebo změny ve vašem způsobu života by měly být konzultovány a prodiskutovány s vaší manželkou.

%newday
\newpage
\section{Den 3 - POSUNOUT SE VPŘED }
\zacatekPrvniTyden
\subsection*{Čtení na den}
\textbf{Exodus 1, 15-22}
\newline
\textit{
\textsuperscript{15}Egyptský král poručil hebrejským porodním bábám, z nichž jedna se jmenovala Šifra a druhá Púa:
\textsuperscript{16}„Když budete pomáhat Hebrejkám při porodu a při slehnutí zjistíte, že to je syn, usmrťte jej; bude-li to dcera, aťsi je naživu.“
\textsuperscript{17}Avšak porodní báby se bály Boha a rozkazem egyptského krále se neřídily. Nechávaly hochy naživu.
\textsuperscript{18}Egyptský král si porodní báby předvolal a řekl jim: „Co to děláte, že necháváte hochy naživu?“
\textsuperscript{19}Porodní báby faraónovi odvětily: „Hebrejky nejsou jako ženy egyptské; jsou plné života. Porodí dříve, než k nim porodní bába přijde.“
\textsuperscript{20}Bůh pak těm porodním bábám prokazoval dobrodiní a lid se množil a byl velmi zdatný.
\textsuperscript{21}Protože se porodní báby bály Boha, požehnal jejich domům.
\textsuperscript{22}Ale farao všemu svému lidu rozkázal: „Každého syna, který se jim narodí, hoďte do Nilu; každou dceru nechte naživu.“
}

\subsection*{Reflexe}

Faraon měl tak velké obavy z toho, aby se Izraelité neosvobodili ze svého zotročení, že nařídil porodním bábám, aby udělaly něco naprosto zvláštního - aby zabily každého chlapce, kterého porodí, a tím aby udusily budoucnost Izraele. Zatímco porodní báby se tomuto požadavku hrdinně vyhýbají, faraon je neúprosný a vyzývá k utopení dětí mužského pohlaví v řece Nil.

Svatý Metoděj považuje faraóna za „předobraz ďábla“. Stejně jako faraón nařídil zabití izraelských chlapců, ďábel se pokouší zabít lidskou ctnost. Vysvětluje, že vody Nilu jsou obrazem našich vášní a ten zlý chce, aby se naše duše vrhly do těchto vod, aby se utopily. Každý člověk zná bolest vnitřního udušení: osamělost pornografie, prázdnota alkoholu, nuda neprozkoumaného života a nekonečná snaha o zábavu.

Dnes je třetí den vašeho odhodlání osvobodit se od těchto věcí. To, co vás kdysi zotročovalo, je nyní prostředkem, díky němuž se stáváte svobodnými, pokud se toho snažíte zbavit.

Svatý Augustin si všímá jisté ironie v příběhu Exodus: Izraelci kráčeli vodami Rudého moře na svobodu. Ti, kteří byli zotročeni a odsouzeni k utonutí, nyní procházejí mořem na cestě ke svobodě.

Naše kultura nás obklopuje neustálým vybízením k bezduchým a ničivým požitkům. I když se od nich vzdalujeme, připadá nám, jako bychom jimi procházeli. V tom se podobáme Izraelitů, kteří procházeli Rudým mořem a měli dvě obří vodní stěny - po levici a pravici.

Ale když Boží síla otevírá cestu, naším jediným úkolem je pokročit vpřed.
Děkujte Pánu, že vám dnes otevřel cestu, a získejte odvahu jít kupředu.


%newday
\newpage
\section{Den 4 - DAR NOVÉHO ŽIVOTA}
\zacatekPrvniTyden
\subsection*{Čtení na den}
\textbf{Exodus 2, 1-10}
\newline
\textit{
\textsuperscript{1}Muž z Léviova domu šel a vzal si lévijskou dceru.
\textsuperscript{2}Žena otěhotněla a porodila syna. Když viděla, jak je půvabný, ukrývala ho po tři měsíce.
\textsuperscript{3}Ale déle už ho ukrývat nemohla. Proto pro něho připravila ze třtiny ošatku, vymazala ji asfaltem a smolou, položila do ní dítě a vložila do rákosí při břehu Nilu.
\textsuperscript{4}Jeho sestra se postavila opodál, aby zvěděla, co se s ním stane.
\textsuperscript{5}Tu sestoupila faraónova dcera, aby se omývala v Nilu, a její dívky se procházely podél Nilu. Vtom uviděla v rákosí ošatku a poslala svou otrokyni, aby ji přinesla.
\textsuperscript{6}Otevřela ji a spatřila dítě, plačícího chlapce. Bylo jí ho líto a řekla: „Je z hebrejských dětí.“
\textsuperscript{7}Jeho sestra se faraónovy dcery otázala: „Mám jít a zavolat kojnou z hebrejských žen, aby ti dítě odkojila?“
\textsuperscript{8}Faraónova dcera jí řekla: „Jdi!“ Děvče tedy šlo a zavolalo matku dítěte.
\textsuperscript{9}Faraónova dcera jí poručila: „Odnes to dítě, odkoj mi je a já ti zaplatím.“ Žena vzala dítě a odkojila je.
\textsuperscript{10}Když dítě odrostlo, přivedla je k faraónově dceři a ona je přijala za syna. Pojmenovala ho Mojžíš (to je Vytahující) . Řekla: „Vždyť jsem ho vytáhla z vody.“}

\subsection*{Reflexe}

Když se v našem životě nebo v životě Božího lidu stane něco nepředvídaného a radikálního, můžeme si být téměř jisti, že to jedná Bůh. Vidíme to i v dnešním úryvku z Písma.

Izraelité jsou zotročeni v Egyptě a nemají téměř žádnou naději, že se někdy dočkají svobody, když tu najednou Bůh vzbudí osvoboditele. Jak uvidíme v následujících dnech, narození Mojžíše a jeho povolání jako osvoboditele je předzvěstí velkého osvoboditele, který přijde: Ježíše Krista.

Již nyní můžeme začít vidět podobnosti mezi těmito dvěma postavami:
Mojžíš je zplozen nejmenovaným mužem, je to "hodné" dítě a je na tři dny umístěn v koši na řece. Připomeňme si, že u Marie bylo zjištěno, že "čeká dítě" bez pomoci svého snoubence.

Oba jsou hodné děti: o Ježíšovi čteme, že poté, co byl nalezen v chrámu, „poslouchal je (Josefa a Marii) a prospíval na duchu i na těle a byl milý Bohu i lidem.“ (Lk 2,51-52).

Všimněte si také, že obě děti byly zachráněny před bezohlednými a paranoidními vůdci, kteří se je snažili raději zabít, než aby ztratili svou moc (Mt 2,16).
Číslo tři by mělo vyvolat vzpomínku na mnoho událostí v Kristově životě: tři dny, kdy se dítě ztratí a najde v chrámu, tři dny, kdy je Kristus v hrobě, jeho veřejné působení začalo v jeho třicátém roce.

Ježíš se zjeví jako nový osvoboditel a bude hrát velkou roli v našem vlastním hledání svobody.
A konečně, a to je nejvýznamnější, Mojžíš dostává své jméno, protože faraonova dcera prohlásila: "Vytáhla jsem ho z vody." Mojžíš je tedy v tomto případě zjevením, které je v souladu se skutečností. I vy jste byli vytaženi "z vody", když jste byli pokřtěni, nejspíše jako nemluvně. Byli jste zachráněni z tyranie Zlého a bylo vám dáno vše, co je třeba k tomu, abyste byli synem Nejvyššího.

Mojžíš byl zachráněn skrze vodu, Izraelité byli zachráněni skrze vodu (Ex 14) a i vy jste byli zachráněni skrze vodu křtu.
Křest je dnes často opomíjeným přechodovým rituálem. Na milost a moc svátosti křtu se většinou zapomíná. Svatý Pavel přesto trval na svém významu: „Nevíte snad, že všichni, kteří jsme pokřtěni v Krista Ježíše, byli jsme pokřtěni v jeho smrt? Byli jsme tedy křtem spolu s ním pohřbeni ve smrt, abychom – jako Kristus byl vzkříšen z mrtvých slavnou mocí svého Otce – i my vstoupili na cestu nového života.“ (Řím 6,3-4)

Svoboda, kterou prostřednictvím tohoto exodu hledáte, pramení z nového života, který jste obdrželi při křtu.
Ježíš to řekl jasně: "Kdo uvěří a dá se pokřtít, bude spasen...". Udělali bychom tedy dobře, kdybychom si připomněli milost vlastního křtu a "oživili Boží dar, který je v tobě" (2 Tim 1,6), abyste měli vše, co je nezbytné k získání skutečné a trvalé svobody!

Připomeňte si dnes milosti svého křtu a podívejte se na toho, kdo vám tyto milosti dal. Touží ve vás ještě jednou vzbudit dar nového života. Mluvte s ním dnes otevřeně. Poděkujte mu za pozvání na toto duchovní cvičení a zeptejte se ho, proč vám tak velkoryse znovu nabízí dar nového života. Odpověď bude velká láska.

%newday
\newpage
\section{Den 5 - MUŽ PRO OSTATNÍ}
\zacatekPrvniTyden
\subsection*{Čtení na den}
\textbf{Exodus 2,11-25}
\newline
\textit{
\textsuperscript{11}V oněch dnech, když Mojžíš dospěl, vyšel ke svým bratřím a viděl jejich robotu. Spatřil nějakého Egypťana, jak ubíjí Hebreje, jednoho z jeho bratří.
\textsuperscript{12}Rozhlédl se na všechny strany, a když viděl, že tam nikdo není, ubil Egypťana a zahrabal do písku.
\textsuperscript{13}Když vyšel druhého dne, spatřil dva Hebreje, jak se rvali. Řekl tomu, který nebyl v právu: „Proč chceš ubít svého druha?“
\textsuperscript{14}Ohradil se: „Kdo tě ustanovil nad námi za velitele a soudce? Máš v úmyslu mě zavraždit, jako jsi zavraždil toho Egypťana?“ Mojžíš se ulekl a řekl si: „Jistě se o věci už ví!“
\textsuperscript{15}Farao o tom vskutku uslyšel a chtěl dát Mojžíše zavraždit. Ale Mojžíš před faraónem uprchl a usadil se v midjánské zemi; posadil se u studny.
\textsuperscript{16}Midjánský kněz měl sedm dcer. Ty přišly, vážily vodu a plnily žlaby, aby napojily stádo svého otce.
\textsuperscript{17}Tu přišli pastýři a odháněli je. Ale Mojžíš vstal, ochránil je a napojil jejich stádo.
\textsuperscript{18}Když přišly ke svému otci Reúelovi, zeptal se: „Jak to, že jste dnes přišly tak brzo?“
\textsuperscript{19}Odpověděly: „Nějaký Egypťan nás vysvobodil z rukou pastýřů. Také nám ochotně navážil vodu a napojil stádo.“
\textsuperscript{20}Reúel se zeptal svých dcer: „Kde je? Proč jste tam toho muže nechaly? Zavolejte ho, ať pojí chléb!“
\textsuperscript{21}Mojžíš se rozhodl, že u toho muže zůstane, a on mu dal svou dceru Siporu za manželku .
\textsuperscript{22}Ta porodila syna a Mojžíš mu dal jméno Geršóm (to je Hostem-tam) . Řekl: „Byl jsem hostem v cizí zemi.“
\textsuperscript{23}Po mnoha letech egyptský král zemřel, ale Izraelci vzdychali a úpěli v otročině dál . Jejich volání o pomoc vystupovalo z té otročiny k Bohu.
\textsuperscript{24}Bůh vyslyšel jejich sténání, Bůh se rozpomněl na svou smlouvu s Abrahamem, Izákem a Jákobem,
\textsuperscript{25}Bůh na syny Izraele pohleděl, Bůh se k nim přiznal.
}
\subsection*{Reflexe}

Muži jsou v tom nejlepším, když jsou skutečně „muži pro druhé“. Zralý a sebeovládající se muž, který je zformovaný Boží rukou, má velkou moc, která vychází z nového života v něm.
Ale to, co dělá muže skutečně velkým, je jeho ochota sloužit druhým, používat svou moc/sílu pro druhé - ať už je to jeho žena, děti, bratři, sousedé, církev nebo země. V moderní době jsme upadli do zlozvyku dávat na první místo své vlastní potřeby a touhy, a až na druhé, pokud vůbec, brát v úvahu potřeby ostatních.

V dnešním úryvku z Písma vidíme Mojžíše, jak je solidární s chudými a utlačovanými a jak využívá svou mladickou moc k tomu, aby zasáhl a pomohl druhým. Vidí, že dcery Reuela (Jethra) se nedokážou samy o sebe postarat a postavit se proti darebáckým pastýřům.
Když se zaměříme na péči, podporu a obranu slabších a potřebných, Bůh si nás a naši sílu může použít a použije pro dobro druhých. Brzy se o tom přesvědčíme, až Bůh udělá z Mojžíše velkého osvoboditele a soudce svého lidu.

Dnes a denně se musíme odpoutávat od současných kulturních zvyklostí a rozhodnout se (v případě potřeby každý den) překonat své vlastní potřeby ve prospěch lidí kolem nás.
Během dnešní modlitby se ptejte sami sebe: „Kdo je na vás závislý? Kdo od vás hledá ochranu nebo pomoc tváří v tvář životním nespravedlnostem a nebezpečím? Kdo se vám svěřil v naději, že mu budete pevnou oporou, mužem, na kterého je spolehnutí? Kdo vám věří, že nebudete jen přihlížet, když bude potřeba se angažovat, a to i když to pro vás bude nepohodlné?

Život má skutečně smysl a význam, když se velkoryse věnujeme druhým, nějaké věci nebo církvi.
Jste ochotni takový život žít? Jste ochotni vykročit vpřed a být mužem pro ostatní?

%newday
\newpage
\section{Den 6 - BŮH SI NÁS VOLÍ PŘEDTÍM, NEŽ SI MY ZVOLÍME JEJ}
\zacatekPrvniTyden
\subsection*{Čtení na den}
\textbf{Exodus 3, 1-6}
\newline
\textit{\textsuperscript{1}Mojžíš pásl ovce svého tchána Jitra, midjánského kněze. Jednou vedl ovce až za step a přišel k Boží hoře, k Chorébu.
\textsuperscript{2}Tu se mu ukázal Hospodinův posel v plápolajícím ohni uprostřed trnitého keře. Mojžíš viděl, jak keř v ohni hoří, ale není jím stráven.
\textsuperscript{3}Řekl si : „Zajdu se podívat na ten veliký úkaz, proč keř neshoří.“
\textsuperscript{4}Hospodin viděl, že odbočuje, aby se podíval. I zavolal na něho Bůh zprostředku keře: „Mojžíši, Mojžíši!“ Odpověděl: „Tu jsem.“
\textsuperscript{5}Řekl: „Nepřibližuj se sem! Zuj si opánky, neboť místo, na kterém stojíš, je půda svatá.“
\textsuperscript{6}A pokračoval: „Já jsem Bůh tvého otce, Bůh Abrahamův, Bůh Izákův a Bůh Jákobův.“ Mojžíš si zakryl tvář, neboť se bál na Boha pohledět.
}

\subsection*{Reflexe}

Všimněte si způsobu, jakým Bůh a Mojžíš začali své hluboké přátelství. Ne Mojžíš šel hledat a najít Boha. To se stává velmi zřídka, pokud vůbec. Ale byl to Bůh, kdo přišel s Mojžíšem jako první.  Mojžíš řeší své každodenní starosti, když se mu Bůh zjeví a dovolí mu odpovědět.  

Svatá Terezie z Avily často používala k popisu tohoto jevu příměr se slunečnicí. Když ráno vyjde slunce, jeho paprsky zalijí krajinu a slunečnice k němu otočí hlavu. Může to však udělat pouze tehdy, když na ni svítí slunce. Podobně když se duše obrací k Bohu, je to proto, že Bůh udělal první krok. To je základ duchovního života.

Může existovat tisíc důvodů, proč jste se rozhodli naplnit Exodus 90. Ale nebyli jste to vy, kdo se rozhodl to udělat - byl to Bůh, kdo vás k tomu povolal. Touto výzvou, tímto pozváním k vykonání těchto duchovních cvičení vám Bůh otevřel cestu k hlubšímu vztahu s ním. Během těchto 90 dní je Božím záměrem zjevit se vám více. Tento krok je Jeho, ale je na vás, jak na něj zareagujete. Využijte tento požehnaný čas k tomu, abyste "obrátili hlavu k Bohu" a objevili Ho tak, jak se vám dovolil zjevit.

Volejte dnes k Pánu v modlitbě. Požádejte Ho, aby se vám zjevil více než kdykoli předtím.

%newday
\newpage
\section{Den 7 - BŮH DÁVÁ ČLOVĚKU SÍLU}
\zacatekPrvniTyden
\subsection*{Čtení na den}
\textbf{}
\newline
\textit{
\textsuperscript{7}Hospodin dále řekl: „Dobře jsem viděl ujařmení svého lidu, který je v Egyptě. Slyšel jsem jeho úpění pro bezohlednost jeho poháněčů. Znám jeho bolesti.
\textsuperscript{8}Sestoupil jsem, abych jej vysvobodil z moci Egypta a vyvedl jej z oné země do země dobré a prostorné, do země oplývající mlékem a medem, na místo Kenaanců, Chetejců, Emorejců, Perizejců, Chivejců a Jebúsejců.
\textsuperscript{9}Věru, úpění Izraelců dolehlo nyní ke mně. Viděl jsem také útlak, jak je Egypťané utlačují.
\textsuperscript{10}Nuže pojď, pošlu tě k faraónovi a vyvedeš můj lid, Izraelce, z Egypta.“
\textsuperscript{11}Ale Mojžíš Bohu namítal: „Kdo jsem já, abych šel k faraónovi a vyvedl Izraelce z Egypta?“
\textsuperscript{12}Odpověděl: „Já budu s tebou! A toto ti bude znamením, že jsem tě poslal: Až vyvedeš lid z Egypta, budete sloužit Bohu na této hoře.“
\textsuperscript{13}Avšak Mojžíš Bohu namítl: „Hle, já přijdu k Izraelcům a řeknu jim: Posílá mě k vám Bůh vašich otců. Až se mě však zeptají, jaké je jeho jméno, co jim odpovím?“
\textsuperscript{14}Bůh řekl Mojžíšovi: „JSEM, KTERÝ JSEM.“ A pokračoval: „Řekni Izraelcům toto: JSEM posílá mě k vám.“
\textsuperscript{15}Bůh dále Mojžíšovi poručil: „Řekni Izraelcům toto: ‚Posílá mě k vám Hospodin, Bůh vašich otců, Bůh Abrahamův, Bůh Izákův a Bůh Jákobův.‘ To je navěky mé jméno, jím si mě budou připomínat od pokolení do pokolení.
\textsuperscript{16}Jdi, shromažď izraelské starší a pověz jim: ,Ukázal se mi Hospodin, Bůh vašich otců, Bůh Abrahamův, Izákův a Jákobův, a řekl: Rozhodl jsem se vás navštívit, vím , jak s vámi v Egyptě nakládají,
\textsuperscript{17}a prohlásil jsem: Vyvedu vás z egyptského ujařmení do země Kenaanců, Chetejců, Emorejců, Perizejců, Chivejců a Jebúsejců, do země oplývající mlékem a medem.‘
\textsuperscript{18}Až tě vyslechnou, půjdeš ty a izraelští starší k egyptskému králi a řeknete mu: ‚Potkal se s námi Hospodin, Bůh Hebrejů. Dovol nám nyní odejít do pouště na vzdálenost tří dnů cesty a přinést oběť Hospodinu, našemu Bohu.‘
\textsuperscript{19}Vím, že vám egyptský král nedovolí jít, leda z donucení.
\textsuperscript{20}Proto vztáhnu ruku a budu bít Egypt všemožnými svými divy, které učiním uprostřed něho. Potom vás propustí.
\textsuperscript{21}Zjednám tomuto lidu u Egypťanů přízeň. Až budete odcházet, nepůjdete s prázdnou.
\textsuperscript{22}Každá žena si vyžádá od sousedky a spolubydlící stříbrné a zlaté ozdoby a pláště. Vložíte je na své syny a dcery. Tak vypleníte Egypt.“
}

\subsection*{Reflexe}

Bůh dává vysvobození. Vidíme to na příběhu Izraelitů. V dnešním čtení říká Bůh Mojžíšovi, že nejprve propustí
Izraelity z ruky Egypťanů. Poté je dovede do „zaslíbené země“, která je ovšem obývána Kennaanci, Chetejci,
Emorejsi, Perizejsi, Chivejci a Jebúsejci – všemi nepřáteli Izraele. Dokážete si představit, co si Mojžíš mohl myslet?
„Chceš nás osvobodit od našich otrokářů (což nemůže dopadnout dobře) jenom proto, abys nás mohl dovést
doprostřed našich napřátel?“ Copak je divu, že chtěl být Mojžíš sám? Ale Bůh, aby dal Mojžíšovi odvahu, dělá něco
naprosto nemyslitelného. Zjevuje mu svoje svaté jméno: „Jsem, který jsem.“

V moderním světě jsme zapomněli na důležitost teologie jmen. Ve starověku vědět něčí jméno znamenalo mít nad
ním nějakou moc. Proto Adam pojmenoval všechna zvířata v zahradě Edenu. Prohlašoval tím svou nadřazenost nad
zvířecí říší (Gen 2,20). Když dává Bůh poznat své jméno Mojžíšovi, také mu tím propůjčuje svou božskou moc. Jak
by mohl teď Mojžíš pochybovat o příslibu jemu a jeho lidu? Pouze potřebuje jít za Bohem v důvěře.

Také si povšimněte významnosti úkolu danému Mojžíšovi a zprávy, která mu je dána k předání Izraelitům: „Jsem
mě poslal k vám.“ Znovu Mojžíš předznamenává Ježíše Krista. Tak jako byl Mojžíš poslán Bohem k Izraelitům,
Kristus byl poslán Otcem k osvobození nás všech. „Neboj se“, říká Ježíš, „jen věř,“ (Lk 8,50). Když procházíte
disciplínami Exodu 90, zapíráte se a celou dobu bojujete, mějte oči upřeny na Ježíše. Byl poslán, aby vás vykoupil
od vás samotných, vašich hříchů, zotročení. Když bojujete za svobodu v dennodenním boji, Bůh tam bojuje s vámi
a pro vás.

Zavolejte na Boha. Čeká, aby vám dal svou moc a sílu.


% ===============================================
% ===== DRUHY TYDEN
% ===============================================
%ukony
\newpage
\section*{Úkony (ukazatel cesty) pro 2. týden}

\textbf{Místo:} Egypt (jste v Egyptě)

Izraelité jsou daleko od svobody. Mojžíšova a Áronova poslušnost Bohu pouze zhoršila jejich situaci a zvýraznila jejich otroctví více než kdy dříve. Na tomto místě našeho exodu se také naše otroctví stalo viditelnějším. Disciplíny duchovního cvičení zvětštily naši dříve nevědomou náklonnost k lidskému komfortu. Tato rutina nás formuje, ale nejsme ještě vůbec blízko svobody. Proč si stěžujeme život poslušností Bohu? Nebylo by snazší toto duchovní cvičení opustit a zůstat v pohodlí otroctví navěky? Této úvaze čelí Izraelité i my tento týden.

\subsection*{1. Dobře se vyzpovídejte}
Pokoušet se začít Exodus 90, aniž bychom šli nejdříve ke zpovědi, je jako pokoušet se vylézt na vrchol nebezpečné americké sopky Mount Rainier s pětadevadesáti kily kamení v krosně. Proveditelné, ale pošetilé. Kříž, který na sebe musíte brát každý den je dost těžký tak, jak je. Nechte Boha vyndat ono kamení z vaší krosny. Běžte ke zpovědi, abyste mohli zdolat tuto horu a být opravdu svobodní.
\subsection*{2. Držte se denních reflexí}
Pokud denní čtení a reflexe neprovádíte, neděláte vůbec Exodus 90. Denní čtení Písma a rozjímání nad ním dovoluje Ježíši Kristu, Slovu, aby vás vedl na cestě vaším exodem. Neskončíte se studenou sprchou jen proto, že vám nevyhovuje, tak neskončujte ani se čtením Písma jen proto, že se vám nechce. Držte se tohoto rozjímání. Udrží vás a vaše bratrstvo jednotné po dobu exodu.
\subsection*{3. Navštěvujte jednu mši v týdnu navíc}
Na otázku, co by vyzvalo lidi k tomu, aby více rostli ve víře, odpověděl kněz Augustinského Institutu takto: „Ať chodí na další mši svatou během dní v týdnu.“ Nyní nastal ten čas. Zvolte si den v týdnu a konkrétní čas mše, na kterou budete chodit každý týden, vedle povinných mší svatých v neděli a o svátcích. (Pro více informací o účincích navštěvování více mší svatých v týdnu a způsobu, jak toho využívala různá bratrstva v minulosti, nahlédněte do sekce \textit{Posílit své bratrství} pod pilířem Bratrství v příručce Exodu.)
\subsection*{4. Zvažte přečtení \textit{Průvodce terénem}}
Pokud jste si nenašli čas na přečtení \textit{Průvodce terénem} Exodu 90 předtím, než jste Exodus začali, zvažte jeho přečtení dnes, nebo tuto neděli. Tento průvodce rámcuje celou zkušenost Exodu 90 a pomůže vám pochopit důvod každé části vašeho exodu. Porozumění těmto důvodům často pomáhá k tomu, abyste se zavázali k disciplínám s větší radostí a lehčím srdcem. (Nejdůležitější části \textit{Průvodce terénem} jsou: \textit{Začněte zde: Co je vaše proč}, \textit{Pilíře Exodu 90}, a pro ženaté muže \textit{Muž Exodu a jeho manželka.})
\subsection*{5. Uvědomte si (zjistěte), kde je vaše kotva}
Zavázali jste se k denní komunikaci s vaší kotvou. On na vás spoléhá v tom, že budete naplňovat váš závazek. Jestli jste tak činili, skvěle. Jestli ne, teď je čas začít. Brzy budete potřebovat vaši kotvu stejně tak, jako on potřebuje vás. (Myslíte si, že každodenní komunikace s vaší kotvou není důležitá? Přečtěte si sekci \textit{Kotva (Nepřeskakujte: Smrt je pravděpodobná)} pod pilířem Bratrství v \textit{Pilířích Exodu 90.})

\subsection*{Modlitba}
Modlete se, aby Pán osvobodil vás a vaše bratrství \newline
Modleme se za svobodu všech mužů v exodu, stejně tak, jako se oni modlí za vás.\newline
Ve jménu Otce i Syna i Ducha svatého … Otče náš… Ve jménu Otce i Syna i Ducha svatého … Amen.

%newday
\newpage
\section{Den 8 - HOSPODIN JE JEDINÝ BŮH}
\zacatekDruhyTyden
\subsection*{Čtení na den}
\textbf{Exodus 4,1-9}
\newline
\textit{
\textsuperscript{1}Mojžíš však znovu namítal: „Nikoli, neuvěří mi a neuposlechnou mě, ale řeknou: Hospodin se ti neukázal.“
\textsuperscript{2}Hospodin mu řekl: „Co to máš v ruce?“ Odpověděl: „Hůl.“
\textsuperscript{3}Hospodin řekl: „Hoď ji na zem.“ Hodil ji na zem a stal se z ní had. Mojžíš se dal před ním na útěk.
\textsuperscript{4}Ale Hospodin Mojžíšovi poručil: „Vztáhni ruku a chyť ho za ocas.“ Vztáhl tedy ruku, uchopil ho a v dlani se mu z něho stala hůl.
\textsuperscript{5}„Aby uvěřili, že se ti ukázal Hospodin, Bůh jejich otců, Bůh Abrahamův, Bůh Izákův a Bůh Jákobův.“
\textsuperscript{6}Dále mu Hospodin řekl: „Vlož si ruku za ňadra.“ Vložil tedy ruku za ňadra. Když ruku vytáhl, byla malomocná, bílá jako sníh.
\textsuperscript{7}Tu poručil: „Dej ruku zpět za ňadra.“ Dal ruku zpět za ňadra. Když ji ze záňadří vytáhl, byla opět jako ostatní tělo.
\textsuperscript{8}„A tak jestliže ti neuvěří a nedají na první znamení, uvěří druhému znamení.
\textsuperscript{9}Jestliže však neuvěří ani těmto dvěma znamením a neuposlechnou tě, nabereš vodu z Nilu a vyleješ ji na suchou zemi. Z vody, kterou nabereš z Nilu, se stane na suché zemi krev.“
}

\subsection*{Reflexe}
Skrze Starý Zákon pracuje Bůh nepřetržitě, aby svému lidu ukázal, že je jediný Bůh. Ale jeho lid bojuje
s tím, aby uvěřil. Znamení po znamení ukazuje Bůh svou moc svému lidu a ostatním národům jako Egyptu.
Ti, kteří vidí tato znamení a uvěří, zakusí Boží lásku. Ti, kteří vidí a neuvěří, jdou směrem ke své vlastní
zkáze.

Věříte, že Pán je jediný Bůh? Žijete tak, jako byste tomu věřili? Dobrým způsobem, jak to vyzkoušet, je
podívat se na vaši neděli a první a poslední věc, co uděláte každý den. Co je nejdůležitější částí vaší neděle?
Sportovní utkání? Práce na zahradě? Co je první věcí, co uděláte každé ráno, když se probudíte, a každý
večer předtím, než usnete? Zkontrolovat svůj telefon? Zapnout zprávy? Bůh vás hledá. Chce, abyste věděli,
že On je jediný Bůh. Kontrolování telefonu vám nezaručí vysvobození, ale klečení na kolenou každé ráno
a každou noc vedle své postele před tím, kdo má moc vás vysvobodit, ano. Skrze tyto činy vám může dát
Bůh vysvobození.

Když pokračujete se čtením úžasných věcí, které Bůh učinil pro jeho lid Izrael, obraťte pozornost také na
svůj vlastní život. Vidíte ty úžasné věci, které pro vás, ve vašem životě a na oltáři, udělal? Pohleďte na tyto
věci a posilte dnes svou víru v to, že Pán je Bůh. Ano, Hospodin je jediný Bůh.


%newday
\newpage
\section{Den 9 - DŮVĚŘUJ BOHU}
\zacatekDruhyTyden
\subsection*{Čtení na den}
\textbf{Exodus 4,10-17}
\newline
\textit{
\textsuperscript{10}Ale Mojžíš Hospodinu namítal: „Prosím, Panovníku, nejsem člověk výmluvný; nebyl jsem dříve, nejsem ani nyní, když ke svému služebníku mluvíš. Mám neobratná ústa a neobratný jazyk.“
\textsuperscript{11}Hospodin mu však řekl: „Kdo dal člověku ústa? Kdo působí, že je člověk němý nebo hluchý, vidící nebo slepý? Zdali ne já, Hospodin?
\textsuperscript{12}Nyní jdi, já sám budu s tvými ústy a budu tě učit, co máš mluvit!“
\textsuperscript{13}Ale Mojžíš odmítl: „Prosím, Panovníku, pošli si, koho chceš.“
\textsuperscript{14}Tu Hospodin vzplanul proti Mojžíšovi hněvem a řekl: „Což nemáš bratra Árona, toho lévijce? Znám ho , ten umí mluvit. Jde ti už naproti a bude se srdečně radovat, až tě uvidí.
\textsuperscript{15}Budeš k němu mluvit a vkládat mu slova do úst. Já budu s tvými ústy i s jeho ústy a budu vás poučovat, co máte činit.
\textsuperscript{16}On bude mluvit k lidu za tebe, on bude tobě ústy a ty budeš jemu Bohem.
\textsuperscript{17}A tuto hůl vezmi do ruky; budeš jí konat znamení.“
}

\subsection*{Reflexe}
Mojžíš zakouší tíhu úkolu, který před něj byl postaven, a myslí si, že se Bůh spletl. Bůh po něm žádá něco,
čeho je neschopné dosáhnout. Tak jako mnoho lidí i dnes, Mojžíš postrádá důvěru v sebe i v Hospodina,
což, jak praví Písmo, rozněcuje Boží hněv. Bůh se ale jistě nehněvá kvůli nedostatku Mojžíšovy
výmluvnosti. Spíše „Hospodin vzplanul proti Mojžíšovi hněvem,“ protože Mojžíš postrádá důvěru v Boha
a dovoluje, aby jeho naděje vymizela. Bůh nás v průběhu našeho života zkouší mnoha způsoby, přesně tak
jako Mojžíše v dnešním čtení. Samozřejmě to nedělá proto, aby se o nás něco naučil. Zkouší nás, abychom
se sami naučili něco o sobě a o Něm.

Mojžíš postrádá ctnost naděje. Poznává, že jeho nedostatky nejsou hříchem, ale nemyslí si, že je Bůh
schopný dorovnat to, co Mojžíš postrádá. Právě to je problém. Mojžíš se musí naučit, že pokud má vést
Boží lid a dosáhnout nemožného, musí se zcela spoléhat na Boha.

Těchto 90 dní Exodu je pro vás obrovskou zkouškou. Učte se od Mojžíše: selžete v tomto cvičení, pokud
opravdu nesvěříte svou naději v Boha. Tak často ve strachu převezmeme kontrolu z Božích rukou a
zkoušíme ji třímat v těch našich. Když to uděláme a uspějeme, činíme tak za velkou cenu nás, a občas i
těch, které milujeme.

Uvědomte si, že se sami nemůžete dovést ke svobodě. Poté pohleďte na kříž. Naděje v Toho, který porazil
i smrt. On vás dovede ke svobodě. On jediný vám může dát vysvobození. Doufejte v Něho.

%newday
\newpage
\section{Den 10 - JAKÝM ČLOVĚKEM SE STANETE?}
\zacatekDruhyTyden
\subsection*{Čtení na den}
\textbf{Exodus 4,18-31}
\newline
\textit{
\textsuperscript{18}Mojžíš odešel a vrátil se ke svému tchánu Jitrovi. Řekl mu: „Rád bych šel a vrátil se ke svým bratřím, kteří jsou v Egyptě, a podíval se, zda ještě žijí.“ Jitro Mojžíšovi odvětil: „Jdi v pokoji.“
\textsuperscript{19}Hospodin pak řekl Mojžíšovi ještě v Midjánu: „Jen se vrať do Egypta, neboť zemřeli všichni, kteří ti ukládali o život.“
\textsuperscript{20}Mojžíš tedy vzal svou ženu a syny, posadil je na osla a vracel se do egyptské země. A do ruky si vzal Boží hůl.
\textsuperscript{21}Hospodin dále Mojžíšovi poručil: „Až se vrátíš do Egypta, hleď, abys před faraónem udělal všechny zázraky, jimiž jsem tě pověřil. Já však zatvrdím jeho srdce a on lid nepropustí.
\textsuperscript{22}Potom faraónovi řekneš: Toto praví Hospodin: ‚Izrael je můj prvorozený syn.
\textsuperscript{23}Vzkázal jsem ti: Propusť mého syna, aby mi sloužil. Ale ty jsi jej propustit odmítl. Za to zabiji tvého prvorozeného syna.‘“
\textsuperscript{24}Když se na cestě chystali nocovat, střetl se s ním Hospodin a chtěl ho usmrtit.
\textsuperscript{25}Tu vzala Sipora kamenný nůž, obřezala předkožku svého syna, dotkla se jeho nohou a řekla: „Jsi můj ženich, je to zpečetěno krví.“
\textsuperscript{26}A Hospodin ho nechal být. Tehdy se při obřízkách říkalo: „ Jsi ženich, je to zpečetěno krví.“
\textsuperscript{27}Hospodin řekl Áronovi: „Jdi na poušť naproti Mojžíšovi.“ Áron šel, setkal se s ním u Boží hory a políbil ho.
\textsuperscript{28}Mojžíš oznámil Áronovi všechna Hospodinova slova, s nimiž ho poslal, a všechna znamení, kterými ho pověřil.
\textsuperscript{29}Pak šel Mojžíš s Áronem a shromáždili všechny izraelské starší.
\textsuperscript{30}Áron vyřídil všechna slova, která mluvil Hospodin k Mojžíšovi, a Mojžíš učinil před očima lidu ona znamení.
\textsuperscript{31}A lid uvěřil. Když slyšeli, že Hospodin navštívil Izraelce a že pohleděl na jejich ujařmení, padli na kolena a klaněli se.
}

\subsection*{Reflexe}

Uvažujme o dvou hlavních postavách knihy Exodus. První, Mojžíš, zná své postavení před Pánem. Není povýšený
ani arogantní. Zachovává si svou důstojnout a je mu dokonce dána výsada od Boha. Na druhé straně stojí faraon, tak
povýšený a arogantní, že dokonce prohlásí sebe samého za božského a jeho podřízení ho takto musí přijímat. Bůh
nakonec zničí faraonovo potomstvo a jeho dynastii a dokazuje tak faraonovi, že je pouhý člověk.

Dnešní muži často napodobují faraona. Jen málo z nich by se nestoudně prohlásilo za božské, ale mnoho z nich se
tak chová. Určují si svou vlastní pravdu, nastavují svůj vlastní směr, odmítají se spoléhat na Boha a žijí svrchovaný
život. Od Adamova pádu má člověk vzpupné srdce. Izraelci tohoto příběhu nejsou výjimkou. Budou muset být
poučeni: čtyřicet let v poušti není procházka. Bůh pošle svůj lid do poušte, aby se naučil být na Něm závislý, plně se
na Něj spolehnout. Musí pochopit, že Bůh je Bůh a že oni jsou jeho prvorozenými syny.

Člověk je schopen mnoha velkolepých činů, ale Boží syn je schopen mnohem, mnohem více. Těchto devadesát dní
nabízí skvělou příležitost, jak svůj život nasměrovat k Bohu. Vzdejte se svého vzpurného srdce. Oblečte nového
člověka, jako milovaný syn pozoruhodného Otce, který se o vás nejen postará, ale také vám pomůže stát se
spolehlivým, nezištným a svobodným mužem pro druhé.

Zvažte svůj vlastní vztah s Bohem. Vidíte Boha jako svého Otce a žijete svůj vztah k Němu jako takovému?
Obracíte se na něj často, jako se dítě obrací na svého otce? Přineste dnes tyto otázky do své svaté hodiny.

%newday
\newpage
\section{Den 11 - POKORA}
\zacatekDruhyTyden
\subsection*{Čtení na den}
\textbf{Exodus 5,1-4}
\newline
\textit{
\textsuperscript{1}Mojžíš s Áronem pak předstoupili před faraóna a řekli: „Toto praví Hospodin, Bůh Izraele: Propusť můj lid, ať mi v poušti slaví slavnost .“
\textsuperscript{2}Farao však odpověděl: „Kdo je Hospodin, že bych ho měl uposlechnout a propustit Izraele? Hospodina neznám a Izraele nepropustím!“
\textsuperscript{3}Řekli: „Potkal se s námi Bůh Hebrejů. Dovol nám nyní odejít do pouště na vzdálenost tří dnů cesty a přinést oběť Hospodinu, našemu Bohu, aby nás nenapadl morem nebo mečem.“
\textsuperscript{4}Egyptský král je okřikl: „Proč, Mojžíši a Árone, odvádíte lid od jeho prací? Jděte za svými robotami!“
}

\newpage
\subsection*{Reflexe}

Faraon arogantně ignoruje příkazy Nejvyššího Boha. Samozřejmě má své důvody. Je vládcem nad mocným
královstvím. Řídí armády a otroky. Pochází z mocné linie uctívaných mužů. Faraon je mocný. Přesto,
navzdory všem jeho vznešenostem, bude ponížen dokonce pod mouchy a žáby, protože neohne koleno před
svým Tvůrcem.

Jsme Boží synové a jsme stvořeni k Jeho obrazu a podobě. Každý z nás má mnoho silných stránek, a proto
je pýcha neustálým pokušením. Když jsme na sebe moc hrdí, začneme se čím dál více oddávat vlastním
nutkáním. Výsledkem bude vždy náš krach. Dřív nebo později se ocitneme sraženi na kolenou věcmi, které
jsou pod naši důstojnost – závislostmi, rozptýleností, nudou… Jestliže se takto necháme ovládat, pýcha nás
učiní bezmocnými.

Mějte oči upřené k nebi a pokorně si pamatujte, že je to Bůh, kdo žádá vaši svobodu – a je to On, kdo dává
vysvobození. Ve své svaté hodině dnes promluvte s Pánem o své pýše. Požádejte ho, aby vám ukázal, na
co jste příliš hrdí. Požádejte ho, aby vám ukázal pravou pokoru. Pak ho proste, aby vám dal odvahu a milost
žít dnes v opravdové pokoře.

%newday
\newpage
\section{Den 12 - ZKLAMÁNÍ Z HŘÍCHU}
\zacatekDruhyTyden
\subsection*{Čtení na den}
\textbf{Exodus 5,5-21}
\newline
\textit{
\textsuperscript{5}A farao pokračoval: „Hle, lidu země je teď mnoho, a vy chcete, aby nechali svých robot?“
\textsuperscript{6}Onoho dne přikázal farao poháněčům lidu a dozorcům:
\textsuperscript{7}„Propříště nebudete vydávat lidu slámu k výrobě cihel jako dříve. Ať si jdou slámu nasbírat sami!
\textsuperscript{8}A uložíte jim dodat stejné množství cihel, jaké vyráběli dříve. Nic jim neslevujte, jsou líní. Proto křičí: Pojďme obětovat svému Bohu.
\textsuperscript{9}Ať na ty muže těžce dolehne otročina, aby měli co dělat a nedali na lživé řeči.“
\textsuperscript{10}Poháněči lidu a dozorci vyšli a ohlásili lidu: „Toto praví farao: Nedám vám žádnou slámu.
\textsuperscript{11}Sami si jděte nabrat slámu, kde ji najdete. Ale z vaší pracovní povinnosti se nic nesleví.“
\textsuperscript{12}Lid se rozběhl po celé egyptské zemi, aby na strništích sbíral slámu.
\textsuperscript{13}Poháněči je honili: „Plňte svůj denní úkol, jako když sláma byla.“
\textsuperscript{14}Dozorci z řad Izraelců, které nad nimi ustanovili faraónovi poháněči, byli biti. Vytýkalo se jim : „Proč jste v těchto dnech nevyrobili tolik cihel jako dříve?“
\textsuperscript{15}Dozorci z řad Izraelců tedy přišli a úpěli před faraónem: „Proč se svými otroky takhle jednáš?
\textsuperscript{16}Tvým otrokům se nedodává sláma, ale pokud jde o cihly, poroučejí nám: ‚Dělejte!‘ Hle, tvoji otroci jsou biti a tvůj lid bude pykat za hřích.“
\textsuperscript{17}Farao odpověděl: „Jste lenoši líní, proto říkáte: ‚Pojďme obětovat Hospodinu.‘
\textsuperscript{18}Hned jděte dělat! Sláma vám dodávána nebude, ale dodávku cihel odvedete.“
\textsuperscript{19}Dozorci z řad Izraelců viděli, že je s nimi zle, když bylo řečeno: „Nesmíte snížit svůj denní úkol výroby cihel.“
\textsuperscript{20}Když vycházeli od faraóna, narazili na Mojžíše a Árona, kteří se s nimi chtěli setkat.
\textsuperscript{21}Vyčítali jim: „Ať se nad vámi ukáže Hospodin a rozsoudí. Vy jste pokáleli naši pověst u faraóna a jeho služebníků. Dali jste jim do ruky meč, aby nás povraždili.“
}

\subsection*{Reflexe}

Izraelci si právem stěžují na to, že by měli vyrábět stejné množství cihel, i když dostávají méně slámy. Požadavky
otrokářů se staly mnohem nedosažitelnějšími. V jejich případě je otrokářem faraon, zatímco v našem případě je
otrokářem Satan. Naše závislosti a zvyky v nás vytvářejí neuhasitelnou touhu. Jak však ukazuje dnešní čtení,
poddávat se těmto našim pokušením je hluboce neuspokojivé.

Bez ohledu na to, k čemu jsme zotročeni, jsme vždy v pokušení usilovat o rychlou a snadnou nápravu tím, že se
vzdáme naší touhy. Ironické je, že čím více se snažíme se jí vzdát, tím méně to je možné. Pokušení vytvářejí iluzi,
že budeme více naplněni, když se příště jen trochu víc dopřejeme. Po čase nás pokušení vtáhne hlouběji do otroctví
a dál od svobody, takže je pro nás těžší překonat. Každý z nás tak moc dobře ví, jak pravdivě slova svatého Pavla v
jeho dopise Římanům znějí: „Nepoznávám se ve svých skutcích; vždyť nedělám to, co chci, nýbrž to, co
nenávidím,“ (Řím 7,15).

Nikdy nezapomeňte na tuto lež, že k vám neustále mluví vaše pokušení. Udržujte ji v čele své mysli jako hnací sílu
vedoucí ke svobodě. Je to frustrace z této lži, která vás sem přivedla, a bude to frustrace z této lži, díky které budete
ochotni jít do pouště, ochotni následovat tento náročný plán svobody.

%newday
\newpage
\section{Den 13 - VHLED DÍKY SLABOSTI}
\zacatekDruhyTyden
\subsection*{Čtení na den}
\textbf{Exodus 5,22-6,12}
\newline
\textit{
\textsuperscript{22}Mojžíš se obrátil k Hospodinu a řekl: „Panovníku, proč jsi dopustil na tento lid zlo? Proč jsi mě vlastně poslal?
\textsuperscript{23}Od chvíle, kdy jsem předstoupil před faraóna, abych mluvil tvým jménem, nakládá s tímto lidem ještě hůře. A ty svůj lid stále nevysvobozuješ.“
\textsuperscript{1}Hospodin Mojžíšovi odvětil: „Nyní uvidíš, co faraónovi udělám. Donutím ho , aby je propustil; donutím ho , aby je vypudil ze své země.“
\textsuperscript{2}Bůh promluvil k Mojžíšovi a ujistil ho: „Já jsem Hospodin.
\textsuperscript{3}Ukázal jsem se Abrahamovi, Izákovi a Jákobovi jako Bůh všemohoucí. Ale své jméno Hospodin jsem jim nedal poznat.
\textsuperscript{4}Ustavil jsem s nimi také svou smlouvu, že jim dám kenaanskou zemi, zemi jejich putování, kde pobývali jako hosté.
\textsuperscript{5}Rovněž jsem uslyšel sténání Izraelců, které si Egypťané podrobili v otroctví, a rozpomenul jsem se na svou smlouvu.
\textsuperscript{6}Proto řekni Izraelcům: Já jsem Hospodin. Vyvedu vás z egyptské roboty, vysvobodím vás z vašeho otroctví a vykoupím vás vztaženou paží a velkými soudy.
\textsuperscript{7}Vezmu si vás za lid a budu vám Bohem. Poznáte, že já jsem Hospodin, váš Bůh, který vás vyvede z egyptské roboty.
\textsuperscript{8}Dovedu vás do země, kterou jsem přísežně slíbil dát Abrahamovi, Izákovi a Jákobovi. Vám ji dám do vlastnictví. Já jsem Hospodin.“
\textsuperscript{9}Mojžíš to tak Izraelcům vyhlásil, ale ti nebyli pro malomyslnost a tvrdou otročinu s to Mojžíšovi naslouchat.
\textsuperscript{10}Hospodin dále mluvil k Mojžíšovi:
\textsuperscript{11}„Předstup před faraóna, krále egyptského, a vyřiď mu , ať propustí Izraelce ze své země.“
\textsuperscript{12}Mojžíš Hospodinu namítl: „Když mi nenaslouchají Izraelci, jak by mě poslechl farao! Nejsem způsobilý mluvit.“
}

\subsection*{Reflexe}

Často je „Bůh Starého zákona“ obviněn z tvrdosti a krutosti, a dnešní Písmo zdálky potvrzuje tento stereotyp. Ale
podívej se znovu. Kdyby se Bůh unáhlil a vyřešil problémy, kterým čelí Izraelité, čeho by dosáhl? Izraelci by si
povzdechli úlevou a rychle zapomněli na své nesnáze a na to, kdo jim dává svobodu. Nenaučili by se, že Bůh jejich
otců je jejich Bůh. Nenaučili by se nic, a nic by nezískali. Izraelitům je v tíživé situaci dána možnost naučit se něco
velmi cenného: Bůh je vezme za svůj lid a bude jejich Bohem. Povede je ke svobodě a ukončí jejich zajetí.
Bezpochyby uvidí, že Bůh je Bůh a že bez Něj nemohou nic dělat.

Chvíli přemýšlejte o tom, co by byl člověk bez slabostí. Pravděpodobně by byl pyšnou a povýšenou šelmou, která si
myslí, že nepotřebuje Boha, a věří, že se dokáže vysvobodit jak v tomto světě, tak i v příštím. Naše slabost nám tak
poskytuje vhled. Pokud svou slabost necháte, naučí vás, že potřebujete Boha. Slabost vás může naučit obrátit se
k Bohu a požádat ho o svobodu, místo toho abyste věci brali sami do svých rukou.

Vaše slabost vás sem přivedla. Vaše slabost (ne váš hřích) je Božím darem pro vás. Pokud dbáte na vhled, který vám
dává (potřeba Boha), zůstanete na cestě ke svobodě. Poděkujte dnes Pánu za dar své slabosti, který vás neustále
přivádí zpět do Jeho milujícího náručí.

%newday
\newpage
\section{Den 14 - VLIV OTCE}
\zacatekDruhyTyden
\subsection*{Čtení na den}
\textbf{Exodus 6,13-27}
\newline
\textit{
\textsuperscript{13}Ale Hospodin Mojžíšovi a Áronovi domluvil a dal jim příkazy pro Izraelce i pro faraóna, krále egyptského, aby připravili odchod Izraelců z egyptské země.
\textsuperscript{14}Toto jsou představitelé otcovských rodů: Rúbenovci, potomci Izraelova prvorozeného: Chanók a Palú, Chesrón a Karmí. To jsou čeledi Rúbenovy.
\textsuperscript{15}Šimeónovci: Jemúel, Jamín, Ohad, Jakín, Sóchar a Šaul, syn Kenaanky. To jsou čeledi Šimeónovy.
\textsuperscript{16}Toto jsou jména Léviovců podle jejich rodopisu: Geršón, Kehat a Merarí. Lévi byl živ sto třicet sedm let.
\textsuperscript{17}Geršónovci: Libní a Šimeí podle svých čeledí.
\textsuperscript{18}Kehatovci: Amrám, Jishár, Chebrón a Uzíel. Kehat byl živ sto třicet tři léta.
\textsuperscript{19}Meraríovci: Machlí a Muší. To jsou lévijské čeledi podle jejich rodopisu.
\textsuperscript{20}Amrám si vzal za ženu Jókebedu, svou tetu. Ta mu porodila Árona a Mojžíše. Amrám byl živ sto třicet sedm let.
\textsuperscript{21}Synové Jishárovi: Kórach, Nefeg a Zikrí.
\textsuperscript{22}Synové Uzíelovi: Míšael, Elsáfan a Sitrí.
\textsuperscript{23}Áron si vzal za ženu Elíšebu, dceru Amínadabovu, sestru Nachšónovu. Ta mu porodila Nádaba, Abíhúa, Eleazara a Ítamara.
\textsuperscript{24}Synové Kórachovi: Asír, Elkána a Abíasaf. To jsou kórachovské čeledi.
\textsuperscript{25}Eleazar, syn Áronův, si vzal za ženu jednu z dcer Pútíelových. Ta mu porodila Pinchasa. To jsou představitelé lévijských rodů podle svých čeledí.
\textsuperscript{26}Z tohoto pokolení pocházejí ten Áron a Mojžíš, k nimž mluvil Hospodin: „Vyveďte z egyptské země Izraelce seřazené po oddílech.“
\textsuperscript{27}Oni to byli, kdo mluvili k faraónovi, králi egyptskému, že mají vyvést Izraelce z Egypta. To tedy byli Mojžíš a Áron.
}

\subsection*{Reflexe}

Při zdlouhavých biblických rodokmenech většinou ztrácíme chuť číst dál. Jména jsou cizí a význam rodinných linií je dávno
ztracen v historii. Všechno, co je obsaženo v Písmu Svatém, však ukazuje církvi a věřícím důležité pravdy. Rodokmeny nás
konkrétně spojují se sliby a smlouvami, které učinil Bůh svým lidem. Také nám připomínají – jak připomínali předkům – naši
důstojnost a náš budoucí domov.

Vaše vlastní rodinná linie vás naučí mnoho o vás samých. Síla charakteru, temperament, osobnost, vaše osoba je určena vašimi
předky. Přesněji, navzdory tomu, co nám říká naše kultura, se můžete dozvědět mnoho o sobě a o svém životě od svého otce.

Otcové, možná více, než víme, mají významný dopad na jejich potomstvo – někdy pozitivní, někdy negativní. Převážně od
našich otců se učíme sebeovládání, sebedůvěry, způsobu, jakým komunikujeme s vnějším světem. Pokud bojujete v některé z
těchto oblastí, znamená to, že vás otec selhal? To je zásadní otázka, které je každý člověk vystaven.

Tohle není místo pro vyčerpávající pojednání o otcovství. Když však pracujete, abyste si lépe porozuměli a usilovali o svobodu
sebeovládání, budete pravděpodobně považovat za přínosné přemýšlet o svém otci a vašem vztahu s ním. Vyzkoušejte toto
cvičení: pokud je to možné, požádejte svého otce, aby popsal svého vlastního otce (vašeho dědečka), a zjistíte, odkud pocházelo
mnoho rysů vašeho otce (pozitivních i negativních). Můžete také získat představu o tom, co předáváte (nebo co budete předávat)
svým vlastním dětem. Vnímejte toto cvičení pozitivně a neuchylujte se k obviňování, jde tu o pochopení. Pamatujte, že jste na
cestě k rozvoji ctnostných zvyků uprostřed bratrství. 

Tyto návyky budou darem pro vaše blízké, zejména pro vaše děti, bez
ohledu na to, jak staré jsou. Nemůžete změnit minulost, ale nyní máte příležitost pracovat s Pánem, abyste pro svou rodinu
vytvořili novou budoucnost.


% ===============================================
% ===== TRETI TYDEN
% ===============================================
%ukony
\newpage
\section*{Úkony (ukazatel cesty) pro 3. týden}

\textbf{Místo:} Egypt (jste v Egyptě)

Život se stal náročnějším. Pro Izraelity práce nepovolila. Faraon je odmítá nechat jít. I přes prvních pět ran se Mojžíši a Áronovi nedaří získat větší svobodu Izraelitům. První nadšení z bratrského exodu pro vás také možná již vyprchalo. Začínáte si uvědomovat tvrdou realitu mnoha týdnů před vámi. A co hůře, stejně jako Izraelitům, ani pro vás ještě neuběhlo tolik času na to, abyste začali pociťovat zisk svobody. Navzdory tomu všemu, komu se tento týden rozhodnete sloužit: Bohu, nebo faraonovi?

\subsection*{1. Pokračujte ve zkoumání svého dne}
Nejen, že vám pomůže projít skrz těchto 90 dní, ale udrží vás to svobodnými v 91. dnu. Dobře praktikujte noční examen, takže si vytvoříte dobrý zvyk pro 91. den. (Nevíte, jak dělat zkoumání na konci dne? Podívejte se do kapitoly Jak se modlit noční examen v Průvodci terénem.)
\subsection*{2. Přistupujte upřímně k vaší denní svaté hodině}
Čas v kontemplativní modlitbě je rozhodující při exodu. Bůh dává vysvobození. Bůh vás povede, kam potřebujete. Bůh vám bude připomínat svou lásku k vám a vaší pravou hodnotu jako Jeho prvorozeného syna. Ale neuslyšíte žádnou z těchto věcí, pokud si každý den nevyhradíte čas na ztišení své mysli a naslouchání.
\subsection*{3. Neodbývejte úkony/úkoly (don’t cut corners)}
Touto dobou už jste se dobře seznámili s asketickými disciplínami. Víte, co je na nich těžké a co jednoduché. Čím více disciplíny znáte, tím jednodušší bude je odbývat (cut corners-zkracovat si cestu). Chcete zůstat v Egyptě, nebo chcete být osvobozeni? (Pro více informací o důležitosti asketismu nahlédněte do Pilířů Exodu 90 v příručce pod pilířem Bratrství v příručce Exodu.)
\subsection*{4. Pamatujte si své \textit{proč}.}
Vzpomeňte si, proč jste začali Exodus 90. Pokud své proč zapomenete, pravděpodobně nebudete schopni/moci tuto cestu dokončit. Lákavost pohodlí je jednoduše příliš silná, než abyste ji překonali bez proč, které za to stojí. (Nezapsali jste si své proč? Vraťte se zpět do části příručky Exodu 90 nazvané Co je vaše proč a napište si své proč předtím, než budete pokračovat dál.)
\subsection*{5. Zůstaňte radostní}
Pokud budete držet své proč v čele své mysli a zůstanete pevní v modlitbě, vaše naděje na svobodu nebude jiná než vysoká. Jít exodem je náročné. Naštěstí je to Bůh, kdo dává vysvobození, které hledáme. Egyptské rány nám ukazují, že také Bůh pracuje na vaší svobodě.

\subsection*{Modlitba}
Modlete se, aby Pán osvobodil vás a vaše bratrství \newline
Modleme se za svobodu všech mužů v exodu, stejně tak, jako se oni modlí za vás.\newline
Ve jménu Otce i Syna i Ducha svatého … Otče náš… Ve jménu Otce i Syna i Ducha svatého … Amen.

\newpage


%newday
\newpage
\section{Den 15 - BŮH SI PŘEJE VAŠI SVOBODU}
\zacatekTretiTyden
\subsection*{Čtení na den}
\textbf{Exodus 6,28-7,7}
\newline
\textit{
\textsuperscript{28}To bylo tehdy, když Hospodin mluvil k Mojžíšovi v egyptské zemi.
\textsuperscript{29}Hospodin promluvil k Mojžíšovi: „Já jsem Hospodin! Řekni faraónovi, králi egyptskému, všechno, co k tobě mluvím.“
\textsuperscript{30}Mojžíš však Hospodinu namítl: „Nejsem způsobilý mluvit. Jak by mě farao poslechl?“
\textsuperscript{1}Hospodin řekl Mojžíšovi: „Pohleď, ustanovil jsem tě, abys byl pro faraóna Bohem, a Áron, tvůj bratr, bude tvým prorokem.
\textsuperscript{2}Ty mu povíš všechno, co ti přikážu, a Áron, tvůj bratr, bude mluvit s faraónem, aby propustil Izraelce ze své země.
\textsuperscript{3}Já však zatvrdím faraónovo srdce a učiním v egyptské zemi mnoho svých znamení a zázraků.
\textsuperscript{4}Farao vás neposlechne, ale já vložím na Egypt svou ruku. Vyvedu zástupy svého lidu, syny Izraele, z egyptské země, ale ji postihnu velkými soudy.
\textsuperscript{5}Egypťané poznají, že já jsem Hospodin, až vztáhnu svou ruku na Egypt a vyvedu Izraelce z jejich středu.“
\textsuperscript{6}Mojžíš a Áron učinili přesně tak, jak jim Hospodin přikázal.
\textsuperscript{7}Mojžíšovi bylo osmdesát let a Áronovi osmdesát tři léta, když mluvili s faraónem.
}

\subsection*{Reflexe}
Podívejme se na věc z této perspektivy – Bůh žádá Mojžíše, aby čelil nejmocnějšímu vládci světa v samém srdci
jeho království, kde je obklopen svými lidmi. Zde má Mojžíš říci faraonovi, co má dělat. Asi není divu, že se
Mojžíš zachová jako většina lidí, když čelí znepokojivému úkolu: couvá zpět. Mojžíš vysvětluje Bohu, že je vázán
jazykem a není způsobilý mluvit. Bůh však jeho lidskou omluvu nepřijímá.

Místo toho Bůh, vládce veškerého stvoření, říká slabému Mojžíšovi: „Ustanovil jsem tě, abys byl pro faraóna
Bohem.“ Svatý Ambrož nám říká, že Mojžíšova ctnost daleko převyšuje faraonovu moc. Mojžíšovi není dáno
nadmíru. Jeho vášně nad ním nevládnou. Je to muž, který „ostře kritizuje, že jeho tělo bylo ztělesněno autoritou,
která byla téměř královská“.Zatímco Mojžíšova sebedůvěra ochabuje, Hospodin má ve svého syna veškerou
důvěru.

Tak je to i s vámi. Vy, stejně jako Mojžíš, ovládáte své vášně a trpíte bolestmi askeze, ztělesňujete Božího syna,
královského a mocného oproti směšnému faraonovi světskosti a neřestí. Mohl jste to o sobě říci před dvaceti dny?
Způsob vašeho života se mění. Následujte Kristův plán modlitby, askeze a bratrství a budete i nadále dostávat
milost vítězit nad svými vášněmi.

Děkujte Bohu za úspěch, který jste doposud měli. Nepřetržitá vděčnost bude podporovat vaši vděčnost uprostřed
výzev tohoto duchovního cvičení.

%newday
\newpage
\section{Den 16 - SPÁSA SKRZE KŘÍŽ}
\zacatekTretiTyden
\subsection*{Čtení na den}
\textbf{Exodus 7,8-13}
\newline
\textit{
\textsuperscript{8}Hospodin dále řekl Mojžíšovi a Áronovi:
\textsuperscript{9}„Až k vám farao promluví: ‚Prokažte se nějakým zázrakem,‘ řekneš Áronovi: ‚Vezmi svou hůl a hoď ji před faraóna,‘ a stane se drakem.“
\textsuperscript{10}Mojžíš s Áronem tedy předstoupili před faraóna a učinili, jak Hospodin přikázal. Áron hodil svou hůl před faraóna i před jeho služebníky a ona se stala drakem.
\textsuperscript{11}Farao však také povolal mudrce a čaroděje, a egyptští věštci učinili svými kejklemi totéž.
\textsuperscript{12}Hodili každý svou hůl na zem a ony se staly draky. Ale Áronova hůl jejich hole pohltila.
\textsuperscript{13}Srdce faraónovo se však zatvrdilo a neposlechl je, jak Hospodin předpověděl.
}

\subsection*{Reflexe}

V dnešním čtení faraon přikazuje svým věštcům, aby zopakovali znamení, které provedl Áron.
Dosvědčuje se, že je to pouhým pokusem zmást lidi a setřít moc Božího znamení. Faraon se sanží
nabídnout alternativy k Boží cestě. V dnešní době nalézáme moderní „faraony“ všude, snažící se o to
samé.

V Písmu Svatém Áronova hůl předznamenává kříž. Jedním z příkladů je použití Áronovy hole k rozdělení
Rudého moře umožňující přechod Izraelitů ke svobodě. Tento čin předznamenává křest, kde voda, skrze
kterou přecházíme ke svobodě, dostává svou moc z kříže. Neexistuje žádná jiná cesta ke spáse než skrze
kříž, ale stejně jako faraon, i naše kultura zoufale hledá alternativu – něco jednoduššího, něco
příjemnějšího, něco, co člověk dokáže ovládat.

Kolikrát jste hledali jinou cestu ke svobodě, než jste nakonec museli přijmout, že žádná „jiná cesta“
neexistuje?„Jako Mojžíš vyvýšil hada na poušti, tak musí být vyvýšen Syn člověka,“ (J 3,14). Podívejte
se na plán před vámi: modlitba, askeze, bratrství. Nemělo by být žádným překvapením, že cesta ke
svobodě, kterou lidem poskytuje Písmo, je cestou kříže.

Pokud dokážete unést odpověď našeho Pána, podívejte se dnes na kříž a zeptejte se ho, zda existuje jiný
způsob než právě on.

%newday
\newpage
\section{Den 17 - HOJNĚ NAPLNĚNÝ}
\zacatekTretiTyden
\subsection*{Čtení na den}
\textbf{Exodus 7,14-24}
\newline
\textit{ 
\textsuperscript{14}Hospodin řekl Mojžíšovi: „Srdce faraónovo je neoblomné. Nechce lid propustit.
\textsuperscript{15}Jdi k faraónovi ráno. Až půjde k vodě, postav se naproti němu na břehu Nilu a vezmi si do ruky hůl, která se proměnila v hada.
\textsuperscript{16}Řekneš mu: Hospodin, Bůh Hebrejů, mě k tobě posílá se vzkazem: Propusť můj lid, aby mi na poušti sloužil. Ale ty jsi dosud neposlechl.
\textsuperscript{17}Toto praví Hospodin: Podle toho poznáš, že já jsem Hospodin: Holí, kterou mám v ruce, teď udeřím do vody v Nilu, a ta se promění v krev.
\textsuperscript{18}Ryby, které jsou v Nilu, leknou a Nil bude páchnout. Marně budou Egypťané usilovat, aby se mohli napít vody z Nilu.“
\textsuperscript{19}Hospodin dále řekl Mojžíšovi: „Vyzvi Árona: ‚Vezmi svou hůl a vztáhni ruku nad egyptské vody, nad průplavy, nad říční ramena, nad jezera, vůbec nad všechny nahromaděné vody.‘ Stanou se krví. V celé egyptské zemi bude krev, i ve džberech a džbánech.“
\textsuperscript{20}Mojžíš a Áron učinili, jak Hospodin přikázal. Áron pozdvihl hůl a před očima faraóna a jeho služebníků udeřil do vody v Nilu a všechna voda Nilu se proměnila v krev.
\textsuperscript{21}Ryby v Nilu lekly, Nil začal páchnout a Egypťané nemohli vodu z Nilu pít. A krev byla v celé egyptské zemi.
\textsuperscript{22}Ale totéž učinili egyptští věštci svými kejklemi. Faraónovo srdce se zatvrdilo a neposlechl je, jak Hospodin předpověděl.
\textsuperscript{23}Farao se obrátil a vešel do svého domu, a ani toto si nevzal k srdci.
\textsuperscript{24}Všichni Egypťané kopali kolem Nilu, aby přišli na pitnou vodu, protože vodu z Nilu pít nemohli.
 }

\subsection*{Reflexe}

Bůh zjevuje všechny věci takové, jaké skutečně jsou. V této pasáži Bůh začíná řadu deseti ran, aby ukázal svou
moc nad faraonem a falešnými bohy Egypta. Jeho moc se ukazuje tak mocně, že se všechny vody Egypta
proměňují v krev, takže je Nil tak zkažený, že lidé nemají vodu k pití. Krev je zde symbolem tělesné existence –
hmoty lidské přirozenosti tohoto světa. Voda, zdroj lidského života, se stává zdrojem smrti a rozpadu.

V hebrejštině tento text naznačuje, že hrnce a nádoby používané pro vodu (a nyní naplněné krví) byly vyrobeny z
materiálu získaného ze stromů a kamenů, což byly shodou okolností stejné materiály, které Egypťané používali při
stavbě svých model. Jinými slovy, falešní bohové Egypťanů se nyní stali pro ně zdrojem smrti; nemohli je totiž
zachránit.

I dnes jste neustále v pokušení obrátit se k věcem těla –k sexu, moci, penězům– které vás odtrhnou od pravého
života a svobody. Nádoba vašeho života je často naplněna smrtí a úpadkem, spíše než životem. Přesto pohleďte na
jiný úryvek Písma uvedený v evangeliu sv. Jana (2. kapitola). Zde Kristus promění vodu, která je v nádobách k
čištění, na víno, v symbol života a radosti a znamení nového života, který nám nabízí v Duchu. Buďte pevní a
vězte, že pokud mu dovolíte, Ježíš také obrátí vodu vašeho života na víno. Přijďte dnes k Pánu. Požádejte ho, aby
vás tak hojně naplnil životem a radostí, že všichni lidé kolem vás budou moct vidět a vědět, co ve vás On činí.

%newday
\newpage
\section{Den 18 - ŽIJETE PRO OSTATNÍ? }
\zacatekTretiTyden
\subsection*{Čtení na den}
\textbf{Exodus 6,13-27}
\newline
\textit{ 
\textsuperscript{25}To trvalo plných sedm dní poté, co Hospodin zasáhl Nil.
\textsuperscript{26}Potom Hospodin řekl Mojžíšovi: „Předstup před faraóna a řekni mu: Toto praví Hospodin: Propusť můj lid, aby mi sloužil.
\textsuperscript{27}Budeš-li se zdráhat jej propustit, napadnu celé tvé území žábami.
\textsuperscript{28}Nil se bude žábami hemžit, vylezou a vniknou do tvého domu, do tvé ložnice a na tvé lože i do domu tvých služebníků a mezi tvůj lid, do tvých pecí a díží.
\textsuperscript{29}I po tobě, po tvém lidu a po všech tvých služebnících polezou žáby.“
\textsuperscript{1}Hospodin dále řekl Mojžíšovi: „Vyzvi Árona: ‚Vztáhni ruku se svou holí nad průplavy, nad říční ramena i nad jezera a vyveď na egyptskou zemi žáby.‘“
\textsuperscript{2}Áron vztáhl ruku nad egyptské vody a žáby vylézaly, až pokryly egyptskou zemi.
\textsuperscript{3}Ale totéž učinili věštci svými kejklemi a i oni vyvedli na egyptskou zemi žáby.
\textsuperscript{4}Tu povolal farao Mojžíše a Árona a řekl: „Proste Hospodina, aby mě i můj lid zbavil žab. Pak propustím lid, aby obětoval Hospodinu.“
\textsuperscript{5}Mojžíš faraónovi odvětil: „Rač mi sdělit , kdy mám prosit za tebe, za tvé služebníky a za tvůj lid, aby Hospodin vyhladil žáby u tebe i v tvých domech. Zůstanou jen v Nilu.“
\textsuperscript{6}Farao odpověděl: „Zítra.“ Mojžíš řekl: „ Ať je podle tvého slova, abys poznal, že nikdo není jako Hospodin, náš Bůh.
\textsuperscript{7}Žáby se stáhnou od tebe i z tvých domů, od tvých služebníků a od tvého lidu. Zůstanou jen v Nilu.“
\textsuperscript{8}Nato odešel Mojžíš s Áronem od faraóna a Mojžíš úpěnlivě volal k Hospodinu kvůli žábám, kterými faraóna postihl.
\textsuperscript{9}Hospodin učinil podle Mojžíšovy prosby a žáby v domech, ve dvorcích i na polích pošly.
\textsuperscript{10}Shrabali je na hromady a kupy a zápach z nich naplnil zemi.
\textsuperscript{11}Když však farao viděl, že nastala úleva, zůstal v srdci neoblomný a neposlechl je, jak Hospodin předpověděl.
\textsuperscript{12}Hospodin řekl Mojžíšovi: „Vyzvi Árona: ‚Vztáhni svou hůl a udeř do prachu na zemi!‘ Stanou se z něho po celé egyptské zemi komáři.“
\textsuperscript{13}I učinili tak. Áron vztáhl ruku s holí a udeřil do prachu na zemi a na lidech i na dobytku se objevili komáři. Po celé egyptské zemi se ze všeho prachu země stali komáři.
\textsuperscript{14}Když totéž chtěli učinit věštci svými kejklemi, totiž vyvést komáry, nemohli. A komáři byli na lidech i na dobytku.
\textsuperscript{15}Věštci tedy řekli faraónovi: „Je to prst Boží.“ Srdce faraónovo se však zatvrdilo a neposlechl je, jak Hospodin předpověděl.
 }

\subsection*{Reflexe}

Pokud nejste osmiletý chlapec, vyhlídka na nesčetné kvákající žáby ohromující zemi je v nejlepším případě
nepříjemná. Tato slizká scéna odhaluje něco o faraonovi jako člověku. Když mu Mojžíš nabídne úlevu od
obojživelníků, Faraon souhlasí s tím, že Mojžíš by měl zasáhnout a přimluvit se u Boha, ale ne teď – „zítra“. Jeho
arogantní lhostejnost k situaci jeho poddaných odhaluje tvrdost jeho srdce a jeho narcismus. Odmítá pokleknout
před Bohem bohů a být viděn s Ním jakýmkoliv způsobemspolupracovat. Kromě toho, když je tato rána stažena,
faraon nevykazuje žádnou vděčnost Mojžíšovi a ve své nafouklé hrdosti se vůbec nezajímá o Boha.

Jednou z nejzávažnějších chyb, které děláme, je,že nebereme v úvahu nikoho jiného než sebe samého. Příliš snadno
(i když jen obrazně) přehlížíme Boha vesmíru, naše ženy, farníky, syny a dcery. Toto nerespektování a přehlížení
ostatních je opakem toho, co to znamená být mužem. Kromě toho je tato naše nevšímavost Kristova těla – což je
také přehlížení Boha samotného– jistou cestou do pekla.

Chcete dosáhnout svého skutečného potenciálu a používat svou moc od Boha, jakou vám dal při vašem stvoření?
Pak musíte myslet na ostatní. Musítesesadit sami sebe z místa Boha a pozvednout ty kolem sebe. Tak půjdete
svatou cestou. Jak teď Hospodin vidí vaše činy? Nejste si jistí? Zeptejte se ho.

%newday
\newpage
\section{Den 19 - BŮH DÁVÁ VŠECHNY POTŘEBNÉ MILOSTI }
\zacatekTretiTyden
\subsection*{Čtení na den}
\textbf{Exodus 8,16-19}
\newline
\textit{
\textsuperscript{16}Hospodin řekl Mojžíšovi: „Za časného jitra se postav před faraóna, až vyjde k vodě. Řekneš mu: Toto praví Hospodin: Propusť můj lid, aby mi sloužil!
\textsuperscript{17}Jestliže můj lid nepropustíš, pošlu na tebe, na tvé služebníky, na tvůj lid i na tvé domy mouchy. Domy Egypťanů budou plné much, i ta půda, na které žijí .
\textsuperscript{18}Ale zemi Gošen, kde se zdržuje můj lid, v onen den podivuhodně odliším. Tam mouchy nebudou, abys poznal, že já jsem Hospodin i uprostřed této země.
\textsuperscript{19}Učiním rozdíl mezi lidem svým a lidem tvým. Toto znamení se stane zítra.“
  }

\subsection*{Reflexe}

Říká se, že „Bůh nám nikdy nedává víc, než můžeme zvládnout.“ To je daleko od pravdy. Podívej se na Mojžíše.
Bůh mu dal mnohem víc, než by mohl zvládnout. Kvůli tomu má Mojžíš dvě možnosti: může utéct a snažit se utěšit
sebe sama, s vědomím, že se Bůh zeptá jiných, nebo může věřit, že Bůh zasáhne a poskytne mu milost tam, kde
jeho lidská síla nestačí.
Snažit se být dobrým křesťanem může být v dnešní kultuře velmi náročné. Mnozí z nás jsou v pokušení uvěřit lži,
že kdybychom nikdy nepoznali Boha a jeho Církev, byli bychom mnohem šťastnější, protože bychom nemuseli
snášet život mnoha křížů. Myšlenka, že naše kříže nás zotročují, je ale daleko od pravdy.

\begin{minipage}{\dimexpr\textwidth-20pt}
  Dnes se upokojte v těchto slovech svatého Františka Saleského:
\begin{quote}
  \textit{Zapamatujte si tuto jednoduchou pravdu, která je nade všechny pochybnosti: Bůh dovoluje, aby mnoho obtíží trápilo ty, kteří mu chtějí sloužit, ale nikdy je nenechá padnout pod tíhou těchto obtíží, pokud Mu stále důvěřují… Nikdy, za žádných okolností nepodléhejte pokušení znechucení, ani pod lehce uvěřitelnou záminkou pokory.}
\end{quote}
\end{minipage}


Pokud toužíte sloužit Bohu, bude od vás požadovat víc, než dokážete zvládnout. V tomto bodě duchovního cvičení
jsme si této skutečnosti dobře vědomi. Ale Bůh nikdy nedopustí, abyste se potopili pod tíhou svých břemen, dokud
Mu důvěřujete. Znovu se podívejte na své proč. Věříte, že to Bůh dokončí? Pokud ne, teď je vhodný čas o tom
s Ním mluvit. Pokud ano, věnujte čas tomu, abyste Bohu vyjádřili svou vděčnost.

%newday
\newpage
\section{Den 20 - SLUŽTE HOSPODINU}
\zacatekTretiTyden
\subsection*{Čtení na den}
\textbf{Exodus 8,20-28}
\newline
\textit{
\textsuperscript{20}A Hospodin tak učinil. Dotěrné mouchy vnikly do domu faraónova, do domu jeho služebníků a na celou egyptskou zemi. Země byla těmi mouchami zamořena.
\textsuperscript{21}Tu povolal farao Mojžíše a Árona a řekl: „Nuže, přineste oběť svému Bohu zde v zemi.“
\textsuperscript{22}Mojžíš odpověděl: „Nebylo by správné, abychom to učinili. To, co máme obětovat Hospodinu, svému Bohu, je Egypťanům ohavností. Copak by nás neukamenovali, kdybychom před nimi obětovali, co je jim ohavností?
\textsuperscript{23}Odejdeme do pouště na vzdálenost tří dnů cesty a tam budeme obětovat Hospodinu, svému Bohu, jak nám nařídil.“
\textsuperscript{24}Farao řekl: „Propustím vás tedy, abyste obětovali Hospodinu, svému Bohu, na poušti. Jenom neodcházejte příliš daleko. Proste za mne.“
\textsuperscript{25}Mojžíš odvětil: „Až od tebe odejdu, budu prosit Hospodina a zítra odletí mouchy od faraóna, od jeho služebníků i od jeho lidu. Jen ať nás opět farao neobelstí, že by nechtěl propustit lid, aby obětoval Hospodinu.“
\textsuperscript{26}Pak Mojžíš od faraóna odešel a prosil Hospodina.
\textsuperscript{27}A Hospodin učinil, jak Mojžíš řekl. Mouchy odletěly od faraóna, od jeho služebníků i od jeho lidu. Ani jediná nezůstala.
\textsuperscript{28}Ale farao zůstal v srdci neoblomný i tentokrát a lid nepropustil.
  }

\subsection*{Reflexe}
Nyní jsme dospěli ke skutečného smyslu a účelu celé ságy zaznamenané v knize Exodus. Bůh po
faraonovi požaduje: „Propusť můj lid, aby mi mohl sloužit.“ Mojžíš má v úmyslu, na základě Božího
příkazu, vzít lid na třídenní cestu do pouště, kde přinesou oběti, které jsoupro Egypťany „ohavností“.
Jinými slovy vezmou zvířata, která Egypťané uctívají, a povraždí je. Tato oběť má Izraelcům dokázat, že
bohové Egypťanů jsou pouhými tvory, a ne jediným pravým Bohem.

Možná, že my, moderní muži, jsme příliš vzdělaní na to, abychom uctívali ovce a dobytek. V naší
aroganci a kultivovanosti však stále uctíváme falešné bohy či modly. Nejčastěji jsou to modly ve formě
peněz, sexu, moci, sportu a zábavy. Pokud čteme Písmo důkladně, musíme vidět, že Bůh se nejvíce stará
o Izraelity a jejich svobodu. Jeho hlavním zájmem není svoboda od otrokářů, kteří nařizují, co mají celý
den dělat. Jeho zájmem je spíše svoboda jejich duší. Zotročení si však přece nezaslouží zatracení. Ale
svobodné rozhodnutí uctívat modly namísto Boha? To už je vážné. Po 400 letech v Egyptě Izraelité
uctívají egyptské modly. Toto uctívání zotročilo jejich duše a bránilo jim od správného uctívání jediného
pravého Boha.

Povšimněte si také, že i když se faraon unavuje a dává jim povolení jít do divočiny, nechce, aby Izraelité
šli „moc daleko“. Nedovolí, aby se mu jeho pracovní síla vymkla z rukou. Musí zůstat zotročení.
Přemýšlejte o tom, kdy se vzdáváte věcí, které vás zotročují. Satan vám šeptá: „Běžte a dejte si dovolenou
od vašeho otroctví třeba na týden, po dobu postní, nebo dokonce na devadesát dní. Když se pak vrátíte
zpět, otroctví bude pro vás ještě horší.“ Svobodu nelze vyhrát a navždy ochránit ve stanoveném čase.

Těchto devadesát dní má sloužit jako skvělý začátek, čas očištění, a jako připomínka. Potrvá vám však
celý život věrnosti a spoléhání se na Boha, abyste zůstali svobodným člověkem.
Tato skutečnost by vás neměla vést ke smutku nebo zoufalství. Měla by vám přinést větší horlivost hledat
celoživotní svobodu. Pokud jste odrazeni, přineste to Pánu a dejte mu prostor, aby k vám mluvil pravdu.

Pokud jste plní nadšení, chvalte Boha za tento dar. Obraťte se na své bratry, zejména na vaši kotvu. Ne
všichni muži budou tak nadšení a horliví jako vy. Někteří mohou být dokonce v pokušení přestat. Podělte
se s nimi o svou radost a zápal.

%newday
\newpage
\section{Den 21 - ODDĚLENI PRO BOHA}
\zacatekTretiTyden
\subsection*{Čtení na den}
\textbf{Exodus 9,1-7}
\newline
\textit{
\textsuperscript{1}Hospodin řekl Mojžíšovi: „Předstup před faraóna a promluv k němu: Toto praví Hospodin, Bůh Hebrejů: Propusť můj lid, aby mi sloužil!
\textsuperscript{2}Budeš-li se zdráhat jej propustit a zatvrdíš-li se proti nim ještě víc,
\textsuperscript{3}tu na tvá stáda, která jsou na poli, na koně, na osly, na velbloudy, na skot i na brav, dolehne Hospodinova ruka velmi těžkým morem.
\textsuperscript{4}Hospodin však bude podivuhodně rozlišovat mezi stády izraelskými a stády egyptskými, takže nezajde nic z toho, co patří Izraelcům.
\textsuperscript{5}Hospodin také určil lhůtu: Zítra toto učiní Hospodin v celé zemi.“
\textsuperscript{6}A nazítří to Hospodin učinil. Všechna egyptská stáda pošla, ale z izraelských stád nepošel jediný kus .
\textsuperscript{7}Farao si to dal zjistit, a vskutku z izraelských stád nepošel jediný kus ; přesto zůstalo srdce faraónovo neoblomné a lid nepropustil.
  }

\subsection*{Reflexe}

Poselství poslané faraonovi je naprosto jasné: Bůh chce oddělit Izraelity od egyptského království. Izraelci nepatří
faraonovi, ale pouze Bohu. Bůh důrazně varuje faraóna, že bude„rozlišovat mezi stády izraelskými a stády
egyptskými, takže nezajde nic z toho, co patří Izraelcům“. Otec chrání své syny.

Stejně jako Izraelci byli odděleni, tak i vy jste byli odděleni svým křtem. Je snadné zapomenout na sílu a účinky
křtu. Když jste byly křtem ponořeni do Krista, buď jako dítě nebo jako dospělý, stali jste se syny nebeského Otce.
Stali jste se posvátnými a dostali jste všechno, co potřebujete k účasti na Božském životě.

Zkoumejte svůj život. Uvědomujete si, co to znamená, že je váš život posvátný? Kalich, který užívá kněz v
posvátné liturgii, nemůže být nikdy použit pro světskou činnost. Je posvátný a musí se s ním tak zacházet. Stejně
tak vás křest oddělil; zůstáváte uprostřed lidí, ale jste odnynějška odděleni pro službu Pánu. Váš úděl je být s
Bohem. Proto musíte odolat pokušení honbě za pozemskými věcmi, zvláště když vylučují Boží plán pro váš život.

Čím více se vyrovnáte se svou pravou identitou, tím více si uvědomujete iracionalitu honby za pozemskými věcmi,
jako je sex, moc, peníze. Proto musíme v našich srdcích často rozdmýchávat milost našeho křtu. Musíme klást
nárok na naše křestní právo a naše věčné dědictví. Musíme zůstat oddaní Bohu / Musíme zůstat
vyčleněni/odděleni/posvěceni pro Boha

Pohovořte s Bohem dnes o vaší podvátnosti a vašem oddání Jemu. Zamyslete se nad tím, jak žijete „odděleni pro
Boha“ a jak v tom selháváte. Buďte konkrétní, pozorně naslouchejte a buďte ochotni a připraveni se změnit, jak
Pán žádá. Bude to znamenat další pevný krok na cestě ke svobodě.

\end{document}